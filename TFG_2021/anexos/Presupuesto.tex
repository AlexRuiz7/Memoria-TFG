\chapter{Presupuesto} \label{cap:presupuesto}

\section{Recursos físicos}

\begin{table}[H]
\centering
\begin{tabular}{lr}
\hline
\multicolumn{1}{c}{\textbf{Recurso}} & \multicolumn{1}{c}{\textbf{Coste (€)}} \\ \hline
Limitador LM7                        & Gratuito (GranaSAT)                    \\
Limitador LM11 (prototipo)           & Gratuito (GranaSAT)                    \\
Amplificador y/o altavoces           & Gratuito (GranaSAT)                    \\
Sonómetro profesional                & Gratuito (GranaSAT)                    \\
Micrónfono                           & Gratuito (GranaSAT)                    \\
Cables balanceados XLR               & Gratuito (GranaSAT)                    \\
Ordenador portátil                   & 580                                    \\
Ordenador de torre                   & 420                                    \\ \hline
\multicolumn{1}{r}{\textbf{TOTAL}}   & \textbf{1 000}
\end{tabular}%
\caption{Costes hardware.}
\label{tab:presupuesto-hardware}
\end{table}

Los recursos hardware relativos al limitador no han supuesto un coste adicional ya que los proporciona la empresa para la que se realiza el proyecto. El hardware del nuevo prototipo queda fuera del ámbito de este proyecto (además, el coste del prototipo se desconoce completamente). Aún así, se incluyen en el listado de recursos hardware los componentes que han resultado esenciales para el desarrollo del proyecto, de forma simbólica, con un coste de 0 €. \\
Al margen de esto, ha sido necesario un ordenador portátil y otro de sobremesa para trabajar en el proyecto, los cuales se incluyen en el presupuesto como parte de equipamiento técnico.

\section{Recursos humanos}

Para estimar los costes del personal, se ha consultado la página web \url{www.indeed.com} con el fin de obtener datos referente a los salarios de los profesionales requeridos para el proyecto. Esa fuente se considera fiable al proporcionar información salarial estimada a partir de empleados, ofertas de empleo o directamente desde empresas. \\
Última actualización: 13 de agosto de 2021.

Los perfiles buscados se describen en el listado siguiente, mientras que el desglose del coste de recursos humanos se representa en la tabla \ref{tab:presupuesto-humanos}
\begin{enumerate}
	\item Junior Software Engineer:
	\begin{enumerate}
		\item Con un salario de 18 000 € por año (10 €/h en un contrato de 1 800 horas).
		\item Trabajará en el proyecto un total de 600 horas a media jornada, es decir, 5 horas al día.
		\item Implementará el sistema.
	\end{enumerate}

	\item Senior Software Engineer:
	\begin{enumerate}
		\item Con un salario de 45 000 € por año (25 €/h en un contrato de 1 800 horas).
		\item Trabajará en el proyecto un total de 240 horas, es decir, 2 horas al día.
		\item Investigará el ecosistema del proyecto y diseñará el sistema.
	\end{enumerate}
\end{enumerate}

%Junior 600h (5h/día durante 6 meses) * 10€/h = 6000€
%Senior 240h (2h/día durante 6 meses) * 25€/h = 6000€

\begin{table}[H]
\centering

\begin{tabular}{lcr}
\hline
\multicolumn{1}{c}{\textbf{Posición}} & \textbf{Tiempo (h)}     & \multicolumn{1}{c}{\textbf{Coste (€)}} \\ \hline
Junior Software Engineer              & \multicolumn{1}{r}{600} & 6 000                                   \\
Senior Software Engineer              & \multicolumn{1}{r}{240} & 6 000                                   \\ \hline
                                      & \textbf{TOTAL}          & \textbf{12 000}
\end{tabular}%
\caption{Costes humanos}
\label{tab:presupuesto-humanos}
\end{table}

\section{Software}

Para la realización del proyecto se han usado herramientas software gratuitas. La mayor parte de ellas son herramienta de gestión y productividad, como los clientes SSH. En el ámbito de la ingeniería se ha utilizado la herramienta \textit{Visual Paradigm} para la generación del diagramas y el IDE \textit{VSCode} para el desarrollo de código.

La comunicación con el equipo ha tenido lugar de forma presencial principalmente, mediante la realización de reuniones semanales en el laboratorio de GranaSAT. Para la comunicaciones en remoto se ha utilizado la herramienta de mensajería instantánea Telegram, a través de un grupo creado para este propósito y en el que se encuentran los miembros involucrados en el proyecto.

\begin{table}[h]
\centering
\begin{tabular}{lll}
\hline
\multicolumn{1}{c}{\textbf{Software}} & \multicolumn{1}{c}{\textbf{Coste (€)}} & \multicolumn{1}{c}{\textbf{Categoría}} \\ \hline
Visual Studio Code                    & Gratuito  &  IDE            \\
Visual Paradigm CE                    & Gratuito  &  Diagramas      \\
Git                                   & Gratuito  &  Versionado     \\
OpenAPI                               & Gratuito  &  Desarrollo     \\
Docker                                & Gratuito  &  Desarrollo     \\
Snowflake SSH Client                  & Gratuito  &  Conectividad   \\
MobaXterm SSH Client                  & Gratuito  &  Conectividad   \\
Cisco AnyConnect                      & Gratuito  &  Conectividad   \\
MikTex                                & Gratuito  &  Documentación  \\
TexStudio                             & Gratuito  &  Documentación  \\
Inkscape                              & Gratuito  &  Diagramas      \\
Google Drive                          & Gratuito  &  Productividad  \\
Telegram                              & Gratuito  &  Comunicación   \\
\hline
\end{tabular}%
\caption{Costes software}
\label{tab:presupuesto-software}
\end{table}

%\begin{table}[h]
%\centering
%\begin{tabular}{ll}
%\hline
%\multicolumn{1}{c}{\textbf{Software}} & \multicolumn{1}{c}{\textbf{Coste (€)}} \\ \hline
%Visual Studio Code                    & Gratuito                               \\
%Visual Paradigm Community Edition     & Gratuito                               \\
%Git                                   & Gratuito                               \\
%OpenAPI                               & Gratuito                               \\
%Docker                                & Gratuito                               \\
%Snowflake SSH Client                  & Gratuito                               \\
%MobaXtrem SSH Client                  & Gratuito                               \\
%TotalCommander                        & Gratuito                               \\
%MikTex                                & Gratuito                               \\
%TexStudio                             & Gratuito                               \\
%Inkscape                              & Gratuito                               \\
%Google Spreadsheet                    & Gratuito                               \\ \hline
%\end{tabular}%
%\caption{Costes software}
%\label{tab:presupuesto-software}
%\end{table}

\section{Presupuesto final}

Teniendo en cuenta los costes calculados anteriormente, el presupuesto final para poder llevar a cabo este proyecto asciende a un total de \textbf{13 000 €}.