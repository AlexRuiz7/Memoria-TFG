
%Para hacer el glosario

%\newacronym[<options>]{<label>}{<abbrv>}{<long>}

%is a shortcut for:

%\newglossaryentry{<label>}{type=\acronymtype, name={<abbrv>}, description={<long>}, text={<abbrv>}, first={<long> (<abbrv>)}, plural={<abbrv>s}, firstplural={<long>s (<abbrv>s)}, <options>}

% -------------------------- %
%          ACRONIMOS
% -------------------------- %
\newacronym{GPIB}{GPIB}{General Purpose Interface Bus}

\newacronym{ECTD}{ECTD}{Electronics and Computer Technology Department}

\newacronym{PC}{PC}{Personal Computer}

\newacronym{PID}{PID}{Proportional Integral Derivative}

\newacronym{ue}{UE}{Unión Europea}

\newacronym{fortranA}{FORTRAN}{Formula Translating System}

\newacronym{OpenSource}{Open-Source}{Programa de Código Abierto}

\newacronym{GUI}{GUI}{Graphical User Inteface}

\newacronym{UART}{UART}{Universal Asynchronous Receiver-Transmitter}

\newacronym{TTL}{TTL}{Time To Live}

\newacronym{SPI}{SPI}{Serial Peripheral Interface}

\newacronym{USB}{USB}{Universal Serial Bus}

\newacronym{GPIO}{GPIO}{General Purpose Input Output}

\newacronym{LCD}{LCD}{Liquid Crystal Display}

\newacronym{TCP}{TCP}{Transmission Control Protocol}

\newacronym{IP}{IP}{Internet Protocol}

\newacronym{SSL/TLS}{SSL/TLS}{Secure Socket Layer / Transport layer Security}

\newacronym{SSH}{SSH}{Secure SHell}

\newacronym{VNC}{VNC}{Virtual Network Computing}

\newacronym{GPS}{GPS}{Global Positioning System}

\newacronym{JSON}{JSON}{JavaScript Object Notation}

\newacronym{XML}{XML}{eXtensible Markup Language}

\newacronym{ADSL}{ADSL}{Asymmetric Digital Subscriber Line}

\newacronym{GPRS}{GPRS}{General Packet Radio Service}

\newacronym{OSI}{OSI}{Open System Interconnection}

\newacronym{HTTP}{HTTP}{HyperText Tranfer Protocol}

\newacronym{URL}{URL}{Uniform Resource Locator}

\newacronym{CSS}{CSS}{Cascade Style Sheet}

\newacronym{MEAN}{MEAN}{MongoDB + Express + Angular.js + Node.js}

\newacronym{LAMP}{LAMP}{Linux + Apache + MySQL + PHP}

\newacronym{WSGI}{WSGI}{Web Server Gateway Interface}

\newacronym{NoSQL}{NoSQL}{Not Only \acrshort{SQL}}

\newacronym{SQL}{SQL}{Structured Query Language}

\newacronym{ACID}{ACID}{Atomicity, Consistency, Isolation and Durability}

\newacronym{RTC}{RTC}{Real Time Clock}

\newacronym{NTP}{NTP}{Network Time Protocol}

\newacronym{PKI}{PKI}{Public Key Infraestructure}

\newacronym{HMAC}{HMAC}{Hash-based Message Authentication Code}

\newacronym{SHA-2}{SHA-2}{Secure Hash Algorithm 2}

\newacronym{ARM}{ARM}{Advanced \acrshort{RISC} Machine}

\newacronym{RISC}{RISC}{Reduced Instruction Set Computer}

\newacronym{LED}{LED}{Light-Emitting Diode}

\newacronym{HTTPS}{HTTPS}{HyperText Transfer Protocol Secure}

\newacronym{GNU}{GNU}{GNU is Not Unix}

\newacronym{PKCS1}{PKCS1}{Public Key Cryptography Standard 1}

\newacronym{ULDTP}{ULDTP}{Universal Limiter Data Transmission Protocol}

\newacronym{WiFi}{WiFi}{Wireless Fidelity}

\newacronym{UGR}{UGR}{Universidad de Granada}

\newacronym{HTML}{HTML}{HyperText Markup Language}

\newacronym{CSV}{CSV}{Comma Separated Values}

\newacronym{PCI}{PCI}{Peripheral Component Interconnect}

\newacronym{IEEE}{IEEE}{Institute of Electrical and Electronics Engineers}

\newacronym{FPGA}{FPGA}{Field-Programmable Gate Arrays}

\newacronym{LUT}{LUT}{Lookup Table}

\newacronym{LAN}{LAN}{Local Area Network}

\newacronym{IMU}{IMU}{Inertial Measurement Unit}

\newacronym{OS}{OS}{Operative System}

\newacronym{RTOS}{RTOS}{Real-Time Operative System}

\newacronym{ISO}{ISO}{International Organization for Standardization}

\newacronym{CPU}{CPU}{Central Processing Unit}

\newacronym{SPA}{SPA}{Single Page Application}

\newacronym{STL}{STL}{Standard Triangle Library}

\newacronym{LO}{LO}{Local Oscilator}

\newacronym{ADC}{ADC}{Analog to Digital Converter}

\newacronym{AGC}{AGC}{Automatic Gain Control}

\newacronym{EDP}{EDP}{Engineering Design Process}


% -------------------------- %
%          GLOSARIO
% -------------------------- %
\newglossaryentry{SPAG}{name={Single Page Application}, text={Single Page Application},description={Web site which, once downloaded, allows dynamic interaction with the user by rewriting the page instead of requesting the whole page to the server each time}}

\newglossaryentry{websockets}{name={Websockets}, text={Websockets},description={Full-duplex communications protocol which allows multiple channles over a single \acrshort{TCP} connection (ports 80 or 443). Although it is not mandatory, it is usually used along with \acrshort{HTTP}. It enables interaction with a web site (typically a \acrshort{SPA}) with low overhead, easing real-time data transfer in both ways \cite{wiki}}}

\newglossaryentry{stakeholders}{name={Stakeholders}, text={Stakeholders},description={According to \acrshort{ISO}21500, person, group or organization that has interests in, or can affect, be affected by, or perceive itself to be affected by, any aspect of the project}}

\newglossaryentry{payload}{name={Payload}, text={Payload},description={Carrying capacity of an aircraft, normally measured in terms of weight. More specifically, it can be referred to the equipment carried for the performance of a certain mission, e.g. an camera or an star tracker}}

\newglossaryentry{Raspberry Pi HAT}{name={Raspberry Pi HAT},description={Nombre común de las extensiones \textit{hardware} de la comunidad de la plataforma Raspberry Pi. Hace referencia a\textit{ Hardware Attached on Top} y juega con el doble sentido de la palabra \textit{hat} o sombrero en inglés}}

\newglossaryentry{API HTTP REST}{name={API HTTP REST},description={Siglas de Application Programming Interface, HiperText Tranfer Procotol y REpresentational State Transfer. Este concepto hace referencia a un tipo de interfaz web hacia el lado del cliente basada en \acrshort{HTTP} que utiliza los verbos comunes del protocolo HTTP (GET, POST, DELETE...) para acceder y gestionar datos e información existente en el lado del servidor}}

\newglossaryentry{HTTP Websockets}{name={HTTP Websockets},description={Tecnología que permite generar una comunicación \textit{full-duplex} sobre el protocolo de aplicación \acrshort{HTTP} similar a la utilizada por \acrshort{TCP} en la capa de transporte de la pila \acrshort{OSI}}}

\newglossaryentry{MVC}{name={Modelo-Vista-Controlador},description={Paradigma de arquitectura de \textit{software} que separa la lógica del programa del conjunto de datos a manejar, así como de la representación de los mismos. Tal como indica el propio nombre, los tres pilares de eta arquitectura son el modelo de datos, el controlador o lógica de programa y la vista o interfaz de usuario}}

\newglossaryentry{RWD}{name={Responsive Web Design},description={Filosofía de diseño web orientada a hacer diseños adaptables al dispositivo en el que se muestre la web en cuestión. Se tienen en cuenta tanto el tamaño, resolución y ratio de aspecto de la pantalla como el tipo de interfaz de interacción de usuario (ratón, táctil...)}}

\newglossaryentry{FSJ}{name={Full Stack JavaScript},description={Tambien conocido como \acrshort{MEAN} Stack, es un nuevo paradigma de programación web que está sustituyendo al famoso \acrshort{LAMP}. Se basa en el uso de JavaScript y \acrshort{JSON} desde la interfaz de usuario hasta el servidor y la base de datos. Pretende unificar el lenguaje de programación web así como la notación de objetos}}

\newglossaryentry{daemon}{name={Daemon},text={daemon},description={Se conoce como servicio o proceso \textit{daemon} a aquellos procesos que se ejecutan en segundo plano y no tienen interacción con el usuario dentro del conjunto de procesos de un sistema operativo. En programación multihebrada en Python, se conoce como hilo \textsl{daemon} a aquellos hilos secundarios que se ejecutan en segundo plano y que terminan junto con el hilo de ejecución principal, sin necesidad de tener que gestionar su estado desde el mismo}}

\newglossaryentry{ruido acustico}{name={Ruido acústico},description={Se considera ruido acústico toda aquella potencia emitida en forma de onda acústica que no es deseada en un entorno concreto para un cierto grupo de receptores}}

\newglossaryentry{variables ambientales}{name={Variables ambientales},text={variables ambientales},description={En el contexto de este proyecto, se consideran variables ambientales todas aquella características que describen el estado de un sistema ambiental, entendiendo por sistema ambiental un entorno urbano, rural, un local o un emplazamiento público. Son variables ambientales la temperatura, la humedad, la luminosidad o la velocidad del viento o el ruido acústico, entre otras}}

\newglossaryentry{sensor}{name={Sensor},description={En el contexto de este proyecto, se considera un sensor a todo aquel dispositivo capaz de medir el estado de las \glsname{variables ambientales} en un instante concreto}}

\newglossaryentry{GranaSAT}{name={GranaSAT}, text={GranaSAT},description={GranaSAT is an academic project from the University of Granada originally consisting in designing and developing a picosatellite (\glsname{cubesat}). Coordinated by the Professor Andrés María Roldán Aranda, GranaSAT is a multidisciplinary project with students from different degrees, where they can acquire and enlarge the knowledge necessary to face an actual aerospace project}}

\newglossaryentry{fortran}{name=FORTRAN,description={es un lenguaje de programación de alto nivel y procedural, desarrollado para propósitos generales por IBM en 1957 para el equipo IBM 704. Fue el primero desarrollado con estas características. Está fuertemente orientado al cálculo y por ende es uno de los de mayor eficiencia en la ejecución.}}

\newglossaryentry{lenguajeC}{name=lenguaje-C,description={Es un lenguaje de programación de propósito general desarrollado por Dennis Ritchie en 1972 en los Bell Telephone Laboratories para usarlo en el sistema operativo UNIX\copyright.}}

\newglossaryentry{API}{name=API,description={Una interfaz de programación de aplicaciones o API (del inglés Application Programming Interface) es el conjunto de funciones y procedimientos (o métodos, en la programación orientada a objetos) que ofrece cierta biblioteca para ser utilizado por otro software como una capa de abstracción. Son usadas generalmente en las bibliotecas (también denominadas comúnmente \textit{librerías}).}}

\newglossaryentry{HW}{name={HW},text={Hardware},description={Conjunto formado por todas los componentes físicos de un producto electrónico}}

\newglossaryentry{FW}{name={FW},text={Firmware},description={Aplicación lógica que permite el control de los componentes físicos de un producto electrónico}}

\newglossaryentry{WINDOWS}{name={Windows},text={Windows\textsuperscript{\textregistered}},description={Sistema operativo realizado por Microsoft\texttrademark}}

\newglossaryentry{GTK+}{name={GTK+},text={GTK+ \textsuperscript{\textregistered}},description={Librería multiplataforma para creación de interfaces gráficos de usuarios. Dispone de un conjunto completo de \emph{widgets}  http://www.gtk.org/}}

\newglossaryentry{OPENGL}{name={OpenGL},text={OpenGL\textsuperscript{\textregistered}},description={
OpenGL (Open Graphics Library) es una especificación estándar que define una API multilenguaje y multiplataforma para escribir aplicaciones que produzcan gráficos 2D y 3D. La interfaz consiste en más de 250 funciones diferentes que pueden usarse para dibujar escenas tridimensionales complejas a partir de primitivas geométricas simples, tales como puntos, líneas y triángulos. Fue desarrollada originalmente por Silicon Graphics Inc. (SGI) en 1992 y se usa ampliamente en CAD, realidad virtual, representación científica, visualización de información y simulación de vuelo. También se usa en desarrollo de videojuegos, donde compite con Direct3D en plataformas Microsoft Windows.}}

\newglossaryentry{GNU/Linux}{name={GNU/Linux},text={GNU/Linux \textsuperscript{\textregistered}},description={GNU/Linux es uno de los términos empleados para referirse a la combinación del núcleo o \emph{kernel} libre similar a \emph{Unix} denominado \emph{Linux}, que es usado con herramientas de sistema \emph{GNU}. Su desarrollo es uno de los ejemplos más prominentes de software libre; todo su código fuente puede ser utilizado, modificado y redistribuido libremente por cualquiera bajo los términos de la GPL (Licencia Pública General de GNU, en inglés: General Public License) y otra serie de licencias libres.}}

\newglossaryentry{BSD}{name={BSD},text={Distribución de Software Berkeley},description={La licencia BSD es la licencia de software otorgada principalmente para los sistemas BSD (Berkeley Software Distribution). Es una licencia de software libre permisiva como la licencia de OpenSSL o la MIT License. Esta licencia tiene menos restricciones en comparación con otras como la GPL  estando muy cercana al dominio público. La licencia BSD al contrario que la GPL permite el uso del código fuente en software no libre.}}

\newglossaryentry{GPL}{name={GPL},text={Licencia Pública General},description={La Licencia Pública General de \emph{GNU} o más conocida por su nombre en inglés \emph{GNU General Public License} o simplemente sus siglas del inglés \emph{GNU GPL}, es una licencia creada por la \emph{Free Software Foundation} en 1989 (la primera versión), y está orientada principalmente a proteger la libre distribución, modificación y uso de software. Su propósito es declarar que el software cubierto por esta licencia es software libre y protegerlo de intentos de apropiación que restrinjan esas libertades a los usuarios.}}