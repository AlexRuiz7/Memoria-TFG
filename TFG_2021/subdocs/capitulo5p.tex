\chapter{Validation}\label{chap:chapter5}


This fifth chapter is used to validate the solutions proposed in this Master's Thesis and check the accomplishment of the goals of the project. This verification is performed using two resources: on the one hand, different photos of the developed products, which complement the ones introduced in each section of the \autoref{chap:chapter4}; on the other hand, the \glsname{ground} software, COSMOS, is used to verify the functioning of the different subsystems, but simulating several operations using the \acrshort{I2DOS} and the GranaSat-I \glsname{cubesat} prototype.


\section{Final product} 
%
%\begin{wrapfigure}{r}{0.48\textwidth} 
	%\centering
		%\includegraphics[angle=270,page=1,trim={0cm 0cm 0cm 0cm},clip=true,width=60mm]{FigurasTFM/Chapter5/Imagenes/final_pr1.jpg}	%
			%\caption{\acrshort{I2DOS} and GranaSat-I prototype} 	\label{final1}
			%\vspace{1cm}
%\end{wrapfigure}

In this section, different photos of the design simulation platform are shown. For example, \autoref{final1} shows the \acrshort{I2DOS} with the developed 1U \glsname{cubesat}, while using the \acrshort{LED} simulator.

The \glsname{cubesat} is powered with an external power bank, which complies with the the \acrshort{EPS} requirement \textbf{EPS.FoR.4}. The solar panel test board has been kept in order to check the behaviour of the platform when the \textbf{mass distribution} is different to the expected; it is able to deal with the imbalance correctly. \autoref{final2} shows a detail of the \glsname{cubesat} electronics.

	\begin{figure} [H]
			\centering
			\includegraphics[angle=270,page=1,trim={0cm 0cm 0cm 0cm},clip=true,width=70mm]{FigurasTFM/Chapter5/Imagenes/final_pr1.jpg}
			\caption{\acrshort{I2DOS} and GranaSat-I prototype} 	\label{final1}
			\vspace{-0.5cm}
\end{figure}


		\begin{figure} [H]
			\centering
			\includegraphics[angle=270,page=1,trim={0cm 0cm 0cm 0cm},clip=true,width=90mm]{FigurasTFM/Chapter5/Imagenes/finalpr2.jpg}
			\caption{GranaSat-I \glsname{cubesat} detail} 	\label{final2}
			\vspace{-2cm}
\end{figure}

Video \autoref{videofinal} shows a slow motion clip (Adobe Reader or another compatible PDF reader needed) of the platform moving.

\vspace{-0.3cm}

\begin{videoFloat}[H]
\centering
\includemovie[text={\includegraphics[page=1,trim={0cm 0cm 0cm 0cm},clip=true,width=115mm]{figurastfm/Chapter5/Imagenes/finalvideo.png}}]{10cm}{7cm}{FigurasTFM/Chapter5/Video/videofinal.wmv}
%\vspace{0.5cm}
\caption{\acrshort{I2DOS} platform in movement (Adobe Reader needed)} \label{videofinal}
\end{videoFloat}
\vspace{-0.5cm}

%\newpage




\section{OBC} 

As for \acrshort{OBC}, once verified electrical connectivity and supply this section validates the functioning of the sensors performing different test and checking the results with COSMOS.

\vspace{-0.3cm}

\subsection{Sensors} 

As validation for the \acrshort{OBC} sensors, two screenshots of the \glsname{ground} software are included. \autoref{obcfig1} illustrates the state of the sensors in a static position, as inferable from the static illumination received (see ADC\_Measurements plot); the previous oscillations imply facing the \textit{Sun} repeatedly when rotating. Also, pressure is stable so there is no variation in height and it confirms that the \glsname{cubesat} \textbf{is stopped}. The plot \textbf{OBC\_Main\_Temperatures} shows the temperatures measured by two different sensors: the Bosch BMP280 and the internal sensor of the \acrshort{CPU}. It helps to see the \textbf{gradient} of temperature across the \acrshort{OBC}.

On the other hand, \autoref{obcfig2} shows the result sent by the sensors when the \glsname{cubesat} has gone through a complete \textbf{start-and-stop} cycle, i.e., it has started tumbling and the \acrshort{ADCS} has counteracted. Therefore, there is appreciable variations in pressure (height) and in illumination (the Sun sensors have faced it multiple times and  finally has left in eclipse position). Also, because of the higher workload and the operation of the actuators the overall temperature of the system have increased (especially in the \acrshort{CPU}, with a gradient of 10 \textdegree C).

\subsection{Communications} 

The whole telemetry data for this validation testbench is being sent over the air, using the wireless communication system of the \glsname{cubesat}. The \acrshort{OBC} sends several parameters regarding this subsystem, and they are plotted in one of the section designed in COSMOS, see \autoref{commscosmo}. Particularly, this data corresponds to the same start-and-stop cycle mentioned before; it is specially interesting the \textbf{signal level}, which varies according to the position of the antenna with respect to the receiver.


Although this validation is being performed using wireless communications, the \textbf{Ethernet} interface has been also tested, with successful results. In fact, the wired connection is necessary to configure the system the first time. In conjunction with the \glsname{ground} management software, COSMOS, the communications subsystem is able to interact in a swarm of CubeSats, which allows the use of this platform in environments such as classrooms or training sessions. \autoref{figcosoms} depicts this possible utilization.

\vspace{0.5cm}

	\begin{figure} [H]
			\centering
			\includegraphics[page=1,trim={0cm 0cm 0cm 0cm},clip=true,width=165mm]{FigurasTFM/Chapter5/PDF/COMSCOSMOS.pdf}
			\caption{Communication subsystem allows being used in a \acrshort{LAN} with multiple CubeSats} 	\label{figcosoms}
			%\vspace{-0.5cm}
\end{figure}


\begin{landscape}
\vspace*{\fill}
	\begin{figure} [H]
			\centering
			\includegraphics[page=1,trim={0cm 0cm 0cm 0cm},clip=true,width=230mm]{FigurasTFM/Chapter5/Imagenes/OBC_1.png}
			\caption{\acrshort{OBC} measurements with static \glsname{cubesat}} 	\label{obcfig1}
			\vspace{-0.5cm}
\end{figure}
\vspace*{\fill}
\end{landscape}

\begin{landscape}
\vspace*{\fill}
	\begin{figure} [H]
			\centering
			\includegraphics[page=1,trim={0cm 0cm 0cm 0cm},clip=true,width=230mm]{FigurasTFM/Chapter5/Imagenes/OBC_2.png}
			\caption{\acrshort{OBC} measurements with static \glsname{cubesat}} 	\label{obcfig2}
			\vspace{-0.5cm}
\end{figure}
\vspace*{\fill}
\end{landscape}


\begin{landscape}
\vspace*{\fill}
	\begin{figure} [H]
			\centering
			\includegraphics[page=1,trim={0cm 0cm 0cm 0cm},clip=true,width=230mm]{FigurasTFM/Chapter5/Imagenes/comms.png}
			\caption{COMMS subsystem telemetry during rotation} 	\label{commscosmo}
			\vspace{-0.5cm}
\end{figure}
\vspace*{\fill}
\end{landscape}


\section{ADCS}

This section addresses the validation of the \acrshort{ADCS} subsystem. Firstly, the magnetometers are characterized using the \textbf{Helmholtz Cage} available at \glsname{GranaSAT} laboratory. Then, just as with the \acrshort{OBC}, the correct functioning of the attitude-related sensors is verified with COSMOS.

\subsection{Magnetometer characterization} 

\autoref{magneto1} shows the testbench used to perform this characterization, with the GranaSat-I in the center of the cage.

	\begin{figure} [H]
			\centering
			\includegraphics[page=1,trim={0cm 0cm 0cm 0cm},clip=true,width=80mm]{figurastfm/Chapter5/Imagenes/testb2.jpg}
			\caption{Magnetometer characterization testbench} \label{magneto1}
			%\vspace{-0.q5cm}
\end{figure}

It is performed a simple characterization test, to assure the correct functioning of the sensor. In order to get the \glsname{cubesat} ready for a real mission, it is necessary to go through the procedure described in section \ref{magnetotest}, standardized in \cite{ESAmagnetic}. \autoref{magnetictest} shows the results; because timing is not important in this test, X-axis is shown in samples.


\begin{landscape}
\vspace*{\fill}
	\begin{figure} [H]
			\centering
			\includegraphics[page=1,trim={0cm 0cm 0cm 0cm},clip=true,width=230mm]{FigurasTFM/Chapter5/PDF/magnetic.pdf}
			\caption{Magnetic field measurement test} 	\label{magnetictest}
			%\vspace{-0.q5cm}
\end{figure}
\vspace*{\fill}
\end{landscape}
\newpage

The procedure to characterize the magnetometer is composed of 4 cycles, clearly differentiated in \autoref{magnetictest} with colours, described next.

\begin{itemize} [topsep=0pt]

\item The first cycle, in \textcolor{orange}{orange}, is a measurement of the \textbf{Earth's magnetic field}, i.e., the Helmholtz Cage is not generating any field, but measuring the field of the Earth. The coincidence between both measurements is already a good sign in itself.

\item The second cycle, in \textcolor{purple}{purple}, depicts the \textbf{cancellation of the Earth's magnetic field}. Once again, the measurements taken by MPU-9250 magnetometers are pretty accurate, and it is able to detect a null field in the three axes.

\item The third cycle, in \textcolor{lime}{green}, shows the measurements of a certain magnetic field generated by the cage. Particularly, with a magnitude of \SI{35}{\micro T}, which is the module of the components measured [$x_{\text{field}}$, $y_{\text{field}}$, $z_{\text{field}}$] = [\SI{-10}{\micro T}, \SI{27}{\micro T}, \SI{-18}{\micro T}].

\item The fourth cycle, in \textcolor{red}{red}, measures Earth's magnetic field again.

\item The test of the magnetometers can clearly be considered a success, taking into account the high degree of accuracy exhibited. However, it is worth noting the amount of \textbf{noise} present in the measurements, specially in the Z-axis. This spatial component is particularly affected by this phenomena probably because it is in the same plane of the DC-motor and the rest of the electronic components, which produce a non-negligible amount of noise in its plane. It would be recommendable to implement some \textbf{filtering} to improve the results.

\end{itemize}

\vspace{-0.3cm}
\subsection{Sensors} 

Once again, to validate the functioning of the sensors, they are included the results received and plotted in COSMOS during operation, with special interest in the maneuver of \textbf{detumbling}; \autoref{adcs1} shows this operation artificially repeated. It can be seen that the system has to deal with different rotational uncontrolled movements, in different senses of rotation; it is easy to see this situation in the \textit{\textbf{Angular\_Speed}} plot. In all cases, it is able to counteract that movement and get the system \textbf{to a stable position}. 

Regarding the magnetic field, it also varies in the Z-axis (plane of \textbf{rotation}) and \linebreak X-axis (because the platform is not completely \textbf{perpendicular} to the Z-axis) but remains approximately \textbf{constant in the Y-axis}, as the \glsname{cubesat} is at a constant \textbf{elevation}. Finally, the \textbf{Accelerometer} measurements, as expected are approximately null in X and Y axis, and \textbf{1 g in the Z-axis} (1 g = \SI{9.8}{m/s^2}).

On the other hand, \autoref{adcs2} shows a maneuver inverting its sense of rotation. By looking at the \textit{Angular\_Speed} plot, the \glsname{cubesat} is rotating counterclockwise, and the \acrshort{ADCS} instead of stabilizing it, inverts its rotation; although this is not a realistic maneuver, it is useful to check the capabilities and functioning of the control system. The inversion is also easy to see in the \textit{Magnetic\_Field} plot.

\begin{landscape}
\vspace*{\fill}
	\begin{figure} [H]
			\centering
			\includegraphics[page=1,trim={0cm 0cm 0cm 0cm},clip=true,width=230mm]{FigurasTFM/Chapter5/Imagenes/adcs1.png}
			\caption{Detumbling operation} 	\label{adcs1}
			%\vspace{-0.q5cm}
\end{figure}
\vspace*{\fill}
\end{landscape}
\newpage

\begin{landscape}
\vspace*{\fill}
	\begin{figure} [H]
			\centering
			\includegraphics[page=1,trim={0cm 0cm 0cm 0cm},clip=true,width=230mm]{FigurasTFM/Chapter5/Imagenes/adcs2.png}
			\caption{Rotation sense inversion} 	\label{adcs2}
			%\vspace{-0.q5cm}
\end{figure}
\vspace*{\fill}
\end{landscape}
\newpage


