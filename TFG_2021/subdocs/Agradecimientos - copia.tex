%Begin ----  Para que funcione bien el TOC en PDF

%\cleardoublepage

\textbf{\emph{\large }}\\
\vspace{4cm}

\begin{figure}[H]
	\centering
		\includegraphics[scale=0.6]{figurastfm/Chapter1/Imagenes/variosstd.png}
	\label{fig:vostok1}
\end{figure}

\begin{quotation}
\centering 
\textbf{\emph{Poyekhali!}}\\
\textit{{\emph{Yuri Gagarin, 12th April 1961}}}
\end{quotation}

\clearpage

\begin{quotation}
\noindent \begin{center}
\textbf{\emph{\Large Acknowledgments:}}\textbf{\emph{\large }}\\
\textbf{\emph{\large }}\\
\textbf{\emph{\large }}\\
\textbf{\emph{\large }}
%Todos aquellos que no tuvieron la oportunidad.
\par\end{center}{\large \par}
\end{quotation}

During the realization of this Project, I had the need to interact with an enormous amount of people. With a reduced number of whom I did it personally, whereas the majority of them were in pages	of books, manuals and websites.

While the latter have been recognized through references, the first ones deserve special recognition, which I have the pleasure to afford them in these lines. The first of them must be for my family, my parents, José and Amparo, and my brother Francisco Javier. 	Ever since I can remember, its unconditional support in every aspect has been of primary importance so every of my undertaking come to fruition. The support provided during this Project has been only a proof more.

It would be unfair forget my comrades in arms. Not many, but authentic friendships forged at the beginning of this stage, which have persisted through time, as they undoubtedly will do forever, along with the already existent ones.

From a technical perspective, this Bachelor Thesis would not have been released without the collaboration and supervision of my tutor, Andrés Roldán Aranda. His praiseworthy desire to improve and vast knowledge obliged me to answer with a level of work of the same magnitude. Pursuiting for excellence is the only way to be able to, someday, dream with reaching it.

And finally, not in relevance order, but with own space, thanks to Natalia, for appearing at the start of this adventure and staying here while I am writing these lines, tolerating each afternoon my technical frustrations and providing me with an unconditional support which eases every task. Her effort and personal dedication are the mirror to look at when I am losing strength.

Every one of the mentioned are part of this Thesis. To all of them, and to those who could not see it, thank you.



\clearpage
\phantomsection
\addcontentsline{toc}{chapter}{Agradecimientos}


\begin{quotation}
\noindent \begin{center}
\textbf{\emph{\Large Agradecimientos:}}\textbf{\emph{\large }}\\
\textbf{\emph{\large }}\\
\textbf{\emph{\large }}\\
\textbf{\emph{\large }}
%Todos aquellos que no tuvieron la oportunidad.
\par\end{center}{\large \par}
\end{quotation}

Durante el desarrollo de este Proyecto, me ha sido necesario interactuar con una ingente cantidad de personas. Con un número muy reducido de ellas lo he hecho en persona, mientras que una mayoría estaban entre las páginas de libros, manuales y sitios web.

Mientras que estas últimas han obtenido su reconocimiento a través de sendas referencias, las primeras merecen un reconocimiento especial que tengo el placer de brindarles en estas líneas. El primero de ellos ha de ir destinado a mi familia, mis padres, José y Amparo, y mi hermano Francisco Javier. Desde siempre, su apoyo incondicional en todos los aspectos ha resultado de importancia capital para que toda empresa acometida llegara a buen puerto. El apoyo mostrado durante el periodo de desarrollo de este Proyecto no ha sido sino una muestra más del mismo.

Sería injusto no acordarme de mis compañeros de fatigas técnicas. Pocas, pero auténticas amistadas se forjaron al inicio esta carrera de fondo, habiendo perdurado en el tiempo, como sin duda lo harán para siempre, junto a las que ya existían. A los compañeros y amigos de laboratorio, por hacer más amena la tarea diaria, especialmente a Pablo, siempre dispuesto a ayudar en cuanto pueda.

Desde la perspectiva técnica, este Trabajo Fin de Grado no habría visto la luz sin la colaboración y supervisión de mi tutor, Andrés Roldán Aranda. Su encomiable afán de mejora, capacidad de trabajo y conocimiento obligan a responder con un nivel de trabajo de la misma magnitud. La búsqueda de la excelencia es la única manera de poder, algún día, soñar con alcanzarla.

Y, por último, no en orden de relevancia, sino con espacio propio, a Natalia, por aparecer al principio de esta aventura y permanecer aquí mientras escribo estas líneas, soportando cada tarde mis frustraciones técnicas y ofreciéndome un apoyo incondicional que facilita cualquier tarea. Su esfuerzo y dedicación personal son el espejo en el que mirarme cuando las fuerzas flaquean.

Cada uno de los mencionados son partícipes de este Trabajo. A todos ellos, y a los que no pudieron verlo, muchas gracias.
\clearpage


%\bigskip


\newpage
\thispagestyle{empty}
~
\newpage
\thispagestyle{empty}


