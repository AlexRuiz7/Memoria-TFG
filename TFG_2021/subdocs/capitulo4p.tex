\chapter{System Design} \label{chap:chapter4}

After the different stages of analysis and requirements definition performed in former chapters, this forth chapter will address the \textbf{system design}. It will be structured as \autoref{chap:chapter3}, so the flow design coincides as much as possible with the analysis flow. 

Firstly, the mechanical platform is designed and characterized using professional \acrshort{CAD} tools such as \glsname{solid}. Also it is manufactured \textit{in-house} using aluminum milling machine. Regarding the \glsname{ground}, it is developed using an extended framework in space field, even used by \acrshort{NASA}. Finally, some of the most important subsystems of \glsname{cubesat} are developed, concluding with a complete \glsname{engmodel}, expandable and which can be used as a solid base to keep working on participating in the \textit{Fly~your~Satellite!} \cite{flyyour} program.

\section{Inertial 2D Orbit Simulator (I2DOS)} \label{i2dossec}

Among the different platforms analyzed in \autoref{mechplat}, this project will include an \textbf{inertial} one, which has been called Inertial 2D Orbit Simulator (\acrshort{I2DOS}). Particularly, the designed platform will be \textbf{rotational} along Z axis. \acrshort{I2DOS} will be composed of two parts, on the one hand, the rotational platform itself, and on the other hand, the base which will support the platform and will allow including external elements such as irradiance sources, according to the \textbf{formal requirements}.

\subsection{Inertial Platform}

\subsubsection{Design and mechanical characterization}


The inertial platform is designed using \glsname{solid} and sequentially improved through different redesigns. It is fully designed from scratch, with a physical background behind, reasoned later. It must have some kind of support which hosts the \glsname{cubesat} during simulation. Given the circular shape of the platform itself, that support will also be circular. Besides, it must have adequate attachment for a 1U \glsname{cubesat}. \autoref{superior} depicts a high quality render of that piece, including attachments.

\vspace{-2cm}
\begin{figure}[H]
			\centering
			\includegraphics[page=1,trim={0cm 0cm 0cm 0cm},clip=true,width=165mm]{figurastfm/Chapter4/Imagenes/superior.png}
			\vspace{-1.5cm}
			\caption{1U \glsname{cubesat} support} \label{superior} 
						\vspace{-1cm}
		\end{figure}


Top view and bottom view are depicted in Figure\autoref{topplat} and Figure\autoref{botplat} while another 3D render of the final design is shown in \autoref{inertiaplat3d}.


			\begin{figure}[H]
			\centering
			\subfloat[Top view \label{topplat}]{\includegraphics[page=1,trim={0cm 0cm 0cm 0cm},clip=true,width=75mm]{figurastfm/Chapter4/Imagenes/vista_superior.png}}
			 \quad
			\subfloat[Bottom view \label{botplat}]{\includegraphics[page=1,trim={0cm 0cm 0cm 0cm},clip=true,width=75mm]{figurastfm/Chapter4/Imagenes/vista_inferior.png}}
			\caption{Inertial platform} 
						\vspace{-2cm}

\end{figure}


\newpage
\vspace*{\fill}
\begin{figure}[H]
			\centering
			\includegraphics[page=1,trim={0cm 0cm 0cm 0cm},clip=true,width=165mm]{figurastfm/Chapter4/Imagenes/inertial.png}
%			\vspace{-2cm}
			\caption{Inertial platform 3D render} \label{inertiaplat3d} 
		\end{figure}
\vspace*{\fill}
\newpage

Regarding the \textbf{mechanical characterization}, simplifying, the \textit{functioning} is based on the inertia moment stored on the bigger ring at the bottom area ($I_1$) plus the one stored on the \glsname{cubesat} support ($I_2$). The first is considered a thick-walled cylindrical tube with open ends of inner radius $r_1$, outer radius $r_2$, length $h$, mass $m$ and a density $\rho$ whose moment of inertia is given by the first term in \autoref{inertiaring}; the second can be approximated by a solid cylinder of radius $r$, height $h$ and mass $m$ and its moment of inertia is given by the second term in \autoref{inertiaring} \cite{wiki}. The inertia moment stored at the rods is considered negligible as well as the components related to X and Y axis.

\begin{equation} \label{inertiaring}
I_{\text{z}}=I_{\text{z}1}+I_{\text{z}2}=\frac{\pi\rho h}{2}(r^{4}_{2}-r^{4}_{1})+\frac{1}{2}mr^{2}
\end{equation}

It has been considered a density $\rho$ of \SI{1.24}{g/cm^3} corresponding to \acrshort{PLA} material and a mass $m$ given by \glsname{solid} of 453 g.  Substituting with the rest of design parameters, the approximated total moment of inertia stored when rotating around Z axis is \SI{94.6}{kg/cm^2}. \glsname{solid} also calculates the inertia moment exactly, featuring a value of  \SI{108}{kg/cm^2}. It makes a difference of about 13 \% which is coherent with the approximations taken and verifies the result of \autoref{inertiaring}.

\acrshort{I2DOS}, and particularly this inertial platform, simulates the moment of inertia that the \glsname{cubesat} will face when deployed, in \glsname{tumbling}. It will have to use different detumbling mechanisms to counter that uncontrolled movement.

			\vspace{-0.5cm}


\subsubsection{Manufacturing}

The inertial platform is 3D-printed using \acrshort{PLA} and aluminum rods as the ones depicted in \autoref{inertiaplat3d}. The final result is shown in \autoref{fotoinertial}.

\vspace{-0.3cm}
\begin{figure}[H]
			\centering
			\includegraphics[page=1,trim={0cm 0cm 0cm 0cm},clip=true,width=69mm]{figurastfm/chapter4/imagenes/i2dos_b.png}
			\caption{Manufactured Inertial platform} \label{fotoinertial} 
			\vspace{-2cm}
\end{figure}

\subsection{Base}

\subsubsection{Design}

In order for the inertial platform to rotate, it is necessary a base which supports it. It is also designed using \glsname{solid} according to the requirements (see section \ref{frmech}); it must allow including external components such as sun simulators. The high quality 3D render of the base is depicted in \autoref{basemech}.

\begin{figure}[H]
			\centering
			\includegraphics[page=1,trim={0cm 0cm 0cm 0cm},clip=true,width=120mm]{figurastfm/chapter4/imagenes/base.png}
			\caption{Base 3D render} \label{basemech} 
			%\vspace{-2cm}
\end{figure}

The center hole holds the inertial base while the external ones allows adding external components; the attachments of the inertial platform are reused. It is designed to be wooden manufactured and the additional components can be attached using 4 cm aluminum rods. \autoref{basecosas} depicts the base with the different rods.

\begin{figure}[H]
			\centering
			\includegraphics[page=1,trim={0cm 0cm 0cm 0cm},clip=true,width=120mm]{figurastfm/chapter4/imagenes/base_cosas.png}
			\caption{3D render of the base with the aluminum rods} \label{basecosas} 
\end{figure}
\subsubsection{Manufacturing}

The base was manufactured at one of the \textit{makerspace} of the University of Granada in which the base could be cut. Video \autoref{videojulio} shows part of that process (Adobe Reader needed).


\begin{videoFloat}[H]
\centering
\includemovie[text={\includegraphics[page=1,trim={0cm 0cm 0cm 0cm},clip=true,width=90mm]{figurastfm/Chapter4/Imagenes/foto_video.jpg}}]{10cm}{6cm}{FigurasTFM/Chapter4/Video/makerspace.wmv}
\caption{Wooden base manufacturing (double click)} \label{videojulio}
\vspace{-0.5cm}
\end{videoFloat}

Figure\autoref{base1} and Figure\autoref{base2} depict the final result, with an external irradiance source attached, analyzed next in \autoref{irrasources}. 


			\begin{figure}[H]
			\centering
			\subfloat[Front \label{base1}]{\includegraphics[page=1,trim={0cm 0cm 0cm 0cm},clip=true,width=80mm]{figurastfm/Chapter4/Imagenes/frontbase.jpg}}
			 \quad
			\subfloat[Detail \label{base2}]{\includegraphics[page=1,trim={0cm 0cm 0cm 0cm},clip=true,width=80mm]{figurastfm/Chapter4/Imagenes/detailbase.jpg}}
			\caption{Inertial 2D Orbit Simulator (I2DOS)} \label{i2dosfinal}
			\vspace{-0.5cm}
\end{figure}

This completes the Inertial 2D Orbit Simulator (\acrshort{I2DOS}).

\subsection{Irradiance sources characterization} \label{irrasources}

In order to have a simulation platform as realistic as possible, it is needed an irradiance source which emulates the Sun. In this subsection, different lightning options are compared, following the analysis performed in section \ref{irradsources}. As studied before, it is desirable a spectrum as similar as possible to the Sun's, so the spectral response of the solar cells is such that the output power is maximum.

\subsubsection{Xenon Arc Sun Simulator} \label{xenonar}

Firstly, the Xenon Arc Sun Simulator of the \glsname{GranaSAT} laboratory is characterized following the procedures detailed in \autoref{chap:chapter3}. Although it is not usable in the proper mechanical platform designed in \autoref{i2dossec}, it is useful as a reference and can be eventually used in order to perform real missions simulations. This simulator features six different power modes, in increasing order. Figure\autoref{canon1} and Figure\autoref{canon2} show the simulator to be characterized.

\begin{figure}[H]
			\centering
			\subfloat[Side view \label{canon1}]{\includegraphics[page=1,trim={0cm 0cm 0cm 0cm},clip=true,width=85mm]{figurastfm/Chapter4/Imagenes/canon1.png}}
			 \\
			\subfloat[Front detail \label{canon2}]{\includegraphics[page=1,trim={0cm 0cm 0cm 0cm},clip=true,width=50mm]{figurastfm/Chapter4/Imagenes/canon2.png}} 
			\caption{Xenon Arc Sun Simulator at \glsname{GranaSAT} laboratory}
			%\vspace{-2cm}
\end{figure}
\newpage
\paragraph{Spectral response} \label{spectral}

To characterize the spectral response of the simulator, we had the collaboration of the  Department of Optics from the University of Granada. To perform the measurement, it was used an spectrometer THORLABS CCS200/M \cite{spectrometro} as the one depicted in \autoref{spetrometer}.

\begin{figure}[H]
			\centering
			\includegraphics[page=1,trim={0cm 0cm 0cm 0cm},clip=true,width=80mm]{figurastfm/chapter4/imagenes/spectrometer.jpg}
			\caption{Spectrometer THORLABS CCS200/M \cite{spectrometro}} \label{spetrometer} 
						\vspace{-1cm}

\end{figure}


The Sun simulator was measured with the different selectable power levels, getting the plot shown in \autoref{canonspectral}. 

\begin{figure}[H]
			\centering
			\includegraphics[page=1,trim={0cm 0cm 0cm 0cm},clip=true,width=165mm]{figurastfm/chapter4/PDF/Spectrum_sun_Sim.pdf}
			\caption{Spectral response of the different power levels} \label{canonspectral}
			%\vspace{-2cm}
\end{figure}

\autoref{canonspectral} illustrates a relative intensity plot, as the used spectrometer cannot measure absolute values, which will be determined later. It is easy to see that the spectrum remains the same regardless of the chosen level, as expected, featuring an spectral response in the range between 400 and 700 nm. It is worth mentioning that power does not increase uniformly with levels, for instance, while the increase is greater between levels 3 and 4, the difference between levels 5 and 6 is almost imperceptible.  

Recalling that the \glsname{cubesat} will count with solar panels as power source, it is interesting to see how the spectral response of the simulator matches the Sun's. It is depicted in \autoref{suncanon}, using \acrshort{AM}1.5 again.


\begin{figure}[H]
			\centering
			\includegraphics[page=1,trim={0cm 0cm 0cm 0cm},clip=true,width=160mm]{figurastfm/chapter4/PDF/Spectrum_comparison.pdf}
			\caption{Spectrum comparison} \label{suncanon}
			%\vspace{-2cm}
\end{figure}

The simulator exhibits a narrower wavelength range in comparison with Sun's, which features a wide spectral response between 300 nm up to 1000 nm; however, the area with a greater intensity is similar to the simulator's one, between 450 and 600 nm. It must be recalled from \autoref{negrous} that a standard Si solar cell gets excited in the wavelength range 400-1100 nm, so it is reasonable to expect a considerably worse performance when this Xenon Arc Sun Simulator is used.  

\paragraph{Irradiance}

After characterizing the xenon arc simulator in terms of spectral response, this section addresses the irradiance power and its variation with distance. In order to perform that measurement, it is used the Silicon-base pyranometer \textbf{Apogee~SP-110-SS}~\cite{apogee}. Figure\autoref{pyranomapog} depicts it while Figure\autoref{apogeespec} shows its spectral response.

			\begin{figure}[H]
			\centering
			\subfloat[Detail \label{pyranomapog}]{\includegraphics[page=1,trim={0cm 0cm 0cm 0cm},clip=true,width=25mm]{figurastfm/Chapter4/Imagenes/APOGEE.jpg}}
			 \quad
			\subfloat[Spectral response \label{apogeespec}]{\includegraphics[page=1,trim={0cm 0cm 0cm 0cm},clip=true,width=110mm]{figurastfm/Chapter4/PDF/pyranomspec.pdf}}
			\caption{Apogee SP-110-SS \cite{apogee}}
			\vspace{-0.75cm}
\end{figure}

 Using the pyranometer, several measurements are taken at different distances and plotted in \autoref{irradistance}.

\begin{figure}[H]
			\centering
			\includegraphics[page=1,trim={0cm 0cm 0cm 0cm},clip=true,width=140mm]{figurastfm/chapter4/PDF/irradiance_Distance.pdf}
			\caption{Irradiance decay with distance} \label{irradistance}
			\vspace{-2cm}
\end{figure}

As \autoref{irradistance} shows, the irradiance decays with distance approximately in a slightly exponential or power tendency. Marked in green, it is the point in which irradiance matches the amount received at the Earth's surface (\SI{1120}{W/m^2}), approximately at a distance of 140 cm. This shall be the point to perform simulations when emulating the Sun.

\paragraph{Lightning distribution}

As performed in \autoref{chap:chapter3} with the measurements from \cite{catalan}, it is analyzed the lightning distribution of this Sun simulator, in order to detect off-centered beam or any other inconsistencies. This is a complex procedure which may be performed with different instruments. In this case, the previously mentioned pyranometer is used along with a designed \textbf{test template} to be lighted up with the xenon arc simulator. \autoref{plantillacanon} shows that template, designed with AutoCAD.

\begin{figure}[H]
			\centering
			\includegraphics[page=1,trim={0cm 0cm 0cm 0cm},clip=true,width=100mm]{figurastfm/chapter4/PDF/template.pdf}
			\caption{Lightning distribution test template} \label{plantillacanon}
			\vspace{-0.7cm}
\end{figure}

The procedure consists in performing measurements with the pyranometer at the different angles depicted in the test template, until completing the whole circumference. Besides, these measurements must be taken at several distances, marked in red. When completed, it is possible to plot the lightning distribution, just as performed in \autoref{catorigin} and \autoref{catinterp}. The results are shown in \autoref{distrib1} and \autoref{distrib2}.

	\afterpage{
\begin{landscape}
%\vspace*{\fill}
\begin{figure}
			\centering
			\includegraphics[page=1,trim={0cm 0cm 0cm 0cm},clip=true,width=175mm]{figurastfm/Chapter4/PDF/Lightning_distrib_pcolor_jet.pdf}
			\caption{Lightning distribution (raw data)} \label{distrib1}
\end{figure}
%\vfill
\end{landscape}
}
	
	
\afterpage{
\begin{landscape}
%\vspace*{\fill}
\begin{figure}
			\centering
			\includegraphics[page=1,trim={0cm 0cm 0cm 0cm},clip=true,width=175mm]{figurastfm/Chapter4/PDF/Lightning_distrib_contourf_jet.pdf}
			\caption{Interpolated Lightning distribution} \label{distrib2}
\end{figure}
%\vfill
\end{landscape}
}

As in \autoref{chap:chapter3}, the measurements are plotted twice, one with the \textit{raw} data in \autoref{distrib1} and another interpolating the data, allowing a more understandable graphic in \autoref{distrib2}. In order to ease the comparison, they are both plotted together in \autoref{comparison}. The ones above correspond to the simulator at \glsname{GranaSAT} laboratory and the ones below to the one used in \cite{catalan}.

\begin{figure}
			\centering
			\includegraphics[page=1,trim={0cm 0cm 0cm 0cm},clip=true,width=165mm]{figurastfm/Chapter4/PDF/Lightning_distrib_comparison.pdf}
			\caption{Lightning distribution comparison} \label{comparison}
\end{figure}

The first conclusion which can be extracted in comparison with the simulator used in \cite{catalan} is a noticeably less powerful xenon bulb in my case, even at a shorter distance (1.8~m vs. 2.7~m) the maximum irradiance is 33 \% lower than his. By inspection at \autoref{irradistance}, the equivalent irradiance at 2.8 m is about \SI{500}{W/m^2}, once again one third lower. Therefore, although the power of the xenon bulb used is unknown, assuming similar losses, it can be estimated around \textbf{650 W}. Another possibility is that the bulb has reached its end-of-life and power gets lower. 

As stated in \ref{irradsources}, the closer the simulated is used, the greater the beam gets off-centered; indeed, as seen in \autoref{comparison}, the beam of the simulator in our case is clearly off-centered, even more so than the other, at a distance of about 5 cm from the center, within the range between 180\textdegree~and~290\textdegree. Therefore, there are two possibilities in order to use this Sun simulator:
\newpage
\begin{itemize} [noitemsep,topsep=0pt]
\item Use it at greater distances, so the beam gets centered, sacrificing power, which will get below the \SI{1100}{W/m^2}. Allows characterizing the solar panel in the center of the test template (see \autoref{plantillacanon}) but it is not useful for real mission simulations.	\\
\item Use it at closer distances, with an off-centered beam as depicted in previous plots, but with realistic irradiance. Allows emulating Sun's irradiance, needed in order to check functioning in a real mission scenario, but needs to place the solar panel in the area where the beam focus, stated before.
\end{itemize}

The choice will depend on the purpose of the simulation.

\subsubsection{LED Sun Simulator} \label{ledsim}
%\vspace{-0.5cm}

\begin{wrapfigure}{r}{0.28\textwidth} 
	\centering
	\includegraphics[width=0.20\textwidth]{figurastfm/chapter4/imagenes/led.jpg}
	\vspace{0.1cm}
	\caption{\acrshort{LED} proposed} \label{ledchino}
\end{wrapfigure}


Because of its size, the xenon arc simulator from \ref{xenonar} cannot be used along with the inertial platform designed. Therefore, in this section it will be characterized another technology previously analyzed in \autoref{chap:chapter3}, \acrshort{LED}. According to the requirements, it has to be usable with the platform, so it must be small-sized. It is proposed using a low-cost thermally-bonded \acrshort{LED} lamp, such as the one shown in \autoref{ledchino}. \\ \\


%\begin{figure} [H]
			%\centering
			%\includegraphics[page=1,trim={0cm 0cm 0cm 0cm},clip=true,width=40mm]{figurastfm/Chapter4/Imagenes/led.jpg}
			%\caption{\acrshort{LED} proposed} \label{ledchino}
			%\vspace{-0.5cm}
%\end{figure}
\vspace{-1cm}
It is a 100 W lamp to be supplied between 20 V and 32 V, with a maximum consumption of 3 A. It may reach high temperatures so it needs cooling; besides, as it is desired a lightning beam as concentrated as possible, it raises the need for a \textbf{collimator}. To satisfy these needs, it is used a kit designed for this kind of lamps, which includes both, a collimator and a 12 V fan. It is shown in \autoref{collimator}. When it is completely assembled, the result is the one which could be seen in \autoref{i2dosfinal}.

%\vspace{-0.5cm}


			\begin{figure}[H]
			\centering
			\subfloat[Collimator and body]{\includegraphics[page=1,trim={0cm 0cm 0cm 0cm},clip=true,width=50mm]{figurastfm/Chapter4/PDF/cuerpoled.pdf}}
			\quad
			\subfloat[Copper base to cool the \acrshort{LED}]{\includegraphics[page=1,trim={0cm 0cm 0cm 0cm},clip=true,width=50mm]{figurastfm/Chapter4/Imagenes/topled.jpg}}
			 \quad
			\subfloat[Fan]{\includegraphics[page=1,trim={0cm 0cm 0cm 0cm},clip=true,width=50mm]{figurastfm/Chapter4/Imagenes/cooler.jpeg}}
			\caption{Kit used for the \acrshort{LED} lamp} \label{collimator}
			\vspace{-1cm}
\end{figure}


\paragraph{Spectral response}

The spectral response of the \acrshort{LED} is measured using the same procedure detailed in \ref{spectral}. The results are plotted in \autoref{specralled}, along with the Sun's.

\begin{figure} [H]
			\centering
			\includegraphics[page=1,trim={0cm 0cm 0cm 0cm},clip=true,width=165mm]{figurastfm/Chapter4/PDF/Spectrum_comparison_led.pdf}
			\caption{Spectrum comparison} \label{specralled}
\end{figure}

In this case, the \acrshort{LED} spectrum covers the range 400 nm to 650 nm, with a pronounced peak at 500 nm in which almost no intensity is irradiated. Therefore, the range of a typical solar panel is slightly covered and in order to get at least an acceptable efficiency, a really high power would be needed.  

\paragraph{Irradiance and consumption}

Irradiance is measured again using the same pyranometer. As this lamp is intended to be used with the inertial platform at an approximately fixed distance, it will be characterized at that single point, \textbf{20 cm far}. \autoref{irradiancemeas} shows part of the assembly needed to perform the measurement. 

\begin{figure} [H]
			\centering
			\includegraphics[page=1,trim={0cm 0cm 0cm 0cm},clip=true,width=100mm]{figurastfm/Chapter4/Imagenes/irrad_meas.jpg}
			\caption{\acrshort{LED} irradiance measurement} \label{irradiancemeas}
\end{figure}

The measurement needs to control simultaneously two devices: on the one hand, the power source, which must sweep between 20 V and 32 V and on the other hand, the oscilloscope to check for signal integrity first and perform the measurements. Specifically, the power source used is a \textbf{Siglent SPD3303X} while the oscilloscope is an \textbf{Agilent MXO-X-4104A}, both are shown in \autoref{equipos}.

			\begin{figure}[H]
			\centering
			\subfloat[Siglent SPD3303X \cite{Siglent}]{\includegraphics[page=1,trim={0cm 0cm 0cm 0cm},clip=true,width=50mm]{figurastfm/Chapter4/Imagenes/SIGLENT.jpg}}
			\quad
			\subfloat[Agilent MXO-X-4104A \cite{agilent}]{\includegraphics[page=1,trim={0cm 0cm 0cm 0cm},clip=true,width=55mm]{figurastfm/Chapter4/Imagenes/AGILENT.png}}
			\caption{Measurement devices} \label{equipos}
			%\vspace{-1cm}
\end{figure}


Firstly, the irradiance is measured along with the \acrshort{LED} current consumption.  In order to get it done synchronically, the following Python code is developed.


%Python code highlighting
\lstset{  backgroundcolor=\color{white},  commentstyle=\color{codegreen}, keywordstyle=\color{magenta},   numberstyle=\tiny\color{codegray}, stringstyle=\color{codepurple},  basicstyle=\ttfamily\footnotesize,  breakatwhitespace=false,           breaklines=true,                   captionpos=b,                      keepspaces=true,                   numbers=left,                      numbersep=5pt,                    showspaces=false,                  showstringspaces=false,  showtabs=false,                    tabsize=2,  stepnumber=3, numberstyle=\tiny, basicstyle=\linespread{0.4}}

\begin{lstlisting}[language=Python, caption=Polling based measurements script]
import visa
from visa import constants
import vxi11
import csv
import pandas as pd
import time
import math
import os
import numpy as np

# GPIB INIT
# visa.log_to_screen()
SG =  vxi11.Instrument("192.168.1.119")
OSC = visa.ResourceManager('@py').get_instrument('TCPIP0::192.168.1.121::inst0::INSTR')
OSC.timeout=2500000

# IDENTIFYING
print("SG found: " + SG.ask("*IDN?").strip())
print("OSC found: " + OSC.query('*IDN?').strip())

#OSC Setting-up

OSC.write(':CHANnel1:DISPlay ON')
OSC.write(':DISPlay:SIDebar MEASurements')
OSC.write(':MEASure:VPP CHANnel1')

#SG Setting-up

SG.ask("CH1:VOLT 12")
SG.ask("CH2:VOLT 20")

SG.ask("CH1:CURRent 0.3")
SG.ask("CH2:CURRent 3.2")

SG.ask("OUTPut CH1,ON")
SG.ask("OUTPut CH2,ON")

volt_Sweep=np.arange(20,32.5,0.5)
current_Sweep=[]
measured_Osc=[]

#Allow time for the measurement to stabilize at the final values

for V in volt_Sweep:
	
	SG.ask("CH2:VOLT %f" %V)
	
	if V==31 or V==31.5 or V==32:
	
		time.sleep(60)
		
	else:
	
		time.sleep(1)	

	current_Sweep.append(SG.ask("MEASure:CURRent? CH2"))	
	measured_Osc.append(OSC.query(':MEASure:VRMS? CHANnel1'))


voltage_Data=pd.DataFrame(volt_Sweep)
current_Data=pd.DataFrame(current_Sweep)
measured_Voltage_Data=pd.DataFrame(measured_Osc)

pd.concat([voltage_Data,current_Data,measured_Voltage_Data],axis=1).to_csv("data5.csv")

print("Data written to CSV")

SG.ask("OUTPut CH1,OFF")
SG.ask("OUTPut CH2,OFF")

\end{lstlisting}

The results are plotted in \autoref{irracurr}. The tendency is clearly lineal, as expected; the higher the current, the greater the irradiance, until reaching the maximum current allowed by the \acrshort{LED}. The highest irradiance accomplished by the simulator at that distance (20 cm) is \SI{235}{W/m^2}. It is about five times lower than the one received at Earth's surface, therefore, it can be expected a poor performance from the solar panels with this irradiance. Of course it is also way lower than the value produced by the xenon arc lamp at that distance. However, that comparison is not realistic; while this \acrshort{LED} is about 80 W (see \autoref{ivpower}), the xenon arc simulator has an estimated power of about 600 W and is not intended to work at such a short distance. \textbf{Their simulation purposes are different.}

\begin{figure} [H]
			\centering
			\includegraphics[page=1,trim={0cm 0cm 0cm 0cm},clip=true,width=135mm]{figurastfm/Chapter4/PDF/IRRAD_CURR.pdf}
			\caption{Irradiance with \acrshort{LED} current consumption} \label{irracurr}
						\vspace{-0.7cm}
\end{figure}

 However, the irradiance is not linear with \textbf{voltage supply}, because of the exponential relationship between voltage and current in a \acrshort{LED}; indeed, neither the functioning range given by the manufacturer nor the power drawn are completely correct. This can be concluded from the plot in \autoref{ivpower}, which relates I-V curve of the \acrshort{LED} with the power drawn. 

\begin{figure} [H]
			\centering
			\includegraphics[page=1,trim={0cm 0cm 0cm 0cm},clip=true,width=135mm]{figurastfm/Chapter4/PDF/ivpower.pdf}
			\caption{I-V curve and power drawn} \label{ivpower}
			\vspace{-2cm}
\end{figure}

That plot is important because it allows checking the real requirements of the lamp and therefore, of the \acrshort{LED} driver. It can be concluded that the \acrshort{LED} does not start lightning until supplied with 26 V, and the maximum power drawn is about 80 W, unlike the range stated by the manufacturer 20-32 V and 100 W power, respectively. The power losses were obviously expected, but an under consumption of 20 \% is worth noting. Although the expected maximum is not reached, the lamp heats considerably, which makes it necessary to install the fan mentioned before. Even when the fan is on, the temperature borders on 100 \textdegree C, as shown in \autoref{temp}.


\begin{figure} [H]
			\centering
			\includegraphics[page=1,trim={0cm 0cm 0cm 0cm},clip=true,width=40mm]{figurastfm/Chapter4/Imagenes/temp.jpg}
			\caption{Measured temperature on the \acrshort{LED}} \label{temp}
						\vspace{-0.7cm}
\end{figure}
In sum, in order to supply the \acrshort{LED} Sun simulator, it is necessary a \textbf{driver} with an output voltage about 32 V and 100 W of power capability. \autoref{driverchino} shows the chosen one.

\begin{figure} [H]
			\centering
			\includegraphics[page=1,trim={0cm 0cm 0cm 0cm},clip=true,width=100mm]{figurastfm/Chapter4/Imagenes/led_driver.jpeg}
			\caption{Low cost 100 W \acrshort{LED} driver} \label{driverchino}
						%\vspace{-0.7cm}
\end{figure}


\subsubsection{Comparative}

The irradiance sources characterization finalizes comparing the spectrums of the proposed technologies with the Sun's. \autoref{compall} plots this comparison with all the data normalized to the Sun's spectrum irradiance, in order to ease the spectral visualization.


\begin{figure} [H]
			\centering
			\includegraphics[page=1,trim={0cm 0cm 0cm 0cm},clip=true,width=145mm]{figurastfm/Chapter4/PDF/Spectrum_comparison_all.pdf}
			\caption{Spectral comparison of the different irradiance sources} \label{compall}
						%\vspace{-0.7cm}
\end{figure}

Xenon arc and \acrshort{LED} spectrums are actually pretty similar, except for the 500 nm area in the latter. However, from \autoref{compall} and previous plots it is easy to conclude that \textbf{none of them matches the Sun's}, neither spectrum nor power. That is not necessarily a problem, though; the \acrshort{LED} is intended to be used along with \acrshort{I2DOS} so its final purposes are not strictly real simulation. When that is required however, it will be needed to use a different xenon arc simulator. Even if spectral matching is enough, low power and off-centered beam makes it unreliable for real missions simulations.


\section{Ground Station}

Regarding the \glsname{ground}, this Master's Thesis will address the design and launch of the central controller. It will be based on the environment \textbf{Ball~Aerospace~COSMOS} and will count with different user interfaces and a basic telemetry database. It is intended to function as the main point of control and communications with the \glsname{cubesat}.

The designed \glsname{ground} software will be expandable so it can be used as it would in a real mission.

\subsection{Ball Aerospace COSMOS} \label{cosmosgs}

COSMOS is a complete environment designed by Ball Aerospace which is composed of some sample applications and allows defining new interfaces and functionalities, covering the \textbf{full lifecycle} of the mission, as the same interface can be used from test to operation. It is written in \textbf{Ruby} and provides with a command and control system to interact with a variety of embedded systems such as a \glsname{cubesat} but also electronics equipment such as a radio transmitter.  It is also compatible with international standards analyzed before like \acrshort{CCSDS} or \acrshort{XTCE} and is \textbf{open source}.


Using COSMOS allows adding a new abstraction layer to develop the tools needed by the mission focusing on the required functionalities, instead of rewriting existing tools. This makes it a really powerful and productive tool, which has been widely used by international institutions including \acrshort{NASA}, particularly the Goddard Space Flight Center. Implementing it in \glsname{GranaSAT} laboratory gives future students the valuable opportunity to work and train themselves using real cutting-edge technology, which satisfies the academic perspective of this Master's Thesis.

\vspace{-0.5cm}
\subsubsection{Design and configuration}

Once installed, COSMOS design and configuration is based on plain text files which can be either modified directly or using a dedicated tool included with COSMOS, as it is proposed. \autoref{configeditor} displays that configuration editor.


\begin{figure} [H]
			\centering
			\includegraphics[page=1,trim={0cm 0cm 0cm 0cm},clip=true,width=155mm]{figurastfm/Chapter4/Imagenes/configeditor.png}
			\caption{Configuration editor} \label{configeditor}
			\vspace{-1cm}
\end{figure}

Next, some of the most important concepts about COSMOS are briefly introduced and configured. \newpage

\begin{itemize} [noitemsep,topsep=0pt]
	\item \textbf{Target} 
\end{itemize}

	Device or spacecraft intended to establish a communication. As for this project, there will be only one target, the designed \glsname{cubesat}. Firstly, they have to be \textbf{declared}; this can be done in the \textbf{system.txt} file, under the following path.

	
	\begin{figure} [H]
\centering
\begin{minipage}{5cm}
\dirtree{%
.1 /GranaSAT-GroundStation.
.2 +config.
.3 system.
.4 system.txt.
}
\end{minipage}
\end{figure}



Besides, it is necessary to create a folder with the name of the target and replicate the following folder structure:

	\begin{figure} [H]
\centering
\begin{minipage}{5cm}
\dirtree{%
.1 /GranaSAT-GroundStation.
.2 +config.
.3 +targets.
.4 GRANASAT-I.
.5 +cmd-tlm.
.6 target.txt.
}
\end{minipage}
\end{figure}

The file \textit{target.txt} must contain the name of the database with the telemetry and telecommands information.


\begin{itemize} [noitemsep,topsep=0pt]
	\item \textbf{Interface}
\end{itemize}

In COSMOS, interfaces refer to the protocol used to communicate with a certain target, i.e., TCP/IP, serial or any other defined. Particularly, TCP/IP is already defined and is very convenient to use in a local environment such as the one used. They are defined in the file \textit{cmd-tlm-server.txt} under the following path.


	\begin{figure} [H]
\centering
\begin{minipage}{5cm}
\dirtree{%
.1 /GranaSAT-GroundStation.
.2 +config.
.3 +tools.
.4 +cmd-tlm-server.
.5 cmd-tlm-server.txt.
}
\end{minipage}
\end{figure}

\autoref{configinterf} illustrates the information needed in that file. Further information on the different options can be found on the official website \cite{cosmos}.

\begin{figure} [H]
			\centering
			\includegraphics[page=1,trim={0cm 0cm 0cm 0cm},clip=true,width=165mm]{figurastfm/Chapter4/Imagenes/interfaces.png}
			\caption{Interface configuration screenshot} \label{configinterf}
			%\vspace{-0.7cm}
\end{figure}

\subsubsection{Telecommand \& Telemetry database}

The whole framework revolves around the database with the telemetry and telecommand information. Once again, it can be defined using plain text; however, COSMOS is also compatible with \acrshort{XTCE} format, with an XML syntax. Once the implementation is decided, the data structure can be standardized using \acrshort{CCSDS}. Out of simplicity, in this Master's Thesis the database will be implemented using \acrshort{XTCE} and \acrshort{CCSDS}, but this one only partially. Expanding the database using completely that format is proposed as a future line of work, specially when working on a real mission.

There are countless XML editors, any of which can be used to create the needed database. It is proposed \textbf{Altova XMLSpy} because of the variety of different representations it allows. \autoref{codigoxml} shows an example of telemetry packets defined using \acrshort{XTCE}.

%xml code highlighting
\definecolor{gray}{rgb}{0.4,0.4,0.4}
\definecolor{darkblue}{rgb}{0.0,0.0,0.6}
\definecolor{cyan}{rgb}{0.0,0.6,0.6}
%%%%%%%%%%%%%%%%%
\lstdefinelanguage{XML}
{
  morestring=[b]",
  morestring=[s]{>}{<},
  morecomment=[s]{<?}{?>},
  stringstyle=\color{black},
  identifierstyle=\color{darkblue},
  keywordstyle=\color{cyan},
  morekeywords={xtce,encoding,UnitSSet}% list your attributes here
}
\lstset{  backgroundcolor=\color{white},  commentstyle=\color{codegreen}, keywordstyle=\color{magenta},   numberstyle=\tiny\color{codegray}, stringstyle=\color{codepurple},  basicstyle=\fontsize{4}\ttfamily\footnotesize,  breakatwhitespace=false,           breaklines=true,                   captionpos=b,                      keepspaces=true,                   numbers=left,                      numbersep=5pt,                    showspaces=false,                  showstringspaces=false,  showtabs=false,                    tabsize=2,  stepnumber=3, numberstyle=\tiny, basicstyle=\linespread{0.4}}

\begin{lstlisting}[language=XML, caption=Example of telemetry implemented using XTCE, label={codigoxml}]
<xtce:FloatParameterType name="OBC_Bosch_Temperature" shortDescription="OBC_Bosch_Temperature" signed="true">
				<xtce:UnitSet>
					<xtce:Unit description="deg">C</xtce:Unit>
				</xtce:UnitSet>
				<xtce:FloatDataEncoding sizeInBits="32" encoding="signed" bitOrder="LITTLE_ENDIAN">
					</xtce:FloatDataEncoding>
			</xtce:FloatParameterType>
			<xtce:FloatParameterType name="ESP_Data_Rate" shortDescription="ESP_Data_Rate" signed="true">
				<xtce:UnitSet>
					<xtce:Unit description="Mb/s">Mb/s</xtce:Unit>
				</xtce:UnitSet>
				<xtce:FloatDataEncoding sizeInBits="32" encoding="signed" bitOrder="LITTLE_ENDIAN">
					</xtce:FloatDataEncoding>
			</xtce:FloatParameterType>
			<xtce:FloatParameterType name="ESP_Signal_Level" shortDescription="ESP_Signal_Level" signed="true">
				<xtce:UnitSet>
					<xtce:Unit description="dBm">dBm</xtce:Unit>
				</xtce:UnitSet>
				<xtce:FloatDataEncoding sizeInBits="32" encoding="signed" bitOrder="LITTLE_ENDIAN">
					</xtce:FloatDataEncoding>

\end{lstlisting}

\autoref{schemaxml} shows part of the implemented database with one of the visual representations provided by Altova.

\begin{figure} [H]
			\centering
			\includegraphics[page=1,trim={0cm 0cm 0cm 0cm},clip=true,width=165mm]{figurastfm/Chapter4/Imagenes/xtce.png}
			\caption{Implemented telemetry and telecommand database using \acrshort{XTCE} format} \label{schemaxml}
			%\vspace{-0.7cm}
\end{figure}



\subsubsection{Interface}

Once the configuration parameters are set, COSMOS allows defining different user interfaces following the same procedure. For example, \autoref{maingui} displays the main interface of the developed \glsname{ground} software.

It shows the different tools developed in this Master's Thesis. It is divided into three differentiated areas: the first one, which initiates the controller, the second one 'Commanding and Scripting´, with the tools necessary to send commands to the \glsname{cubesat} and the last one 'Telemetry´, with tools to visualize and post-process the telemetry. Some of them are briefly described next.

\begin{figure} [h]
			\centering
			\includegraphics[page=1,trim={0cm 0cm 0cm 0cm},clip=true,width=85mm]{figurastfm/Chapter4/Imagenes/gui.png}
			\caption{Main interface of the \glsname{GranaSAT} \glsname{ground} controller} \label{maingui}
			\vspace{0.5cm}
\end{figure}

\begin{itemize} [noitemsep]
	\item \textbf{Command and Telemetry Server:} Used to connect to the different targets. It shows information regarding the number of packages received, errors detected and so forth.
\end{itemize}


\begin{itemize} [noitemsep,topsep=0pt]
	\item \textbf{Replay:} Used to \textit{repeat} a past mission. By loading the log of a certain session, it allows repeating the processing performed in real time, plot the information again with a temporal reference, etc. \autoref{replay} shows the interface of this tool.
\end{itemize}

\begin{figure} [H]
			\centering
			\includegraphics[page=1,trim={0cm 0cm 0cm 0cm},clip=true,width=85mm]{figurastfm/Chapter4/Imagenes/replay.png}
			\caption{Replay tool} \label{replay}
			%\vspace{-0.7cm}
\end{figure}


\begin{itemize} [noitemsep,topsep=0pt]
	\item \textbf{Packet viewer:} Used to visualize the telemetry data as received, either in real time or using the \textbf{Replay} tool. \autoref{viewer} shows an example of received packet in this tool.
\end{itemize}

\begin{figure} [H]
			\centering
			\includegraphics[page=1,trim={0cm 0cm 0cm 0cm},clip=true,width=85mm]{figurastfm/Chapter4/Imagenes/packetview.png}
			\caption{Packet viewer} \label{viewer}
			%\vspace{-0.7cm}
\end{figure}

Telemetry Grapher is not described because it will be widely seen in \autoref{chap:chapter5} to verify the functioning of the system.

\subsection{Real time 3D viewer}

Besides the COSMOS environment developed and explained in the latter section, it has been developed a web-based real time 3D viewer. Once again, it is intended to be in a local environment; in this case the reason is obvious: the communication delay existing in a real in orbit communication makes it impossible a real time display. However, it is an attractive tool for students learning about CubeSats with this platform. 

The diagram of \autoref{3dview} depicts the functioning and technologies used in the viewer.

\begin{figure} [H]
			\centering
			\includegraphics[page=1,trim={0cm 0cm 0cm 0cm},clip=true,width=165mm]{figurastfm/Chapter4/PDF/3dviewer.pdf}
			\caption{3D viewer architecture} \label{3dview}
			%\vspace{-0.7cm}
\end{figure}

The proper \glsname{cubesat} hosts in the \acrshort{OBC}, along with the rest of components (\acrshort{OBDH}, Flight Software...), a \textbf{Node.js} \cite{nodejs} server. When the attitude-related sensors (accelerometer, gyroscope and magnetometer) have data available, they pass it directly to the web server. Once the server has the attitude information, it communicates with the user interface using secure \glsname{websockets}. The user interface is designed as a \glsname{SPAG}~(\acrshort{SPA}) so it is fully downloaded just once. On the other hand, the server is listening at port 8000, waiting for connections.

The designed SPA uses three technologies to plot a real time 3D view of the \glsname{cubesat}. The one in the bottom is \textbf{WebGL} \cite{webgl}, graphic libraries optimized for web applications, in charge of rendering the 3D visualization. Right above it, \textbf{D3.js} \cite{d3} which adds an abstraction layer over WebGL, easing the process by managing \acrshort{STL} models, controlling light, cameras, background, etc; in sum, it is a 3D graphic library. Finally, on the top of the stack,  the library \textbf{C3.js} \cite{c3js} is used to generate real time 2D plots.

The website can be seen in \autoref{plotter3d}.

\begin{figure} [H]
			\centering
			\includegraphics[page=1,trim={0cm 0cm 0cm 0cm},clip=true,width=165mm]{figurastfm/Chapter4/PDF/3dplotter_.pdf}
			\caption{3D viewer web application} \label{plotter3d}
			%\vspace{-0.7cm}
\end{figure}

\section{Simulation CubeSat}

The design phase finalizes with the simulation \glsname{cubesat}, which will lay the foundations to complete a mission-ready device in the future. This section addresses all the aspects around \glsname{cubesat} design, from the mechanical design of the structure to the electronics, which makes it \textbf{the most complex part of the whole platform}.

\subsection{Mechanical structure}

As described in \autoref{cubesatdefin}, one of the most extended \glsname{cubesat} standards was defined by the California Politechnical State University \cite{calpoly}, depicted in \autoref{cubesatstd}. Therefore, the design performed in this Master's Thesis will stick as much as possible to it.

\subsubsection{Design}

Once again, the mechanical structure is designed using \glsname{solid}. \autoref{drawings} shows the drawings of the design. On the other hand, \autoref{cubesatrender} shows another high quality 3D render.


\begin{figure} [H]
			\centering
			\includegraphics[page=1,trim={0cm 0cm 0cm 0cm},clip=true,width=135mm]{figurastfm/Chapter4/Imagenes/csat3d.png}
			\caption{Mechanical structure 3D render} \label{cubesatrender}
			%\vspace{-1cm}
\end{figure}

Besides the outer edge design, the non-standard inner is important because it will constrain the \acrshort{PCB} edge.

\begin{landscape}
\begin{figure} [H]
			\centering
			\includegraphics[page=1,trim={0cm 0cm 0cm 0cm},clip=true,width=220mm]{figurastfm/Chapter4/PDF/soliddrw.pdf}
			\caption{Mechanical structure design} \label{drawings}
			%\vspace{-0.7cm}
\end{figure}
\end{landscape}

\subsubsection{Manufacturing}

The structure is made of aluminum and manufactured using a milling machine (see \autoref{milling}) in our own laboratory. 

\begin{figure} [H]
			\centering
			\includegraphics[page=1,trim={0cm 0cm 0cm 0cm},clip=true,width=80mm]{figurastfm/Chapter4/Imagenes/milling.jpg}
			\caption{Aluminum milling machine} \label{milling}
			\vspace{-0.5cm}
\end{figure}

After an iterative process, an optimized structure is accomplished, displayed in \autoref{manufactu}. Some testing solar panels \acrshort{PCB} have been attached in order to check the correctness of the design.

			\begin{figure}[H]
			\centering
			\subfloat[1]{\includegraphics[page=1,trim={0cm 0cm 0cm 0cm},clip=true,width=83mm]{figurastfm/Chapter4/Imagenes/manuf1.jpg}}
			\quad
			\subfloat[2]{\includegraphics[page=1,trim={0cm 0cm 0cm 0cm},clip=true,width=75mm]{figurastfm/Chapter4/Imagenes/manuf2.jpg}}
			\caption{Mechanical structure manufactured} \label{manufactu}
			\vspace{-2cm}
\end{figure}


\subsection{On-board computer (OBC)} \label{obcdesig}

As stated in \autoref{chap:chapter3}, the \acrshort{OBC} is probably the most complex part of the \glsname{cubesat}. It is in charge of managing and synchronizing the rest of the system \textbf{in-time} and acting consequently. This section is divided into different subsections according to the main functioning modules of the \acrshort{OBC}. 

\subsubsection{Central Processing Unit}

The Central Processing Unit of a \glsname{cubesat}, typically abbreviated as \acrshort{CPU}, can indeed be understood in the same manner of a standard computer. It is some kind of circuitry, typically a microprocessor, which performs certain instructions consisting in arithmetic, I/O operations, logic, etc.

This task can be addressed by a variety of architectures. One of the most common options is using a \textbf{microcontroller}, for example the MSP430 by Texas Instruments \cite{msp}. However, in this project the design will go one step beyond using a so-called \linebreak micro-computer or more specifically an embedded \textbf{Single-Board Computer} (\acrshort{SBC}). They are a complete computer built on a single \acrshort{PCB}, including not only microprocessor, but also memory, RAM, or I/O capabilities. Unlike microcontrollers, \acrshort{SBC} allows storage memory and can run an operating system completely. On the other hand, they can be more difficult to implement.  

\acrshort{SBC} can be found in a variety of formats; for instance, \autoref{sbcformats} shows a couple of examples in \textit{standalone} format.

			\begin{figure}[H]
			\centering
			\subfloat[Raspberry Pi 4 B+]{\includegraphics[page=1,trim={0cm 0cm 0cm 0cm},clip=true,width=83mm]{figurastfm/Chapter4/Imagenes/rpi.jpg}}
			\quad
			\subfloat[Orange Pi]{\includegraphics[page=1,trim={0cm 0cm 0cm 0cm},clip=true,width=75mm]{figurastfm/Chapter4/Imagenes/orangepi.jpg}}
			\caption{Different \acrshort{SBC} available on the market} \label{sbcformats}
			%\vspace{-2cm}
\end{figure}



\textit{Standalone} \acrshort{SBC} just need a loaded operating system and a power supply. Although that format is convenient for hobby or \textit{domestic} projects, as for a \glsname{cubesat} project is too heavy and furthermore really difficult to fit into the structure. For these reason, this design will make use of \textbf{industrial} \acrshort{SBC}, characterized by a more flexible form  factor with standard DDR2 SODIMM connector. It  increases substantially the complexity of the design but also allows a much more professional result and gets the project closer to the ticket-to-fly in the \textit{Fly Your Satellite!} program, described in \ref{fys}.

\paragraph{Comparative}

In order to choose the \acrshort{SBC} which fits best the requirements needed, it is performed a comparative between different options available in the market. The comparative is also needed because of the cost of this part, which is usually the highest of the whole \glsname{cubesat}.  Normally, this is also the component with the greatest power consumption, therefore an extensive analysis is performed on the option chosen.

\subparagraph{Apalis TK1}

The Apalis TK1 by Toradex \cite{apalistk1} is based on the NVIDIA Tegra K1 \acrshort{SoC}. It counts with a Cortex A15 quad-core CPU up to 2.1 GHz and a powerful GPI Geforce Kepler, also from NVIDIA, as well as 2 GB RAM.

Additionally, the Apalis TK1 features a low power ARM Cortex M4 up to 100 MHz which extends the \acrshort{ADC}, \acrshort{GPIO} and several other interfaces. It fits into a standard 200 pins SODIMM connection. \autoref{tk1} shows this \acrshort{SBC}.


\begin{figure} [H]
			\centering
			\includegraphics[page=1,trim={0cm 0cm 0cm 0cm},clip=true,width=80mm]{figurastfm/Chapter4/Imagenes/tk1.jpg}
			\caption{Apalis TK1} \label{tk1}
			%\vspace{-1cm}
\end{figure}


\subparagraph{Raspberry Pi 3 Compute Module}

The second option is one of the most known in the \acrshort{SBC} field: the Raspberry Pi 3 Compute Module. It is the industrial equivalent to the standard Raspberry Pi displayed in \autoref{sbcformats}; therefore, it is based on the same \acrshort{SoC}, the Broadcom BCM2837 up to 1.2 GHz and features 1 GB RAM, all integrated on a 67.6 mm x 31 mm board with a SODIMM connector. As for storage capabilities, it allows both 4 GB integrated flash memory and \acrshort{SD} card interface.

One of the greatest advantages of this solution is the amount of drivers already available when Linux is used as \acrshort{OS}, because of the broad usage of this board. \autoref{rpicm} shows the \acrshort{PCB}.


\begin{figure} [H]
			\centering
			\includegraphics[page=1,trim={0cm 0cm 0cm 0cm},clip=true,width=80mm]{figurastfm/Chapter4/Imagenes/rpicm.jpg}
			\caption{Raspberry Pi 3 Compute Module \cite{cm3}} \label{rpicm}
			%\vspace{-1cm}
\end{figure}


\subparagraph{Colibri VF50}


This solution is provided by Toradex \cite{colibrivf50} and is based on the NXP Freescale Vybrid \acrshort{SoC}, featuring a microprocessor Cortex A5 at 400 MHz which delivers cost effective processing and graphic performance. It counts with 128 MB RAM and 128 MB of storage.

Its major advantage is its low power consumption and value for money, however its storage capabilities are not enough for this design. \autoref{colibri} shows it.

\begin{figure} [H]
			\centering
			\includegraphics[page=1,trim={0cm 0cm 0cm 0cm},clip=true,width=80mm]{figurastfm/Chapter4/Imagenes/COLIBRI.jpg}
			\caption{Colibri VF50 \cite{colibrivf50}} \label{colibri}
			%\vspace{-1cm}
\end{figure}

\subparagraph{CL-SOM-iMX8X}

Based on the i.MX8X processor family by NXP, features a quad-core ARM Cortex-A35 up to 1.2 GHz and an integrated GPU. It also counts with 4 GB RAM and WiFi 802.11ac capabilities. Another interesting point is its wide temperature range, -40 \textdegree C to 85 \textdegree C.

Regarding connectivity, it counts with 96 \acrshort{GPIO}, 4 \acrshort{UART} and is \acrshort{USB} 3.0 ready. It is shown in \autoref{cls}.

\begin{figure} [H]
			\centering
			\includegraphics[page=1,trim={0cm 0cm 0cm 0cm},clip=true,width=80mm]{figurastfm/Chapter4/Imagenes/cls.jpg}
			\caption{CL-SOM-iMX8X} \label{cls}
			%\vspace{-1cm}
\end{figure}


\autoref{tablacm} sums up the main features of each solution.
% Please add the following required packages to your document preamble:
% \usepackage{booktabs}
\begin{table}[H]
\centering
\begin{tabular}{@{}cccccc@{}}
\toprule
\textbf{SBC}      & \textbf{System-On-Chip} & \multicolumn{1}{l}{\textbf{RAM (GB)}} & \multicolumn{1}{l}{\textbf{Flash (GB)}} & \multicolumn{1}{l}{\textbf{GPIO}} & \multicolumn{1}{l}{\textbf{Price (€)}} \\ \midrule
Apalis TK1        & NVIDIA Tegra K1         & 2                                     & 16                                      & 87                                & 211.25                                 \\ \midrule
Raspberry Pi 3 CM & Broadcom BCM2837        & 1                                     & 4                                       & 48                                & 30.30                                  \\ \midrule
Colibri VF50      & NXP Freescale Vybrid    & 0.128                                 & 0.128                                   & 103                               & 40.40                                  \\ \midrule
CL-SOM-iMX8X      & ARM Cortex A-35         & 4                                     & 4                                       & 96                                & 75                                     \\ \bottomrule
\end{tabular}
\caption{SBC comparative}
\label{tablacm}
\end{table}

The first option, Apalis TK1 has a powerful GPU which is not likely to be used for the \glsname{cubesat} purposes; therefore, taking into account its high cost, it is not a viable solution. On the hand, the Colibri VF50 is not powerful enough, although it counts with the highest number of \acrshort{GPIO} it lacks RAM and storage; the Raspberry Pi 3 CM exhibits better features at a lower price. Among the two possibilities left, besides a better value for money, the Raspberry solution, as mentioned, counts with one of the greatest communities because of its extended usage. For all these reasons, it is the option chosen.

\paragraph{Raspberry Pi 3 CM consumption analysis} \label{rpipower}

As described before, the \acrshort{CPU} is one of the most power consuming components of the \glsname{cubesat}. Therefore, it is necessary to perform a consumption analysis in different situations which allows characterizing the device completely, so the \acrshort{EPS} can be adequately designed, increasing reliability. 

To perform this analysis, different performance tests will be executed using the Raspberry Pi 3 CM. To ease these tests, the operating system \textbf{DietPi} \cite{diet} is used; it is a lightweight system which already includes different stress tests. By executing those tests, the device will increase its consumption until different ranges, including maximum before powering itself off. 

Regarding the data gathering, it is used the DC Power Analyzer Keysight N6705B, shown in \autoref{dcana}, by establishin a \acrshort{TCP} connection over a \acrshort{LAN} network, using \glsname{matlab}.

\begin{figure} [H]
			\centering
			\includegraphics[page=1,trim={0cm 0cm 0cm 0cm},clip=true,width=90mm]{figurastfm/Chapter4/Imagenes/keysight.png}
			\caption{Keysight N6705B \cite{agilent}} \label{dcana}
			%\vspace{-1cm}
\end{figure}



The structure of the tests will be the following: as the CM allows being supplied with two voltages (3.3 V and 5 V), both must be tested in order to decide. On the other hand, DietPi allows choosing the number of cores of the microprocessor used (1 to 4) as well as the CPU \textbf{governor} (\textit{ondemand, conservative, powersave or perfomance}). With all these parameters, the CM will be analyzed under three situations: at the boot (power peaks are common), in stationary state and when dealing with high CPU loads (maximum consumption).

Of course, this schema has produced a large number of plots; out of simplicity, only a selection of the most representative ones are plotted. Particularly, it is compared the power consumption for each voltage supply in every three situations. Additionally, they are compared twice: with the \textit{Powersave} governor and 1 core enabled (minimum consumption) and with \textit{Perfomance} governor and 4 core enabled (maximum consumption). This setting-up will allow determining the whole consumption range, from the minimum to the maximum.


\subparagraph{\textit{Powersave} governor - 1 core}



\begin{figure} [H]
			\centering
			\includegraphics[page=1,trim={0cm 0cm 0cm 0cm},clip=true,width=135mm]{figurastfm/Chapter4/PDF/powersboot.pdf}
			\caption{Power Consumption at \textbf{Boot}} %\label{dcana}
			\vspace{-1cm}
\end{figure}

\begin{figure} [H]
			\centering
			\includegraphics[page=1,trim={0cm 0cm 0cm 0cm},clip=true,width=135mm]{figurastfm/Chapter4/PDF/powersstat.pdf}
			\caption{Power Consumption at \textbf{Stationary State}} % \label{dcana}
			\vspace{-2cm}
\end{figure}


\begin{figure} [H]
			\centering
			\includegraphics[page=1,trim={0cm 0cm 0cm 0cm},clip=true,width=135mm]{figurastfm/Chapter4/PDF/powershigh.pdf}
			\caption{Power Consumption at \textbf{High load}} % \label{dcana}
			\vspace{-0.5cm}
\end{figure}

\subparagraph{\textit{Performance} governor - 4 cores}


\begin{figure} [H]
			\centering
			\includegraphics[page=1,trim={0cm 0cm 0cm 0cm},clip=true,width=135mm]{figurastfm/Chapter4/PDF/performboot.pdf}
			\caption{Power Consumption at \textbf{Boot}} %\label{dcana}
			\vspace{-2cm}
\end{figure}

\begin{figure} [H]
			\centering
			\includegraphics[page=1,trim={0cm 0cm 0cm 0cm},clip=true,width=135mm]{figurastfm/Chapter4/PDF/performstat.pdf}
			\caption{Power Consumption at \textbf{Stationary State}} % \label{dcana}
			%\vspace{-1cm}
\end{figure}


\begin{figure} [H]
			\centering
			\includegraphics[page=1,trim={0cm 0cm 0cm 0cm},clip=true,width=135mm]{figurastfm/Chapter4/PDF/performhigh.pdf}
			\caption{Power Consumption at \textbf{High load}} % \label{dcana}
			\vspace{-2cm}
\end{figure}

From the previous plots, some conclusions can be extracted:

\begin{itemize} [topsep=0pt]
	
	\item At \textbf{boot}, the \textbf{minimum} power consumption peak (with \textit{Powersave} governor and 1~core enabled) is \textbf{2.25 W, with 5 V supply}.
	\item At \textbf{boot}, the \textbf{maximum} power consumption peak (with \textit{Performance} governor and 4 core enabled) is \textbf{2.5 W}, regardless of the voltage supply; 11.1 \% higher than the minimum.
	\item At \textbf{stationary state}, power consumption is higher in the \textit{Performance} configuration, regardless of the voltage supply (0.49 W vs 0.61 W with 3.3 V supply and 0.62 W vs 0.78 W with 5 V supply).
	\item When dealing with a \textbf{high work load}, the power consumption is again higher with 5 V supply (1.5 W and 3.5 W with \textit{Powersave} and \textit{Performance} configuration, respectively) than with 3.3 V supply (1.2 W and 3.25 W with \textit{Powersave} and \textit{Performance} configuration, respectively). This case features the greatest increase in power consumption between the voltages supply: \textbf{170.83 \%} and \textbf{133.33 \%} increase with 3.3 V and 5 V respectively.
	\item Considering equal configuration, \textbf{power consumption with 3.3 V supply is always lower} than the equivalent with 5 V supply.
	
\end{itemize}

Therefore, in order to save the most energy possible (given that the system will be supplied by batteries) 3.3 V supply is chosen. All this information will be used in \ref{battscharac} to estimate the operating time of the \acrshort{OBC} depending on the battery used.


\subsubsection{Communications}

As described in previous sections, the Communications subsystem is included within the \acrshort{OBC}. As for this part of the \glsname{cubesat}, only the simulation purposes are considered. In a real mission, COMMS area will count with different radio-frequency systems, antennas and the rest of elements mentioned in section \ref{radiofr}. Once  again, that task can suppose a complete project by itself, so in this section, the focus will be on communications system which allows simulation-related tasks such as programming the device, downloading data, connecting to the Internet and so on.
\paragraph{Wired}

Wired connections will be the easiest way of communication when working with the platform. Although Wireless communications are intented to be used under operation, some tasks need to be performed over a wire, for example, the first programming.

\subparagraph{USB \& Ethernet}

\acrshort{USB} is used to first program the device, perform low-level tasks and \textbf{charge} the batteries. This is important considering the poor performance expected from the solar panels, as described in \autoref{ledsim}; although that charging procedure can be used to test the \acrshort{EPS}, when the simulations are related to another subsystems there must be some quick charge method, in this case provided through \acrshort	{USB}.  Several connections will be disposed, allowing not only the mentioned tasks but also connecting external devices for prototyping, such as cameras. 

Both \acrshort{USB} and Ethernet connections are complex to design as they are \textbf{differential transmission lines}. Differential signaling is a technique to transmit information using complementary signals (positive and negative) using different wires, in contrast to the more common \textbf{single-ended signaling}. \autoref{diffsignal} shows a couple of differential lines with a representation of each signal when traveling across each of the lines.

\begin{figure} [H]
			\centering
			\includegraphics[page=1,trim={0cm 0cm 0cm 0cm},clip=true,width=105mm]{figurastfm/Chapter4/PDF/diff_Signaling.pdf}
			\caption{A differential pair with a signal propagating on each line \cite{bogatin}} \label{diffsignal}
			%\vspace{-2cm}
\end{figure}


In an impedance-matched circuit, external \acrshort{EMI} affects both wires equally. Since the receiver only detects the difference between the conductors, this technique resists electromagnetic noise better than a single-ended conductor, as well as improving \acrshort{SNR} and minimizing crosstalk \cite{wiki}. \autoref{diffsch} depicts the advantages of this method.


\begin{figure} [H]
			\centering
			\includegraphics[page=1,trim={0cm 0cm 0cm 0cm},clip=true,width=115mm]{figurastfm/Chapter4/Imagenes/Diff.png}
			\caption{Differential signaling functioning \cite{wiki}} \label{diffsch}
			%\vspace{-2cm}
\end{figure}

In order to design a \acrshort{USB} line, it is necessary a \textbf{Host Controller Interface} \acrshort{HCI} or simply a controller. It is an integrated circuit which allows the computer to communicate with the \acrshort{USB} device. Indeed, Ethernet connections need another too. That is the reason for this common section: it will be used a common controller for both, particularly the \textbf{Microchip LAN9514} \cite{lan9514}. It is a dual \acrshort{USB} 2.0 and 10/100~Ethernet (full-duplex) Controller which features four \acrshort{USB} downstream ports and one upstream. \autoref{laninten} depicts the internal block diagram of the chip.


\begin{figure} [H]
			\centering
			\includegraphics[page=1,trim={0cm 0cm 0cm 0cm},clip=true,width=125mm]{figurastfm/Chapter4/PDF/lan9514_.pdf}
			\caption{LAN9514 internal block diagram \cite{lan9514}} \label{laninten}
			%\vspace{-2cm}
\end{figure}

This common usage has drawbacks though. First, the Raspberry Pi 3 Compute Module has a single connection point (which is actually composed of two differential pins) for both, Ethernet and \acrshort{USB} connections. However, if the controller is directly connected to the \acrshort{SBC}, the Compute Module cannot be turned into \textbf{slave mode}, which means that \textbf{it cannot be programmed}. In this context, \textit{programmed} means mounting the memory storage of the Compute Module as \acrshort{USB} slave and then \textbf{flash} it with an operating system image. Therefore, it is necessary a \textbf{\acrshort{USB} switch} that allows both modes depending on the connector used and the configuration selected. For this purpose, it is chosen the integrated circuit \textbf{ON Semiconductor FSUSB42} \cite{onsemi}; it is a bi-directional two \acrshort{USB} 2.0 ports switch in charge of turning the Compute Module into slave mode when a \acrshort{USB} is connected to the programming port. \autoref{swusb} depicts the functioning of this circuitry. 

\begin{figure} [H]
			\centering
			\includegraphics[page=1,trim={0cm 0cm 0cm 0cm},clip=true,width=165mm]{figurastfm/Chapter4/PDF/swusb.pdf}
			\caption{LAN9514 and \acrshort{USB} switch mechanism} \label{swusb}
			%\vspace{-2cm}
\end{figure}
Ethernet connections are designed for a differential impedance of \SI{100}{\text{ \ohm}}. Differential impedance depends on a variety of factors, featuring: track width, differential tracks separation, substrate used, substrate height and frequency. This kind of design is a complex issue which usually needs to assume different tradeoffs. There are a variety of methods to compute differential impedance; in short, every couple of differential impedance exhibits two different impedances: on the one hand, the differential one and, on the other hand, the single-ended corresponding to each of the lines (i.e., the \textbf{characteristic impedance of the line}). The equations under differential impedance are experimental, as there is no exact solution. Some of the most typically used are shown in \autoref{diffeqs}, where $\epsilon_\text{r}$ is the \glsname{permittivity}, $h$ is the substrate's height and $w$ and $t$ the width and thickness of the line, respectively. It has been used the values given by the manufacturer which will be in charge to produce the \acrshort{PCB}.

\begin{equation}
\begin{split}
Z_{\text{0}} &= \frac{87}{\sqrt{\epsilon_{\text{r}}+1.41}} \ln \frac{5.98h}{0.8w+t} \\
Z_{\text{diff}} &= 2\times Z_{\text{0}} (1-0.48\mathrm{e}^{-0.96\frac{d}{h}})\\
\end{split}\label{diffeqs}
\end{equation}


To illustrate the complexity under this operation, the differential impedance corresponding to an Ethernet line is graphically calculated, using the equations above. Particularly, \autoref{difftrace} depicts the differential impedance depending on the trace width, for three different space lengths between the lines. The first noticeable thing is that differential impedance decreases as trace width increases. The horizontal discontinuous lines depicts the target impedance for both, differential and single-ended, in the Ethernet case \textbf{100 $\Omega$} and \textbf{50 $\Omega$}, respectively. 
 
\begin{figure} [H]
			\centering
			\includegraphics[page=1,trim={0cm 0cm 0cm 0cm},clip=true,width=145mm]{figurastfm/Chapter4/PDF/diff_trace.pdf}
			\caption{Graphical calculation of differential impedance} \label{difftrace}
			\vspace{-2cm}
\end{figure}

The differential impedance varies with the spacing between lines, while the single-ended is obviously independent of this parameter. By plotting vertical lines (in discontinuous black) at the point in which each series matches its target is easy to see that \textbf{it is not possible to simultaneously have 100 $\Omega$ as differential impedance and 50~$\Omega$ as single-ended}. A certain \textbf{tradeoff} has to be assumed. In this case, the closest option is using traces of \textbf{10.8 mils} width with a separation of \textbf{14 mils} (purple series). With that design, single-ended impedance would be about 55 $\Omega$ while differential would be about 100 $\Omega$. 

While the latter example helps to understand the complexity under this kind of designs, the calculation is usually performed using dedicated and more accurate tools such as \textbf{Saturn PCB Design Toolkit} \cite{saturn}. \autoref{saturnpic} shows the calculation using that tool. It allows checking that the result accomplished by analytical calculation before is really close to the one estimated by Saturn, \textbf{105 $\Omega$}/\textbf{59 $\Omega$}, which falls into a 10 \% tolerance, enough for the design to work properly.


\begin{figure} [H]
			\centering
			\includegraphics[page=1,trim={0cm 0cm 0cm 0cm},clip=true,width=165mm]{figurastfm/Chapter4/Imagenes/saturn.png}
			\caption{Calculation of differential impedance using Saturn PCB Design Toolkit} \label{saturnpic}
			\vspace{-2cm}
\end{figure}

In order for the signals to reach the end of the line at the same time, it is important for the lines to be \textbf{length matched}. It is accomplished by tweaking the lines as shown in \autoref{ethernetlines}.

\begin{figure} [H]
			\centering
			\includegraphics[page=1,trim={0cm 0cm 0cm 0cm},clip=true,width=120mm]{figurastfm/Chapter4/PDF/eth_pair.pdf}
			\caption{Lines tweaking to achieve length matching} \label{ethernetlines}
			%\vspace{-2cm}
\end{figure}

Just like Ethernet lines, \acrshort{USB} designs must be based on differential signaling, particularly with a 90 $\Omega$ differential impedance and 45 $\Omega$ single-ended, which is accomplished using the same procedure.

Besides the embedded Ethernet design described above, in the \acrshort{OBC} design it is included a ENC28J60 \cite{enc28j60break} Ethernet breakout which provides with direct Ethernet connection by \acrshort{SPI}, shown in Figure\autoref{enc}. Its obvious drawback is its size, as seen in Figure \autoref{encobc}, with the breakout on, the \acrshort{PCB} cannot be fitted into the \glsname{cubesat} structure; also the data rate accomplished with it is significantly inferior. However, in case the embedded design fails, this could be used as \textbf{contingency option}.

			\begin{figure}[H]
			\centering
			\subfloat[Breakout\label{enc}]{\includegraphics[page=1,trim={0cm 0cm 0cm 0cm},clip=true,width=40mm]{figurastfm/Chapter4/Imagenes/enc.jpg}}
			\quad
			\subfloat[Placed in the \acrshort{OBC}, impeding its placement in the \glsname{cubesat}\label{encobc}]{\includegraphics[page=1,trim={0cm 0cm 0cm 0cm},clip=true,width=90mm]{figurastfm/Chapter4/PDF/enc.pdf}}
			\caption{ENC28J60} \label{sbcformats}
			%\vspace{-2cm}
\end{figure}

\paragraph{Wireless}

The formal requirement \textbf{OBC.FoR.15} described in \autoref{forobc}, states the need for a wireless communication system, provided by an integrated device fully compatible with \acrshort{IEEE} 802.11g standard. After designing the wired communications subsystem, this section addresses the wireless ones, specially important in order to allow a realistic simulation using the \acrshort{I2DOS}: wired communications allow neither a free rotation nor a convenient telemetry sending. Therefore, it will be crucial to accomplish a reliable communication over the air.

\subparagraph{WiFi}
Another communications related requirement was wireless communications. It will be provided with the typical technology family \textbf{WiFi}, based on the \acrshort{IEEE}~802.11~standards.

In order to include this feature, it is used the \textbf{ESP8266} microcontroller by Espressif~\cite{espressif}. It is a highly integrated Wi-Fi programmable \acrshort{SoC} solution, really power-efficient with an output power of up to 17.5 dBm. It will be used as slave to the Raspberry~Pi~3~Compute~Module with which will be completely integrated as a network interface. Figure\autoref{espfoto} shows the module while Figure\autoref{espblocks} depicts the functional block diagram.




\begin{figure}[H]
			\centering
			\subfloat[External aspect\label{espfoto}]{\includegraphics[page=1,trim={0cm 0cm 0cm 0cm},clip=true,width=55mm]{figurastfm/Chapter4/Imagenes/esp.jpg}}
			\\
			\subfloat[Functional block diagram\label{espblocks}]{\includegraphics[page=1,trim={0cm 0cm 0cm 0cm},clip=true,width=165mm]{figurastfm/Chapter4/PDF/esp.pdf}}
			\caption{ESP8266 \cite{espressif}} 
			%\vspace{-2cm}

\end{figure}

\subsubsection{Payload. Sensors}

As defined in previous sections, the payload is the \textbf{useful equipment} carried by the \glsname{cubesat}, able to provide with valuable information. Besides dedicated payloads such as cameras, also sensors and equipment necessary for the correct functioning of the \glsname{cubesat} itself  can be considered as a payload, from \glsname{startracker} to magnetometers or similar. This Master's Thesis provides with the basis for a simulation \glsname{cubesat} which includes the majority of the sensors needed to successfully perform a mission, allowing adding more elements as payload in the future. In this section, the design of the different sensors implemented is outlined. Although most of them are used by the \acrshort{ADCS} in its functioning, they can be considered part of the \acrshort{OBC} and will be placed on its board.


\paragraph{Inertial Measurement Unit}


\acrshort{IMU} have been previously defined in this project. It is in charge of measuring different parameters such as gravitational acceleration, angular rate and orientation of the device, by using accelerometers, gyroscopes and magnetometers. This design will use the \textbf{MPU9250} by Invensense \cite{mpu9250}.


%\begin{wrapfigure}{r}{0.18\textwidth} 
	%\centering
	%\includegraphics[width=0.13\textwidth]{figurastfm/chapter4/imagenes/mpu.png}
	%\vspace{-0.1cm}
	%\caption{MPU9250} %\label{MPU9250PIC}
%\end{wrapfigure}

\begin{figure} [H]
			\centering
			\includegraphics[page=1,trim={0cm 0cm 0cm 0cm},clip=true,width=40mm]{figurastfm/Chapter4/imagenes/mpu.png}
			\caption{MPU9250} \label{mpupack}
			\vspace{-0.5cm}
\end{figure}


MPU9250  is a multi-chip module consisting of two dies integrated into a single package, displayed in \autoref{mpupack}; while the first hosts the 3-Axis gyroscope and the 3-Axis accelerometer, the second houses the 3-Axis magnetometer, resulting in a 9-axis MotionTracking
device that combines a 3-axis gyroscope, 3-axis accelerometer, 3-axis magnetometer in a small 3x3x1 mm package, challenging to solder. It allows communicating using both \acrshort{I2C} and \acrshort{SPI} and features nine 16-bit \acrshort{ADC} to digitize the output of each sensor.

In order to gather the data from the sensors, it is implemented a Python library in which the different registers addressed are assigned a variable name, as well as the default \acrshort{I2C} slave addresses. Also, the different methods to read from the sensors are coded. For example, \autoref{coderead} shows the method to read the accelerometer data, using pseudocode.

\begin{lstlisting}[language=Python, caption=Method to read from the accelerometer sensors, label=coderead]
    def readAccelerometer(self):
 
        x = round(data[0]*resolution, 3)
        y = round(data[1]*resolution, 3)
        z = round(data[2]*resolution, 3)

        return {x,y,z}
\end{lstlisting}

A similar procedure is followed for the rest of the sensors, which eases accessing to the data from the main script.


\paragraph{Barometer \& Thermometer}

Specially useful for local simulation purposes, implementing a sensor easy to understand such as this gets students closer to the device and increases its interest. In this case, the integrated circuit used is the tiny Bosch BMP280 \cite{bosch}, shown in \autoref{bmp}.

\begin{figure} [H]
			\centering
			\includegraphics[page=1,trim={0cm 0cm 0cm 0cm},clip=true,width=55mm]{figurastfm/Chapter4/Imagenes/BMP.jpg}
			\caption{Bosch BMP280 \cite{bosch}} \label{bmp}
			\vspace{-0.5cm}
\end{figure}



Bosch BMP280 exhibits high accuracy and linearity, as well as an almost negligible power consumption (with an average current of \SI{2.74}{\micro A}. Besides, it allows a wide range of voltage supply, between 1.71 V and 3.6 V. Its technical parameters comply with the requirements defined in \autoref{chap:chapter3}.

Again, it is developed a Python library to  read from the integrated sensors, thermometer and barometer. This chip is also able to estimate the altitude above sea level from the pressure detected. Also, it is able to be programmed with different over-sampling settings for both barometer and temperature sensor; each oversampling step reduces noise and increases the output resolution by one bit. \autoref{bmppcb} shows the BMP280 placed in the \acrshort{OBC} \acrshort{PCB}.

\begin{figure} [H]
			\centering
			\includegraphics[page=1,trim={0cm 0cm 0cm 0cm},clip=true,width=100mm]{figurastfm/Chapter4/PDF/bosch.pdf}
			\caption{Bosch BMP280 placed in the \acrshort{OBC} \acrshort{PCB}} \label{bmppcb}
			\vspace{-2cm}
\end{figure}


\paragraph{Sun sensors. ADC}

In order to determine the position of the real or simulated Sun, it is necessary to include the so-called \textbf{Sun sensors}. As for every component implied in a space-oriented design, there are qualified Sun sensors, intended to be used in a real mission, \autoref{ssensor} shows an example from \cite{csatshop}.


\begin{figure} [H]
			\centering
			\includegraphics[page=1,trim={0cm 0cm 0cm 0cm},clip=true,width=70mm]{figurastfm/Chapter4/Imagenes/ssensor.png}
			\caption{NCSS-SA05 Sun Sensor} \label{ssensor}
				%\vspace{-2cm}
\end{figure}

As usual, the \textit{problem} with space qualified components is the cost: the NCSS-SA05 costs around 3300 €. That value makes it unfeasible for this project so, instead, the Sun sensors used will be based on \acrshort{LDR}. 

\acrshort{LDR} will make it possible, depending on the intensity measured, to determine with enough accuracy the position of the light. Given that \acrshort{LDR} are ultimately resistances, it is necessary to arrange some circuitry to determine the Sun's position. The easiest way to accomplish that point is designing a voltage divider between the \acrshort{LDR} itself and another resistor. When supplied, a certain voltage falls in the measurement point, which is gathered using an \acrshort{ADC}. \autoref{ldrcircuit} depicts the circuit described.


\begin{figure} [H]
			\centering
			\includegraphics[page=1,trim={0cm 0cm 0cm 0cm},clip=true,width=70mm]{figurastfm/Chapter4/PDF/ldr_circuit.pdf}
			\caption{Sun sensor based on \acrshort{LDR} circuit} \label{ldrcircuit}
			\vspace{-2cm}
\end{figure}

As seen in \autoref{ldrcircuit}, it is needed another resistor to complete the voltage divider. Given that a certain \acrshort{LDR} will exhibit a fixed resistance range, the \textbf{excursion~output~range} of the voltage divider (i.e., the difference between the minimum and the maximum voltage measured at the point $ADC\_READ$) will depend on the resistor magnitude chosen. Therefore, it is necessary to choose that resistor in order to \textbf{optimize} the excursion output range and get the most resolution possible.

In the \autoref{ldrcircuit} above, \acrshort{LDR} is denoted as $R^*$ and $R$ is another resistance completing the voltage divider. Depending on the light falling upon the \acrshort{LDR}, $R^*$ varies, so does the voltage $V$. That voltage is directly connected to the \acrshort{ADC} so it is converted into a discrete value. With this circuit, $V$ will increase when light level is high (facing Sun) whereas it will decrease when light reduces (eclipse). 

Ideally, $U$ voltage should go through the range $[0 \text{ V},U_0]$ to maximize distinguishable values. Considering that the only degree of freedom is given by $R$ value, it must be the point of the optimization. On the other hand, extreme resistance values of the \acrshort{LDR} must be known so the whole resistance of the voltage divider can be calculated. Therefore, for a certain resistance $R^* \in [R^*_{\text{min}}, R^*_{\text{max}}]$:

\begin{align*}
U(R)=U_0 \frac{R}{R^* + R}
\end{align*}

Since the aim is maximizing the voltage margin for $U$, it will be denoted as $\Delta U(R)$, which is the electric potential differential across $R^*$:

% <![CDATA[
\begin{align*}
\Delta U(R) &= U_{\text{max}}(R) - U_{\text{min}}(R) \\
&= U_0 \Bigg[\frac{R}{R^*_{\text{min}} + R} - \frac{R}{R^*_{\text{max}} + R} \Bigg] \\
&= U_0 \Bigg[\frac{R(R^*_{\text{max}} - R^*_{\text{min}})}{(R^*_{\text{min}} + R)(R^*_{\text{max}} + R)} \Bigg] \\
\end{align*}


In order to maximize the function $\Delta U(R)$, the derivative $\frac{\partial}{\partial R} \Delta U $ is calculated:

\begin{align*}
\centering
\frac{\partial}{\partial R} \Delta U &= U_0\Bigg[\frac{(R^*_{\text{min}} - R^*_{\text{max}})\big[R^2-R^*_{\text{min}}R^*_{\text{max}}\big]}{(R^*_{\text{min}} + R)^2(R^*_{\text{max}} + R)^2}\Bigg]
\end{align*} %]]>

Then, it is solved $\frac{\partial}{\partial R} \Delta U = 0$ to find extreme point or \textbf{extrema}:

\begin{align*}
\centering
U_0\Bigg[\frac{(R^*_{\text{min}} - R^*_{\text{max}})\big[R^2-R^*_{\text{min}}R^*_{\text{max}}\big]}{(R^*_{\text{min}} + R)^2(R^*_{\text{max}} + R)^2}\Bigg] =  0 \rightarrow 
(R^*_{\text{min}} - R^*_{\text{max}})\big[R^2-R^*_{\text{min}}R^*_{\text{max}}\big] =  0
\end{align*} %]]>

Assuming $R^*_{\text{max}}>R^*_{\text{min}}$, solving the last equation yields to:

\begin{align*}
R &= \sqrt{R^*_{\text{min}}R^*_{\text{max}}} \\
\end{align*} %]]>


Therefore, the optimum resistance in order to get the greatest excursion in the output voltage is given by the square root of the product of the  \acrshort{LDR} extreme values, which means that the optimum resistor depends on the chosen \acrshort{LDR}. In order to make that decision, \autoref{ldrcomp} shows the behavior of the function $\Delta U(R)$ depending on the extreme resistance values of different photoresistor. Because of availability reasons, the \acrshort{LDR} corresponding to the \textbf{blue} series is chosen which, with an associated optimum resistance $ R = $ \SI{1}{k\ohm}, reaches around 80 \% of the possible excursion output.
		
		\begin{figure} [H]
			\centering
			\includegraphics[page=1,trim={0cm 0cm 0cm 0cm},clip=true,width=140mm]{FigurasTFM/Chapter4/PDF/ldr_comp.pdf}
			\caption{$\Delta{U}$ ratio for different LDR extreme values}      		\label{ldrcomp}
			\vspace{-1cm}
\end{figure}

Once it has been accomplished an optimum excursion output, that value has to be read using an \acrshort{ADC}. As for this design, the one chosen is the \textbf{Texas~Instruments~ADS1015}~\cite{ads15}. It is a low-power \acrshort{I2C} compatible \acrshort{ADC} with a maximum resolution of 12 bits and up to four measurements; therefore, considering that there is one Sun Sensor per face of the \glsname{cubesat}, two \acrshort{ADC} are needed. They must be assigned two different \acrshort{I2C} addresses in order to allow communication with both.

Following the same procedure, a simple \acrshort{I2C} communication library has been developed, allowing data polling. This \acrshort{ADC} also features the \textit{ALERT} function, which uses a digital comparator that can issue an alert on the corresponding pin when conversion data exceeds the limits set in  threshold registers. 

\paragraph{Real-Time Clock}

A Real-Time Clock or \acrshort{RTC} is an integrated circuit which keeps track of the time. They must be distinguished from the typical hardware clocks, which are simple signals not related to time in typical units.

\acrshort{RTC} are based on crystal oscillators, typically at a frequency of \textbf{32.768 kHz}, because of its convenience with binary counter circuits. In this case, it is used in conjunction with the \acrshort{CPU}, so it is able to be aware of the current time; this is not needed when it is connected to the Internet, as it gets time by that via, but in any other case the system will start in \textbf{Unix Time}, the first of January, 1970. 

As for this design, the \acrshort{RTC} chosen is the PCF8523 by NXP \cite{rtcnxp}, along with a CR2032 battery which ensures it keeps tracking time when the central system is unpowered. \autoref{rtcpic} shows the placement of the ADS1015 with the crystal oscillator in the \acrshort{PCB}.

		\begin{figure} [H]
			\centering
			\includegraphics[page=1,trim={0cm 0cm 0cm 0cm},clip=true,width=100mm]{FigurasTFM/Chapter4/PDF/rtc.pdf}
			\caption{\acrshort{RTC} and crystal oscillator in the \acrshort{OBC} \acrshort{PCB}}      		\label{rtcpic}
			\vspace{-2cm}
\end{figure}


\newpage

\subsubsection{Flight Software. On-Board Data Handling}

As stated in section \ref{flight}, Flight Software is in charge of controlling the whole system and dealing with all the possible situations during a mission. In this section, it is addressed the software under the whole \glsname{cubesat} as well as implemented a basic \acrshort{OBDH}.


\vspace{-0.5cm}
\paragraph{Operating System} \label{rtosdes}

\acrshort{OS} were divided into Real-time and Non-real-time. The solution taken in this project is somewhere in the middle of both: it is used an initially Non-\acrshort{RTOS} such as Linux, which is patched to be turned into a \acrshort{RTOS}. This way, it is possible to maintain the great drivers compatibility of Linux but adding the deterministic nature of the \acrshort{RTOS}.

In order to perform this change, it is necessary to \textbf{recompile the Linux kernel}. By default, Linux kernel is non-preemptive. To turn it into a preemptive scheduling, it is applied the \textbf{PREEMPT\_RT patch} \cite{linuxrtos}, which provides with faster response times and removes unbounded latencies. 

The first task is downloading the desired version of the kernel. Then, the corresponding version of the patch is also downloaded and unpacked. It can be patched following the instructions in \cite{linuxpat}. After patching, the kernel is rebuilt and configured as \textbf{preemptive}.


\paragraph{Telemetry System}

As for this first prototype, it is designed a simple telemetry sending system using \textbf{Python}, particularly the \textbf{socket} library. It connects to a certain IP address and port with the instructions shown in \autoref{socket}. 


\begin{lstlisting}[language=Python, caption=Connection to server using the \textbf{socket} library, label=socket]
  
	s = socket.socket(socket.AF_INET, socket.SOCK_STREAM)
	host = '192.168.1.118'
	port = 8445
	s.connect((host, port)) 	
	
\end{lstlisting}

Particularly, the tuple IP:Port must match the existing at the other end of the communication, in which \textbf{the COSMOS \glsname{ground} is listening}, as seen in \autoref{cosmosgs}. Once the connection is established, the flow of communication with telemetry and telecommands can start. %As for the downlink (to the \glsname{ground}), the information is packed and sent down using the established connection.

\subsubsection{PCB design}

This section includes the different schematics and information related to the \acrshort{PCB} designed for the \acrshort{OBC}. In order to allow its production, this design must comply with the constraints given by the manufacturer; these depends on the number of layers of the \acrshort{PCB}, while the \acrshort{OBC} is a 4-layers board, the \acrshort{ADCS} \acrshort{PCB} (addressed in section \ref{adcsdes}) is a 2-layers board and the constraints \textbf{change}. Particularly, \autoref{conschino} depicts the different design requirements expected by the manufacturer: the first one for \textbf{2-layers} and the second one for\textbf{ 4-layers}.

			\begin{figure}[H]
			\centering
			\subfloat[2-layers]{\includegraphics[page=1,trim={0cm 0cm 0cm 0cm},clip=true,width=165mm]{figurastfm/Chapter4/PDF/pcbcons.pdf}}
			\\
			\subfloat[4-layers]{\includegraphics[page=1,trim={0cm 0cm 0cm 0cm},clip=true,width=165mm]{figurastfm/Chapter4/PDF/pcbcons4layer.pdf}}
			\caption{\acrshort{PCB} Manufacturing constraints \cite{pablotesis}} \label{conschino}
			\vspace{-1cm}
\end{figure}

\newpage

\autoref{final} shows a 3D render of the top and bottom layer in the final design, while \autoref{persp} shows a perspective view from different angles. 			
%\vspace{1cm}





			\begin{figure}[H]
			\centering
			\subfloat[Top]{\includegraphics[page=1,trim={0cm 0cm 0cm 0cm},clip=true,width=80mm]{figurastfm/Chapter4/PDF/obc_3d.pdf}}
			\quad
			\subfloat[Bottom]{\includegraphics[page=2,trim={0cm 0cm 0cm 0cm},clip=true,width=80mm]{figurastfm/Chapter4/PDF/obc_3d.pdf}}
			\caption{\acrshort{OBC} \acrshort{PCB} final design} \label{final}
			%\vspace{-0.5cm}
\end{figure}

\vspace{1cm}


		\begin{figure}[H]
			\centering
			\subfloat[]{\includegraphics[page=1,trim={0cm 0cm 0cm 0cm},clip=true,width=80mm]{figurastfm/Chapter4/PDF/3d_persp.pdf}}
			\quad
			\subfloat[]{\includegraphics[page=1,trim={0cm 0cm 0cm 0cm},clip=true,width=80mm]{figurastfm/Chapter4/PDF/3d_persp_3.pdf}}
			\caption{\acrshort{OBC} \acrshort{PCB} 3D perspective} \label{persp}
			\vspace{-2cm}
\end{figure}
%
%\includepdf[pages=-,link=true,landscape,linkname=schematics]{FigurasTFM/Chapter4/PDF/sch_OBC.PDF}
%\includepdf[pages=-,link=true,landscape,linkname=schematics]{FigurasTFM/Chapter4/PDF/OBC_PCB.PDF}
\includepdf[pages=-,link=true,landscape,linkname=schematics]{FigurasTFM/Chapter4/PDF/obc_3dview.PDF}


\newpage

\subsubsection{Manufacturing and soldering} \label{obcmanuf}

Once the \acrshort{PCB} are designed, they are sent to a professional manufacturer. The result is shown in \autoref{obcpcb}.

\begin{figure}[H]
			\centering
			\subfloat[Top]{\includegraphics[page=1,trim={0cm 0cm 0cm 0cm},clip=true,width=80mm]{figurastfm/Chapter4/Imagenes/obc_top.png}}
			\quad
			\subfloat[Bottom]{\includegraphics[page=1,trim={0cm 0cm 0cm 0cm},clip=true,width=80mm]{figurastfm/Chapter4/Imagenes/obc_bot.png}}
			\caption{\acrshort{OBC} \acrshort{PCB} manufactured} \label{obcpcb}
			%\vspace{-2cm}
\end{figure}

Because of certain components packages (for instance, the LAN9514 or the Compute Module), this \acrshort{PCB} cannot be soldered by hand. It is needed a technique called \textbf{oven reflow}. It is a procedure in which the first step consists in covering the \acrshort{PCB} with \textbf{solder paste} using a \glsname{stencil} and then the components are placed onto the board. \autoref{obcsolder} shows different moments of this process.

Once the \acrshort{PCB} is fully assembled and all the components are in place, it undergoes the reflow process, which in this case is performed with a domestic oven which had been previously characterized with a trial-and-error approach. Video \autoref{reflow} shows the \acrshort{PCB} undergoing the process (video available with Adobe Acrobat Reader).


\begin{figure}[H]
			\centering
			\subfloat[Solder paste transfer using a \glsname{stencil}]{\includegraphics[page=1,trim={0cm 0cm 0cm 0cm},clip=true,width=81.5mm]{figurastfm/Chapter4/Imagenes/stencil.jpg}} 
			\quad
			\subfloat[QFN package placement]{\includegraphics[page=1,trim={0cm 0cm 0cm 0cm},clip=true,width=76mm]{figurastfm/Chapter4/Imagenes/detail_place.jpg}}
			%\\
			%\subfloat[\acrshort{PCB} partially assembled]{\includegraphics[page=1,trim={0cm 0cm 0cm 0cm},clip=true,width=75mm]{figurastfm/Chapter4/Imagenes/detail_place2.jpg}}
			\caption{Soldering process} \label{obcsolder}
			\vspace{2cm}
\end{figure}

\begin{videoFloat}[H]
	\centering
\includemovie[text={\includegraphics[page=1,trim={0cm 0cm 0cm 0cm},clip=true,width=90mm]{figurastfm/Chapter4/Imagenes/inoven.jpg}}]{10cm}{6cm}{FigurasTFM/Chapter4/Video/makerspace.wmv}
\vspace{2.8cm}
\caption{Oven reflow (video available with Adobe Acrobat Reader)} \label{reflow}
%\vspace{-2cm}
\end{videoFloat}

\newpage

\autoref{topfinalobc} and \autoref{botfinalobc} show the final product after completing soldering.


		\begin{figure} [H]
			\raggedright
			\begin{minipage}{10cm}
 			\includegraphics[page=1,trim={0cm 0cm 0cm 0cm},clip=true,width=100mm]{figurastfm/Chapter4/Imagenes/final_obc.png}
			\centering
			\caption{Top layer of the \acrshort{OBC}}     \label{topfinalobc}
			  \end{minipage}
			\vspace{-1cm}
\end{figure}


		\begin{figure} [H]
		\raggedleft
			\begin{minipage}{10cm}
			\includegraphics[page=1,trim={0cm 0cm 0cm 0cm},clip=true,width=100mm]{figurastfm/Chapter4/Imagenes/final_obc_bot.png}
			\caption{Bottom layer of the \acrshort{OBC}}     \label{botfinalobc}
						  \end{minipage}
			\vspace{-2cm}

\end{figure}


\newpage
\subsection{Attitude Determination and Control System (ADCS)}

This section addresses the design of the \acrshort{ADCS}, in charge of keeping the \glsname{cubesat} correctly oriented and managing external torques properly. At this point, some clarifications must be made with respect to the \acrshort{ADCS} analysis in \autoref{chap:chapter3}. For instance, this subsection will include the \textbf{Co-processing Programmable Core} initially analyzed in the \acrshort{OBC} section; this is due to the need of an independent processing unit which frees the \acrshort{CPU} of this tasks and allows having a real \acrshort{ADCS} managing the issues regarding this matter. On the  other hand, the \textbf{tachometer} is the only sensor which is included in this section, as it is placed in the \acrshort{ADCS} board and is explicitly necessary for its operation. The \textbf{actuators} included in the design are also considered here.

\subsubsection{Co-processing Programmable Core}

As described in section \ref{copropo}, this architectures are usually implemented using \acrshort{FPGA}. Using an \acrshort{FPGA} provides with \textbf{time-critical} capabilities which goes in the same line of the \acrshort{RTOS} introduced in \autoref{rtosdes}: while with the operating system that capability is provided inherently by software, an \acrshort{FPGA} allows expanding that concept to the hardware parcel. Time accuracy and management is a vital issue in space field, which justifies the different designs addressed in this project on said line. 

Therefore, this Master's Thesis will make use of an \acrshort{FPGA}, particularly the ICE40UP5K by Lattice \cite{up5k}. It counts with 5000 \acrshort{LUT} and is provided in a variety of packages, in this case, \textbf{QFN} is chosen. Typically,  \acrshort{FPGA} are programmed from an external memory in a \textbf{master-slave} configuration in which the \acrshort{FPGA} acts as master. In this project, however, given the presence of a central processing unit (the Raspberry Pi 3 Compute Module), it is designed the other way around: it acts as master while the \acrshort{FPGA} does as \textbf{slave}. This way, each time the \acrshort{SBC} boots, it sends (as master) a \textbf{bitstream} through an \acrshort{SPI} channel which contains the programming code. Additionally, the same channel is used to transfer the information during the \acrshort{FPGA} operation, detailed next.

\paragraph{PWM generation}

In this project, the \acrshort{FPGA} main functionality will be generating different \acrshort{PWM} signals to control the actuators, given that the \acrshort{CPU} cannot produce the amount needed; however, it could be used for a variety of purposes, which is the idea under the Co-processing Programmable Core. As described before, the same \acrshort{SPI} channel is used to configure each of the \acrshort{PWM}. \autoref{fpgadiag} depicts the basis of this operation.

		\begin{figure} [H]
			\centering
			\includegraphics[page=1,trim={0cm 0cm 0cm 0cm},clip=true,width=165mm]{FigurasTFM/Chapter4/PDF/fpga.pdf}
			\caption{\acrshort{FPGA} functioning as \acrshort{PWM} generator }      		\label{fpgadiag}
			%\vspace{-2cm}
\end{figure}

The \textbf{bitstream} is composed of two bytes; while the first of them chooses the particular \acrshort{PWM} to be set up, the second one states the configuration of it, in terms of \textbf{frequency} and \textbf{duty cycle}. The \acrshort{PWM} are generated by using a 25 MHz  oscillator, which is divided as needed to accomplish the desired frequency. Finally, the generated \acrshort{PWM} outputs in \textbf{one of the pins} of the \acrshort{GPIO} from the banks. As \textbf{Hardware Description Language} it has been used \textbf{Verilog}.


\subsubsection{Actuator: reaction wheel}

Actuators are used to change or control the attitude of the \glsname{cubesat}. As for this prototype, it has been implemented a \textbf{reaction wheel}. However, the \acrshort{ADCS} is capable of managing up to three \glsname{magnetorquers} which can be added in the future, using also the different \acrshort{PWM} signals available. 

As reaction wheel, it has been reused the one available at \glsname{GranaSAT} laboratory, shown in \autoref{3dreact}. In order to propel the wheel, it is necessary a DC motor. As usual, it is not possible to supply a DC motor directly; in this case, for example, it can provide with just a few mA, so an \textbf{H-bridge} is needed. It is a device composed of four switches that control the flow of large currents to a load, which is precisely its usefulness: allows, with a low-power control signal (in this case, the \acrshort{PWM}) to drive high loads, such as a motor. It is composed of a simple circuit, depicted in \autoref{hbridge}.

		\begin{figure} [H]
			\centering
			\includegraphics[page=1,trim={0cm 0cm 0cm 0cm},clip=true,width=65mm]{FigurasTFM/Chapter4/PDF/H_bridge.pdf}
			\caption{H-bridge basic circuit \cite{wiki}}      		\label{hbridge}
			%\vspace{-2cm}
\end{figure}

In sum, the \acrshort{PWM} generated by the \acrshort{FPGA} attack the H-bridge (control signals) and it supplies the load with the needed current (larger than the one the \acrshort{FPGA} can provide). The H-bridge could attack the motor directly and supply it with all the current demanded, however, this implies two drawbacks: on the one hand, the motor would be driven at the maximum speed continuously, which reduces its control dramatically; on the other hand, it would be propelled in the same fixed direction. An H-bridge allows controlling both parameters.

As for this design, several H-bridges are used, listed next:

\begin{itemize} [noitemsep,topsep=0pt]
\item\textbf{ ST L298N}
\end{itemize}

The L298N by ST \cite{l298n} is an integrated high voltage, high current Dual Full-Bridge Driver. It allows an operating supply voltage of up to 46 V @ 4 A. Additionally, it features two \textit{enable} inputs that allow the L298 to be enabled or disabled independently of the input signals. Because of its high power capabilities, it is used to control the motor driving the reaction wheel. It will be used a single unit, which provides with \textbf{four outputs}; as the motor only needs a couple, the other two are used for the X-axis magnetorquer.


		\begin{figure} [H]
			\centering
			\includegraphics[page=1,trim={0cm 0cm 0cm 0cm},clip=true,width=50mm]{FigurasTFM/Chapter4/Imagenes/l298.jpg}
			\caption{ST L298N \cite{l298n}}  %    		\label{hbridge}
			\vspace{-1cm}
\end{figure}
\newpage



\begin{itemize} [noitemsep,topsep=0pt]
\item \textbf{ROHM BD6211F-E2}
\end{itemize}

As stated before, the \acrshort{ADCS} will count with the capability to use \glsname{magnetorquers} in the future. They must be supplied with an H-bridge, because of the same reasons of a motor, essentially. However, the current needed for \glsname{magnetorquers} is much lower, so a smaller and cheaper bridge is used, particularly the BD6211F-E2 by ROHM \cite{rohm}. It is able to supply with up to 1 A with 5.5 V. 

		\begin{figure} [H]
			\centering
			\includegraphics[page=1,trim={0cm 0cm 0cm 0cm},clip=true,width=45mm]{FigurasTFM/Chapter4/Imagenes/ROHM.jpg}
			\caption{ROHM BD6211F-E2 \cite{rohm}}  %    		\label{hbridge}
			%\vspace{-2cm}
\end{figure}

Each BD6211F-E2 exhibits two outputs; considering that the X-axis magnetorquer is supplied by the L928N, each of the axis left needs another couple of outputs so they are necessary two BD6211F-E2. 



\subsubsection{Sensor: Tachometer}

In order to adequately control the reaction wheel, it is implemented a \textbf{tachometer}, able to measure the rotation speed of the wheel. To design the tachometer, it is used a \textbf{Reflective Optical Sensor}. These devices produce an infrared light onto a certain surface and the reflection is measured; when there is no reflection the sensor will show a \textbf{0} at the output, while a certain reflection will produce a \textbf{1}. Particularly, the sensor used is the CNY70 by Vishay \cite{cny}, shown in Figure\autoref{cny70}.


\begin{figure}[H]
			\centering
			\subfloat[Package \label{cny70}]{\includegraphics[page=1,trim={0cm 0cm 0cm 0cm},clip=true,width=35mm]{figurastfm/Chapter4/Imagenes/cny70.jpg}} 
			\quad
			\subfloat[Top view]{\includegraphics[page=1,trim={0cm 0cm 0cm 0cm},clip=true,width=45mm]{figurastfm/Chapter4/PDF/cny70.pdf}}
			\caption{Vishay CNY70 \cite{cny}} %\label{obcsolder}
			%\vspace{-2cm}
\end{figure}

The maximum distance at which the CNY70 is able to measure reflection depends on the \textbf{forward current} of the transistor, which in turn is subject to the emitter diode current. Therefore, the resistors needed to bias the device have to be optimized. The maximum measurement distance is accomplished when the forward current is about \textbf{20 mA}, so regarding the \textbf{diode}, the following equations can be considered:


\begin{equation}
\centering
I_{\text{forward}}=\frac{V_{\text{cc}}-V_{\text{forward}}}{R_{\text{diode}}}
\end{equation} %]]>


Considering a supply of 3.3 V and taking into account that, according to the datasheet \cite{cny}, the typical forward voltage is about\textbf{ 1.25 V}, it yields to:


\begin{equation} 
\centering
20 \text{ mA}=\frac{5 \text{ V}- 1.25 \text{ V}}{R_{\text{D}}} \rightarrow R_{\text{D}} = \SI{102.5}{\ohm}
\end{equation} %]]>

As for the phototransistor, for a typical collector current of 0.1 mA, at a distance of 0.3 mm, the maximum $V_{\text{CE}}$ under saturation is 0.3 V. So, substituting in \autoref{eqtrans}:



\begin{align} \label{eqtrans}
\centering
I_{\text{C}}=\frac{V_{\text{cc}}-V_{\text{CE}}}{R_{\text{collector}}} \rightarrow R_{\text{collector}} = \SI{30}{k\ohm}
\end{align} %]]>

In order to test the accuracy of the system, the CNY70 is used with the configuration determined to measure the rotation rate of a reaction wheel. At the same time, it is measured using a professional tachometer. The results are plot together at \autoref{tachom}.

		\begin{figure} [H]
			\centering
			\includegraphics[page=1,trim={0cm 0cm 0cm 0cm},clip=true,width=137mm]{FigurasTFM/Chapter4/PDF/rpm.pdf}
			\caption{Rotation rate measured with a professional tachometer and with the CNY70 circuit designed} 	\label{tachom}
			\vspace{-2cm}
\end{figure}

As expected, the rotation rate is linear with the voltage supply. On the other hand, the error made is completely negligible, with an almost exact coincidence between both measurements. This validates the design of the tachometer based on the CNY70.


\subsubsection{Control law}

The control law is the software implementation governing the hardware above. It deals with the variety of situations along a mission and acts consequently. As starting point for this prototype, it is implemented a control law based on a \acrshort{PID}, described in section \ref{pid3}. Particularly, it has been used the library \textbf{simple\_pid} in Python~\cite{simplepid}. \autoref{pidop} shows an example of the \acrshort{PID} counteracting the movement of the  \glsname{cubesat}.

\begin{figure}[H]
			\centering
			\subfloat[Long operation]{\includegraphics[page=1,trim={0cm 0cm 0cm 0cm},clip=true,width=147mm]{figurastfm/Chapter4/PDF/pid3.pdf}} 
			\quad
	\subfloat[Detail]{\includegraphics[page=1,trim={0cm 0cm 0cm 0cm},clip=true,width=147mm]{figurastfm/Chapter4/PDF/pid2.pdf}} 
			\caption{\acrshort{PID} counteracting Z-axis rotation} \label{pidop}
			\vspace{-2cm}
\end{figure}


\newpage
\subsubsection{PCB design} \label{adcsdes}
\autoref{finaladcs} shows a 3D render of the top and bottom layer in the final design, while \autoref{perspadcs} shows a perspective view from different angles. 			
%\vspace{1cm}





			\begin{figure}[H]
			\centering
			\subfloat[Top]{\includegraphics[page=1,trim={0cm 0cm 0cm 0cm},clip=true,width=80mm]{figurastfm/Chapter4/PDF/adcs_top.pdf}}
			\quad
			\subfloat[Bottom]{\includegraphics[page=1,trim={0cm 0cm 0cm 0cm},clip=true,width=80mm]{figurastfm/Chapter4/PDF/adcs_bot.pdf}}
			\caption{\acrshort{ADCS} \acrshort{PCB} final design} \label{finaladcs}
			%\vspace{-0.5cm}
\end{figure}

\vspace{1cm}


		\begin{figure}[H]
			\centering
			\subfloat[]{\includegraphics[page=1,trim={0cm 0cm 0cm 0cm},clip=true,width=80mm]{figurastfm/Chapter4/PDF/adcs_persp.pdf}}
			\quad
			\subfloat[]{\includegraphics[page=1,trim={0cm 0cm 0cm 0cm},clip=true,width=80mm]{figurastfm/Chapter4/PDF/adcs_persp2.pdf}}
			\caption{\acrshort{ADCS} \acrshort{PCB} 3D perspective} \label{perspadcs}
			\vspace{-2cm}
\end{figure}

%\includepdf[pages=-,link=true,landscape,linkname=schematics]{FigurasTFM/Chapter4/PDF/ADCS.PDF}
%\includepdf[pages=-,link=true,landscape,linkname=schematics]{FigurasTFM/Chapter4/PDF/ADCS_PCB.PDF}
\includepdf[pages=-,link=true,landscape,linkname=schematics]{FigurasTFM/Chapter4/PDF/adcs_3d.PDF}

\subsubsection{Manufacturing and soldering}

In this case, the \acrshort{PCB} is composed of 2 layers. The design is sent to a professional manufacturer too, resulting in the \acrshort{PCB} shown in \autoref{adcspcb}.

\begin{figure}[H]
			\centering
			\subfloat[Top]{\includegraphics[page=1,trim={0cm 0cm 0cm 0cm},clip=true,width=80mm]{figurastfm/Chapter4/Imagenes/adcs_top.png}}
			\quad
			\subfloat[Bottom]{\includegraphics[page=1,trim={0cm 0cm 0cm 0cm},clip=true,width=80mm]{figurastfm/Chapter4/Imagenes/adcs_bot.png}}
			\caption{\acrshort{ADCS} \acrshort{PCB} manufactured} \label{adcspcb}
			\vspace{-0.5cm}
\end{figure}

As for the soldering process, the procedure is the same as the one described in section \ref{obcmanuf}; the result is shown in \autoref{adcspcbsolder}.

		\begin{figure} [H]
			\centering
			\includegraphics[page=1,trim={0cm 0cm 0cm 0cm},clip=true,width=115mm]{FigurasTFM/Chapter4/Imagenes/adcs_final.png}
			\caption{\acrshort{ADCS} \acrshort{PCB} after assembling and soldering} 	\label{adcspcbsolder}
			\vspace{-2cm}
\end{figure}



\newpage
\subsection{Electrical Power System (EPS)}

Regarding the \acrshort{EPS}, this Master's Thesis addresses some of the crucial choices to be made when facing the design of this kind of system. While the electronics design is left for future projects, this one will lay the foundations for the photovoltaic subsystem of the \textbf{GranaSAT-I}, by analyzing and characterizing different solar panels in order to determine the optimum one according to the rest of the \glsname{cubesat}. On the other hand, a set of batteries are characterized at different charge/discharge rates, solving part of the decisions regarding the energy storage system. 


\subsubsection{Solar panels}

Solar cells were profusely analyzed in \autoref{chap:chapter3} so in this section, the \textbf{testbench} designed for its characterization is presented first and then different solar panels are characterized.

\paragraph{Measurements testbench}

Characterizing solar panels is a complex matter. As described in previous sections, generally speaking, solar cells behavior is described by its I-V curve. However, obtaining said curve is not as simple as it could seem: if the output of a solar panel is measured when it is under light, a certain voltage is obtained, the $V_{\text{OC}}$, which is useless as it cannot deliver any power at that point (see section \ref{solarcellsexpl}). So in order to get the complete curve it is necessary to sweep a voltage range when the solar panel is \textbf{loaded}, this is, emulating a real operation environment. It is easy to understand under the basis stated in previous explanations: if the solar panel is forced to work at a certain voltage, against a certain load, it will deliver a given power; if that voltage varies, so does the rest, getting the complete I-V curve. Particularly, the voltage sweep must cover the whole expected functioning range. 

Simulating all these variables requires an adequate \textbf{testbench} composed of different equipment, depicted in \autoref{testbench}. The testbench has been designed in a way which allows measuring in both, \textbf{outdoor} (solar light) and \textbf{indoor} (simulators) environments. Firstly, the solar panel is connected to the \textbf{load}, which is composed of two devices: the \textbf{HP 6063B} \cite{hpdata1} and the \textbf{HP E3631A} \cite {hpdata2}. While the first is an electronic load, used to perform the \textbf{voltage sweep}, the second is a simple DC power source used to impose a certain voltage offset to the HP 6063B output, needed for correct functioning. Both, acting as load, forces the solar panel output a certain voltage (\textbf{$V_{\text{sweep}}$}); as it is exposed to an irradiance source (whether actual Sun light or simulated), the solar panel provides with the \textbf{current} corresponding to the impedance seen from the panel, which varies with the voltage sweep. As the sweep is performed, both the voltage imposed and the current provided is measured using the multimeter SIGLENT~SDM3065X \cite{siglentmult}. Additionally, the testbench counts with a pyranometer to be placed along with the solar panel, so the irradiance present at the test can be taken into account; it is also connected to the multimeter, and the voltage measured is converted into an irradiance measurement.

The whole equipment is controlled using a \acrshort{matlab} library developed progressively throughout the years at the \glsname{GranaSAT} laboratory, which also allows downloading the data. Finally, the I-V curve can be plotted.

		\begin{figure} [H]
			\centering
			\includegraphics[page=1,trim={0cm 0cm 0cm 0cm},clip=true,width=165mm]{FigurasTFM/Chapter4/PDF/testbench.pdf}
			\caption{Solar panels characterization testbench} 	\label{testbench}
			%\vspace{-0.5cm}
\end{figure}


\paragraph{Characterizations}

In this section, different solar panels are characterized using the testbench just described. Firstly,  in order to assure the correct functioning of the characterization bench, it is performed a test under Sun's illumination, on a high quality solar panel with a known IV curve. It is a 5 W polycristalline solar panel with an open-circuit voltage of 22 V \cite{panelrs}, shown in \autoref{rs}.

The irradiance measured outdoors is \SI{1000}{W/m^2} and it slightly varied for all the characterizations with solar lightning performed.


		\begin{figure} [H]
			\centering
			\includegraphics[page=1,trim={0cm 0cm 0cm 0cm},clip=true,width=70mm]{FigurasTFM/Chapter4/PDF/panel_rs.pdf}
			\caption{5 W solar panel \cite{panelrs}} 	\label{rs}
			\vspace{-0.5cm}
\end{figure}

\autoref{rs1} shows the IV curve of the panel, along with the power delivered by in those conditions.


\begin{figure}[H]
			\centering
			\includegraphics[page=1,trim={0cm 0cm 0cm 0cm},clip=true,width=145mm]{figurastfm/Chapter4/PDF/plot_panel_rs.pdf}
			\caption{Characterization of the RS solar panel} \label{rs1}
			\vspace{-0.5cm}
\end{figure}

The panel behaves exceptionally well, with an output even \textbf{above 5 W} and an \textbf{MPP~=~[17~V,~0.31~A]}. This \textbf{validates the functioning of the testbench}, so in the next pages different small-sized solar panels are also characterized.

The solar panels candidates to be used in the \glsname{cubesat} will be characterized facing both Sun's and \acrshort{LED} lightning. \autoref{montaje} shows the assembly for the measurement using the \acrshort{LED} simulator.

\begin{figure}[H]
			\centering
			\subfloat[]{\includegraphics[page=1,trim={0cm 0cm 0cm 0cm},clip=true,width=80mm]{figurastfm/Chapter4/Imagenes/ledpanel1.jpg}}
			\quad
			\subfloat[]{\includegraphics[page=1,trim={0cm 0cm 0cm 0cm},clip=true,width=80mm]{figurastfm/Chapter4/Imagenes/ledpanel2.jpg}}
			\caption{Assembly to characterize solar panels with \acrshort{LED} lightning} \label{montaje}
			%\vspace{-0.5cm}
\end{figure}

The results are divided into the three solar panels characterized, each one with two plots depending on the illumination. For each panel, it is indicated the theoretical performance according to the manufacturer. \autoref{fotospanel} shows the panels under test.
\begin{landscape}


\vspace*{\fill}
\begin{figure}[H]
			\centering
			\subfloat[1: 5 V @ 1.3 W]{\includegraphics[page=1,trim={0cm 0cm 0cm 0cm},clip=true,width=65mm]{figurastfm/Chapter4/Imagenes/panel1.png}}
			\quad
			\subfloat[2: 12 V @ 1.5 W]{\includegraphics[page=1,trim={0cm 0cm 0cm 0cm},clip=true,width=65mm]{figurastfm/Chapter4/Imagenes/panel2.png}}
			\quad
				\subfloat[3: 5.5 V @ 1 W]{\includegraphics[page=1,trim={0cm 0cm 0cm 0cm},clip=true,width=65mm]{figurastfm/Chapter4/Imagenes/panel3.png}}			
			\caption{Solar panels characterized} \label{fotospanel}
			%\vspace{-0.5cm}
\end{figure}
\vspace*{\fill}


\end{landscape}


\begin{itemize} [noitemsep,topsep=0pt]
\item \textbf{Solar panel 1: 5 V @ 1.3 W}
\end{itemize}


\begin{figure}[H]
			\centering
			\subfloat[Solar lightning - MPP = (11 V, 0.097 A)]{\includegraphics[page=1,trim={0cm 0cm 0cm 0cm},clip=true,width=135mm]{figurastfm/Chapter4/PDF/panel2_sol.pdf}}
			\\
			\subfloat[\acrshort{LED} lightning - MPP = (11.5 V, 0.0197 A)]{\includegraphics[page=1,trim={0cm 0cm 0cm 0cm},clip=true,width=135mm]{figurastfm/Chapter4/PDF/panel2_led.pdf}}
			\caption{Solar panel 1. Characterization results} %\label{adcspcb}
			\vspace{-2cm}
\end{figure}

\newpage


\begin{itemize} [noitemsep,topsep=0pt]
\item \textbf{Solar panel 2: 12 V @ 1.5 W}
\end{itemize}

\begin{figure}[H]
			\centering
			\subfloat[Solar lightning - MPP = (5 V, 0.165 A)]{\includegraphics[page=1,trim={0cm 0cm 0cm 0cm},clip=true,width=135mm]{figurastfm/Chapter4/PDF/panel3_sol.pdf}}
			\\
			\subfloat[\acrshort{LED} lightning - MPP = (5 V, 0.035 A)]{\includegraphics[page=1,trim={0cm 0cm 0cm 0cm},clip=true,width=135mm]{figurastfm/Chapter4/PDF/panel3_led.pdf}}
			\caption{Solar panel 2. Characterization results} %\label{adcspcb}
			\vspace{-2cm}
\end{figure}



\newpage

\begin{itemize} [noitemsep,topsep=0pt]
\item \textbf{Solar panel 3: 5.5 V @ 1 W}
\end{itemize}

\begin{figure}[H]
			\centering
			\subfloat[Solar lightning - MPP = (5 V, 0.1 A)]{\includegraphics[page=1,trim={0cm 0cm 0cm 0cm},clip=true,width=135mm]{figurastfm/Chapter4/PDF/panel1_sol.pdf}}
			\\
			\subfloat[\acrshort{LED} lightning - MPP = (5.3 V, 0.024 A)]{\includegraphics[page=1,trim={0cm 0cm 0cm 0cm},clip=true,width=135mm]{figurastfm/Chapter4/PDF/panel1_led.pdf}}
			\caption{Solar panel 3. Characterization results} %\label{adcspcb}
			\vspace{-2cm}
\end{figure}


\newpage

From the previous plots, several parameters can be obtained; they are arranged in \autoref{tablepanel} along with the figures of merit of each panel. Definitions and formulas for the parameters exposed can be found in section \ref{solarcellsexpl}.


\begin{table}[]
\centering
\begin{tabular}{cccccccc}
\hline
\textbf{Panel}              & \textbf{Lightning} & \textbf{$V_{\text{MP}}$ (V)} & \textbf{$I_{\text{MP}}$ (A)} & \textbf{$P_{\text{MP}}$ (W)} & \multicolumn{1}{l}{\textbf{Fill Factor}} & \textbf{Incident power (W)} & \textbf{$\eta$} \\ \hline
\multicolumn{1}{c|}{1} &                    &                              &                              &                              & \multicolumn{1}{l}{}                     &                             &                 \\
\multicolumn{1}{c|}{}  & Sun                & 11                           & 0.097                        & 1.067                        & 0.70                                     & 8.8                         & 12 \%           \\
\multicolumn{1}{c|}{}  & \acrshort{LED}     & 11.5                         & 0.0197                       & 0.23                         & 0.82                                     & 2.07                        & 11 \%           \\ \hline
\multicolumn{1}{c|}{2} &                    &                              &                              &                              &                                          &                             &                 \\
\multicolumn{1}{c|}{}  & Sun                & 5                            & 0.165                        & 0.825                        & 0.75                                     & 9.78                        & 8 \%            \\
\multicolumn{1}{c|}{}  & \acrshort{LED}     & 5                            & 0.035                        & 0.175                        & 0.64                                     & 2.30                        & 8 \%            \\ \hline
\multicolumn{1}{c|}{3} &                    &                              &                              &                              &                                          &                             &                 \\
\multicolumn{1}{c|}{}  & Sun                & 5                            & 0.1                          & 0.5                          & 0.81                                     & 9.03                        & 6 \%            \\
\multicolumn{1}{c|}{}  & \acrshort{LED}     & 5.3                          & 0.024                        & 0.13                         & 0.80                                     & 2.12                        & 6 \%            \\ \hline
\end{tabular}
\caption{Results of the solar panels characterized}
\label{tablepanel}
\end{table}

Additionally, several conclusions can be extracted from the previous characterization, summarized next.



\begin{itemize} [topsep=0pt]
	
	\item In order to get realistic results, the incident power has been calculated for each panel, i.e., the power received by the solar panel is not the one irradiated by the source (\SI{1000}{W/m^2} for the Sun or \SI{235}{W/m^2} for the \acrshort{LED}), but the \textbf{proportional part} considering the panel's size. Indeed, the \textbf{packing density} also has to be taken into account, with an estimated value of \textbf{0.8}.
	\item Although, at first sight, the panels seem to exhibit a significantly worse performance when the \acrshort{LED} simulator is used, actually \textbf{the efficiency is almost the same}. Particularly, panels 2 and 3 features the same \textbf{power conversion efficiency} regardless of the lightning. 
	\item On the other hand, the power delivered is about a fifth when the \acrshort{LED} simulator is lightning, but that is logical, considering that the incident power is also about \textbf{20~\%} of the one provided by the Sun.
	\item The $V_{\text{MP}}$ is also similar regardless of the irradiance source used, which means that, for an equivalent load, \textbf{the spectrum of both power sources is reasonably similar too}. The decrease in power is then a consequence of the decrease in the \textbf{output current}, $I_{\text{MP}}$, directly related to the incident power. 
	\item In the same way, the \textbf{Fill Factor} slightly differs with the kind of lightning, indeed, the panel 3 features almost an identical factor. Furthermore, \textbf{in all cases the factor is pretty acceptable}, with a minimum value of 0.64 when the panel 2 is lighted with the \acrshort{LED} simulator. With a mean value of about 0.75, this implies that about that percentage of the power in shortcircuit is provided as \textbf{real power}. 
	\item Regarding the \textbf{power conversion efficency} or $\eta$ is really poor in all cases; this issue was expectable considering that these are low-cost solar panels. However, while the efficiency exhibited by panels 2 and 3 is even lower than typical values, the efficiency associated to the panel 1 (12 \%) \textbf{can be deemed acceptable.}
	
	\item In sum, this testbench has allowed, on the one hand: characterizing the behavior of different solar panels in terms of \textbf{power and figures of merit}; on the other hand, \textbf{validating} the \acrshort{LED} simulator, as the results are coherent with the ones obtained using solar lightning, considering the \textbf{difference in incident power}.
	
\end{itemize}

The final choice will depend on the overall design of the \acrshort{EPS} as well as on the battery used.

\subsubsection{Batteries characterization} \label{battscharac}

As for the energy storage subsystem, because this Master's Thesis has addressed the power consumption of some elements, such as the \textbf{Central Processing Unit} (see section \ref{rpipower}), some Li-ion batteries are characterized, so it can be used as a reference for the future \acrshort{EPS}. The definitions of the most relevant \textbf{figures of merit} related to batteries, mentioned next, can be found in section \ref{energystr}.

Particularly, several batteries with different \textbf{capacities} (1100 and 1800 mAh) are tested using a wide range of values for the \textbf{dimensional abuse}. This test is performed using the battery charging centre \textbf{VOLCRAFT ALC-8500}, shown in \autoref{volcraft}.


\begin{figure}[H]
			\centering
			\includegraphics[page=1,trim={0cm 0cm 0cm 0cm},clip=true,width=80mm]{figurastfm/Chapter4/Imagenes/alc.jpg}
			\caption{VOLCRAFT ALC-8500} \label{volcraft}
			%\vspace{-0.5cm}
\end{figure}

Once again, the results allow both, characterizing the batteries and making a first approximation to the energy needs of the system and the battery \textbf{run-time} when supplying it.  The results are shown in the following pages.

\includepdf[pages=-,link=true,landscape,linkname=schematics]{FigurasTFM/Chapter4/PDF/completo_1100.PDF}

\includepdf[pages=-,link=true,linkname=tp,landscape,linkname=schematics]{FigurasTFM/Chapter4/PDF/charge_cap_1800.PDF}

Some conclusions can be extracted from the previous results:

\begin{itemize} [topsep=0pt]
	
	\item \textbf{Neither voltage drop nor increase is linear} with discharge or charge respectively. Therefore, neither battery duration can be expected to behave linearly, e.g., with a \textbf{discharge rate} C/10, the run-time of the 1100 mAh battery is about 10 hours, and with C/20, instead of 20 hours, is slightly 19 hours.
	\item As expected, the battery behaves as a \textbf{constant current source}, whose voltage varies. This can be seen clearly in the plot analyzing the current drop when discharging. However, when charging the behavior is different, although current is constant for a while, at the end of the charge, \textbf{it decreases almost exponentially}.
	\item Because current drop is constant, capacity drop is too, shown at the third plot on the left. It is interesting to point out another phenomena: \textbf{the batteries do not completely discharge}. For instance, see the capacity increase in the 1100 mAh battery; it clearly \textbf{surpasses its theoretical capacity}, reaching more than 1200 mAh. However, the capacity drop reaches the 1100 mAh stated by the manufacturer. This is actually a \textbf{protection mechanism} of the charging centre, which prevents the battery voltage from dropping below a certain value (about 3.1 V), as this would permanently \textbf{damage the battery.}
	\item It is worth noting that, when charging the 1800 mAh battery, the capacity slightly surpasses 1500 mAh, but it should reach almost 1900 mAh, according to the behavior seen before. Of course, this supposes that the capacity drop is about 1500 mAh too, instead of 1800 mAh. In this case, the manufacturer has provided a \textbf{fake total capacity.}
	\item Another interesting point is that batteries \textbf{cannot work at every dimensional abuse}; it depends on the specific battery. If a certain battery is forced to work at an inadequate rate, its behavior will be somewhat erratic. An example of this can be seen when the 1800 mAh battery is charged at a C/2 rate. In fact, the current increase with charge is not constant at any point, after reaching the maximum it immediately start falling exponentially. The voltage increase with this rate differs from the others too. Although not strange behavior is noticed when discharged at this rate,\textbf{ it is inadvisable to use the battery with this configuration}, to prevent any damage.
\end{itemize}


In order to get a rough approximation of the battery run-time when supplying the \glsname{cubesat}, it can be applied the \textbf{Pareto principle}: the 80 \% of the whole power consumption is produced by only a 20 \% of the components, in this case, the Raspberry~Pi~3~Compute~Module. Therefore, it is estimated the battery run-time assuming that it supplies only that component, using a discharge \textbf{dimensional abuse} coherent with the results gathered in section \ref{rpipower}. The results are summarized in \autoref{runtime1} and \autoref{runtime2}, for the 1800 mAh battery.

% Please add the following required packages to your document preamble:
% \usepackage[table,xcdraw]{xcolor}
% If you use beamer only pass "xcolor=table" option, i.e. \documentclass[xcolor=table]{beamer}
\begin{table}[H]
\centering
\begin{tabular}{ccccccccccc}
\multicolumn{11}{c}{\textbf{3.3 V supply - 1800 mAh battery run-time estimation - Stationary state (C/10, 5.51 Wh)}}                                                                                                                                                                                                                                                                         \\ \hline
\textbf{Cores}         & \textbf{Mode}                    & \multicolumn{3}{c}{\textbf{Current (A)}}                                                                  & \multicolumn{3}{c}{\textbf{Power (W)}}                                                                    & \multicolumn{3}{c}{\textbf{Time (h)}}                                                                    \\ \hline
\textbf{}              & \textbf{}                        & \textbf{Mean}             & \textbf{RMS}              & \textbf{Peak}                                     & \textbf{Mean}             & \textbf{RMS}              & \textbf{Peak}                                     & \textbf{Mean}             & \textbf{RMS}              & \textbf{Peak}                                    \\ \cline{3-11} 
\multicolumn{1}{c|}{1} & Powersave                        & 0.147                     & 0.147                     & \multicolumn{1}{c|}{{\color[HTML]{FE0000} 0.156}} & 0.484                     & 0.484                     & \multicolumn{1}{c|}{{\color[HTML]{FE0000} 0.514}} & 11.38                     & 11.38                     & {\color[HTML]{FE0000} 10.72}                     \\
\multicolumn{1}{c|}{}  & Ondemand                         & 0.147                     & 0.147                     & \multicolumn{1}{c|}{{\color[HTML]{FE0000} 0.155}} & 0.485                     & 0.485                     & \multicolumn{1}{c|}{{\color[HTML]{FE0000} 0.511}} & 11.36                     & 11.36                     & {\color[HTML]{FE0000} 10.77}                     \\
\multicolumn{1}{c|}{}  & Performance                      & 0.188                     & 0.188                     & \multicolumn{1}{c|}{{\color[HTML]{FE0000} 0.213}} & 0.620                     & 0.620                     & \multicolumn{1}{c|}{{\color[HTML]{FE0000} 0.702}} & 8.88                      & 8.89                      & {\color[HTML]{FE0000} 7.85}                      \\
\multicolumn{1}{l|}{}  & \multicolumn{1}{l}{Conservative} & \multicolumn{1}{l}{0.146} & \multicolumn{1}{l}{0.146} & \multicolumn{1}{l|}{{\color[HTML]{FE0000} 0.155}} & \multicolumn{1}{l}{0.482} & \multicolumn{1}{l}{0.482} & \multicolumn{1}{l|}{{\color[HTML]{FE0000} 0.510}} & \multicolumn{1}{l}{11.43} & \multicolumn{1}{l}{11.43} & \multicolumn{1}{l}{{\color[HTML]{FE0000} 10.80}} \\ \hline
\multicolumn{1}{l}{}   & \multicolumn{1}{l}{}             & \multicolumn{1}{l}{}      & \multicolumn{1}{l}{}      & \multicolumn{1}{l|}{{\color[HTML]{FE0000} }}      & \multicolumn{1}{l}{}      & \multicolumn{1}{l}{}      & \multicolumn{1}{l|}{{\color[HTML]{FE0000} }}      & \multicolumn{1}{l}{}      & \multicolumn{1}{l}{}      & \multicolumn{1}{l}{{\color[HTML]{FE0000} }}      \\
\multicolumn{1}{c|}{4} & Powersave                        & 0.145                     & 0.145                     & \multicolumn{1}{c|}{{\color[HTML]{FE0000} 0.149}} & 0.477                     & 0.477                     & \multicolumn{1}{c|}{{\color[HTML]{FE0000} 0.491}} & 11.54                     & 11.54                     & {\color[HTML]{FE0000} 11.23}                     \\
\multicolumn{1}{c|}{}  & Ondemand                         & 0.148                     & 0.147                     & \multicolumn{1}{c|}{{\color[HTML]{FE0000} 0.168}} & 0.487                     & 0.486                     & \multicolumn{1}{c|}{{\color[HTML]{FE0000} 0.554}} & 11.32                     & 11.34                     & {\color[HTML]{FE0000} 9.94}                      \\
\multicolumn{1}{c|}{}  & Performance                      & 0.186                     & 0.186                     & \multicolumn{1}{c|}{{\color[HTML]{FE0000} 0.197}} & 0.612                     & 0.612                     & \multicolumn{1}{c|}{{\color[HTML]{FE0000} 0.651}} & 9.00                      & 9.00                      & {\color[HTML]{FE0000} 8.46}                      \\
\multicolumn{1}{c|}{}  & Conservative                     & 0.146                     & 0.145                     & \multicolumn{1}{c|}{{\color[HTML]{FE0000} 0.150}} & 0.481                     & 0.479                     & \multicolumn{1}{c|}{{\color[HTML]{FE0000} 0.495}} & 11.46                     & 11.50                     & {\color[HTML]{FE0000} 11.13}                     \\ \hline
\end{tabular}
\caption{1800 mAh battery run-time estimation in stationary state, with a dimensional abuse C/10}
\label{runtime1}
\end{table}
\vspace{-0.8cm}


As seen in the battery analysis performed, the capacity can vary depending on the dimensional abuse used to charge it; in this case, the battery exhibits a stored energy of 5.51 Wh. Recalling the results in section \ref{rpipower}, the \acrshort{CPU} current drain in stationary state can be approximated by C/10 for this battery. With these considerations, the battery-runtime is estimated taking three values: mean, \acrshort{RMS} and Peak. The results are really satisfactory as, for the maximum performance configuration, the battery is estimated to supply enough power for more than 8 hours. \autoref{runtime2} summarizes the results of this test under a high workload.


% Please add the following required packages to your document preamble:
% \usepackage[table,xcdraw]{xcolor}
% If you use beamer only pass "xcolor=table" option, i.e. \documentclass[xcolor=table]{beamer}
\begin{table}[H]
\centering
\begin{tabular}{ccccccccccc}
\multicolumn{11}{c}{\textbf{3.3 V supply - 1800 mAh battery run-time estimation - High load (C/5, 5.25 Wh)}}                                                                                                                                                                                                                                                                              \\ \hline
\textbf{Cores}         & \textbf{Mode}                    & \multicolumn{3}{c}{\textbf{Current (A)}}                                                                  & \multicolumn{3}{c}{\textbf{Power (W)}}                                                                    & \multicolumn{3}{c}{\textbf{Time (h)}}                                                                 \\ \hline
\textbf{}              & \textbf{}                        & \textbf{Mean}             & \textbf{RMS}              & \textbf{Peak}                                     & \textbf{Mean}             & \textbf{RMS}              & \textbf{Peak}                                     & \textbf{Mean}            & \textbf{RMS}             & \textbf{Peak}                                   \\ \cline{3-11} 
\multicolumn{1}{c|}{1} & Powersave                        & 0.294                     & 0.296                     & \multicolumn{1}{c|}{{\color[HTML]{FE0000} 0.370}} & 0.971                     & 0.976                     & \multicolumn{1}{c|}{{\color[HTML]{FE0000} 1.221}} & 5.41                     & 5.38                     & {\color[HTML]{FE0000} 4.30}                     \\
\multicolumn{1}{c|}{}  & Ondemand                         & 0.461                     & 0.459                     & \multicolumn{1}{c|}{{\color[HTML]{FE0000} 0.568}} & 1.521                     & 1.515                     & \multicolumn{1}{c|}{{\color[HTML]{FE0000} 1.874}} & 3.45                     & 3.47                     & {\color[HTML]{FE0000} 2.80}                     \\
\multicolumn{1}{c|}{}  & Performance                      & 0.461                     & 0.465                     & \multicolumn{1}{c|}{{\color[HTML]{FE0000} 0.567}} & 1.523                     & 1.533                     & \multicolumn{1}{c|}{{\color[HTML]{FE0000} 1.872}} & 3.45                     & 3.42                     & {\color[HTML]{FE0000} 2.80}                     \\
\multicolumn{1}{l|}{}  & \multicolumn{1}{l}{Conservative} & \multicolumn{1}{l}{0.456} & \multicolumn{1}{l}{0.459} & \multicolumn{1}{l|}{{\color[HTML]{FE0000} 0.563}} & \multicolumn{1}{l}{1.504} & \multicolumn{1}{l}{1.515} & \multicolumn{1}{l|}{{\color[HTML]{FE0000} 1.858}} & \multicolumn{1}{l}{3.49} & \multicolumn{1}{l}{3.47} & \multicolumn{1}{l}{{\color[HTML]{FE0000} 2.83}} \\ \hline
\multicolumn{1}{l}{}   & \multicolumn{1}{l}{}             & \multicolumn{1}{l}{}      & \multicolumn{1}{l}{}      & \multicolumn{1}{l|}{{\color[HTML]{FE0000} }}      & \multicolumn{1}{l}{}      & \multicolumn{1}{l}{}      & \multicolumn{1}{l|}{{\color[HTML]{FE0000} }}      & \multicolumn{1}{l}{}     & \multicolumn{1}{l}{}     & \multicolumn{1}{l}{{\color[HTML]{FE0000} }}     \\
\multicolumn{1}{c|}{4} & Powersave                        & 0.524                     & 0.524                     & \multicolumn{1}{c|}{{\color[HTML]{FE0000} 0.598}} & 1.729                     & 1.731                     & \multicolumn{1}{c|}{{\color[HTML]{FE0000} 1.974}} & 3.04                     & 3.03                     & {\color[HTML]{FE0000} 2.66}                     \\
\multicolumn{1}{c|}{}  & Ondemand                         & 1.016                     & 0.999                     & \multicolumn{1}{c|}{{\color[HTML]{FE0000} 1.082}} & 3.353                     & 3.298                     & \multicolumn{1}{c|}{{\color[HTML]{FE0000} 3.569}} & 1.57                     & 1.59                     & {\color[HTML]{FE0000} 1.47}                     \\
\multicolumn{1}{c|}{}  & Performance                      & 0.986                     & 0.978                     & \multicolumn{1}{c|}{{\color[HTML]{FE0000} 1.042}} & 3.254                     & 3.229                     & \multicolumn{1}{c|}{{\color[HTML]{FE0000} 3.437}} & 1.61                     & 1.63                     & {\color[HTML]{FE0000} 1.53}                     \\
\multicolumn{1}{c|}{}  & Conservative                     & 0.979                     & 0.981                     & \multicolumn{1}{c|}{{\color[HTML]{FE0000} 1.055}} & 3.231                     & 3.238                     & \multicolumn{1}{c|}{{\color[HTML]{FE0000} 3.483}} & 1.63                     & 1.62                     & {\color[HTML]{FE0000} 1.51}                     \\ \hline
\end{tabular}
\caption{1800 mAh battery run-time estimation under high load stress, with a dimensional abuse C/5}
\label{runtime2}
\end{table}
\vspace{-0.8cm}

When the \acrshort{CPU} is dealing with a high load stress, the battery run-time falls drastically, as expected. However, it reaches an hour and a half, which is a great value taking into account that this workload will not be usual.