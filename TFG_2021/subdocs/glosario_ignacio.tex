\cleardoublepage
\pagestyle{plain}%
\phantomsection \label{Glossary}
\addcontentsline{toc}{chapter}{Glossary}

\chapter*{Glosario}

%
\gloss[nocite]{*}
\printgloss{glsbase,./referencias/glosario_BIB_ignacio}


%
%
%\textbf{\LARGE{ }}
%\begin{description}
%\item[ ] 
%\end{description}
%
%\textbf{\LARGE{A}}
%
%\begin{description}
  %
	%\item[AMR] Magnetorresistencia Anisotrópica. Es un tipo de material cuya resistencia eléctrica depende del ángulo entre la dirección de la corrriente que los atraviesa y la dirección del vector de magnetización del propio material.
%
	%\item[Avellanado] Hundimiento cónico en torno a un agujero o fresado donde cabe la cabeza de un tornillo para que quede enrasada en la superficie y no sobresalga de la misma.
	%
%\end{description}
%
%
%
%\textbf{\LARGE{B}}
%
%\begin{description}
	%\item[BNC] El conector BNC (del inglés Bayonet Neill-Concelman) es un tipo de conector para uso con cable coaxial usado en aplicaciones de RF que precisaban de un conector rápido, apto para UHF y de impedancia constante a lo largo de un amplio espectro. Muy utilizado en instrumentación electrónica (p.ej osciloscópios y generadores) por su versatilidad.
	%
	%\item[Bobina de Helmholtz] Son dos bobinas idénticas situadas distantes de forma simétrica una de la otra respecto del eje que une sus centros, rodeando el área experimental en la que producen un campo magnético homogéneo al alimentarlas con corriente eléctrica. Su nombre hace honor al físico alemán Hermann von Helmholtz. 
%
	%\item[Bobinado] Acción de bobinar, mediante la cual arrollamos el hilo de cobre esmaltado alrededor de la estructura o carcasa de la bobina, formando las espiras por las que circula la corriente.	
	%
%\end{description}
%
%
%\textbf{\LARGE{C}}
%
%\begin{description}
  %
	%\item[CMR] Magnetorresistencia Colosal. Es un tipo de material magnetorresistivo en el que se emplean determinados óxidos mixtos para conseguir estructuras sensibles al campo magnético, con índices magnetorresistivos muy elevados ($103-110\%$).
%
	%\item[Coercividad magnética] es la intensidad del campo magnético que debe aplicarse a un determinado material para reducir su magnetización remanente hasta un valor cero, después de haber sido magnetizada en sentido opuesto.
	%
%\end{description}
%
%
%\textbf{\LARGE{D}}
%
%\begin{description}
  %
	%\item[DIP] DIP o Dual In-Line por sus siglas en inglés, es una forma de encapsulamiento que consiste en un bloque con dos hileras paralelas de pines. La nomenclatura normal para designarlos es DIPn donde n es el número de pines totales del circuito.
%
	%\item[DUT] DUT o Device Under Test por sus siglas en inglés, es el dispositivo al que se le está sometiendo a pruebas a fin de poder caracterizar su comportamiento.
	%
%\end{description}
%
%
%\textbf{\LARGE{E}}
%
%\begin{description}
  %
	%\item[Engraving] Es la práctica de la incisión de un diseño sobre una superficie generalmente dura y homogénea. Se emplean tornos de control numérico y brocas de fresado de distintos materiales en función de la dureza de la superficie de grabado.
%
	%\item[Espira] Es una línea curva cerrada de material conductor a través del cual se hace pasar una corriente eléctrica para inducir un campo magnético por la Ley de Ampère.
	%
%\end{description}
%
%
%
%\textbf{\LARGE{F}}
%
%\begin{description}
  %
	%\item[Ficha de calibración] Es una pieza pequeña, generalmente plana, delgada y con una graduación métrica que se usa para establecer una relación de proporcionalidad entre el tamaño de un elemento y la captura fotográfica digitalizada obtenida por medio de un microscopio.
%
	%\item[Front-End] Es la parte del software que interactúa con el usuario y que es responsable de recolectar los datos de entrada, que pueden ser de muchas y variadas formas.
	%
	%\item[FR4] Es la designación del material compuesto por un tejido de láminas de fibra de vidrio ligadas entre sí mediante resina epoxi. Su utilización en placas de circuito impreso es muy frecuente.
	%
%\end{description}
%
%
%\newpage
%\textbf{\LARGE{G}}
%
%\begin{description}
  %
	%\item[Gaussímetro] Dispositivo electrónico utilizado para la medición de radiación magnética.
%
	%\item[GMR] Magnetorresistencia Gigante. Es un tipo de estructura litográfica laminada sensible al campo magnético, que varía su resistencia eléctrica en función de la alineación de los vectores de magnetización de sus capas ante la presencia de un campo magnético externo.
	%
	%\item[GPIB] del inglés General Purpose Interface Bus. Es un bus de comunicación de datos cuyo protocolo IEEE 488.2 se encuentra estandarizado. Se utiliza frecuentemente para controlar diferentes equipos de instrumentación electrónica de forma remota.
	%
	%\item[Grid] Es una cuadrícula tipográfica formada por la intersección de los ejes vertical y horizontal, utilizada para estructurar un contenido.
	%
	%\item[GUIDE] Siglas de Graphical User Interface Development Environment, o entorno de desarrollo de interfaces gráficas de usuario. Es un entorno de programación de interfaces front-end utilizada por el software de programación científica Matlab.
	%
%\end{description}
%
%
%\textbf{\LARGE{H}}
%
%\begin{description}
  %
	%\item[Histéresis] Tendencia de un material a conservar una de sus propiedades, en ausencia del estímulo que la ha generado.
%
%\end{description}
%
%
%\textbf{\LARGE{L}}
%
%\begin{description}
  %
	%\item[Litofrafía] Arte de dibujar o grabar en una superficie preparada al efecto, para reproducir, mediante impresión, lo dibujado o grabado en otra superficie a través de sustancias hidrofóbicas e hidrofílicas. Es una técnica muy usada durante la fabricación de la estructura laminar de una magnetorresistencia GMR o CMR.
%
%\end{description}
%
%
%\textbf{\LARGE{M}}
%
%\begin{description}
  %
	%\item[Máscara] Plantilla en dos dimensiones realizada mediante software de diseño asistido por computador para mostrar el área de trazado o deposición de materiales sobre una determinada superficie activa. En litografía suele especificarse por colores la correspondencia entre diferentes capas superpuestas.
%
	%\item[Magnetorresistencia] Material cuya estructura morfológica laminada es sensible e interacciona al campo magnético aplicado sobre su volumen haciendo variar su resistencia eléctrica según la incidencia del mismo.
	%
	%\item[Mesa de puntas] Mesa empleada para disponer sobre ella determinados dispositivos electrónicos para realizar test paramétricos. Usualmente, estas mesas incorporan mandos de control manual y automático muy precisos para asistir el posicionamiento de pequeñas agujas sobre los contactos de los dispositivos, pudiendo así excitarlos eléctricamente.
	%
	%\item[Metacrilato] Producto de polimerización del ácido acrílico o de sus derivados. Es un sólido transparente, rígido y resistente a los agentes atmosféricos, y uno de los materiales plásticos más utilizados.
	%
	%\item[Microposicionador] Dispositivo mecánico con un brazo regulable en las tres coordenadas cartesianas, que se acopla sobre una mesa de puntas para añadir un mayor grado de libertad en el posicionamiento sobre el dispositivo electrónico final. Estos dispositivos están concebidos para ser regulados manualmente, por lo que tienen una relación de micras de desplazamiento por revolución de su rosca manual, para cada coordenada.
	%
	%\item[Mils] Unidad de longitud referente a la milésima parte de una pulgada.
	%
%\end{description}
%
%
%
%\textbf{\LARGE{P}}
%
%\begin{description}
  %
	%\item[Pad] Superficie de contacto con un material conductor o semiconductor dispuesto sobre una oblea de silicio. El tamaño de dicha superficie suele ser por lo general del orden de $\mathrm{mm^2}$ o inferior, según el tipo de dispositivos, por lo que será necesario una mesa de puntas y microposicionadores para poder situarse sobre él.
%
	%\item[Palometa] Tuerca con dos extensiones laterales en que se apoyan los dedos para darle vueltas mientras se rosca sobre un tornillo.
	%
	%\item[Permalloy] Es una marca comercial que designa a una aleación magnética compuesta por níquel y hierro. Fue descubierta en 1914 por Gustav Elmen de los Laboratorios Bell.
	%
	%\item[PCB] PCB o circuito impreso (Printed Circuit Board), es un medio para sostener mecánicamente y conectar eléctricamente componentes electrónicos, a través de rutas o pistas de material conductor, grabados en hojas de cobre laminadas sobre un sustrato no conductor, comúnmente baquelita o fibra de vidrio.
	%
	%\item[PVC] El policloruro de vinilo o PVC es un polímero termoplástico que tiene muy buena resistencia eléctrica así como a la llama.
	%
	%\item[Mils] Unidad de longitud referente a la milésima parte de una pulgada.
	%
%\end{description}
%
%
%\textbf{\LARGE{S}}
%
%\begin{description}
  %
	%\item[Sala blanca] Cuarto limpio o sala limpia (en inglés, clean room). Es una sala especialmente diseñada para obtener bajos niveles de contaminación así como ejercer un estricto control de los parámetros ambientales como temperatura, humedad, flujo de aire, presión, iluminación, etc. Se utilizan para desarrollar y fabricar materiales quirúrgico plásticos, aleaciones equiatómicas y semiconductores. 
%
	%\item[SCPI] La norma SCPI (del ingés Standard Commands for Programmable Instruments), es una norma de estandarización de los comandos de control y el formato de los datos de los instrumentos controlados mediante conexión GPIB bajo el protocolo IEEE 488.2.
	%
	%\item[SMU] SMU o Source/Monitor Unit es una fuente de tensión y/o corriente eléctrica capaz al mismo tiempo de ejercer como amperímetro o voltímetro respectivamente. Es comúnmente integrada en dispositivos analizadores de parámetros para caracterizar dispositivos semiconductores en fase de pruebas o fabricación.
	%
%\end{description}
%
%
%\textbf{\LARGE{T}}
%
%\begin{description}
  %
	%\item[Teflón] El Teflón (PTFE) es un polímero similar al polietileno, en el que los átomos de hidrógeno han sido sustituidos por átomos de flúor. 
%
	%\item[Termoretráctil] Material de plástico destinado a guarnecer grilletes, enganches rápidos y demás útiles susceptibles de soportar golpes o sufrir enredos. Se aplica mediante una fuente de calor.
	%
	%\item[Triaxial] Es un tipo de línea de transmisión similar al cable coaxial pero que añade una capa extra de aislante y un segundo conductor interno. Proporciona mayor ancho de banda y mayor rechazo a la interferencia que el cable coaxial simple.
	%
%\end{description}
%
%
%\textbf{\LARGE{V}}
%
%\begin{description}
  %
	%\item[Varilla roscada] Pieza cilíndrica, metálica y con resalte en hélice roscado sobre su cara lateral, utilizada para crear cohesión entre piezas mediante sujeciones roscadas a la misma.
%
	%\item[Vector de magnetización] Se denomina domo el Momento Dipolar Magnético por unidad de volumen de un material.
	%
%\end{description}