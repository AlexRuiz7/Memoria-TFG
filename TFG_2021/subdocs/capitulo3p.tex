\chapter{Space System Engineering}\label{chap:chapter3}

This third chapter tackles another phase related to analysis and requirements definition. It resembles the more general 'System Analysis` and will cover the whole Platform, as well as the most common \textbf{testing procedures} demanded in space according to \acrshort{ECSS}. The Platform will be analyzed following the previously defined structure, divided into Mechanical Simulation Platform, Ground Station and Simulation \glsname{cubesat}.

After deeply studying every subsystem and technologies associated to each, \textbf{formal~requirements} are defined (in contrast with the objectives and preliminary requirements defined in \autoref{chap:chapter1} and \autoref{chap:chapter2}) according to the Design Process followed in this project, described in \autoref{engprocess}. At that point, the project may go through the \textbf{System Design} process, detailed in \autoref{chap:chapter4}.

%%%%%%%%%%%%%%%%%%%%%%%%%%%%%%%%%%%%%%%%%

%%\section{Introduction to System Engineering}  \label{sec:syseng}
%
%Systems engineering is an interdisciplinary field of engineering and engineering management that focuses on how to design and manage complex systems over their life cycles \cite{wiki}.
%
%A system is an integrated set of elements to accomplish a defined objective. These elements include  hardware, software, firmware, human resources, information, techniques, facilities services, and other support elements.
%
%OJO AJUSTAR Within space sector, as seen in \autoref{missioneng}, the highest level is often called \textit{mission level}, which usually consists of one or more segments (space, ground, user...). From this perspective, requirements originate from the next upper level (the customer) and elements are procured from the next lower level (the suppliers). 
%
%Figure \autoref{fig:sysengfig} shows, according to ECSS-E-ST-10C \cite{ESASYSENG}, the boundaries of system engineering, its relationship with production, operations, product assurance and management disciplines and its internal partition into the following system engineering sub-functions:
%
%\begin{itemize}
%
	%\item{Requirement engineering, which consist on requirement analysis and validation, requirement allocation, and requirement maintenance;}
	%\item{Analysis, which is performed for the purpose of resolving requirements conflicts, decomposing and allocating requirements;}
	%\item{Design and configuration which results in a physical architecture, and its complete system functional, physical and software characteristics;}
	%\item{Verification, which objective is to demonstrate that the deliverables conform to the specified requirements;}
	%\item{System engineering integration and control, coordinating the various engineering disciplines and participants;}
%
	%\end{itemize}
	%
	%
			%\begin{figure}[h]
			%\centering
			%\includegraphics[page=1,trim={0cm 0cm 0cm 0cm},clip=true,width=150mm]{figurastfm/Chapter2/PDF/syseng.pdf}
			%\caption{System engineering, sub-functions and boundaries \cite{ESASYSENG}}
			%\label{fig:sysengfig}
			%\end{figure}
%
	%The system engineering process is in turn applied by each system engineering function of each supplier of the elements of the product decomposition.This process consists of activities to be performed by the system engineering function within each project phase according to the designated lifecycle model. The objective is to obtain a product which satisfies the customer technical requirements within pre-established objectives of cost time and quality. Through this process the system engineering function performs a multidisciplinary functional decomposition to obtain logical lower level products.	
	%
	%The functional decomposition defines, for each level of the system, the technical requirements for the procurement of subassemblies or lower level products as well as the requirements for the verification of the final characteristics of each product. The system engineering process uses the results of these lower level verification activities to build a bottom-up multi-layered evidence that the customer requirements have been met.
%
	%This project will follow a methodology as close as possible to this system engineering perspective, according to \cite{ecss} standards.

%%%%%%%%%%%%%%%%%%%%%%%%%%%%%%%%%%%%%%%%5
\section{Testing procedures} \label{testing}

CubeSats intended to be in orbit must pass a series of comprehensive testing programs in order to get the qualification to fly. Through this analysis, the different tests performed are introduced, so the constraints can be taken into account in \autoref{chap:chapter4}, and especially in future designs. In a real mission project, in which a qualification to fly is needed, it is vital to know which are the technical goals the system shall comply with as soon as possible.

This project does not intend to perform a complete testing-oriented design, because its double perspective does not allow to set space-related requirements specifically, needed to pass the demanding testing stage. Instead, the most relevant tests are introduced; they can be considered when setting the requirements and the design can stick to them as much as possible, enabling to design in complete accordance with them in the future; the aforementioned work will suppose less requirements change and their associated costs when that re-engineering process is performed.

Testing environment and requirements are given by standard ECSS-E-ST-10-03C \cite{ESAtest}. Particularly, this standard splits the process into two levels of decomposition: requirements for \textbf{space segment equipment test} and \textbf{space segment element test}. While \textit{space segment} refers to a part of a space system intended to fulfil the goals of the mission, \textit{space segment elements} are elements within a space segment.

\subsection{Space segment equipment test requirements} \label{spacesegment}

\subsubsection{General tests}
\paragraph{Humidity test}

In order to prevent the space segment of the system from the humidity effects (see more information on this in ECSS‐Q‐ST‐70‐01 \cite{ESAhumid}). This test shall be performed if space segment equipment can be exposed to humidity level above 65 \% during its life time. To qualify humidity test, the space segment equipment shall be installed in a chamber with temperature at room ambient conditions. Temperature shall be increased to +35\textdegree C and humidity to at least 95 \% over an hour. This process must be repeated during 4 different cycles according to the mentioned standard. Finally, the equipment must be visually inspected for deterioration or damage. An example of humidity chamber is shown in \autoref{tempch}.

			\begin{figure} [H] 				
				\centering
				\includegraphics[page=1,trim={0cm 0cm 0cm 0cm},clip=true,width=85mm]{figurastfm/Chapter3/Imagenes/temp_chamber.png}
				\caption{Temperature and humidity chamber \cite{tempchamber}} \label{tempch}
			\end{figure}


\paragraph{Life test}

The life test for space segment equipment qualification shall be designed to demonstrate the ability of the space segment equipment to withstand the maximum operating time and the maximum number of predicted operational cycles during the “product lifetime” by providing the required performance at the end of life. Obviously, the test must be performed under the environmental conditions expected during actual operation, including ambient, thermal, vacuum or a combination of these. 

\subsubsection{Mechanical and structural integrity tests}
\paragraph{Physical properties measurements}

Mechanical integrity is one of the most crucial aspects in every spatial system. Therefore, according to ECSS‐Q‐ST‐70‐01 \cite{ESAhumid}, there must be determined the following physical properties:

\begin{itemize} [noitemsep,topsep=0pt]
	
	\item{Dimensions and interfaces} \\
	\item{Mass} \\
	\item{Centre of gravity with respect to a given coordinate system for three mutually perpendicular axes} \\
	\item{Momentum of inertia with respect to the given coordinate system}

\end{itemize}
%\vspace{-\topsep}

Because of the intermediate character of the simulation platform designed in this project, between real simulation and training platform, this testing will be applied not only to the CubeSat itself, but to the whole physical simulation platform, as widely analyzed in \autoref{chap:chapter4}.

\paragraph{Acceleration test}

In order to ensure uniform force distribution on the space segment equipment, it is centrifuged with an arm whose length shall be at least five times the dimension of the space segment equipment measured along the arm. 

%meter imángenes como https://www.google.es/url?sa=i&source=images&cd=&cad=rja&uact=8&ved=2ahUKEwj7q4iLpI7iAhWNlhQKHe3BAF0Qjxx6BAgBEAI&url=http%3A%2F%2Fshiken.jaxa.jp%2Fen%2Ffacility13_e.html&psig=AOvVaw3h76l0b7_1W2AS1ktcZjk7&ust=1557485469670442

\paragraph{Random and sinusoidal vibration test}

Random and sinusoidal tests shall be both conducted in the launch configurations for all axes. In order to evaluate the space segment equipment integrity, a resonance search shall be performed before and after both tests. The equipment shall not suffer a shift in frequency above 5 \% and 10 \% for amplitude shift, respectively. \autoref{vibration} shows a typical vibration test equipment from \acrshort{ESA}.

			\begin{figure} [H] 				
				\centering
				\includegraphics[page=1,trim={0cm 0cm 0cm 0cm},clip=true,width=100mm]{figurastfm/Chapter3/Imagenes/vibration_test.jpg}
				\caption{Vibration test \cite{ESA}} \label{vibration}
			\end{figure}
			
\paragraph{Shock test}

The shock tests demonstrate the ability of the space segment equipment to withstand the shocks encountered during its lifetime, e.g.: fairing separation or solar arrays and antennas deployment. The equipment shall be powered during the test and the selected shock test method shall achieve the specified \textbf{ Shock Response Spectrum }with a representative transient, comparable in shape and duration to the expected in‐flight shock.

Once again, visual inspection and hardware integrity review shall be performed after test. \autoref{shock} shows a standard shock test equipment.

	\begin{figure} [H] 				
				\centering
				\includegraphics[page=1,trim={0cm 0cm 0cm 0cm},clip=true,width=100mm]{figurastfm/Chapter3/Imagenes/SHOCK_TEST.jpg}
				\caption{Shock test equipment \cite{isispace}} \label{shock}
			\end{figure}

\paragraph{Leak test}

The leak test shall demonstrate the ability of sealed or pressurized space segment equipment to conform to the leak rates stated in the specifications. They are performed only on sealed or pressurized space segment equipment, sensitive to loss of pressure or vacuum. Additionally, leak tests shall be performed prior to and following the completion of space segment equipment thermal and mechanical tests. \autoref{leaktest} shows a leaking test performed to a CubeSat at \acrshort{ESA}.

	\begin{figure} [H] 				
				\centering
				\includegraphics[page=1,trim={0cm 0cm 0cm 0cm},clip=true,width=100mm]{figurastfm/Chapter3/Imagenes/vcam.jpg}
				\caption{Leak test \cite{ESA}} \label{leaktest}
			\end{figure}

\subsubsection{Thermal-Vacuum tests}

While ECSS-E-ST-10-03C \cite{ESAtest} is the general standard for testing, ECSS-E-HB-31-03A \cite{ESAthermal1} and ECSS-Q-ST-70-04C \cite{ESAthermal2} particularly address Thermal-Vacuum testing and analysis; this kind of test subjects the system to a series of temperature cycles in vacuum conditions as present in space. The latter of them include among the deleterious effects to be anticipated during the thermal cycling test under vacuum:  

\begin{itemize}

	\item{\glsname{outgassing}}
	\item{Cracking, contraction or fracture of materials or assemblies due to sudden dimensional changes by expansion.}
	\item{Overheating of materials or assemblies due to change in convection and conductive heat transfer characteristics.}


\end{itemize}

This standard also specifies conditions for work area, such as the need to be nominally clean with minimum dust, filtered ventilation or ambient conditions relative to temperature (25 \textdegree	C $\pm$ 3) or humidity (25 \%	$\pm$ 10). A typical thermal cycling test has the following steps:

\begin{itemize} [noitemsep,topsep=0pt]
	

	\item{The initial characteristics are analyzed at (22 \textdegree C	$\pm$ 3) or humidity (55 \%	$\pm$ 10) within six hours prior to the beginning of the thermal cycling.} \\
	\item{Thermal cycling begins after a working vacuum of \SI{1e-5}{\pascal} to $\pm$ 5 \% has been reached.} \\
	\item{The chamber is thermally cycled between temperatures of ($\pm$100 \textdegree C $\pm$ 5) unless otherwise specified, at a nominal heating or cooling rate of +(10~\textdegree C~$\pm$~2) per minute.} \\
	\item{A minimum of 100 thermal cycles shall be performed on each sample.} \\
	\item{Within six hours after the completion of thermal cycling in vacuum, final inspection and testing of the sample shall be conducted at the initial ambient conditions.} \\

\end{itemize}

This test is commonly performed in a so-called \textbf{Thermal Vacuum Chamber} or \acrshort{TVAC}, as shown in \autoref{tvac}. 

\vspace{1cm}
	\begin{figure} [H] 				
				\centering
				\includegraphics[page=1,trim={0cm 0cm 0cm 0cm},clip=true,width=100mm]{figurastfm/Chapter3/Imagenes/tvac.jpg}
				\caption{Thermal Vacuum Chamber \cite{isispace}} \label{tvac}
			\end{figure}

\newpage
\glsname{GranaSAT} laboratory counts with a self-made \acrshort{TVAC} with a full \textit{in-house} design; details are available at \cite{juanma}. \autoref{tvacjuanma} shows some photos of the chamber. In the future, the testing procedures described above will be performed using that equipment, allowing to determine the accuracy of a certain system to the standard, before going through the official tests of the \textit{Fly Your Satellite!} \cite{flyyour}.

			\begin{figure}[H]
			\centering
			\subfloat[Front]{\includegraphics[page=1,trim={0cm 0cm 0cm 0cm},clip=true,width=100mm]{figurastfm/Chapter3/Imagenes/tvac2.png}}
			 \quad
			\subfloat[Side]{\includegraphics[page=1,trim={0cm 0cm 0cm 0cm},clip=true,width=100mm]{figurastfm/Chapter3/Imagenes/tvac3.png}}
			\caption{Thermal Vacuum Chamber at \glsname{GranaSAT} laboratory} \label{tvacjuanma}
\end{figure}


\newpage
\subsubsection{Electrical/RF Tests}

The following Electrical/RF Tests are crucial for an adequate functioning of the whole system and are specified in ECSS-E-ST-20-07C Rev. 1 \cite{ESAmagnetic}.


\paragraph{EMC}

Electromagnetic compatibility (\acrshort{EMC}) of a space system or equipment is the ability to function satisfactorily in its electromagnetic environment without introducing intolerable electromagnetic disturbances to anything in that environment. For acceptance stage, the space segment equipment shall be subjected to the
following tests:

\begin{itemize} [noitemsep,topsep=0pt]
	

	\item{\textbf{Bonding verification:} also called ground continuity test, consists in testing whether the ground points of a device are well connected in between each other.} \\
	\item{\textbf{Power lines isolation.}} \\
	\item \textbf{Inrush current:} consisting in the surge or momentary burst of current that flows into the device when powered on, due to the high initial currents required to charge the capacitors and inductors or transformers. These currents can be as high as 20 times the steady state currents. Even though it only lasts for about 10 ms it takes between 30 and 40 cycles for the current to stabilize to its steady state. It must be limited in order to prevent the equipment from any damage. \autoref{inrush} illustrates this phenomena. \\
	
	\begin{figure} [H] 				
				\centering
				\includegraphics[page=1,trim={0cm 0cm 0cm 0cm},clip=true,width=100mm]{figurastfm/Chapter3/Imagenes/inrush_current.jpg}
				\caption{Inrush current phenomena \cite{sunpower}} \label{inrush}
				\vspace{-0.3cm}
	\end{figure}	
	
	\item Conducted emission time \& frequency domain on power lines in the operating mode.
	
\end{itemize}
\vspace{-1cm}


\paragraph{Magnetic test} \label{magnetotest}

This test is performed in order to obtain an estimation of the \glsname{mmoment} of the device under test. A common constraint is measuring the magnetic field at distances typically more than three times the size of the device, as well as assigning to its geometric center a right‐handed orthogonal coordinate system XYZ. In order to perform this test, the device should be placed in an \textbf{earth field compensated area} providing zero‐field conditions for the intrinsic moment determination i.e., a set-up which is able to \textit{cancel} Earth's magnetic field, so it does not alter the measurement. This is usually ensured by using a \textbf{Helmholtz Cage}. \glsname{GranaSAT} Laboratory counts with one of these systems developed by one of its members throughout the years, see \cite{ramsel}. Figure\autoref{cage1} shows the developed Helmholtz Cage while Figure\autoref{cagedetail} shows the detail of the wires along the cage.

			\begin{figure}[H]
			\centering
			\subfloat[Developed Helmholtz Cage\label{cage1}]{\includegraphics[page=1,trim={0cm 0cm 0cm 0cm},clip=true,width=70mm]{figurastfm/Chapter3/Imagenes/cage_side.jpg}}
			 \quad
			\subfloat[Wires detail\label{cagedetail}]{\includegraphics[page=1,trim={0cm 0cm 0cm 0cm},clip=true,width=60mm]{figurastfm/Chapter3/Imagenes/cage_detail.jpg}}
			\caption{GranaSAT Helmholtz Cage}
			\vspace{-0.5cm}
\end{figure}

Once the setup is ready, the following test sequence must be followed:

\begin{itemize}  [noitemsep,topsep=0pt]
	
	\item{\acrshort{DUT} not operating, initial measurements on the six semi‐axes at the reference distances.} \\
	 \item Deperm or demagnetization, in order to get an earth field compensated area in the Helmhotz Cage, at a frequency of 3 Hz, maximum amplitude between \SI{4000}{ \micro\tesla} and \SI{5000}{ \micro\tesla}, successively on each XYZ axis of the \acrshort{DUT}. \\
	\item{It is applied a perm field of \SI{300}{ \micro\tesla} on each axis and measurements are repeated.}

\end{itemize}

Once again, this procedure can be tested using our own equipment, shown before. \autoref{deperm} shows graphically how this procedure shall be performed. Once the device is inside a demagnetizated area, measurements are taken at reference points.

	\begin{figure}[H]
			\centering
			\includegraphics[page=1,trim={0cm 0cm 0cm 0cm},clip=true,width=160mm]{figurastfm/Chapter3/PDF/deperm.pdf}
			\caption{Smooth deperm procedure \cite{ESAmagnetic}}
			\label{deperm}
		\end{figure}




\subsection{Space segment element test requirements}

Sometimes, a satellite is used as service module on one end and as payload module test on the other. In these cases when it is not feasible to test a space segment element as a single entity, it may be tested individually. On the other hand, sometimes testing as \textit{segment} is not possible just due to the size of the \acrshort{DUT}, which exceeds the capacities of the testing facility needed. When some of the described situations apply, instead of performing a \textbf{segment equipment testing}, an \textbf{element equipment testing} is chosen. As far as the scope of this project concerns, descriptions of \autoref{spacesegment} apply. %However, as a reference again, some interesting tables regarding this procedures are included in AÑADIR REF AL ANEXO QUE SEA, YA VEREMOS.

To finish this section, allowable tolerances and test accuracies allowed by this standard are included for the most representative tests, \autoref{tabletolerances} and \autoref{tableaccuracies}.


	\begin{figure}[H]
			\centering
			\includegraphics[page=1,trim={0cm 0cm 0cm 0cm},clip=true,width=135mm]{figurastfm/Chapter3/PDF/tabletolerances.pdf}
			\caption{Allowable tolerances \cite{ESAtest}}
			\label{tabletolerances}
		\end{figure}


\begin{figure}[H]
			\centering
			\includegraphics[page=1,trim={0cm 0cm 0cm 0cm},clip=true,width=135mm]{figurastfm/Chapter3/PDF/tableaccuracies.pdf}
			\caption{Tests accuracies \cite{ESAtest}} 			\label{tableaccuracies}
			\vspace{-1cm}
		\end{figure}

This finalizes the analysis of the most relevant testing procedures needed to qualify a design to fly, according to \acrshort{ECSS}. It must be used as basis not only for this project but especially for future designs, which shall be performed sticking to them as much as possible since the early stages of the design, taking advantage of the previous work developed here.

At this point, where the main challenges to overcome for the design have been outlined, the main subsystems that integrate this project are analyzed next, following with the formal requirements definition.

%\section{Inertial 2D Orbit Simulator (I2DOS)} este nombre en diseño it has been called \textbf{Inertial~2D~Orbit~Simulator} or \acrshort{I2DOS}

\section{Mechanical Platform Analysis} \label{mechplat}

As seen in \autoref{testing} CubeSats can be tested in a large variety of situations and circumstances. Since their outbreak, different kinds of non-professional platforms have emerged, which allow enthusiasts and university developers to test their designs before official assessment,  as seen in previous section. As commented before, this Master's Thesis proposes a planar platform to emulate in orbit dynamics in a 2D plane; the formal requirements will be treated along this section. Previously, some of the most featured kinds of platforms commented are reviewed, by way of \textit{Prior Art} in this particular subsystem.

\subsection{Air bearing}

Air bearing platforms have been used for satellite \acrshort{ADCS} testing for decades. The primary objective of air bearing tests is the faithful representation of spacecraft dynamics, so they try to get a minimal-torque environment, as in space, which is particularly difficult to duplicate.  This frictionless nature partially simulates a zero-g environment, allowing the pitch, roll, and yaw  (see section \ref{sec:eulerangles}) control systems of the satellite to work as they would in space.


It allows manipulating the system under more realistic conditions. It is one of the most popular technologies for this goal. \autoref{testbedtfm} illustrates a 3D model example of an air bearing testbed.

\begin{figure}[H]
			\centering
			\includegraphics[page=1,trim={0cm 0cm 0cm 0cm},clip=true,width=75mm]{figurastfm/Chapter3/PDF/testbed.pdf}
			\caption{Spherical air bearing testbed 3D model \cite{tfmtestbed}}
			\label{testbedtfm}
			\vspace{-1cm}
		\end{figure}
		
		Additionally, air bearings are intended to enable \acrshort{DUT} to experience some level of rotational and translational freedom. Pressurized air passes through small holes in the grounded section of the bearing and establishes a thin film that supports the weight of the moving section. These planar air-bearing systems provide one rotational and two translational degrees of freedom \cite{airbearingreview}.
		
		One of the most common types of air bearing is the \textbf{spherical}, as the one depicted in \autoref{testbedtfm}. As shown, the two sections of the bearing are portions of concentric spheres, one of which rotates on an air film bounded by the other section in three degrees of freedom. Another \glsname{GranaSAT} member developed one of these systems, see in detail in \cite{testbedgranasat}. \autoref{tbedgs} shows that implementation.

\begin{figure}[h]
			\centering
			\subfloat[3D model]{\includegraphics[page=1,trim={0cm 0cm 0cm 0cm},clip=true,width=145mm]{figurastfm/Chapter3/PDF/testbed3d.pdf}} \\ 
			\subfloat[Front]{\includegraphics[page=1,trim={0cm 0cm 0cm 0cm},clip=true,width=89mm]{figurastfm/Chapter3/PDF/testbedburgos1.pdf}}
			 \quad
			\subfloat[With \glsname{cubesat} structure]{\includegraphics[page=1,trim={0cm 0cm 0cm 0cm},clip=true,width=70.5mm]{figurastfm/Chapter3/PDF/testbedburgos2.pdf}}
			\caption{Air bearing testbed developed at \glsname{GranaSAT} \cite{testbedgranasat}} \label{tbedgs}
\end{figure}


\subsection{Inertial platforms}

Another technology to simulate orbital dynamics is an inertial platform. It must allow \glsname{cubesat} to rotate in three axes. Just as air bearings could be spherical or other type, inertia platforms design can also vary. One of the most extended is the so-called gyroscopic mechanical platform, because of its simplicity. Figure\autoref{gyroscopicplat} depicts a 3D model example. The platform is designed and constructed to be adjusted to the \glsname{cubesat}size. 


			\begin{figure}[H]
			\centering
			\subfloat[Gyroscopic mechanical inertia platform \cite{inertiaplat} \label{gyroscopicplat}]{\includegraphics[page=1,trim={0cm 0cm 0cm 0cm},clip=true,width=80mm]{figurastfm/Chapter3/PDF/inertiaplat.pdf}}
			 \quad
			\subfloat[Rotational inertia platform \cite{rotinet} \label{rotinertia}]{\includegraphics[page=1,trim={0cm 0cm 0cm 0cm},clip=true,width=80mm]{figurastfm/Chapter3/PDF/rotinertia.pdf}}
			\caption{Inertial platforms examples} 
			%\vspace{-2cm}
\end{figure}


%\begin{figure}[H]
			%\centering
			%\includegraphics[page=1,trim={0cm 0cm 0cm 0cm},clip=true,width=90mm]{figurastfm/Chapter3/PDF/inertiaplat.pdf}
			%\caption{}
			%\label{gyroscopicplat}
		%\end{figure}
		
		

Another typical inertia platform is the \textbf{rotational} one. During testing, the \glsname{cubesat} is aligned with the centre of the rotation axis of the board. Figure\autoref{rotinertia} shows another 3D model example.

%\begin{figure}[H]
			%\centering
			%\includegraphics[page=1,trim={0cm 0cm 0cm 0cm},clip=true,width=90mm]{figurastfm/Chapter3/PDF/rotinertia.pdf}
			%\caption{Rotational inertia platform \cite{rotinet}} \label{rotinertia}
		%\end{figure}


\subsection{Formal Requirements Definition}

As described when analyzing \acrshort{EDP}, after an iterative refinement of the preliminary and technical requirements stated in previous chapters, the formal requirements can be defined. These are the ones which will allow the system to comply with the needs of the mission and therefore, satisfy the objectives of all the \glsname{stakeholders}. This section is included for each subsystem. Once again, they are shown in tabular format, see \autoref{frmech}.

\begin{table}  [H]
\centering

\begin{tabularx}{\linewidth}{lX}

\multicolumn{1}{c}{\textbf{Ref.}}                      & \multicolumn{1}{c}{\textbf{Formal Requirements}}                    \tabularnewline \specialrule{1.1pt}{1pt}{1pt}
MP.FoR.1                                              & The platform shall be 10x10x10 cm dimensions and have adequate attachment capabilities, in order to secure the \glsname{cubesat}. \tabularnewline \midrule
MP.FoR.2                                              & The platform shall allow a consistent and balanced rotation movement, initiated by hand.     \tabularnewline \midrule
MP.FoR.3                                            & The platform shall weight less than 3~kg without the \glsname{cubesat} unit. \tabularnewline \midrule
MP.FoR.4                                                   & The platform shall take up a surface inferior to \SI{0.4}{m^2}.  \tabularnewline \midrule
MP.FoR.5                                                   & The platform shall have three pair of holes separated by 1.50 cm, in order to incorporate external elements. The nearest to the center shall be at least 20~cm far.    \tabularnewline \midrule
MP.FoR.6                                                   & The rotating part of the platform shall exhibit a diagonal inertia matrix.   \tabularnewline \midrule

\end{tabularx}
\caption{Mechanical Platform - Formal Requirements}
\label{frmech}

\end{table}

\newpage
\section{Ground Station Analysis}

Every mission must count with a \glsname{ground}. As described in section \ref{gstation}, in a real mission, it is in charge of communicating with the spacecraft, assuring the correct functioning of the system and so forth. As for this project, its tasks will be similar, but in the local environment designed for the simulation platform. Besides, the system must be expandable enough so it can be used in a real mission.

The concept \glsname{ground} is so wide that this project cannot go into detail as it comprises a huge amount of equipment, software, procedures, communications system and so on. Instead, formal requirements will be defined based on the expected objectives and technical requirements stated. This is one of the examples in which this Master's~Thesis must focus on its academic orientation; nevertheless, as mentioned, the designed \glsname{ground} shall suppose the basis for future improvements which allow to comply with the requirements for a real mission scenario. Additionally, a couple of relevant standards are briefly introduced next. %\autoref{forgstation} contains the formal requirements needed.

\subsection{Data-management standards in Ground Station}

The packetization format of the information exchanged between the spacecraft and the \glsname{cubesat} is also standardized. Indeed, not only the packet format is based on a standard but also the mechanism to exchange the definition of the control commands and telemetry itself. Both are addressed next.

\subsubsection{Consultative Committee for Space Data Systems} \label{ccsds}

This is a governmental organism founded in 1982 to develop standards for space data and foster interoperability. It is composed of the most important space agencies in the world, \acrshort{NASA}, \acrshort{ESA}, \acrshort{JAXA}...

\begin{figure}[H]
			\centering
			\includegraphics[page=1,trim={0cm 0cm 0cm 0cm},clip=true,width=75mm]{figurastfm/Chapter3/PDF/ccsdslogo.pdf}
			\caption{\acrshort{CCSDS} logo} 
		\end{figure}


One of its most important contributions is the so-called space packet procotol, typically known just as the organism, \acrshort{CCSDS}. It specifies a communications protocol to be used by space missions to transfer space application data over a network that involves a ground-to-space or space-to-space communications link. It is defined in \cite{ccsds1}, \cite{ccsds2} and \cite{ccsds3}. \autoref{ccsdspacket} shows an example of \acrshort{CCSDS} telemetry packet format.

\begin{figure}[H]
			\centering
			\includegraphics[page=1,trim={0cm 0cm 0cm 0cm},clip=true,width=165mm]{figurastfm/Chapter3/PDF/ccsdspacket.pdf}
			\caption{\acrshort{CCSDS} format example \cite{ccsds2}} \label{ccsdspacket}
			\vspace{-1cm}
		\end{figure}


\subsubsection{XML Telemetric and Command Exchange} \label{xtce}

XML Telemetric and Command Exchange is an XML based data exchange format and spacecraft telemetry and command meta-data \cite{wiki}. For a given mission there are a number of lifecycle phases supported by a variety of systems and organizations; telemetry and command definitions must be exchanged among all of these phases, systems, and organizations.

A typical example of this process is between the spacecraft manufacturer and spacecraft-operating agency. The first defines the telemetry and command data in a format much different than the one used in the ground segment. This creates the need for database translation, increasing customization and probability of error. Using a common exchange format streamlines the process of transferring definitions from the satellite integrator to the operations team and between ground systems supporting the same satellite. This reduces the need to develop mission-specific database import/export tools and enables the reutilization of command and telemetry database tools \cite{xtce}. \acrshort{XTCE} has also been adopted as a recommendation by the \acrshort{CCSDS}.

\subsection{Formal Requirements Definition}


\begin{table} [H]
\centering

\begin{tabularx}{\linewidth}{lX}

\multicolumn{1}{c}{\textbf{Ref.}}                      & \multicolumn{1}{c}{\textbf{Formal Requirements}}                    \tabularnewline \specialrule{1.1pt}{1pt}{1pt}
GS.FoR.1                                              & The \glsname{ground} shall have a centralized system \acrshort{CCSDS} and \acrshort{XTCE} compatible. \tabularnewline \midrule
GS.FoR.2                                              & The \glsname{ground} shall allow connectivity with standard radio stations. \tabularnewline \midrule
GS.FoR.3                                            & The \glsname{ground} shall be able to use both \acrshort{TCP} or \acrshort{UDP} for the local environment operation.  \tabularnewline \midrule
GS.FoR.4                                                   & The \glsname{ground} shall have a text-based logging system which continuously registers events, along with timing information.\tabularnewline \midrule
GS.FoR.5                                                   & The \glsname{ground} shall allow connectivity with both  \acrshort{IEEE}802.3 and  \acrshort{IEEE}802.11 standards.   \tabularnewline \midrule
GS.FoR.6                                                   & The \glsname{ground} shall count with a \acrshort{LED} lamp exhibiting an irradiance of at least 200 \SI{200}{W/m^2}.   \tabularnewline \midrule

\end{tabularx}
\caption{\glsname{ground} - Formal Requirements}
\vspace{-0.5cm}
\label{forgstation}

\end{table}



\section{Simulation CubeSat Analysis} \label{cubesatanaly}

CubeSats were introduced at \autoref{cubesatdefin}. In this section, the subsystems addressed in this Master's Thesis are profusely analyzed and just as done before, formal requirements are defined. It is important to keep in mind the big picture illustrated in \autoref{cubesatsub} when analyzing a \glsname{cubesat} as a whole system. As it was specified at that point, this Master's Thesis will focus on \acrshort{OBC}, \acrshort{ADCS} and in the power source area of the \acrshort{EPS}.

\subsection{On-board computer (OBC)}\label{sec:OBC}

On-board computer is probably the most important subsystem of a \glsname{cubesat}. It is in charge of controlling the whole system, controls I/O and coordinates the different subsystems to successfully perform the tasks needed at every moment. As seen in \ref{cubesatdefin}, sometimes \acrshort{OBC} inherently includes another subsystem such as \textbf{On-board~Data~Handling}~(\acrshort{OBDH}), while in others systems it tackles functions related to a different subsystem (e.g. \acrshort{TCC}) out of simplicity. In order to perform all these tasks, the \acrshort{OBC} can be composed of more than one processing core, which can be used to free the main one. This wide scope makes this subsystem to require a strong effort of integration with different specialists.

Additionally, sensors can also be considered part of the \acrshort{OBC} given that as a last resort it is in charge of processing the information received from them. Besides, a significant number of them are physically placed at the \acrshort{OBC}

Just as previous subsystems analyzed, \acrshort{OBC} can be subdivided into different modules, which are broken down next.

\subsubsection{Central Processing Unit}

The Central Processing Unit  can be seen as the microprocessor of a standard computer. It is typically an \textbf{embedded system} and the most complex module of the system. It works as interface with the rest of the modules and \acrshort{HID}.

It is recommendable for this module to be as flexible as possible and to allow a wide range of programming possibilities. It must be able to deal with hardware interruptions, act as \textbf{Scheduler}, manage Write/Reading operations on memory, prevent the system from failures and recover it from faulty situations... in sum it is the global coordinator of the system.

Central Processing Units can be either a single \acrshort{SOC} or a complete module \acrshort{COTS} and in aerospace sector they are usually able to have a complete \textbf{Operating System} installed, normally Linux-based. %The diagram in FIGURE illustrates some of the duties and relations with the rest of the system that this module usually has.

For all the reasons exposed, a high-level analysis of this subsystem is impossible due to the vast amount of possibilities. Therefore, although the main requirements are listed at \ref{obcreqs} just as with the rest of the sections, to go deeper on the possibilities and distinctive features of this kind of modules requires the scope to be somewhat more constrained; that task will be performed and exposed in \autoref{obcdesig}.

\subsubsection{Co-processing Programmable Core} \label{copropo}

One of the featured characteristics of the cutting-edge designs in a variety of sectors is including a reconfigurable or programmable hardware module, which can be used to act as auxiliary processor, performing a variety of tasks. \textbf{Field-Programmable Gate Arrays} (\acrshort{FPGA}) are usually used for this purpose. The task to be developed by this module is generally specified using some Hardware Description Language such as \acrshort{VHDL} or Verilog.

An \acrshort{FPGA} can be used to deal with any computable problem, featuring its speediness for some applications because of their parallel nature. Some of the common usages nowadays are aerospace, digital processing, wireless communications or image processing. All of those usages are potentially useful in a \glsname{cubesat} like the one under development; that is the reason why Co-processing Programmable Core are increasingly implemented, its versatility allows to solve different problems with a single module.

This brief analysis will focus on \acrshort{FPGA} because of their wide usage for this purpose, as mentioned before. The most common \acrshort{FPGA} architecture consists of an array of \textbf{logic blocks} and routing channels with I/O capabilities. A certain application circuit must be mapped into an \acrshort{FPGA} with the adequate resources, which may vary considerably. Generally speaking, a logic block consists of a number of logical cells which are typically composed of four input \textbf{Lookup tables} \acrshort{LUT}, a \textbf{full adder} and \textbf{D-type flip-flop}. \autoref{fpgafig} shows this structure.	Also, as depicted in \autoref{fpgafig}, a clock signal is needed as most of the electronics inside of an \acrshort{FPGA} is synchronous.
 


			\begin{figure} [H] 				
				\centering
				\includegraphics[page=1,trim={0cm 0cm 0cm 0cm},clip=true,width=85mm]{figurastfm/Chapter3/Imagenes/fpga.png}
				\caption{Typical logic cell \cite{wiki}} \label{fpgafig}
			\end{figure}
			
			Among the concepts introduced before, \textbf{Lookup tables} (\acrshort{LUT}) have special importance. They are arrays which replace runtime computation with simple array indexing which is time-saver in processing terms, as retrieving a value from memory is often faster than I/O operation \cite{wiki}. \acrshort{LUT} are in charge of storing the truth table of the any boolean function. It is a parameter sometimes used to estimate the \textit{capacity} of an \acrshort{FPGA}.
			
			Regarding the programming, the user normally provides a certain design using some Hardware Description Language, as mentioned. A netlist is generated by a procedure called \textbf{synthesis}, which will be fitted to the \acrshort{FPGA} architecture using a process called \textbf{place-and-route}. As a result of this process, the user will obtain a performance report exposed via timing analysis or simulation. Finally, the generated binary file is used to program the \acrshort{FPGA}.
			
		\acrshort{FPGA} are particularly suitable for this Co-processing module because of their versatility; the variety of implementations possible is countless; for example, one of the most featured applications is \textbf{embedding a processor} inside an \acrshort{FPGA}. It has many advantages, e.g.: specific peripherals can be chosen based on the application, mitigates obsolescence, reduces costs and allows impressive customization. For instance, there is a complete flexibility to select any combination of peripherals and controllers which can be directly connected to the processor's bus. This point allows meeting non-standard requirements, for example a \acrshort{COTS} processor with ten \acrshort{UART} may be impossible to find, but it is easy to implement in an \acrshort{FPGA} embedded processor \cite{fpgaproc}. An embedded processor implementation example can be found on \cite{fpgacalvo}. Another possible implementations using \acrshort{FPGA} are generating digital interfaces such as \acrshort{I2C} or generating \acrshort{PWM}.
	
\subsubsection{Communications}

For obvious reasons, Communications subsystem or, more specifically, Telecommand~\&~Telemetry~(\acrshort{TCC}) are vital in CubeSats. Communications must be understood not only as the ones between the \glsname{cubesat} and the \glsname{ground} but also inside the \glsname{cubesat} itself. Next, a brief analysis of the useful technologies for CubeSats is introduced.

\paragraph{Wired}

Normally, wired connections are necessary even in CubeSats intended to fly to be able to program data, perform test and so forth. In this project, because of the simulation parcel, wired communications will be specially important.

\subparagraph{Serial}

This in an historical wired interface. Serial communication is based on sequentially sending data one bit at a time. Serial ports are typically identified as such which comply with RS-232 standard. One of the main drawbacks of serial communications is their slowness.

Although it is considered to be deprecated, it is still usual in many electronics systems, as debugging port or just as a contingency way of communication. In fact, these have could its usefulness in a \glsname{cubesat}. Throughout years, serial ports have led to new derived technologies, such as \acrshort{USB}. \autoref{serial} depicts DE-9 connectors, to be used with RS-232 standard.

			\begin{figure} [H] 				
				\centering
				\includegraphics[page=1,trim={0cm 0cm 0cm 0cm},clip=true,width=65mm]{figurastfm/Chapter3/Imagenes/serial.jpeg}
				\caption{DE-9 connector \cite{wiki}} \label{serial}
				\vspace{-1cm}
			\end{figure}


\subparagraph{Universal Serial Bus}

One of the most extended interfaces in industry. Derived from serial ports, they were initially intended to standardize the connection of peripherals, and finally have largely replaced its preceding interfaces.

\acrshort{USB} has plenty of advantages, improved ease of use as it is self-configurable and hot pluggable. It is also much more faster than serial ports, reaching 5 Gb/s in the 3.0 version, one of the latest \cite{wiki}. They are also extensively used as power ports. \autoref{usbfig} shows some of the standard connectors pinout.

			\begin{figure} [H] 				
				\centering
				\includegraphics[page=1,trim={0cm 0cm 0cm 0cm},clip=true,width=80mm]{figurastfm/Chapter3/Imagenes/usb.png}
				\caption{USB standard connectors pinout \cite{wiki}} \label{usbfig}
				\vspace{-0.5cm}
			\end{figure}

Because of their versatility, \acrshort{USB} interfaces may have numerous applications on CubeSats, from programming to power ports or to downloading data.

\subparagraph{Ethernet}

First standardized in 1983 as \acrshort{IEEE}802.3 is by far the most used wired technologies in \acrshort{LAN}. Systems communicating over this interface divide streams of data into \textbf{frames} which contain source and destination addresses and allow error-checking. 

In CubeSats, Ethernet interfaces may be useful to allow communications using high-level protocols such as \acrshort{SSH}, which ease communications and configuration of the system.

			\begin{figure} [H] 				
				\centering
				\includegraphics[page=1,trim={0cm 0cm 0cm 0cm},clip=true,width=60mm]{figurastfm/Chapter3/Imagenes/ethernet.png}
				\caption{RJ45 connector \cite{ethernet}}
				\vspace{-2cm}
			\end{figure}


\paragraph{Wireless}

Wireless communications are vital in real CubeSats, for obvious reasons. In orbit CubeSats communicates with \glsname{ground} using radio-links and dedicated antennas. It is normally a whole subsystem within Communications system. It goes beyond the scope of this project analyzing that kind of implementations; extended radio-link analysis and calculation can be found on \cite{tfg}. In this section, it will be briefly introduced local area wireless communications, which can be used as base for future developments focused on real long-distance links.

\subparagraph{Radio-frequency} \label{radiofr}

Communications over the air are the only choice when dealing with space communications. There is a variety of frequencies used in radio-links, depending on distance, power and numerous additional factors. Dependingo on the frequency used, the link will have certain particularities which shall be taken into account. \autoref{tfgradio} schematically shows an example of radio path between a \glsname{ground} based on Granada and \acrshort{ISS}; it illustrates some of the aspects to be considered in a long-distance radio-frequency communication such as elevation angle or effective distance, see \autoref{tfgradio}.

			\begin{figure} [H] 				
				\centering
				\includegraphics[page=1,trim={0cm 0cm 0cm 0cm},clip=true,width=130mm]{figurastfm/Chapter3/PDF/radio_l.pdf}
				\caption{Radio-path basic elements \cite{tfg}} \label{tfgradio}
				\vspace{-0.5cm}
			\end{figure}


Although this project will not tackle this kind of communications, short-distance radio link may be considered useful for the Simulation Platform branch of the project. Contrary to the links designed for larger distances, short-distance communications over the air exhibits way lower complexity and cost due to the minor requirements in relation with power, losses constraints, among others.


\subparagraph{IEEE802.11}

Commonly known as Wi-Fi (a trademark from the Wi-Fi Alliance), it is based on \acrshort{IEEE}802.11 standard and it is the most extended wireless communication protocol for local areas. Once again, this makes it specially suitable for this Simulation Platform because of its ease of use and extended use. On the other hand, it cannot be used for space radio-links.

Since its origins, back in 1997, the standard has gone through a significant number of reviews. Nowadays the most usual versions are \acrshort{IEEE}802.11g/n/ac reaching data rates up to 1300 Mbit/s in the 5 GHz band \cite{wiki}.

As a result of the wide use of \acrshort{IEEE}802.11, the majority of the systems related to it are highly standardized (\acrshort{IC}, antennas, amplification subsystems, etc.) which reduces complexity and eases integration.

\subsubsection{Payload. Sensors.}

The \glsname{payload} is generally known as the amount of cargo capacity of an aircraft, including fuel and people. However, it may also refer to the equipment specifically intended to perform a certain mission while in orbit, for instance a camera or a \glsname{startracker}.

In CubeSats field, a common \glsname{payload} example are \textbf{sensors}: they are typically a crucial part of any mission, not only as part of the \glsname{payload} but also a necessary part for the correct functioning of the whole system. Normally they are some kind of electronic device which takes some measurement or perform a certain action depending on the inputs it receives. Some examples of sensors have already been treated before, see \ref{imus}. Another possible sensors are: barometer, thermometer, magnetometer, lightmeter, tachometer... the list is countless. Many of them will be analyzed and implemented in \autoref{chap:chapter4}. \autoref{sensores} shows some examples of them.

While this analysis considers sensors as part of the \acrshort{OBC}, given that eventually the data will be treated there, they can belong to a different subsystem such as \acrshort{ADCS}, see section \ref{imus}. 
			\vspace{-0.5cm}

			\begin{figure}[H]
			\centering
			\subfloat[\acrshort{IMU}]{\includegraphics[page=1,trim={0cm 0cm 0cm 0cm},clip=true,width=55mm]{figurastfm/Chapter3/Imagenes/MPU.png}}
			 \quad
			\subfloat[Barometer and thermometer]{\includegraphics[page=1,trim={0cm 0cm 0cm 0cm},clip=true,width=55mm]{figurastfm/Chapter3/PDF/BMP.pdf}}
			\caption{Sensor examples \cite{hobbyking}} \label{sensores}
			\vspace{-2cm}
\end{figure}	


\subsubsection{Flight Software. On-Board Data Handling} \label{flight}

In charge of all the exposed, there must be a \textbf{governor}, a coordinator which controls the hardware and deals with the different stages and situations of a mission, this is the \textbf{Flight Software} (FSW). As in a significant number of the subsections analyzed through this Master's Thesis, Flight Software is such a large field that it allows a complete project focused on it. Therefore, hereby only a basic classification is introduced; this will be an important issue at the design stage and also for future improvements of the system.

As far as this project concerns, considered flight software can be subdivided into the following.

\paragraph{Non-real-time Operating System}

Non-real-time Operating Systems is basically a general purpose \acrshort{OS} to be used with personal computers, servers, etc. The main difference with \acrshort{RTOS} is \textbf{determinism}. Non-\acrshort{RTOS} are not deterministic as tasks will not run at a certain time and for a certain time, there are no guarantees for critical tasks, exhibits high latency because of using unpredictable virtual memory, as well as \glsname{jitter}.

The previous characteristics translate into the use of \textbf{non-preemptive schedulers}. Examples of these kind of \acrshort{OS} are the widely used Windows\,\textsuperscript{\textregistered}, Debian, etc.

\paragraph{Real-time Operating System}

In contrast with the latter, \acrshort{RTOS} are completely deterministic, this is, how and when a task will run given whatever conditions defined for it to do so, it is \textbf{guaranteed}. They are intended to process data as it comes in, without buffer delays. There are a huge amount of systems nowadays which must use \acrshort{RTOS}: cars, \textbf{spacecrafts}, avionics, critical systems... In sum, \acrshort{RTOS} must be able to compute a task in a \textbf{limited or predictable amount of time} i.e., is time-bounded. However, this behaviour has nothing to do with processing speed but with a known deadline (a second, an hour or a month) and reduced \glsname{jitter}.

From a technical perspective, this implies not to use virtual memory, strict scheduling (preemptive) and avoid non-deterministic elements. \acrshort{RTOS} are usually much smaller than general purpose \acrshort{OS} in order to ease maintenance and find sources of delay. One of the most used \acrshort{RTOS} is \textbf{VxWorks} \cite{vxworks}; Linux non-preemptive kernel can also be \textbf{patched} to allow real-time behaviour.


			\begin{figure} [H] 				
				\centering
				\includegraphics[page=1,trim={0cm 0cm 0cm 0cm},clip=true,width=70mm]{figurastfm/Chapter3/Imagenes/vxworks.png}
				\vspace{0.3cm}
				\caption{VxWorks logo \cite{vxworks}} 
				\vspace{-2cm}
			\end{figure}

\subsubsection{Formal Requirements Definition} \label{obcreqs}


\begin{table} [H]
\centering

\begin{tabularx}{\linewidth}{lX}

\multicolumn{1}{c}{\textbf{Ref.}}                      & \multicolumn{1}{c}{\textbf{Formal Requirements}}                    \tabularnewline \specialrule{1.1pt}{1pt}{1pt}
OBC.FoR.1                                              & The \acrshort{OBC} shall have an \acrshort{I2C} interface which allows communication with the rest of subsystems and components. \tabularnewline \midrule
OBC.FoR.2                                              & The \acrshort{OBC} shall have an \acrshort{SPI} interface which allows communication with the rest of subsystems and components. \tabularnewline \midrule
OBC.FoR.3                                            & The \acrshort{OBC} shall have a programmable \acrshort{FPGA} of at least 2500 \acrshort{LUT}. \tabularnewline \midrule
OBC.FoR.4                                                   & The \acrshort{OBC} shall have one RJ-45 connector. \tabularnewline \midrule
OBC.FoR.5                                                   & The \acrshort{OBC} shall have at least two \acrshort{USB} Micro-B device-ports and capability to deal with four.  \tabularnewline \midrule
OBC.FoR.6                                                   & The \acrshort{OBC} shall be programmable with an external computer using a \acrshort{USB} Micro-B connector. \tabularnewline \midrule
OBC.FoR.7                                                   & The \acrshort{OBC} shall have a \acrshort{LED} system to check the correct functioning of the \acrshort{EPS}  \tabularnewline \midrule
OBC.FoR.8                                                   & The \acrshort{OBC} shall have at least two \acrshort{USB} device-ports and capability to deal with four.  \tabularnewline \midrule
OBC.FoR.9                                                   & The \acrshort{OBC} shall have a barometer with a resolution of at least 0.1 hPa.  \tabularnewline \midrule
OBC.FoR.10                                                   & The \acrshort{OBC} shall have a thermometer with a resolution of at least 0.1\textdegree~C.  \tabularnewline \midrule
OBC.FoR.11                                                   & The \acrshort{OBC} shall have at least 8 \acrshort{ADC} channels. \tabularnewline \midrule
OBC.FoR.12                                                   & The \acrshort{OBC} shall have 6 analog sun sensors, one per each face of the \glsname{cubesat}. \tabularnewline \midrule
OBC.FoR.13                                                   & The \acrshort{OBC} shall have a Real Time Clock (\acrshort{RTC}) which provides the whole system with time information. \tabularnewline \midrule
OBC.FoR.14                                                   & The \acrshort{OBC} shall have an \acrshort{SD} card holder, to store telemetry data.  \tabularnewline \midrule
OBC.FoR.15                                                   & The \acrshort{OBC} shall have an integrated \acrshort{IEEE}802.11g device which allows wireless communications. \tabularnewline \midrule
OBC.FoR.16                                                   & The \acrshort{OBC} shall implement a linux-based \acrshort{RTOS}. \tabularnewline \midrule

\end{tabularx}
\caption{\acrshort{OBC} - Formal Requirements}
\vspace{-0.5cm}
\label{forobc}

\end{table}

\newpage

\subsection{Attitude Determination and Control System (ADCS)}\label{sec:ADCS}

%mirar que son esas peliculas
%$Magnetic Controllers: B-dot, Y-spin and Cross-Product$
%Wheel Controllers: Y-Momentum and Quaternion Feedback
%Estimators: Rate Kalman Filter (RKF), TRIAD algorithm, Extended Kalman Filter (EKF)

The Attitude Determination and Control System (\glsname{ADCS}) (or \glsname{AOCS}) is the subsystem in charge of assuring that the CubeSat is correctly oriented and is able to deal with external torques and forces properly. It consists of a series of sensors and actuators, which work along with different algorithms, allowing proper pointing to the objective, like sun pointing (power) or receiving antennas pointing (communications). All these factors will be extensively treated along this and \autoref{chap:chapter4}, but first it is necessary to analyze and put into context the need of this subsystem, as well as the basic concept regarding \textbf{Spacecraft dynamics} and mathematical tools required to deal with it.

\subsubsection{Spacecraft dynamics}

Spacecraft dynamics is again an extensive topic which cannot be addressed in a few lines in this project. Nevertheless, this section will introduce some of the most important concepts used in this Master's Thesis, vital for the design stage; particularly, rotation representation and concepts such as angular momentum or \textbf{inertia~matrix} are briefly introduced. Out of simplicity, no demonstrations will be included and the analysis will be kept as simple as possible.

\paragraph{Rotation representations}\label{sec:rotationrepres}

The attitude of a three-dimensional body is most conveniently defined with a set of axes fixed to the body. This set of axes is generally a triad of orthogonal coordinates, and is normally called a \textit{body coordinate frame}. The attitude of a body is thought of as a coordinate transformation that transforms a defined set of reference coordinates into the body coordinates of the spacecraft. 

Below, the most important three-axis attitude frames are summarily exposed. For further details, see \cite{sabroff} and \cite{wertz}.

\vspace{-0.2cm}
\subparagraph{Direction cosine matrix}\label{sec:directioncosine}

The basic three-axis attitude transformation is based on the direction cosine matrix, also called \textbf{attitude matrix}. Any attitude transformation in space is actually converted to this essential form. It has the important property of mapping vectors from the reference frame to the body frame, describing the transformation from coordinate system \textit{a} to \textit{b}.

This system has no singularities, which is its main advantage, but on the other hand, it supposes propagating nine elements (three unit vectors, each with three components) defining an orthogonal right-handed triad.

\subparagraph{Euler angles}\label{sec:eulerangles}

The Euler angle rotation is defined as successive angular rotations about the three orthogonal axes of the body frame. Typically, these are defined by \textit{i}, \textit{j}, and \textit{k}, and those of the reference frame by \textit{I}, \textit{J}, and \textit{K}. There is a multitude of order combinations by which the rotation can be performed.  It is common to define the Euler~\textbf{roll~angle} ($\phi$) as a rotation about the $X$ body axis, the \textbf{pitch angle} ($\theta$) about the $Y$~body~axis, and the \textbf{yaw angle} ($\psi$) about the $Z$ body axis. However, any other definition is acceptable as long as it remains consistent with the analytical development. \autoref{fig:eulerangles} shows an illustration of this rotation representation system. 

			\begin{figure}[h]
				\centering
				\includegraphics[page=1,trim={0cm 0cm 0cm 0cm},clip=true,width=90mm]{figurastfm/Chapter3/PDF/euler_angles.pdf}
				\caption{Three-axis Euler angles around \glsname{cubesat} \cite{ESA}}
				\label{fig:eulerangles}
			\end{figure}

Euler angles are intuitive and often relevant for requirements specification, but computer implementation is not straightforward and presents numerics singularities with 90\textdegree{} rotations, a phenomena called \textbf{gimbal lock}; it consists in the loss of one degree of freedom in a three-dimensional gimbal system, when the axes of two of the three gimbals are driven into a parallel configuration, \textit{locking} the system into rotation and degenerating into a two-dimensional space \cite{wiki}. \autoref{gimbal} illustrates this problem.
%\vspace{0.5cm}
	
Besides, attitude rotations derived on the basis of Euler angles necessitate dealing with nine elements of the direction cosine matrix, and each element may include several trigonometric functions. These are some of the reasons for the wide use of quaternions, explained next.

	\begin{figure}[H]
				\centering
				\includegraphics[page=1,trim={0cm 0cm 0cm 0cm},clip=true,width=140mm]{figurastfm/Chapter3/Imagenes/gimballock.jpg}
				\caption{Gimbal lock pheonomena \cite{gimbal}}
				\label{gimbal}
				%\vspace{-2cm}
			\end{figure}


\subparagraph{Quaternions}\label{sec:quaternions}

Quaternions are a direct consequence of the properties of the direction cosine matrix. It can be shown with some linear algebra that a proper orthogonal 3x3 matrix has at least one \glsname{eigenvector} with eigenvalue of unity. That eigenvector $e_{i}$ has the same component along the body axes and along the reference frame axes. It can also be demonstrated that any attitude transformation by a series of consecutive rotation about the three orthogonal unit vectors of the coordinate system can be achieved by a single rotation about the eigenvector with unity value. The quaternion is defined as a vector as follows:

\begin{equation} 
q = q_{4} + iq_{1}+jq_{2}+zq_{3}
\end{equation}



With the unit vectors, $i, j, k$ satisfying:

\begin{equation} 
i^{2}=j^{2}=k^{2}=-1
\end{equation}


The main disadvantages of quaternions are being non-linear kinematics and not very intuitive. However, they \textbf{have no singularities} and are easy to implement in software, reaching a great efficiency in computations; while representing the attitude of a body in a reference frame by a direction cosine matrix requires knowing nine parameters $a_{ij}$, quaternions only require four $q_{i}$ parameters. Besides, the elements of the direction cosine matrix, in contrast to those of the quaternions, are trigonometric functions, which are much more cumbersome to compute. \autoref{quatplot} depicts the quaternions concept graphically.

			\begin{figure}[H]
				\centering
				\includegraphics[page=1,trim={0cm 0cm 0cm 0cm},clip=true,width=65mm]{figurastfm/Chapter3/Imagenes/quaternion.png}
				\caption{Quaternion graphical representation \cite{euclidean}}
				\label{quatplot}
			\end{figure}
			
\paragraph{Inertia matrix and angular momentum}\label{imatrix}

These are concepts constantly utilized when dealing with physical designs, and are closely related to another basic concept: the \textbf{center of mass}; given a certain distribution of mass in space, it is defined as the unique point where the weighted relative position of the distributed mass sums to zero \cite{wiki}. \autoref{centermass} illustrates where the center of mass (C) locates in a block toy.

	\begin{figure}[H]
				\centering
				\includegraphics[page=1,trim={0cm 0cm 0cm 0cm},clip=true,width=50mm]{figurastfm/Chapter3/PDF/centermass.pdf}
				\caption{Center of mass example \cite{wiki}}
				\label{centermass}
			\end{figure}
			
			On the other hand, the \textbf{moment of inertia} of a body determines the amount of \glsname{torque} needed to reach a certain angular acceleration about a rotational axes. It depends on the body's mass distribution and chosen axis; the larger the moment, the greater the \glsname{torque} needed to change the rotation rate. Depending on the number of axis of rotation, the different moments of inertia are arranged into a NxN matrix, with N that number of axis, mutually orthogonal; this is the so-called \textbf{inertia matrix}, and represents how the mass is distributed in the body. A desirable body design is that whose inertia matrix is diagonal; non-zero values are called the \textbf{principal axes of inertia}. They are always symmetric.
			
			These concepts are important in order to understand the basics of the rotational dynamics. Particularly, CubeSats attitude varies according to the fundamental equations of motion for rotational dynamics or \textbf{Euler equations}:

\begin{equation}\label{eulerrot}
\dot{H} = T-\omega\times H
\end{equation}

\autoref{eulerrot} represents the equations for the conservation of the angular momentum, denoted by \textbf{H}. Angular momentum is the rotational motion of a body that will continue unless changed by a \glsname{torque}, and it is calculated as the body's \textbf{moment of inertia} times its angular rate. It is clear from that expression that the body's angular momentum will remain constant in absence of external torques, \textbf{even if some parts of the body moves with respect to another}, for example, a reaction wheel (see \autoref{actuators}) spinning. Therefore, if that happens, the rest of the body will have to spin in the opposite direction in order to conserve the total angular momentum \cite{smad}. Further manipulations to \autoref{eulerrot} allow understanding how attitude can change due to multiple causes, closely related to concepts as \textbf{inertia matrix}, among others, described before.
			
			
			%This short introduction is enough for the reader to understand the use of these concepts along this Master's Thesis.

\subsubsection{Inertial Measurement 
Units} \label{imus}

If there is an element of particular importance in the \acrshort{ADCS}, it is the \textbf{Inertial Measurement Unit} (\acrshort{IMU}). They are electronic devices which measure gravitational acceleration, angular rate and orientation of the device, by using accelerometers, gyroscopes and magnetometers \cite{wiki}, normally with a set of these three per axis, see \autoref{fig:eulerangles}. \acrshort{IMU} are widely used in a variety of spacecrafts, including planes and satellites. However, they are also useful in everyday products such as mobiles phones or wearable devices, see \autoref{movil}.


\begin{figure}[H]
				\centering
				\includegraphics[page=1,trim={0cm 0cm 0cm 0cm},clip=true,width=60mm]{figurastfm/Chapter3/Imagenes/movil.jpg}
				\caption{Mobile phones also count with an \acrshort{IMU} \cite{sentinel}}
				\label{movil}
				\vspace{-2cm}
			\end{figure}
			
In spacecrafts field, an \acrshort{IMU} is usually part of the Inertial Navigation System, which uses its measurements to calculate attitude or angular speed. Specifically, angular speed is normally integrated to get angular position. In combination with the gravitational acceleration given by the accelerometers, attitude can be estimated using predictors such as a Kalman filter. If that estimation is used to transform acceleration measurements into a inertial reference frame and integrated once, linear velocity can be achieved, linear position if integrated twice. As it could not be any other way, under this complex electronic system basic physics works, stated at \autoref{eqmovim}.


%\begin{equation} \label{eqmovim}
%\begin{aligned}
%\text{Acceleration measurements} \rightarrow & a_\text{g}(t) \\
%&v_\text{linear}(t) = \int_{0}^{t} a_\text{g}(t) dt \\
%&r_\text{linear}(t) = \int_{0}^{t} v_\text{linear}(t) dt
%\end{aligned}
%\end{equation}

\begin{equation} \label{eqmovim}
%\begin{aligned}
a_\text{g}(t) \rightarrow v_\text{linear}(t) = \int_{0}^{t} a_\text{g}(t) dt \rightarrow r_\text{linear}(t) = \int_{0}^{t} v_\text{linear}(t) dt
%\vspace{-2cm}
%\end{aligned}
\end{equation}

\autoref{diagmov} illustrates this process. 

\begin{figure}[H]
				\centering
				\includegraphics[page=1,trim={0cm 0cm 0cm 0cm},clip=true,width=150mm]{figurastfm/Chapter3/PDF/attcalc.pdf}
				\caption{Position, velocity and attitude calculation process}
				\label{diagmov}
			\end{figure}

\newpage
\subsubsection{Control laws}

As described before, the \textbf{attitude determination} process consists in combining available sensor inputs to provide an accurate solution for the attitude state as a function of time. The term \textbf{control law} is the name normally used in \acrshort{ADCS} systems to refer to the \textbf{algorithms} which, using the data from the sensors, control the actuators available as needed in order to reach the target \textbf{attitude}. \autoref{controlaw} depicts this continuous process.

\begin{figure}[h]
				\centering
				\includegraphics[page=1,trim={0cm 0cm 0cm 0cm},clip=true,width=155mm]{figurastfm/Chapter3/PDF/controlaw.pdf}
				\caption{Control laws interaction with the rest of the \glsname{cubesat}}
				\label{controlaw}
			\end{figure}
			
			There is an enormous range of control laws, with different complexities and accuracies. Most spacecrafts use some kind of active control loop, as shown in \autoref{controlaw}. It depends on the actuator used for the attitude maneuver (see section \ref{actuators}). For systems in which spacecraft rates will be small, 3-axis control can normally be decoupled into three independent axes. Two of the most used ones are briefly introduced next.
			
			
\paragraph{Proportional-Integral-Derivative Controller} \label{pid3}

In the simplest form, each axis of the \glsname{cubesat} can be controlled by a \textbf{Proportional-Derivative controller}, with a control \glsname{torque} given by \autoref{pideq}.

\begin{equation} \label{pideq}
T_C=K_P\theta_E+K_D\omega_E
\end{equation}

Where $\theta_{\text{E}}$ is attitude error angle and $\omega_{\text{E}}$ is the attitude rate error. The most important design parameter is $K_{\text{P}}$ or proportional gain, representing the amount of control torque desired from a unit of attitude error. It determines the bandwidth of the system, closely related to the speed of response. To improve performance, sometimes an \textbf{integrator} is included, completing the Proportional-Integral-Derivative Controller or simply \acrshort{PID}. The error signals are normally measured as \textbf{Euler angles}. One of the keys for a correct functioning is \textbf{tuning}; there are different methods in the literature to get an optimal tuning depending on the requirements.

\paragraph{B-dot Controller} \label{bdotsec}

It is an specific control law dedicated to the \textbf{detumbling} stage. \glsname{tumbling} occurs right after deployment from the \acrshort{PPOD}. In that moment, \glsname{cubesat} is unusable because of the uncontrolled free spinning. Therefore, the first step before trying to control attitude using any control law such as the \acrshort{PID} seen before, must be reaching a controlled rotation speed. That is the detumbling process.

B-dot is based on the usage of \glsname{magnetorquers}, introduced in \ref{actuators}. They are used to generate a \glsname{torque} opposed to the rotation of the \glsname{cubesat}, by applying an alternating positive/negative current to the coil, which produces a certain magnetic moment. It only requires knowing the evolution of the magnetic field measured by the sensors. The functioning is simple, when the satellite is rotating around a given axis, progressively pointing in the same direction as the Earth's magnetic field, the magnetic field measured for that axis increases and so does the derivative of the magnetic field, being positive. As a response, an opposite signal to that derivative is sent through the coil, that is, a negative current. Once the magnetic field of that axis turns negative, the control signal is reversed, so the resulting torque is conserved. In sum, B-dot algorithm creates a magnetic moment in the opposite direction to the change in the magnetic field measured, expressed by \autoref{bdot}.

\begin{equation}\label{bdot}
M_\text{i}=-k_\text{i}\dot{B}_\text{i}
\end{equation}

Where $i$ is one of the axes, $M_\text{i}$ the generated magnetic moment, $B_\text{i}$ the measured magnetic moment at that axis and $k_\text{i}$ is the proportional constant calculated. 
			
\subsubsection{Actuators} \label{actuators}

CubeSats may implement a variety of actuators to be used to control attitude. The most important ones are \textbf{reaction wheels} and \textbf{magnetorquers}, both of which have been mentioned before. Next, they are briefly introduced.

\paragraph{Reaction wheels}

Reaction wheels are a particular kind of flywheel used actively in attitude control. It is normally operated at a constant rotation rate which makes the \glsname{cubesat} store a large angular momentum, tending to stabilize the satellite and allowing high pointing accuracy. They generate \glsname{torque} by turning the wheel in the opposite direction of the rotation. Reaction wheels are usually controlled using \acrshort{PWM} signals. They allow generating considerably more torque than \glsname{magnetorquers}, however, current consumption is way higher and increases the weight of the \glsname{cubesat}, as well as severely impacting the mass distribution, if it is not positioned correctly. \autoref{reactw} shows two typical \glsname{cubesat} reaction wheels.

			\begin{figure}[H] 
				\centering
				\includegraphics[width=75mm]{figurastfm/Chapter3/Imagenes/reactionw.png}
				\caption{Reaction wheels \cite{csatshop}}      		
				\label{reactw}
				%\vspace{-0.5cm}
  		\end{figure}
			
			\glsname{GranaSAT} counts with different reaction wheels designs from previous works, particularly \cite{testbedgranasat}. \autoref{3dreact} shows the manufactured reaction wheel.
			
				\begin{figure}[H] 
				\centering
				\includegraphics[width=85mm]{figurastfm/Chapter3/PDF/burgoswheel.pdf}
				\caption{Reaction wheel manufactured in \glsname{GranaSAT} laboratory \cite{testbedgranasat}}      		
				\label{3dreact}
				%\vspace{-0.5cm}
  		\end{figure}
			
\newpage
\paragraph{Magnetorquers}

Also known as torque rods, they are systems widely used to control attitude or going out of \glsname{tumbling}. They are electromagnetic coils, arranged in different ways, which create a magnetic dipole and produces a certain \glsname{torque} when interacting with Earth's magnetic field. Depending on the performance required and available area, it is built with a number of turns. They are reliable and energy-efficient but the \glsname{torque} that \glsname{magnetorquers} are able to provide is very limited. An important point is that the \glsname{torque} can be generated only perpendicularly to the Earth's magnetic field vector. As advanced in \ref{bdotsec}, they are applied an alternating current depending on the desired attitude, producing a \glsname{torque} $\tau$ given by \autoref{magnetoeq}.
		
		\begin{equation}\label{magnetoeq}
		\tau=nIA\times B
		\end{equation}
		
		Where $n$ is the number of turns, $I$ the current provided, $A$ the area of the coil and $B$ the magnetic field vector. \autoref{magnetor} shows two types of magnetorquers, the rod one, in which a copper wire is wrapped around a ferromagnetic core and an embedded coil based on the \acrshort{PCB} design.
			%\vspace{-0.5cm}

			\begin{figure}[H]
			\centering
			\subfloat[Rod format \cite{isispace}]{\includegraphics[page=1,trim={0cm 0cm 0cm 0cm},clip=true,width=85mm]{figurastfm/Chapter3/Imagenes/magnetorquer.png}}
			 \quad
			\subfloat[\glsname{GranaSAT} \acrshort{PCB} design]{\includegraphics[page=1,trim={0cm 0cm 0cm 0cm},clip=true,width=70mm]{figurastfm/Chapter3/PDF/pattern.pdf}}
			\caption{Magnetorquers} \label{magnetor}
			\vspace{-0.5cm}
\end{figure}

\newpage
\subsubsection{Formal Requirements Definition}

\begin{table} [H]
\centering

\begin{tabularx}{\linewidth}{lX}

\multicolumn{1}{c}{\textbf{Ref.}}                      & \multicolumn{1}{c}{\textbf{Formal Requirements}}                    \tabularnewline \specialrule{1.1pt}{1pt}{1pt}
ADCS.FoR.1                                              & Shall have a functional \acrshort{IMU}. \tabularnewline \midrule
ADCS.FoR.2                                              & Shall measure angular speed with an accuracy of at least 1\textdegree/s \tabularnewline \midrule
ADCS.FoR.3                                            & Shall measure gravitational acceleration with an accuracy of at least 0.1 g. \tabularnewline \midrule
ADCS.FoR.4                                                   & Shall measure magnetic field intensity with an accuracy of at least \SI{10}{\micro T}. \tabularnewline \midrule
ADCS.FoR.5                                                   & Shall implement at least a \acrshort{PID} control law.   \tabularnewline \midrule
ADCS.FoR.6                                                   & Shall count at least with a 4 cm diameter Z-axis reaction wheel.  \tabularnewline \midrule
ADCS.FoR.7                                                   & Shall include an adequate DC motor to accelerate the reaction wheel, between 3 V and 6 V supply.\tabularnewline \midrule
ADCS.FoR.8                                                  & Shall allow generating at least 5 independent \acrshort{PWM} signal with a minimum frequency of 10 Hz. \tabularnewline \midrule
ADCS.FoR.9                                                   & Shall include terminals which allow connecting the system to external magnetorquers. \tabularnewline \midrule

\end{tabularx}
\caption{\acrshort{ADCS} - Formal Requirements}
%\vspace{-2cm}
%\label{forgstation}

\end{table}


\subsection{Electrical Power System (EPS)} \label{sec:EPS}

Electrical Power System (\glsname{EPS}) is the subsystem which provides, stores, distributes, and controls spacecraft electrical power. It is the most crucial subsystem, as a lack of energy in orbit would inevitably result in the end of the mission. Some of the most typical top-level \acrshort{EPS} functions are listed below:

\begin{itemize}  [noitemsep,topsep=0pt]

	\item{Supply a continuous source of power during the mission life.} \\
	\item{Control and distribution of electrical power.} \\
	\item{Deal with both average and peak consumption requirements.} \\
	\item{Allow command and telemetry capabilities for \acrshort{EPS} status, remote control, etc.} \\
	\item{Suppress transient voltages or spikes in the bus, which may damage the system.} \\


\end{itemize}

Nowadays, the challenges around space power systems focus on maximizing efficiency and reliability while minimizing mass and costs. In this section, a functional breakdown of a typical spacecraft \acrshort{EPS} will be studied in detail in order to define technical requirements.

Once more, this project will address this part of the system from the double perspective mentioned (\autoref{perspc}). On the other hand, \autoref{fig:epsfunct} depicts the main components and functions of a typical \acrshort{EPS}.

			\begin{figure}[H] 
				\centering
				\includegraphics[width=160mm]{figurastfm/Chapter3/PDF/provisional_power_sub.pdf}
				\caption{Electrical Power System main components and functions}      		
				\label{fig:epsfunct}
  		\end{figure}

\subsubsection{Power Source}

As seen in \autoref{fig:epsfunct} power source is the first of the functions to be covered by \acrshort{EPS}. As every electrical system, spacecrafts will require enough energy in order to function properly. Typically, when designing a real space system different power sources are considered depending on the energy magnitude needed for the mission, which in turn will depend on its expected duration. 

\autoref{sourcesduration} illustrates that point, by plotting together the most usual power sources capabilities along with the expected duration of that technology. Some of the most common power sources depicted are briefly analyzed next, including some with an increasing interest in CubeSats sector in recent times.

			\begin{figure} [h] 
				\centering
				\includegraphics[page=1,trim={0cm 0cm 0cm 0cm},clip=true,width=100mm]{figurastfm/Chapter3/PDF/sourcesduration.pdf}
				\caption{Optimum energy sources for various power levels and mission durations \cite{spacecraftspower}}
				\label{sourcesduration}
			\end{figure}


\paragraph{Radio-isotope thermoelectric generator} 
\label{rtg}


Radio-isotope thermoelectric generators (\acrshort{RTG}) can be considered similar to a battery and they are based on \glsname{seebek}. They are usually used when large voltages are needed in a unmaintained term longer than common batteries or fuel cells (see section \ref{fuelcell}), chemical-based, allow; therefore, they have been used as main power source in a variety of situations, from satellites to simply provide isolated facilities with electricity.

\acrshort{RTG}, as most nuclear processes, make use of \textbf{thermocouples}, a device which is able to transform thermal energy into electrical energy, due to the \glsname{seebek} mentioned before; it makes this technology part of the so-called \textit{static sources}. Two different semiconductors are mutually connected, flowing an electric current when there is a temperature gradient (produced by decay of the radioactive source)  between the p-n junction of individual thermoelectric cells connected in a series-parallel arrangement to provide the desired DC electrical output from each converter. \autoref{thermocoup} shows an example of thermocouples.

			\begin{figure}[H] 
				\centering
				\includegraphics[width=50mm]{figurastfm/Chapter3/Imagenes/thermocouple.jpg}
				\caption{Thercouple \cite{wiki}}      		
				\label{thermocoup}
				\vspace{-2cm}
  		\end{figure}

Normally, plutonium-238 is used, providing 0.54 Watts/g \cite{rtgefec} and typically up to a few kW. \acrshort{RTG} are really inefficient, varying between 3-7 \%, and this technology is gradually falling into disuse. A notable curiosity is the numerous missions which flew to the moon using those systems\cite{rtgefec}. \autoref{apollo} shows a photography of one of these modules deployed on the Moon in Apollo 14 mission.

			\begin{figure}[H] 
				\centering
				\includegraphics[width=80mm]{figurastfm/Chapter3/Imagenes/apollo.jpg}
				\caption{\acrshort{RTG} used in Apollo 14 mission \cite{wiki}}      		
				\label{apollo}
				%\vspace{-2cm}
  		\end{figure}
\paragraph{Nuclear reactor}

Nuclear reactor is an example of \textit{dynamic source}, which uses a heat source and a heat exchanger to drive an engine in a thermodynamic power cycle. It is in charge of initiating a controlled self-sustained chain reaction which results into heat, transferred to a working fluid, which drives an energy-conversion heat engine. They are typically based on uranium-235 or plutonium-238 and one of its main advantages is avoiding thermal energy storage, as the source provides continuous heat. 

With an average efficiency of 35 \% and capability to produce up to a few hundred kW, it has been extensively used by Soviet Union and Russia, in contrast with the US.

\paragraph{Fuel cells} \label{fuelcell}
 
Fuel cells convert the chemical energy of an oxidation reaction to electricity, i.e., they are based on \textbf{REDOX reactions}, normally using hydrogen as fuel and oxygen as oxidizing agent \cite{wiki}. Contrary to solar cells (see section \ref{solarcellsexpl}) they can operate without sunlight, but must carry their own reactant supply, which allows them to produce electricity continuously for as long as fuel and oxygen are supplied.


			\begin{figure}[H] 
				\centering
				\includegraphics[width=85mm]{figurastfm/Chapter3/PDF/fuel_cell.pdf}
				\caption{Fuel cell chemical basis\cite{wiki}}      		
				\vspace{-0.5cm}
  		\end{figure}
			
			
A typical single cell produces a voltage of 0.8 VDC, however, when combined, a fuel cell unit is able to generate up to tens of kilowatts of power with efficiencies as high as 80 \%, with an average of 50-60 \% for larger currents, still way higher than the rest of the power sources analyzed. 

\paragraph{Solar Photovoltaic energy and Photoelectric effect} \label{solarcellsexpl}

Despite of the variety of power sources commonly used in spacecrafts, so far the vast majority of CubeSats missions (although there are proposals pointing to another systems, see \cite{glenn} or \cite{bristol}) have utilized \textbf{Photovoltaic energy}, mainly because of size and weight constraints, as well as a lower energy requirements than bigger spacecrafts which use any of the others power sources possibilities. Therefore, with the aim of keeping this simulation platform as real as possible, it will be the one chosen.  All the more reason, and contrary to the others, in this phase of the Engineering Design Process this technology is studied in depth in this section so the main drawbacks and constraints are outlined, in order to achieve more accurate formal requirements.	


Solar Photovoltaic energy functioning is based on \textbf{Photoelectric effect}. It was first described by Max Planck in 1900 and later in 1905 Albert Einstein went in depth, for which he was granted with the Nobel Prize. Basically, Photoelectric effect consists in the emission of electrons when light hits a material, with a variable kinetic energy depending on light frequency. The modern model which explains light behaviour was beyond the traditional conception of it as a wave, and proposed that light sometimes behaves as particles of electromagnetic energy called nowadays \textbf{photons}. According to Plank's equation:

	\begin{equation} \label{eq:plank}
			E_{\text{photon}}=\hslash v
	\end{equation}
	
In \autoref{eq:plank} $E_{\text{photon}}$ is the energy of a photon, $\hslash$ is the Planck's constant and $v$ is the frequency of the light in \SI{}{\hertz}. According to it, the energy of a photon is proportional to the frequency of the light \cite{photoelec}. Experimental tests show that if the incident light had a frequency lower than a certain frequency $v_{0}$ or \textbf{threshold frequency} no electrons were ejected regardless of the light amplitude, while for frequencies greater for $v_{0}$, they were. That value depends on the metal; \autoref{photoeffect} illustrates this point.


			\begin{figure}[H] 
				\centering
				\includegraphics[width=145mm]{figurastfm/Chapter3/Imagenes/photovolt.png}
				\caption{Threshold frequency in Photoelectric effect \cite{wiki}}      		
				\label{photoeffect}
  		\end{figure}

In the example of \autoref{photoeffect}, the red light frequency is lower than threshold frequency of the metal, so no electrons are ejected. On the other hand, green and blue lights do produce this ejection, given its higher frequency when compared to $v_{0}$. Besides, the higher energy of the blue light makes electrons be ejected with a higher \textbf{kinetic energy} than green does.

As it will be seen next, and also during \autoref{chap:chapter4}, this phenomena has a capital importance in functioning and performance of \textbf{solar cells}.

To conclude this subsection, \autoref{comptech} depicts a duration comparison of the different technologies analyzed.


			\begin{figure}[H] 
				\centering
				\includegraphics[width=105mm]{figurastfm/Chapter3/PDF/comp_tech.pdf}
				\caption{Duration comparison depending on technology \cite{mit}}      		
				\label{comptech}
				\vspace{-0.5cm}
  		\end{figure}
		

\subparagraph{Solar cells} \label{scellssect}


Solar cells are electrical devices that convert the energy of light directly into electricity due to the Photovoltaic effect previously described. They are composed of several P/N junctions monolithically connected in series. Briefly, in the n-type layer, there is an excess of electrons, and in the p-type layer, there is an excess of positively charged holes (which are vacancies due to the lack of valence electrons). Near the junction of the two layers, the electrons on one side of the junction (n-type layer) move into the holes on the other side of the junction (p-type layer). This creates an area around the junction, called the \textbf{depletion region}, in which the electrons fill the holes. \autoref{solarcells} illustrates this phenomena.

	\begin{figure}[h] 
				\centering
				\includegraphics[width=140mm]{figurastfm/Chapter3/Imagenes/solarcellphenom.jpg}
				\caption{Solar cell functioning \cite{solarcell}}      		
				\label{solarcells}
				%\vspace{-0.5cm}
  		\end{figure}

When all the holes are filled with electrons in the depletion zone, the p-type side of the depletion zone (where holes were initially present) now contains negatively charged ions, and the n-type side of the depletion zone (where electrons were present) now contains positively charged ions. The presence of these oppositely charged ions creates an internal electric field that prevents electrons in the n-type layer to fill holes in the p-type layer.

When light strikes a solar cell, electrons in the silicon are ejected, which results in the formation of 'holes' (the vacancies left behind by the escaping electrons). If this happens in the electric field, the field will move electrons to the n-type layer and holes to the p-type layer. When n-type and p-type layers are electrically connected the electrons travel from the n-type layer to the p-type layer by crossing the depletion zone and then go through the connection back to the n-type layer, creating a flow of electricity \cite{solarcell}.

			Although that is the basic behaviour, performance of solar cells varies depending on its type and material; the first noteworthy distinction is \textbf{Monocrystalline} and \textbf{Polycrystalline} Solar cells. While the first of them are composed of a single crystal which is fit into a solar panel, the second are made pouring the material into molds where it cools and solidifies. Monocrystalline are easily recognizable because of their darker colour and chopped off corners. They have a longer life compared to polycrystalline, which are also less space efficient than mono. \autoref{monopoly} shows an example of each of these solar cells.
			
			\begin{figure}[H]
			\centering
			\subfloat[Monocrystalline Solar Panel]{\includegraphics[page=1,trim={0cm 0cm 0cm 0cm},clip=true,width=50mm]{figurastfm/Chapter3/Imagenes/mono.png}}
			 \quad
			\subfloat[Polycrystalline Solar Panel]{\includegraphics[page=1,trim={0cm 0cm 0cm 0cm},clip=true,width=50mm]{figurastfm/Chapter3/Imagenes/poly.png}}
			\caption{Visual aspect solar panels comparison \cite{monopolyimg}} \label{monopoly}
			\vspace{-2cm}
\end{figure}
			
			
			Another important division is considered between \textbf{single-junction} and \textbf{multi-junction} solar cells, depending on the use of one or more materials for the p-n~junctions; the latter are composed of two or more junctions in layers one on top of each other, which allows to maximize the spectral range from which energy can be collected. This point is relevant because of the frequency dependency stated before, see \autoref{eq:plank}. \autoref{triplejunction} graphically illustrates the advantages of this technique: by using different alloys with \textbf{Gallium}, the spectral range is increased and so does efficiency. This figure is also useful to compare the spectral response of this triple-junction solar cell with Sun spectrum, by plotting AM1.5 spectrum; the Sun as irradiance source will be analyzed in \ref{irradsources}. Besides the ones mentioned, there are several more ways to classify solar cells (according to their generation, biohybrids, among others) but they are not relevant for this project.
			
				\begin{figure} [H] 
				\centering
				\includegraphics[page=1,trim={0cm 0cm 0cm 0cm},clip=true,width=120mm]{figurastfm/Chapter3/PDF/triplejunction.pdf}
				\caption{Structure of a GaInP-GaInAs-Ge solar cell. Spectral range covered \cite{phandphoto}}
				\label{triplejunction}
				\vspace{-1cm}
			\end{figure}
			
						
			
			Finally, regarding manufacturing materials, while single-junction \textbf{silicon} with efficiencies about 24.7 \% \cite{phandphoto} are the most used, later techniques make use of materials such as \textbf{GaAs}, which due to its higher light absorption coefficient is much more efficient than silicon, specially when using multi-junction panels. However, they also have higher cost, which makes them appropriate only when high efficiency is needed, e.g. in space applications \cite{wiki}. Particularly, \textbf{GaAs triple-junction} solar cells mentioned before are widely used in that sector, reaching efficiencies up to 30 \% \cite{azurspace}. Of course, those efficiencies are related to high quality solar panels, considerably diminishing when low cost manufacturing techniques are used (see IB-3 \cite{phandphoto}). 
 			 
			As a reference, \autoref{nrel} shows the historical evolution of solar cells efficiency depending on the materials used, issued by the \textbf{National Renewable Energy Laboratory} in April 2019. It is worth mentioning the higher efficiencies accomplished under laboratory conditions (up to \textbf{46 \%} using four-junction cells) when compared with the ones commercially available.  For a more realistic perspective of the solar panels market nowadays, see \autoref{efficienciesreal} \cite{phandphoto}; it also shows some parameters which will be introduced later. Intensive research and the advance in techniques and materials assures a favourable future for photovoltaic technology. 	
						\afterpage{
\begin{landscape}
\begin{figure}[H]
			\centering
			\includegraphics[page=1,trim={0cm 0cm 0cm 0cm},clip=true,width=230mm]{figurastfm/Chapter3/PDF/nrel.pdf}
			\caption{Best Research-Cell Efficiencies. NREL. \cite{nrelpage}} \label{nrel}
\end{figure}
\end{landscape}
}
			
			\begin{figure}[H]
			\centering
			\includegraphics[page=1,trim={0cm 0cm 0cm 0cm},clip=true,width=130mm]{figurastfm/Chapter3/PDF/solarcell_efficiencies.pdf}
			\caption{Typical optimum performance with the most common materials \cite{phandphoto}} \label{efficienciesreal}
\end{figure}
			

			Regardless of materials and manufacturing process, from a modelling point a view, a solar cell can be initially considered as a current source in parallel with a couple of diodes and a pair of resistors, as shown in \autoref{circuitmodelsolarcell}.
			
						
				\begin{figure} [H] 
				\centering
				\includegraphics[page=1,trim={0cm 0cm 0cm 0cm},clip=true,width=160mm]{figurastfm/Chapter3/PDF/solarcellcircuit.pdf}
				\caption{Solar cell circuit model} 		
				\label{circuitmodelsolarcell}
			\end{figure}
			
			Photons in striking light are absorbed by a semiconducting matrial, such as silicon, producing a current proportional to the amount of illumination, and adjusting the output voltage as necessary to provide that current. This is by definition the behaviour of a constant current source, so in \autoref{circuitmodelsolarcell} is represented by the current source ($I_{\text{Light}}$), which is diminished by parasitics effects given by diodes and resistances ($I_{\text{o1}},I_{\text{o2}},I_{\text{Sh}}$), which will be studied later. Let us analyze the circuit; by inspection, the current flowing out ($I_{\text{out}}$) is given by:
			
			\begin{equation} \label{eq:solarcelleqshort}
				I_{\text{out}}=I_{\text{Light}}-I_{\text{o1}}-I_{\text{o2}}-I_{\text{Sh}}
			\end{equation}
			
			Current through diodes can be substituted by \textbf{Shockley diode equation}, while $I_{\text{Sh}}$ is found using Ohm's law. Therefore, equation \autoref{eq:solarcelleqshort} yields to \autoref{eq:solarcelleq}.
			
			\begin{equation} \label{eq:solarcelleq}
		I_{\text{out}}=I_{\text{Light}}-I_{\text{o1}}(e^{qV_{\text{out}}/kT}-1)-I_{\text{o2}}(e^{qV_{\text{out}}/2kT}-1)-\frac{V_{\text{out}}+I_{\text{out}}R_{\text{S}}}{{R_{\text{Sh}}}}
		\end{equation}
			
			On the other hand, when the solar cell is short-circuited:
			
				\begin{equation} \label{ilightisc}
					V_{\text{out}}=0 \Rightarrow I_{\text{out}} \approx I_{\text{Light}}=I_{\text{SC}}
				\end{equation}
			
						
			So finally, there are four currents: $I_{\text{SC}}$ is the \textbf{short-circuit current}, $I_{\text{o1}}$ and  $I_{\text{o2}}$ are \textbf{dark saturation currents}, and $I_{\text{Sh}}$  is the \textbf{shunt current} due to ohmic losses. While the latter, as said before, is given by Ohm's law, the three first are analitycally given by complex expressions which go beyond the scope of this work (detailed demonstrations can be found at \cite{photovoltaichandbook}); however, they are not needed in order to understand the basic functioning of a solar cell operation. Briefly, the short-circuit current is the sum of the contributions from each of the three regions: \textbf{the n-type region, the depletion region, and the p-type region}. On the other hand, the dark saturation current is the current generated due to recombination in the quasi-neutral regions (apparently neutral regions where electric field is zero). Hence, $D_1$, in parallel, represents the recombination current in the quasi-neutral regions ($ I_{\text{o1}} \propto e^{qV/kT} $), while $D_2$ represents recombination current in the depletion region ($ I_{\text{o2}}~\propto~e^{qV/2kT} $); a common and reasonable assumption is to ignore the dark current due to the depletion region $ I_{\text{o2}} $. \autoref{ivmodel} shows at the front a graphical representation of \autoref{eq:solarcelleq} for typical values, while at the background the corresponding Power-Voltage curve.

\begin{figure} [h] 				
				\centering
				\includegraphics[page=1,trim={0cm 0cm 0cm 0cm},clip=true,width=120mm]{figurastfm/Chapter3/PDF/IV_model.pdf}
				\caption{Typical IV curve for standard solar cell} \label{ivmodel}
			\end{figure}
			
			At small applied voltages, diodes currents ($ I_{\text{o1}} $ and $  I_{\text{o2}}  $) are negligible and the solar cell behaves as a constant current source with an output current equivalent to the short-circuit current, $ I_{\text{SC}} $, as stated in \autoref{ilightisc}. When the applied voltage is high enough so that diodes currents (recombination current) become significant, the solar cell current drops quickly.
			
			Finally, shunt current is due to ohmic losses, modelled with a couple of resistors with a varying influence: as can also be seen in \autoref{eq:solarcelleq}, shunt resistance $R_{\text{Sh}}$ has no effect on the short-circuit current, but reduces the open-circuit voltage, $V_{\text{OC}}$. Conversely, the series resistance $R_{\text{S}}$ has no effect on the open-circuit voltage, but reduces the short-circuit current, $I_{\text{SC}}$; sources of series resistance include the metal contacts, particularly the front grid, and the transverse flow of current in the solar cell emitter to the front grid. This phenomena is illustrated in \autoref{rshuntseriescomp} which shows the behaviour of a solar cell depending on parasitics resistances; in\autoref{noseries}, shunt resistance value is varied assuming that series resistance is zero, while in\autoref{noshunt}, it is performed an analysis the other way around. Indeed, it is immediate to see how shunt resistances shift open-voltage circuit with respect to no shunt resistor, while series resistances shift short-circuit current. Notice that some extreme values have been used to illustrate the phenomena. 
			
					
			\begin{figure}
					\centering		
					\subfloat[Depending on $R_{Sh}$ $(R_{series} = 0)$\label{noseries}]{\includegraphics[page=1,trim={0cm 0cm 0cm 0cm},clip=true,width=135mm]{figurastfm/Chapter3/PDF/IV_Curves_noseries.pdf}} 
							\\[3ex]
					\subfloat[Depending on $R_{series}$ $(R_{Sh}\rightarrow\infty)$\label{noshunt}]{\includegraphics[page=1,trim={0cm 0cm 0cm 0cm},clip=true,width=135mm]{figurastfm/Chapter3/PDF/IV_Curves_noshunt.pdf}}	
					\caption{I-V Curves analysis with parasitics effects} \label{rshuntseriescomp}
			\end{figure}
			
			
 Going back to \autoref{ivmodel}, it illustrates several important \textbf{figures of merit} for solar cells, which help to understand its behaviour, some of which have already been tackled.

\begin{itemize}

	\item{\textbf{Open-circuit voltage ($V_{\text{OC}}$):} voltage $V_{\text{out}}$ across the output terminals when the cell is operated at open circuit. It is not possible to extract any power from the cell at this point.}

	\item{\textbf{Short-circuit current ($ I_{\text{SC}}$):} current $ I_{\text{out}} $ at the output when the cell is operated at short-circuit. It is not possible to extract any power from the cell at this point.}
	
	\item{\textbf{Maximum power point ($MPP (V_{\text{MP}},I_{\text{MP}})$):} point on the I-V curve where the power produced is at a maximum. For any given set of operational conditions, cells have a single operating point where the values of the current and voltage of the cell result in a maximum power output. These values correspond to a particular load resistance given by Ohm's law. This point defines a rectangle whose area is the largest for any point on the I-V curve, given by $P_{\text{MP}}=V_{\text{MP}}I_{\text{MP}}$.} The maximum power point is found by solving: \cite{photovoltaichandbook}
	

			\begin{equation} 
				\frac{\partial P}{\partial V}\Bigr|_{\substack{V=V_{\text{MP}}}}=\frac{\partial (IV)}{\partial V}\Bigr|_{\substack{V=V_{\text{MP}}}}=\Big[ I+V\frac{\partial I}{\partial V}\Big]\Bigr|_{\substack{V=V_{\text{MP}}}}=0
			\label{mppequation}
			\end{equation}
			
			The current at the maximum power point, $I_{\text{MP}}$, is then found by evaluating \autoref{eq:solarcelleq}

	\item{\textbf{Fill factor ($FF$):} measure of the \textit{squareness} of the I-V curve, given by the ratio of the areas of the two rectangles shown in figure \autoref{ivmodel}, always lower than one. Mathematically:}
	
		\begin{equation} \label{fillfactorequation}
			FF=\frac{V_{\text{MP}}I_{\text{MP}}}{V_{\text{OC}}I_{\text{SC}}}=\frac{P_{\text{MP}}}{V_{\text{OC}}I_{\text{SC}}}
	\end{equation}
	
	\item{\textbf{Packing Density:}  refers to the area of the panel which is covered with solar cells compared to that which is empty. It affects the output power of the module as well as its operating temperature; it mainly depends on the shape (round, squared...) of the solar cells used. It is usually taken to be about 0.8.	}
	
	\item{\textbf{Power conversion efficiency $(\eta)$:} relates the power obtained at the maximum power point with the incident power $P_{\text{in}}$, where the incident power is determined by the properties of the light spectrum incident upon the solar cell}:
	
	\begin{equation} 
		\eta=\frac{P_{\text{MP}}}{P_{\text{in}}}=\frac{V_{\text{MP}}I_{\text{MP}}}{P_{\text{in}}} \label{efficencysolarcell}
	\end{equation}

\end{itemize}

Obviously, the goal of a system getting energy from a solar cell, will be operating at \textbf{Maximum Power Point}. Apparently, it should not a problem: as illustrated in \autoref{ivmodel}, every solar cell will have a certain point in which power is maximized, according to \autoref{mppequation}; it may be higher or lower depending on the quality of the solar cell (i.e, parasitic effects importance, as seen before), but it is just a matter of finding it. The problem is, reality is not that simple; in a real scenario, there is a limit to how much voltage a certain panel can produce, i.e., it will act as a constant current source only as far as the connected load allows a voltage below that maximum. Now, let us suppose the illumination changes: according to the stated, for a given connected load, output current will vary, so does voltage; if for that load, the generated current tries to impose a certain voltage \textbf{above} the maximum the panel is able to produce, the \textbf{knee} of the curve is reached and generated power will turn unstable (see \autoref{ivmodel}).

 Using the concepts exposed, the latter means that \textit{MPP} will move constantly through the I-V curve, particularly depending on two factors: \textbf{irradiance} and \textbf{temperature}. While the first of them is obvious (as stated before, the more lightning striking the solar cell, the more current), the second is due to the temperature dependence of the intrinsic carrier concentration. It can be demonstrated (\cite{photovoltaichandbook}) that temperature rising yields to an increase of it, which then increases the \textbf{dark saturation currents}. Because open-circuit voltage is proportional to the reverse of them, it will decrease. This is usually \textbf{the most important performance loss.} \autoref{tempeffectimg} shows graphically this degradation in performance.

\begin{figure} [H] 				
				\centering
				\includegraphics[page=1,trim={0cm 0cm 0cm 0cm},clip=true,width=100mm]{figurastfm/Chapter3/PDF/temp_effects.pdf}
				\caption{Standard I-V curve depending on extreme temperatures \cite{spacecraftspower}} \label{tempeffectimg}
			\end{figure}

	It is immediate to see the degradation mentioned, which must be adequately taken into account when dimensioning a photovoltaic system. On the other hand, it is worth to mention the efficiency increase of a solar cell when exposed to extremely low temperatures; indeed, it is coherent with theoretical behaviour explained before: the lower the temperature, the lower dark saturation currents will be, and with them, the larger open-circuit voltage, as precisely shows \autoref{tempeffectimg}.

Another significant reason for performance degradation is the point in cell's life, that is, generated power is expected to be lower at its \textbf{End Of Life (\acrshort{EOL})} than at its \textbf{Beginning~Of~Life (\acrshort{BOL})}, as shown in \autoref{eoleffects}.

\begin{figure} [H] 				
				\centering
				\includegraphics[page=1,trim={0cm 0cm 0cm 0cm},clip=true,width=90mm]{figurastfm/Chapter3/PDF/eol_effect.pdf}
				\caption{I-V curves with cell's life \cite{spacecraftspower}} \label{eoleffects}
			\end{figure}


Finally, the last noteworthy reason of performance decrease is \textbf{radiation}. The impinging particles produce defects in the crystalline structure of the PV cells. The resulting defects degrade the voltage and current outputs of the cell. Low-energy particles create damage close to the surface, and therefore lower the open circuit voltage. On the other hand, high-energy particles penetrate deeper in the base and lower the lifetime of electron hole pairs, thus decreasing the short circuit current. \autoref{doses} plots an I-V typical curve affected by different radiation doses accumulated during its life service.

\begin{figure} [H] 				
				\centering
				\includegraphics[page=1,trim={0cm 0cm 0cm 0cm},clip=true,width=135mm]{figurastfm/Chapter3/PDF/ivradiation.pdf}
				\caption{I-V Curve degradation with radiation doses \cite{spacecraftspower}} \label{doses}
			\end{figure}

The power generation capability continues degrading as the radiation dose accumulates over the years \cite{spacecraftspower}. \autoref{lossesyears} depicts the consequences of radiation, showing that, although it oscillates, power generation have a decreasing tendency with time. 

\begin{figure} [H] 				
				\centering
				\includegraphics[page=1,trim={0cm 0cm 0cm 0cm},clip=true,width=100mm]{figurastfm/Chapter3/PDF/radiationyears.pdf}
				\caption{Losses throughout the years \cite{spacecraftspower}} \label{lossesyears}
				\vspace{-1cm}
			\end{figure}



	%\begin{figure}
					%\centering		
					%\subfloat[ \label{}]{\includegraphics[page=1,trim={0cm 0cm 0cm 0cm},clip=true,width=100mm]{figurastfm/Chapter3/PDF/radiationyears.pdf}} 
							%\\[3ex]
					%\subfloat[\label{}]{\includegraphics[page=1,trim={0cm 0cm 0cm 0cm},clip=true,width=135mm]{figurastfm/Chapter3/PDF/ivradiation.pdf}}	
					%\caption{Performance degradation due to radiation \cite{spacecraftspower}} \label{radiationeffects}
			%\end{figure}

	
			To finalize this comprehensive analysis of solar cells, there is another vital point in solar cells response, apart from the systems to optimize delivered power, intrinsic efficiency of the solar panels and so forth; this is \textbf{spectral response}. Indeed, solar cells behave differently depending on manufacturing materials, given that each one may cover different sections of the \textbf{spectrum}, affecting to efficiency and extracted power, as mentioned before. \autoref{negrous} shows spectral response of a typical solar cell manufactured with the most usual materials nowadays, some of which have been mentioned before. I must personally express my gratitude to Dr. Hamadani, from the \textbf{National Institute of Standards and Technology} in the US, for providing me with the data shown; additional information on that study can be found on \cite{hamadani}.
			
						
	\begin{figure} [h] 				
				\centering
				\includegraphics[page=1,trim={0cm 0cm 0cm 0cm},clip=true,width=120mm]{figurastfm/Chapter3/PDF/spect_material.pdf}
				\caption{Spectral response depending on solar cells material} \label{negrous}
				%\vspace{-2cm}
			\end{figure}			
			
				As stated in the legend, solar cells made of different materials are characterized: average silicon Si, high efficiency silicon, gallium arsenide GaAs, cadmium telluride CdTe, copper indium gallium selenide CIGS and organic solar cell. Although some of them have not been mentioned in this Master's Thesis because of its low utilization in space sector, they are included as general reference for photovoltaic technology. As plotted, while the organic solar cell presents the shorter range \textbf{350-630 nm} the solar cell manufactured with high efficiency silicon expands its spectral response between the \textbf{350-1150 nm} range, clearly the best out of the analyzed. It allows us to conclude that even the material which shows the best spectral response by its own, behaves worse than multi-junction architectures which in the end cover a larger portion of the spectrum.
			


	\subparagraph{Irradiance sources} \label{irradsources}
	
	As explained in previous section, spectral response of solar cells depends on the manufacturing material. In the same way, for a certain material, maximum efficiency will be accomplished only when the \textbf{whole spectral range of the solar cell} is covered by the irradiation source. This is usually not a problem when dealing with Sun spectrum (see AM1.5 series in \autoref{triplejunction}) as current technology allows a reasonable covering of it; however, it may not satisfy the simulation platform under development in this project, as the Sun is not likely to be used as irradiation source indoors. Therefore, it is worth considering whether solar cells behaviour will be the same when excited with a non-solar illumination source. Through this section, different irradiance sources, including the Sun, will be analyzed and compared with typical solar cells expected response.
	
			 \textbullet\ \textbf{Sun}

		The Sun is, for obvious reasons, the most common irradiance source when dealing with photovoltaic energy. Solar irradiance is the energy emitted by the Sun as a result of its nuclear fusion reactions in the range 250-2500 nm. The Sun radiates approximately as a \glsname{blackbody} at an effective temperature of 6000 K, with a total amount of energy \textbf{above the atmosphere} of about \SI[separate-uncertainty = true]{1366}{W/m^2 \pm{} \num{6.9}\percent} depending on the varying distance from the sun. That irradiance is distributed along a wide portion of the spectrum, including non-visible. However, a significant portion of solar radiation incident on the atmosphere is not received at the ground, being absorbed by atmospheric constituents such as \ce{H2O} vapour, \ce{CO2}, \ce{O3}, and \ce{O2}, and resulting on an average irradiance \textbf{at the Earth's surface} of about \SI{1120}{ W/m^2} \cite{sunspectral}. \autoref{spectrumsolaraltura} illustrates Sun's spectral behavior depending on the measurement point. 
		
		\begin{figure} [H] 				
				\centering
				\includegraphics[page=1,trim={0cm 0cm 0cm 0cm},clip=true,width=120mm]{figurastfm/Chapter3/PDF/Solar_spectrum_en.pdf}
				\caption{Solar spectrum above atmosphere and at surface \cite{wiki}} \label{spectrumsolaraltura}
			\end{figure}
			
			As stated, the terrestrial spectrum varies, so in order to count with a standard spectrum as representative as possible, it is commonly taken so-called \textbf{\acrshort{AM}1.5 spectrum}. Instead of a particular measured spectrum, it is calculated from the reference AM0 spectrum under representative geometric and atmospheric conditions. \acrshort{AM} stands for \textit{Air mass}; given that both absorption and scattering depend rather strongly on the path length of sunlight through the atmosphere, i.e. on the Sun’s elevation angle above the horizon. When the sun comes closer to the horizon, its light passes through more air. \acrshort{AM}1.5 refers to an spectrum taken with an air mass of 1.5 (Sun 41\textdegree{} above the horizon) with atmospheric conditions from the US standard atmosphere, representative for most middle latitudes, among others \cite{pvlight}. It will be used extensively during this project.
			
			In the same topic, solar irradiance is usually measured using devices called \textbf{pyranometers}; they translate incident irradiation into a voltage, which can be later mapped into irradiation units. Briefly, there are three different types: \textbf{thermopile pyranometers}, adequate for high accuracy, exhibiting flat spectral response on large range between 300-3000 nm; they are based on the thermocouple principle already mentioned on \ref{rtg}. Secondly, \textbf{Silicon-based pyranometers} also known as photodiode-based or photovoltaic are a more cost-effective option, with lower accuracy specially under cloudy conditions. Its main drawback is their limited spectral range, between 350-1100 nm. They are based on Photoelectric effect, also studied before in \ref{solarcellsexpl}. \autoref{comparpyranom} compares \acrshort{AM}1.5 reference with typical spectral response of each technology. 
			
				\begin{figure} [H] 				
				\centering
				\includegraphics[page=1,trim={0cm 0cm 0cm 0cm},clip=true,width=120mm]{figurastfm/Chapter3/Imagenes/solarpyranom.png}
				\caption{Solar spectrum \acrshort{AM}1.5 reference and pyranometers expected spectral range.} \label{comparpyranom}
			\end{figure}
			
				As mentioned, thermopile pyranometers exhibit a completely flat response, much wider than the rest of technologies.
			
		\textbullet\ \textbf{Xenon Arc Sun Simulator}
	

			Considering the simulation parcel of this project, it is needed to count with irradiance sources usable indoors. One of the most typically used are high power xenon lamps, as sun simulators. They are gas-discharge-based lamps which produce light when passing electricity through ionized xenon gas at high pressure \cite{wiki}. Although it depends on the lamp, life duration and some additional factors, xenon arc lamp may match closely solar spectrum, as desired. \autoref{xenonspect} illustrates the typical spectral behaviour of this kind of lamps, compared with \acrshort{AM}1.5 standard reference again; it shows a pretty matched spectrum considering \acrshort{AM}1.5 standard which confirms this technology as a really good candidate to simulate the Sun; however, as it will be analyzed later, that matching accuracy will depend on several factors (life of the lamp, quality, filters...) which typically will make it difficult to get such a good result.
			
				\begin{figure} [H] 				
				\centering
				\includegraphics[page=1,trim={0cm 0cm 0cm 0cm},clip=true,width=120mm]{figurastfm/Chapter3/Imagenes/xenon_sun.png}
				\caption{Xenon arc lamp typical spectrum compared with \acrshort{AM}1.5 reference} \label{xenonspect}
			\end{figure}
	
An example can be found at the Aerospace Engineering School of the University of Catalu\textbf{ñ}a \cite{catalan},  where it was characterized one of these simulators of the brand \textbf{Anmingli}, shown in figure\autoref{canoncat}, able to deliver up to 4 kW. Figure\autoref{xenondetail} shows the detail of the xenon arc lamp.
	
			\begin{figure}[h]
			\centering
			\subfloat[General view\label{canoncat}]{\includegraphics[page=1,trim={0cm 0cm 0cm 0cm},clip=true,width=83mm]{figurastfm/Chapter3/PDF/canon_cat.pdf}}
			 \quad
			\subfloat[Xenon lamp detail \label{xenondetail}]{\includegraphics[page=1,trim={0cm 0cm 0cm 0cm},clip=true,width=65mm]{figurastfm/Chapter3/PDF/bomb_cat.pdf}}
			\caption{Xenon arc lamp Sun simulator used \cite{catalan}} 
\end{figure}
			
			
		Using a silicon-based pyranometer, they characterized lightning distribution of the simulator. Based on that data the plot has been rebuilt and it is shown in \autoref{catorigin}. Despite the fact that the mentioned work only considered interpolated plot, original measurements have also been represented using the data available, i.e., while \autoref{catorigin} shows discrete measurements \textit{as are}, being difficult to interpret the distribution, \autoref{catinterp} shows an interpolated version of the data, which helps to understand in a more meaningful way how the simulator beam behaves. Indeed, interpolated plot \autoref{catinterp} allows determining that the beam is off-centred to the left, reaching a maximum irradiation on it of \SI{1500}{W/m^2} at a distance of 2.7~m. This is a usual problem with these kind of low-cost simulators, designed to work at larger distances, normally related to suboptimal lens design. This fact should be taken into account when used to characterize solar cells, trying to get maximum irradiation from the simulator. In the same way, it is usually a good practice to work with an irradiation value as close as possible to the expected in a real scenario; assuming \acrshort{AM}1.5 reference again, as stated before, a reasonable value to be expected at the surface is \SI{1120}{W/m^2}. In order to get that irradiation value, the distance must be increased in a factor given by \autoref{eqlight}:
	
	
	\begin{equation} 
		\frac{I_{1}}{I_{2}}=\frac{d^{2}_{2}}{d^{2}_{1}} \rightarrow d_{2} = \sqrt{d_{1}\frac{I_{1}}{I_{2}}}\label{eqlight}
	\end{equation}

	
	Effectively, recalling that the intensity of light is inversely proportional to the square of the distance, standard \acrshort{AM}1.5 irradiation value can be reached at \textbf{3.13 m}.
	
	In addition, in the same work \cite{catalan} spectral response was measured and compared with both, Sun's and a high quality simulator of the brand \textbf{MKS}, see \cite{oriel}. It allows them to conclude that their low-cost simulator matches close enough not only the more expensive one, but also Sun's spectrum. More information on this can be found on the cited work. 
	
	\afterpage{
\begin{landscape}
\vspace*{\fill}
\begin{figure}
			\centering
			\includegraphics[page=1,trim={0cm 0cm 0cm 0cm},clip=true,width=205mm]{figurastfm/Chapter3/PDF/original_catalan_memoria.pdf}
			\caption{Original Lightning distribution. Rebuilt from \cite{catalan}} \label{catorigin}
\end{figure}
\vfill
\end{landscape}
}
	
	
\afterpage{
\begin{landscape}
\vspace*{\fill}
\begin{figure}
			\centering
			\includegraphics[page=1,trim={0cm 0cm 0cm 0cm},clip=true,width=205mm]{figurastfm/Chapter3/PDF/interpolated_cat_2_7.pdf}
			\caption{Interpolated Lightning distribution. Rebuilt from \cite{catalan}} \label{catinterp}
\end{figure}
\vfill
\end{landscape}
}
	
	\newpage
	\begin{itemize} [noitemsep,topsep=0pt]
			 \item \textbf{LED Sun Simulator}
	\end{itemize}

	\acrshort{LED}-based lamps are another possibility to simulate the Sun. They are usually less similar to the Sun's spectrum but they are more power-efficient, which may make them appropriate to be used in a simulation platform like the one under development in this project. Additionally they are easier to buy and are available in smaller packages than xenon. Figure\autoref{classic} shows an example of a classic \acrshort{LED} bulb while\autoref{thbonded} depicts a \acrshort{COB} \acrshort{LED} thermally bonded. \acrshort{LED} lamps are based on \textbf{light-emitting diodes}. \acrshort{LED} lamps run on \acrshort{DC} with voltages way lower than usual \acrshort{AC} which entails the need of some electronics to convert usual \acrshort{AC} supply into an appropriate one for the \acrshort{LED}; these circuits are usually known as \textbf{\acrshort{LED} drivers}. Another important issue related to this technology is thermal management: high temperatures can lead to failure and output light reduction, so typically this kind of lamps also include heat dissipation elements or cooling fans \cite{wiki}. 
	
				\begin{figure}[H]
			\centering
			\subfloat[Classic bulb\label{classic}]{\includegraphics[page=1,trim={0cm 0cm 0cm 0cm},clip=true,width=65mm]{figurastfm/Chapter3/Imagenes/led1.jpg}}
			 \quad
			\subfloat[Thermally-bonded\label{thbonded}]{\includegraphics[page=1,trim={0cm 0cm 0cm 0cm},clip=true,width=78mm]{figurastfm/Chapter3/Imagenes/led2.jpeg}}
			\caption{\acrshort{LED} \cite{wiki}} 
		\vspace{-0.5cm}
\end{figure}	
	
	Regarding spectral behaviour and \acrshort{AM}1.5 comparison, \autoref{ledspec} shows both as well as another example of xenon arc lamp spectral response, which makes it possible to compare them at the same time.
	
			\begin{figure} [H] 				
				\centering
				\includegraphics[page=1,trim={0cm 0cm 0cm 0cm},clip=true,width=140mm]{figurastfm/Chapter3/PDF/led_Xenon_am_spec.pdf}
				\caption{\acrshort{LED} and xenon spectrum compared with \acrshort{AM}1.5 reference \cite{xenonled}} \label{ledspec}
							\vspace{-2cm}

			\end{figure}
			
			
			As previously mentioned, \acrshort{LED} spectral match differs from Sun's in a more significative way than xenon does. However, once more, this point will vary depending on each lamp and its conditions and should be adequately characterized before going into production stage. 
			
			Some research has been previously performed, studying the possibility to use a \acrshort{LED} solar simulator to characterized solar cells, for example \cite{indios} and \cite{xenonled}. The first of them concludes that their \acrshort{LED}-based simulator spectrum matches closely \acrshort{AM}1.5 for the visible spectral range. On the other hand, the second one characterizes I-V curves of certain solar cells using both \acrshort{LED} and xenon simulators; although spectral match was not perfect in any case (see \autoref{ledspec}), the author concludes that it did not affect the gross properties of the curve. Actually, he finishes featuring that \acrshort{LED} simulator behaves better than xenon, given that there were no discernable differences in the I-V response from a number of solar cell, maybe due to the extra radiation generated by the latter and therefore additional heat which, as seen in \ref{solarcellsexpl}, shifts the open circuit voltage of the I-V curve. Nevertheless, energy losses at certain parts of the spectrum are also mentioned (around 700 nm).


\subsubsection{Power Regulation and Control}

As deducible from past sections, spacecrafts usually carry an array of photovoltaic cells (i.e., a solar panel) which powers the load and supplies the whole system. Given the volatility of the power source in a real scenario (e.g., an eclipse) it is not hard to understand that energy must be somewhat stored so the system can keep working properly when the power source is not available. Regarding the simulation platform also under development, this point should also be considered to keep its accuracy. This need will be deeply analyzed in \ref{energystr} and in few words, it is satisfied using \textbf{batteries}, as shown in \autoref{fig:epsfunct} too.  

Therefore, during eclipse the battery will be in charge of powering the load. If the solar panel, the battery and the load were operated at the same constant voltage, there would be no need for any kind of power regulation. Nevertheless, as detailed in \ref{scellssect}, the solar panel output voltage is higher at the beginning of life (see \autoref{eoleffects}), and when the array is cold for several minutes after each eclipse (see \autoref{tempeffectimg}). Also the battery voltage changes. Moreover, typically the spacecraft will be composed of a variety of components with different voltage needs. Since the system is required to provide power to the load at a voltage regulated within specified limits, see \ref{testing}, a power regulation is always needed to match voltages of various power components during the entire operation time \cite{spacecraftspower}.

This power regulation is accomplished by battery charge and discharge converters, a shunt dissipator to control the bus voltage during sunlight and a controller in charge of adequately cope with the bus voltage error signal. The solar panel as power source, and the battery as load, have both their respective I-V curves, shown in \autoref{shuntcontrol}. The system can work at either of the two intersection points A or B. However, point A is inherently unstable because the load slope is lower than the source slope. Point B, on the other hand, is inherently stable. Without a shunt control, the system would operate at point B, producing a \textbf{lower power}. Nevertheless, with adequate shunts regulating the sunlight voltage, the system will pull back from point B to point C, \textbf{shunting} the excess current $I_{shunt}$ (difference between the source current at D and the load current at C) to ground, and producing more power.



\begin{figure} [H] 				
				\centering
				\includegraphics[page=1,trim={0cm 0cm 0cm 0cm},clip=true,width=88mm]{figurastfm/Chapter3/PDF/shuntcontrol.pdf}
				\caption{Stability of operating point and shunt control during sunlight \cite{spacecraftspower}} \label{shuntcontrol}
			\end{figure}


As for the controller, depending on the voltage error signal (the difference between actual bus voltage and desired), sends a control signal to one of the regulators in order to keep the bus voltage within the specified limits, allowing the battery to be charged (sunlight) or to provide energy (eclipse), with no damage for the system.

Although the latter is mainly a safety issue, power regulation is also closely related to performance. Indeed, as exposed in \ref{scellssect}, $V_{\text{out}}$ and $I_{\text{out}}$ will constantly vary, probably mismatching the \textit{MPP}. As \autoref{ivmodel} shows, for the majority of its useful curve, solar cells act as a constant current source, but when reaching \textit{MPP} boundaries (\textit{knee} of the curve), the curve has an approximately inverse exponential relationship between current and voltage. At every moment (depending on striking radiation, temperature etc.) these pair of values will imply a different load $R=V_{\text{out}}/I_{\text{out}}$ given by Ohm's law;  for an external device to draw maximum power from the solar cell, it	should \textit{see} a load with such a resistance equal to the inverse of this value; other way, and as \autoref{efficencysolarcell} states, \textbf{efficiency will decrease} and could eventually reach one of the \textit{No power} points. Thus by \textbf{varying the impedance seen by the panel}, the operating point can be moved towards the \textit{MPP}.

In sum, the desirable operation point varies considerably and this is not only an efficiency problem by itself, but it is also a problem for the following element in the power chain: \textbf{the batteries}. The following example will illustrate the problem: let us assume a set-up where the solar panel is directly connected to a standard 3.7 V Li-on battery. According to the manufacturer, the solar cell is able to provide a power of 1 W at the \textit{MPP}~(5~V,~0.2~A); however, given that it is directly connected to a 3.7 V source (the battery, whose voltage is taken as constant) the output voltage of the solar cell is fixed to that value, therefore operating at (3.7 V, 0.2 A) and producing 0.74 W in the best scenario, implying an \textbf{unexpected loss of 26 \%} (actually, it would be even worse given that the battery voltage is not constant but it will decrease along with its available energy, see section \ref{energystr}). This is another reason why an adequate power regulation is highly desirable.

%Therefore, it would be highly desirable some kind of system which tries to set the operating point as close as possible to the \textit{MPP} given by the load each moment, especially considering the pretty low efficiency described. Both scenarios are considered and analyzed in the following sections.

The photovoltaic power system, therefore, primarily consists of a solar array, a rechargeable battery, and a power regulator which regulates power flow between various components to control the bus voltage. This section deals with the power regulation stage, particularly the different architectures normally used and its efficiency.


	\paragraph{Direct Energy Transfer} \label{det}
	
	The set-up described before, in which the solar panel is directly connected to a battery, is known as \textbf{Direct Energy Transfer} (\acrshort{DET}) architecture. In this case, the power bus is said to be \textit{dominated} by the battery voltage, which implies that the solar panel must operate at the same voltage of the power bus, potentially not delivering the full power it is capable of at all times. Since the solar array is designed to never exceed a voltage past the \textit{MPP}, it will reach its full power power producing capability only when the battery is at its highest voltage, which occurs when the battery is completely charged. However, when the battery is at its minimum voltage, at the beginning of its charge cycle, the solar array will operate well below the \textit{MPP}. This phenomena is graphically illustrated in\autoref{ivexample} which shows the voltage shift produced when the \textit{MPP} is displaced due to mismatch load imposed by the battery. Figure\autoref{pexample} depicts the power losses due to that issue.

This architecture can be subdivided into two classes which are briefly analyzed next.


	\begin{figure} 
					\centering		
					\subfloat[I-V curves in open-circuit and loaded\label{ivexample}]{\includegraphics[page=1,trim={0cm 0cm 0cm 0cm},clip=true,width=140mm]{figurastfm/Chapter3/PDF/IV_Curves_example_lat.pdf}} 
					\\[3ex]	
					\subfloat[Power curves in open-circuit and loaded\label{pexample}]{\includegraphics[page=1,trim={0cm 0cm 0cm 0cm},clip=true,width=140mm]{figurastfm/Chapter3/PDF/P_Curves_example_lat.pdf}}	
					\caption{Solar panel performance directly connected to a battery} \label{compexample}
			\end{figure}

	\subparagraph{Sunlight regulated bus} \label{sunlightregulated}
	
	This sub-architecture is normally used when the objective is to minimize complexity; to achieve that power from both sources available --- the solar panel output and the battery~--- directly to the load. An in-depth analysis of this architecture goes beyond the scope of this project, however typical sun regulated bus architecture is depicted in \autoref{sunregul} so minimal references can be made. 
	
	\begin{figure} [H] 				
				\centering
				\includegraphics[page=1,trim={0cm 0cm 0cm 0cm},clip=true,width=105mm]{figurastfm/Chapter3/PDF/sunregul.pdf}
				\caption{Sun regulated \acrshort{DET} architecture \cite{spacecraftspower}} \label{sunregul}
			\end{figure}
	
	In this architecture, the bus is said to be 'regulated' by \textbf{shunt control} during sunlight which is accomplished by a battery charge regulator to control charge rate. On the other hand, during eclipse, the battery charges directly to the bus through the \textit{battery discharge diode}, $d$ in \autoref{sunregul}; this diode only allows discharge from the battery, blocking any uncontrolled charge current coming to the battery, leaving this task to the charge regulator, and disconnecting the battery from the bus during sunlight. Without this kind of regulation, the solar panel output voltage would settle at operating point B in \autoref{shuntcontrol}, which would be unsafe at some points of operation, for example at \acrshort{BOL} and after eclipse (the solar panel would be cold and produce a higher output voltage, as seen in \autoref{tempeffectimg}).
	
	It is simple and reliable but implies variations in bus voltage up to $\pm25 \%$ around nominal value \cite{spacecraftspower}. This architecture finds application mostly in relatively low power needs, such as CubeSats, for example.
	
	\subparagraph{Fully regulated bus} \label{fullyregulated}

	This sub-architecture is commonly known as \textit{regulated bus} and it is mainly characterized by a controlled bus voltage within a few percent during orbit period. Contrary to sunlight regulated, it requires a battery discharge converter which is expensive. On the other hand, it allows great flexibility in battery choice.
	
	It is typically used when the load has high power requirements or the spacecraft loads require a lower variation around bus voltage.


	\paragraph{Maximum Power Point Tracking} \label{mppt}
	
	Section \ref{scellssect} clearly exposed the advantages of a mechanism able to get the system to the optimum performance point continuously. It can be accomplished with the \textbf{Maximum Power Point Tracking} (\acrshort{MPPT}) architecture, which makes use of switching regulation between the solar panel and the load, as \autoref{mpptarc} shows.
	
		\begin{figure} [H] 				
				\centering
				\includegraphics[page=1,trim={0cm 0cm 0cm 0cm},clip=true,width=105mm]{figurastfm/Chapter3/PDF/mpptarc.pdf}
				\caption{Maximum Power Point Tracking architecture \cite{spacecraftspower}} \label{mpptarc}
			\end{figure}
	
	
	The peak power tracker senses the \textit{MPP} and is in charge of keeping the series regulator input voltage at the maximum power producing level; then the output voltage is stepped down to the load voltage,  i.e. the battery, by varying the duty cycle as needed.
	
	Noteworthy advantages of this architecture are that it maximizes the solar panel output power at every moment without neither shunt nor battery charge regulator. As main drawback, it may result in low efficiency due to power losses in the tracking process itself.
	
		\autoref{proscons} summarizes pros and cons of the architectures analyzed. 

	
			\begin{figure} [h] 				
				\centering
				\includegraphics[page=1,trim={0cm 0cm 0cm 0cm},clip=true,width=135mm]{figurastfm/Chapter3/PDF/proscons_mppt.pdf}
				\caption{Pros and cons of the analyzed architectures \cite{spacecraftspower}} \label{proscons}
			\end{figure}
			
				
\subsubsection{Energy storage. Batteries.} \label{energystr}


In an energy-limited scenario as in orbit, energy storage is a vital issue. For example, when in eclipse, as there is no energy input available, in order for the system to keep working, there must be some kind of reserve; also when the demand exceeds the power generation at
any time. This storage is usually accomplished by using \textbf{batteries} which store energy in an electrochemical form.

As in previous topics, some battery basic concepts are introduced before going into detail.

\begin{itemize}

	\item \textbf{Energy density:} electrical energy per unit mass (Wh/kg) available when fully
discharged from a fully charged state at a given rate and temperature.
	\item \textbf{Roundtrip efficiency:} also called `Wh efficiency', is the ratio of energy provided between full charge and the following full discharge at a given rate and temperature.
	\item \textbf{Depth of discharge (\acrshort{DoD}):} amount of energy taken out of the battery per cycle. Looking in the opposite direction, the \acrshort{SoC} (state of charge) is 100 - \%\acrshort{DoD}.
	\item \textbf{Nominal capacity (Ah):} typically stated by the manufacturer. This capacity may or may not represent the amount of Ah available upon discharge down to a technology dependent \acrshort{EOL} voltage. 
	\item \textbf{Dimensional abuse:} it is defined as the charge the cell can deliver at room temperature until it reaches a cut-off voltage of about 2/3 of the fully charged voltage. Charge and discharge currents (A) are expressed as a multiple or fraction of the nominal capacity which is called `C' and depends on physical size. (e.g. 2C, C/2, C/100...). It is widely used and its practical meaning is the possibility to deliver C amperes for 1 h or C/n amperes for n hours.
	\item \textbf{Cycle life:} defined as the number of charge/discharge (C/D) cycles the
battery can deliver while maintaining the cut-off voltage.
	

\end{itemize}

Batteries are made of numerous electrochemical cells assembled in a series–parallel circuit combination to obtain the required voltage and current. It has positive and negative electrode plates with insulating separator and a chemical electrolyte in-between. The two electrode plates
are connected to two external terminals mounted on the casing. The cell stores electrochemical energy at a low electrical potential. The cell voltage depends solely on the electrochemistry, and not on the physical size. \autoref{emfcell} shows the typical electrochemical cell construction \cite{spacecraftspower}
			\begin{figure} [H] 				
				\centering
				\includegraphics[page=1,trim={0cm 0cm 0cm 0cm},clip=true,width=105mm]{figurastfm/Chapter3/PDF/emfcells.pdf}
				\caption{Typical electrochemical cell construction \cite{spacecraftspower}} \label{emfcell}
				\vspace{-0.5cm}
			\end{figure}

Batteries cells can be thought of as voltage sources with small internal resistance which respectively decreases and increases linearly with the nominal capacity discharge. Ignoring parasitics, it is as a \acrshort{DC} source representing the \textbf{electromotive force} (\acrshort{EMF}); this is the typical term used when a voltage is generated by a battery and represents the \textbf{energy} per unit charge which has been made available by the generating mechanism, in this case, the electrochemical cell. \acrshort{EMF} depends on the \acrshort{SoC} and ageing. On the other hand, the cell resistance in series can be compared with the traditional output series resistance in power sources; besides \acrshort{SoC} and ageing, it also depends on temperature. Next, the most used batteries technologies are briefly introduced.

\paragraph{Nickel Hydrogen}

\ce{NiH2} batteries have been the most widely used in space sector during the last 25 years, which makes them highly reliable. They combine some of the best characteristics from other technologies, such as NiCd or the fuel cell (see \ref{fuelcell}) and can tolerate some over-charge or over-discharge without damage; it also has a greater charge/discharge cycle life and low internal resistances than older technologies such as NiCd and do not exhibit a noticeable \glsname{memeffect}. However, it has a high self-discharge rate, around 0.5 \%, and also high loss of capacity. In the typical configuration, each cell develops 1.25~V. \autoref{nih} shows a common \ce{NiH2} batteries stack.

				\begin{figure} [H] 				
				\centering
				\includegraphics[page=1,trim={0cm 0cm 0cm 0cm},clip=true,width=80mm]{figurastfm/Chapter3/Imagenes/nih.jpg}
				\caption{\ce{NiH2} batteries stack \cite{upsnih}} \label{nih}
			\end{figure}
	
\paragraph{Lithium-Ion}	

With a great expansion throughout the last years, Li-ion technology exhibits pretty higher energy density than \ce{NiH2}. It is corrosion-free but lithium is highly reactive and must be stabilized. Li-ion has a cut-off voltage of 2.7 V, average discharge voltages of 3.5 V and end-of-charge voltage of 4.2 V. The average discharge voltage of 3.5 V contrasts with the 1.25 V for \ce{NiCd} and \ce{NiH2} requires about one third of the cells in series for a given battery voltage, allowing a smaller assembly and test costs \cite{spacecraftspower}. Another significant advantage is its capacity to deliver peak power without negative consequences on life duration.

As for the negative side, Li-ion cells are sensitive to over-charge and over-discharge and exhibit low performance at low temperature, due to associated high internal resistance under those conditions. \autoref{lion} depicts a common Li-ion battery.

			\begin{figure} [H] 				
				\centering
				\includegraphics[page=1,trim={0cm 0cm 0cm 0cm},clip=true,width=75mm]{figurastfm/Chapter3/Imagenes/lion.jpg}
				\caption{Li-ion battery \cite{sparkfun}} \label{lion}
				\vspace{-2cm}
			\end{figure}
\newpage
\paragraph{Lithium Polymer}

This short comparison finalizes with lithium polymer batteries. These batteries are increasingly common these days, because of their wide use in drones and radio-controlled devices. Their electrochemical basis are similar to that of Li-ion but they are more delicate and have the same general problems of Li-ion e.g., over-charge, over-discharge, over-temperature, short circuit, crush and nail penetration may all result in a catastrophic failure.\autoref{lipo} shows a standard LiPo battery while\autoref{lionhinchada} illustrates the consequences of an incorrect usage.

			\begin{figure}[H]
			\centering
			\subfloat[Standard LiPo battery \cite{hobbyking}\label{lipo}]{\includegraphics[page=1,trim={0cm 0cm 0cm 0cm},clip=true,width=65mm]{figurastfm/Chapter3/Imagenes/lipo.jpg}}
			 \quad
			\subfloat[Expanded Lithium-based battery \cite{wiki}\label{lionhinchada}]{\includegraphics[page=1,trim={0cm 0cm 0cm 0cm},clip=true,width=65mm]{figurastfm/Chapter3/Imagenes/lipohin.jpg}}
			\caption{LiPo batteries} 
\end{figure}

\subsubsection{Formal Requirements Definition}
			
\begin{table} [H]
\centering

\begin{tabularx}{\linewidth}{lX}

\multicolumn{1}{c}{\textbf{Ref.}}                      & \multicolumn{1}{c}{\textbf{Formal Requirements}}                    \tabularnewline \specialrule{1.1pt}{1pt}{1pt}
EPS.FoR.1                                              & Solar cells shall produce at least 1 W of power. \tabularnewline \midrule
EPS.FoR.2                                              & Solar cells shall be silicon monocrystalline.\tabularnewline \midrule
EPS.FoR.3                                            & Batteries shall be Li-ion and have a capacity of at least 1500 mAh.  \tabularnewline \midrule
EPS.FoR.4                                                   & The \acrshort{EPS} shall have a contingency system to provide 3.3 V in absence of the main regulation subsystem.\tabularnewline \midrule
EPS.FoR.5                                                   & The \acrshort{EPS} shall have \acrshort{USB} connectors which allow charging batteries from external supply.   \tabularnewline \midrule

\end{tabularx}
\caption{\acrshort{EPS} - Formal Requirements}
\vspace{-0.5cm}
%\label{forgstation}

\end{table}

%\newpage
%\subsection{On-board computer (OBC)}\label{sec:OBC}
%
%On-board computer is probably the most important subsystem of a \glsname{cubesat}. It is in charge of controlling the whole system, controls I/O and coordinates the different subsystems to successfully perform the tasks needed at every moment. As seen in \ref{cubesatdefin}, sometimes \acrshort{OBC} inherently includes another subsystem such as \textbf{On-board~Data~Handling}~(\acrshort{OBDH}), while in others systems it tackles functions related with a different subsystem (e.g. \acrshort{TCC}) out of simplicity. In order to perform all these tasks, the \acrshort{OBC} can be composed of more than one processing core, which can be used to free the main one. This wide scope makes this subsystem to require a strong effort of integration with different specialists.
%
%Additionally, sensors can also be considered part of the \acrshort{OBC} given that as a last resort it is in charge of processing the information received from them. Besides, a significant number of them are physically placed at the \acrshort{OBC}
%
%Just as previous subsystems analyzed, \acrshort{OBC} can be subdivided into different modules, which are breakdown next.
%
%\subsubsection{Central Processing Unit}
%
%The Central Processing Unit  can be seen as the microprocessor of a standard computer. It is typically an \textbf{embedded system} and the most complex module of the system. It work as interface with the rest of the modules and \acrshort{HID}.
%
%It is recommendable for this module to be as flexible as possible and to allow a wide range of programming possibilities. It must be able to deal with hardware interruptions, act as \textbf{Scheduler}, manage Write/Reading operations on memory, prevent the system from failures and recover it from faulty situations... in sum it is the global coordinator of the system.
%
%Central Processing Units can be either a single \acrshort{SOC} or a complete module \acrshort{COTS} and in aerospace sector they are usually able to have a complete \textbf{Operative System} installed, normally Linux-based. %The diagram in FIGURE illustrates some of the duties and relations with the rest of the system that this module usually has.
%
%For all the reasons exposed, a high-level analysis of this subsystem is impossible due to the vast amount of possibilities. Therefore, although the main requirements are listed at \ref{obcreqs} just as with the rest of the sections, to go deeper on the possibilities and distinctive features of this kind of modules requires the scope to be somewhat more constrained; that task will be performed and exposed in section \autoref{obcdesig}.
%
%\subsubsection{Co-processing Programmable Core}
%
%One of the featured characteristics of the cutting-edge designs in a variety of sectors is including a reconfigurable or programmable hardware module, which can be used to act as auxiliary processor, performing a variety of tasks. \textbf{Field-Programmable Gate Arrays} (\acrshort{FPGA}) are usually used for this purpose. The task to be developed by this module is generally specified using some Hardware Description Language such as \acrshort{VHDL} or Verilog.
%
%An \acrshort{FPGA} can be used to deal with any computable problem, featuring its speediness for some applications because of their parallel nature. Some of the common usages nowadays are aerospace, digital processing, wireless communications or image processing. All of those usages are potentially useful in a \glsname{cubesat} like the one under development; that is the reason why Co-processing Programmable Core are increasingly implemented, its versatility allows to solve different problems with a single module.
%
%This brief analysis will focus on \acrshort{FPGA} because of their wide usage for this purpose, as mentioned before. The most common \acrshort{FPGA} architecture consists of an array of \textbf{logic blocks} and routing channels with I/O capabilities. A certain application circuit must be mapped into an \acrshort{FPGA} with the adequate resources, which may vary considerable. Generally speaking, a logic block consists of a number of logical cells which are typically composed of four input \textbf{Lookup tables} \acrshort{LUT}, a \textbf{full adder} and \textbf{D-type flip-flop}. \autoref{fpgafig} shows this structure.	Also, as depicted in \autoref{fpgafig}, a clock signal is needed as most of the electronics inside of an \acrshort{FPGA} is synchronous.
 %
%
%
			%\begin{figure} [H] 				
				%\centering
				%\includegraphics[page=1,trim={0cm 0cm 0cm 0cm},clip=true,width=85mm]{figurastfm/Chapter3/Imagenes/fpga.png}
				%\caption{Typical logic cell \cite{wiki}} \label{fpgafig}
			%\end{figure}
			%
			%Among the concepts introduced before, \textbf{Lookup tables} (\acrshort{LUT}) have special importance. They are arrays which replaces runtime computation with simple array indexing which is time-saver in processing terms, as retrieving a value from memory is often faster than I/O operation \cite{wiki}. \acrshort{LUT} are in charge of storing the truth table of the any boolean function. It is a parameter sometimes used to estimate the \textit{capacity} of an \acrshort{FPGA}.
			%
			%Regarding the programming, the user normally provides a certain design using some Hardware Description Language, as mentioned. A netlist is generated by a procedure called \textbf{synthesis}, which will be fitted to the \acrshort{FPGA} architecture using a process called \textbf{place-and-route}. As a result of this process, the user will obtain a performance report exposed via timing analysis or simulation. Finally, the generated binary file is used to program the \acrshort{FPGA}.
			%
		%\acrshort{FPGA} are particularly suitable for this Co-processing module because of their versatility; the variety of implementations possible is countless; for example, one of the most featured applications is \textbf{embedding a processor} inside an \acrshort{FPGA}. It has many advantages, e.g.: specific peripherals can be chosen based on the application, mitigates obsolescence, reduces costs and allows impressive customization. For instance, there is a complete flexibility to select any combination of peripherals and controllers which can be directly connected to the processor's bus. This point allows meeting non-standard requirements, for example a \acrshort{COTS} processor with ten \acrshort{UART} may be impossible to find, but it is easy to implement in an \acrshort{FPGA} embedded processor \cite{fpgaproc}. An embedded processor implementation example can be found on \cite{fpgacalvo}. Another possible implementations using \acrshort{FPGA} are generating digital interfaces such as \acrshort{I2C} or generating \acrshort{PWM}.
	%
%\subsubsection{Communications}
%
%For obvious reasons, Communications subsystem or, more specifically, Telecommand~\&~Telemetry~(\acrshort{TCC}) are vital in CubeSats. Communications must be understood not only as the ones between the \glsname{cubesat} and the \glsname{ground} but also inside the \glsname{cubesat} itself. One of the simplest ways to communications is between \textbf{wired} and \textbf{wireless}. Next, a brief analysis of the useful technologies for CubeSats is introduced.
%
%\paragraph{Wired}
%
%Normally, wired connections are necessary even in CubeSats intended to fly to be able to program data, perform test and so forth. In this project, because of the simulation parcel, wired communications will be specially important.
%
%\subparagraph{Serial}
%
%This in an historical wired interface. Serial communication is based on sequentially sending data one bit at a time. Serial ports are typically identified as such which comply with RS-232 standard. One of the main drawbacks of serial communications is their slowness.
%
%Although it is considered to be deprecated, it is still usual in many electronics systems, as debugging port or just as a contingency way of communication. In fact, these could its usefulness in a \glsname{cubesat}. Throughout years, serial ports have led to new derived technologies, such as \acrshort{USB}. \autoref{serial} depicts DE-9 connectors, to be used with RS-232 standard.
%
			%\begin{figure} [H] 				
				%\centering
				%\includegraphics[page=1,trim={0cm 0cm 0cm 0cm},clip=true,width=65mm]{figurastfm/Chapter3/Imagenes/serial.jpeg}
				%\caption{DE-9 connector \cite{wiki}} \label{serial}
				%\vspace{-1cm}
			%\end{figure}
%
%
%\subparagraph{Universal Serial Bus}
%
%One of the most extended interfaces in industry. Derived from serial ports, they were initially intended to standardize the connection of peripherals, and finally have largely replaced its preceding interfaces.
%
%\acrshort{USB} has plenty advantages, improved ease of use as it is self-configurable and hot pluggable. It is also much more faster than serial ports, reaching 5 Gb/s in the 3.0 version, one of the latest \cite{wiki}. They are also extensively used as power ports. \autoref{usbfig} shows some of the standard connectors pinout.
%
			%\begin{figure} [H] 				
				%\centering
				%\includegraphics[page=1,trim={0cm 0cm 0cm 0cm},clip=true,width=80mm]{figurastfm/Chapter3/Imagenes/usb.png}
				%\caption{USB standard connectors pinout \cite{wiki}} \label{usbfig}
				%%\vspace{-0.5cm}
			%\end{figure}
%
%Because of their versatility, \acrshort{USB} interfaces may have numerous applications on CubeSats, from programming to power ports or to downloading data.
%
%\subparagraph{Ethernet}
%
%Standardized first in 1983 as \acrshort{IEEE}802.3 is by far the most used wired technologies in \acrshort{LAN}. Systems communicating over this interface divide streams of data into \textbf{frames} which contains source and destination addresses and allows error-checking. 
%
%In CubeSats, Ethernet interfaces may be useful to allow communications using high-level protocols such as \acrshort{SSH}, which ease communications and configuration of the system.
%
			%\begin{figure} [H] 				
				%\centering
				%\includegraphics[page=1,trim={0cm 0cm 0cm 0cm},clip=true,width=70mm]{figurastfm/Chapter3/Imagenes/ethernet.png}
				%\caption{Ethernet standard connector \cite{ethernet}}
				%\vspace{-0.2cm}
			%\end{figure}
%
%
%\paragraph{Wireless}
%
%Wireless communications are vital in real CubeSats, for obvious reasons. In orbit CubeSats communicates with \glsname{ground} using radio-links and dedicated antennas. It is normally a whole subsystem within Communications system. It goes beyond the scope of this project analyzing that kind of implementations; extended radio-link analysis and calculation can be found on \cite{tfg}. In this section, it will be briefly introduced local area wireless communications, which can be used as base for future developments focused on real long-distance links.
%
%\subparagraph{Radio-frequency} \label{radiofr}
%
%Communications over the air are the only choice when dealing with space communications. There is variety of frequencies used in radio-links, depending on distance, power and numerous additional factors. According to the frequency used, the link will have certain particularities which shall be taken into account. \autoref{tfgradio} schematically shows an example of radio path between a \glsname{ground} based on Granada and \acrshort{ISS}; it illustrates some of the aspects to be considered in a long-distance radio-frequency communication such as elevation angle or effective distance, see \autoref{tfgradio}.
%
			%\begin{figure} [H] 				
				%\centering
				%\includegraphics[page=1,trim={0cm 0cm 0cm 0cm},clip=true,width=130mm]{figurastfm/Chapter3/PDF/radio_l.pdf}
				%\caption{Radio-path basic elements \cite{tfg}} \label{tfgradio}
			%\end{figure}
%
%
%Although this Project will not tackle this kind of communications, short-distance radio link may be considered useful for the Simulation Platform branch of the project. Contrary to the links designed for larger distances, short-distance communications over the air exhibits way lower complexity and cost due to the minor requirements in relation with power, losses constraints, among others.
%
%
%\subparagraph{IEEE802.11}
%
%Commonly known as Wi-Fi (a trademark from the Wi-Fi Alliance), it is based on \acrshort{IEEE}802.11 standard and it is the most extended wireless communication protocol for local areas. Once again, this makes it specially suitable for this Simulation Platform because of its ease of use and extended use. On the other hand, it cannot be used for space radio-links.
%
%Since its origins, back in 1997, the standard has going through a significant number of reviews. Nowadays the most usual versions are \acrshort{IEEE}802.11g/n/ac reaching data rates up to 1300 Mbit/s in the 5 GHz band \cite{wiki}.
%
%As a result of the wide use of \acrshort{IEEE}802.11, the majority of the systems related with it are highly standardized (\acrshort{IC}, antennas, amplification subsystems, etc.) which reduces complexity and eases integration.
%
%\subsubsection{Payload. Sensors.}
%
%The \glsname{payload} is generally known as the amount of cargo capacity of an aircraft, including fuel and people. However, it may also refer to the equipment specifically intended to perform a certain mission while in orbit, for instance a camera or a \glsname{startracker}.
%
%In CubeSats field, a common \glsname{payload} example are \textbf{sensors}: they are typically a crucial part of any mission, not only as part of the \glsname{payload} but also a necessary part for the correct functioning of the whole system. Normally they are some kind of electronic device which takes some measurement or perform a certain action depending on the inputs it receives. Some examples of sensors have already been treated before, see \ref{imus}. Another possible sensors are: barometer, thermometer, magnetometer, lightmeter, tachometer... the list is countless. Many of them will be analyzed and implemented in \autoref{chap:chapter4}. \autoref{sensores} shows some examples of them.
%
%While this analysis considers sensors as part of the \acrshort{OBC}, given that eventually the data will be treated there, they can belong to a different subsystem such as \acrshort{ADCS}, see \ref{imus}. 
%
			%\begin{figure}[H]
			%\centering
			%\subfloat[\acrshort{IMU}]{\includegraphics[page=1,trim={0cm 0cm 0cm 0cm},clip=true,width=65mm]{figurastfm/Chapter3/Imagenes/MPU.png}}
			 %\quad
			%\subfloat[Barometer and thermometer]{\includegraphics[page=1,trim={0cm 0cm 0cm 0cm},clip=true,width=65mm]{figurastfm/Chapter3/PDF/BMP.pdf}}
			%\caption{Sensor examples \cite{hobbyking}} \label{sensores}
			%\vspace{-0.55cm}
%\end{figure}	
%
%
%\subsubsection{Flight Software. On-Board Data Handling.}
%
%In charge of all the exposed, there must be a \textbf{governor}, a coordinator which controls the hardware and deals with the different stages and situations of a mission, this is the \textbf{Flight Software} (FSW). As in a significant number of the subsections analyzed through this Master's Thesis, Flight Software is such a large field that it allows a complete project focused on it. Therefore, hereby only a basic classification is introduced; this will be an important issue at the design stage and also for future improvements of the system.
%
%As far as this project concerns, considered flight software can be subdivided into the following.
%
%\paragraph{Non-real-time Operative System}
%
%Non-real-time Operative Systems is basically a general purpose \acrshort{OS} to be used with personal computers, servers, etc. The main difference with \acrshort{RTOS} is \textbf{determinism}. Non-\acrshort{RTOS} are not deterministic as tasks will not run at a certain time and for a certain time, there are no guarantees for critical tasks, exhibits high latency because of using unpredictable virtual memory, as well as \glsname{jitter}.
%
%The previous characteristics translates into the use of \textbf{non-preemptive schedulers}. Examples of these kind of \acrshort{OS} are widely used Windows\,\textsuperscript{\textregistered}, Debian, etc.
%
%\paragraph{Real-time Operative System}
%
%In contrast with the latter, \acrshort{RTOS} are completely deterministic, this is, how and when a task will run given whatever conditions defined for it to do so, it is \textbf{guaranteed}. They are intended to process data as it comes in, without buffer delays. There are a huge amount of systems nowadays which must use \acrshort{RTOS}: cars, \textbf{spacecrafts}, avionics, critical systems...
%
%
%In sum, \acrshort{RTOS} must be able to compute a task in a \textbf{limited or predictable amount of time} i.e., is time-bounded. However, this behaviour has nothing to do with processing speed but with a known deadline (a second, an hour or a month) and reduced \glsname{jitter}.
%
%From a technical perspective, this implies not to use virtual memory, strict scheduling (preemptive) and avoid non-deterministic elements. \acrshort{RTOS} are usually much smaller than general purpose \acrshort{OS} in order to ease maintenance and find sources of delay. One of the most used \acrshort{RTOS} is \textbf{VxWorks} \cite{vxworks}; Linux non-preemptive kernel can also be \textbf{patched} to allow real-time behaviour.
%
%
			%\begin{figure} [H] 				
				%\centering
				%\includegraphics[page=1,trim={0cm 0cm 0cm 0cm},clip=true,width=85mm]{figurastfm/Chapter3/Imagenes/vxworks.png}
				%\vspace{0.3cm}
				%\caption{VxWorks logo \cite{vxworks}} 
				%%\vspace{-2cm}
			%\end{figure}
%
%\subsubsection{Formal Requirements Definition} \label{obcreqs}
%
%
%\begin{table} [H]
%\centering
%
%\begin{tabularx}{\linewidth}{lX}
%
%\multicolumn{1}{c}{\textbf{Ref.}}                      & \multicolumn{1}{c}{\textbf{Formal Requirements}}                    \tabularnewline \specialrule{1.1pt}{1pt}{1pt}
%OBC.FoR.1                                              & The \acrshort{OBC} shall have an \acrshort{I2C} interface which allows communication with the rest of subsystems and components. \tabularnewline \midrule
%OBC.FoR.2                                              & The \acrshort{OBC} shall have an \acrshort{SPI} interface which allows communication with the rest of subsystems and components. \tabularnewline \midrule
%OBC.FoR.3                                            & The \acrshort{OBC} shall have a programmable \acrshort{FPGA} of at least 2500 \acrshort{LUT}. \tabularnewline \midrule
%OBC.FoR.4                                                   & The \acrshort{OBC} shall have one RJ-45 connector. \tabularnewline \midrule
%OBC.FoR.5                                                   & The \acrshort{OBC} shall have at least two \acrshort{USB} Micro-B device-ports and capability to deal with four.  \tabularnewline \midrule
%OBC.FoR.6                                                   & The \acrshort{OBC} shall be programmable with an external computer using an \acrshort{USB} Micro-B connector. \tabularnewline \midrule
%OBC.FoR.7                                                   & The \acrshort{OBC} shall have a \acrshort{LED} system to check the correct functioning of the \acrshort{EPS}  \tabularnewline \midrule
%OBC.FoR.8                                                   & The \acrshort{OBC} shall have at least two \acrshort{USB} device-ports and capability to deal with four.  \tabularnewline \midrule
%OBC.FoR.9                                                   & The \acrshort{OBC} shall have a barometer with a resolution of at least 0.1 hPa.  \tabularnewline \midrule
%OBC.FoR.10                                                   & The \acrshort{OBC} shall have a thermometer with a resolution of at least 0.1\textdegree~C.  \tabularnewline \midrule
%OBC.FoR.11                                                   & The \acrshort{OBC} shall have at least 8 \acrshort{ADC} channels. \tabularnewline \midrule
%OBC.FoR.12                                                   & The \acrshort{OBC} shall have 6 analog sun sensors, one per each face of the \glsname{cubesat}. \tabularnewline \midrule
%OBC.FoR.13                                                   & The \acrshort{OBC} shall have a Real Time Clock (\acrshort{RTC}) which provides the whole system with time information. \tabularnewline \midrule
%OBC.FoR.14                                                   & The \acrshort{OBC} shall have an \acrshort{SD} card holder, to store telemetry data.  \tabularnewline \midrule
%OBC.FoR.15                                                   & The \acrshort{OBC} shall have an integrated \acrshort{IEEE}802.11g device which allows wireless communications. \tabularnewline \midrule
%OBC.FoR.16                                                   & The \acrshort{OBC} shall implement a linux-based \acrshort{RTOS}. \tabularnewline \midrule
%
%\end{tabularx}
%\caption{\acrshort{OBC} - Formal Requirements}
%\vspace{-0.5cm}
%%\label{forgstation}
%
%\end{table}
%
%
%
%
%
%
%
%
%
