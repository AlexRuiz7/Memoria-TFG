
\chapter{Control remoto de la instrumentación electrónica.}

Como ya se comentó inicialmente, se pretende dotar al sistema de un control automático y centralizado de todos los equipos electrónicos necesarios vistos en el capítulo 6, de modo que su configuración se realice por medio del software que implementaremos, evitando la necesidad de manipular cada instrumento individualmente sobre su propio panel frontal de configuración o \textit{setup}.

%Para este objetivo, y aprovechándonos de que la mayoría de equipos que utilizaremos tienen conexión GPIB, crearemos unas librerías de control para cada instrumento, que utilizarán comandos virtuales de configuración remota a través del bus GPIB 488.2 de IEEE. El gaussímetro, que es imprescindible para medir el valor del campo magnético generado por las bobinas y poder así establecer una calibración de las mismas, se controla por medio del puerto serie, por tanto necesitará una librería diferente.
%
%A continuación explicaremos las principales características del bus GPIB de comunicación así como el protocolo empleado con el gaussímetro sobre el puerto serie.
%
%\section{El bus GPIB.}
%
%Es un sistema de inteconexión de instrumentos programables consistente en un bus simple formado por 24 líneas a las que elementos del sistema se conectan en paralelo-serie. La comunicación se realiza mediante el paso de mensajes que transportan datos y mensajes de control. Inicialmente el bus fue desarrollado por \textit{Hewlett-Packard} en 1969 y lo llamo HP-IB (Hewlett-Packard Instrumentation Bus).
%
%En 1978 el IEEE estandarizó el bus en la norma que denominó IEEE-488, para proporcionar u n interfaz estándar para la comunicación entre los instrumentos de diversas fuentes. Originalmente, el objetivo era permitir que sus equipos pudieran conectarse con los de otros fabricantes y con ordenadores funcionando como \textit{hosts}. El interfaz ganó rápidamente renombre en la industria del ordenador debido a su gran versatilidad y entonces el comité del IEE lo rebautizó como GPIB (\textit{General Purpose Interface Bus}).
%
%Dado que esa primera norma no incluía indicaciones sobre la sintaxis o el formato de los comandos a usar, apareció la norma 488.2 que incluía un mínimo de mensajes que debía entender un instrumento o el tipo de formato de datos y comandos.
%
%\subsection{La norma IEEE 488.2 (GPIB).}
%
%A la hora de definir las características de un bus hay que pensar en que éste se puede definir bajo dos conceptos: uno físico y otro lógico. Físicamente consiste en un cierto número de conductores que transportan señales eléctricas en paralelo entre diferentes sistemas que contienen circuitos electrónicos. El concepto lógico de refleja en las normas y formatos de intercambio de datos, en la sincronización y en la temporización.
%
%El bus transportará información en todas direcciones y todos los instrumentos recibirán instrucciones similares y deberán reconocer de forma automática que han sido direccionados. Los parámetros más importantes que describen a un bus son:
%
%\begin{itemize}
%
%\item[-] Datos mecánicos y eléctricos (conector y tecnología empleada).
%
%\item[-] Procesadores compatibles.
%
%\item[-] Espacio de memoria direccionable y cantidad de instrumentos direccionables.
%
%\item[-] transferencia síncrona o asíncrona.
%
%\item[-] Multipleaxión de datos.
%
%\item[-] Frecuencia de operación.
%
%\item[-] Velocidad de transferencia.
%
%\item[-] Protocolo.
%
%\item[-] Numéro de unidades masters.
%
%\end{itemize}
%
%
%De esta manera, los objetivos principales de la norma IEEE 488.1, incorporados por la norma 488.2 fueron:
%
%\begin{itemize}
%
%\item[-] Crear un sistema de control de instrumentación donde dichos instrumentos se encuentran relativamente próximos.
%
%\item[-] Permitir la comunicación directa entre elementos sin necesidad de que los datos pasen por el controlador.
%
%\item[-] Compatibilidad entre equipos de diferente fabricante, funcionalidad y velocidad.
%
%\item[-] No disminuir la funcionalidad de los equipos.
%
%\end{itemize}
%
%
%
%\subsection{Especificaciones físicas y mecánicas del bus.}
%
%
%Finalmente la norma publicó una serie de especificaciones físicas y mecánicas, entre las que podemos destacar:
%
%\begin{itemize}
%
%\item[-] El número máximo de dispositivos conectados es 15, y necesariamente uno de ellos ha de ser el controlador.
%
%\item[-] La longitud del cable de interconexión total de los elementos está restringido a un máximo de 20 metros.
%
%\item[-] Hay 16 líneas de señales, 8 para control y 8 para datos.
%
%\item[-] La transferencia es asíncrona, controlada mediante líneas de \textit{handshaking}.
%
%\item[-] La velocidad máxima de transferencia es de 1MB/s, para distancias muy cortas (<50 cm).
%
%\item[-] Se pueden direccionar hasta un total de 31 direcciones primarias aunque solo se puedan emplear 15 instrumentos. Además de las 31 direcciones primarias, existen otras 31 secundarias.
%
%\item[-] Entre los 15 intrumentos puede haber más de un intrumento de control, sin embargo, solo puede haber uno activo al mismo tiempo.
%
%\end{itemize}
%
%El conector está formado por 24 pines y funciona simultáneamente como conector macho y hembra, facilitando la interconexion de instrumentos entre sí.
%
%Respecto a las especificaciones eléctricas, solo mencionar que tanto los emisores como receptores deben ser compatibles con TTL y que además se utiliza lógica negativa. Actualmente existen gran variedad de integrados especializados en realizar esa función de conexión con el GPIB por lo que dichas especificaciones no deben suponer problema alguno.
%
%En la Figura \ref{fig:pines_conectorGPIB}, podemos ver una representación del conector de 24 pines utilizado para comunicaciones GPIB.
%
%\smallskip
%\begin{figure}[H]%here
%\noindent \begin{centering}
%\includegraphics[scale=0.8]{capitulo7/busGPIB/ieee488_connector_pines}
%\par\end{centering}
%\smallskip
%\caption{\label{fig:pines_conectorGPIB} Conector de 24 pines para bus GPIB \cite{WIKIPED}.}
%\end{figure} 
%
%
%\subsection{Especificaciones funcionales.}
%
%En toda comunicación con intercambio de datos dentro del bus se necesitarán al menos tres elementos funcionales de carácter básico:
%
%\begin{itemize}
%
%\item Un dispositivo actuando \textit{controller}, es decir dirigiendo el flujo de datos de forma adecuada y controlando el sentido de la comunicación. Normalmente la mayoría de los sistemas GPIB consisten en un ordenador y varios instrumentos. Si hay varios ordenadores conectados entre si solo puede haber un \textit{controler} en un mismo instante de tiempo, que se denomina CIC (\textit{Controller In Charge}).
%
%\item Un dispositivo actuando como \textit{listener}, es decir escuchando instrucciones o datos provenientes de bus. Se permite que haya varios \textit{listeners} a la vez.
%
%\item Un dispositivo actuando como \textit{talker}, es decir enviando información al bus. Solo puede haber un \textit{talker} a la vez.
%
%\end{itemize}
%
%
%\subsubsection{Líneas de datos.}
%
%Las líneas de datos son bidireccionales y están destinadas a transportar mensajes usualmente en código ASCII de 7 bits. La información transferida va desde direcciones a órdenes de programación, información sobre el dispositivo o medidas tomadas por un cierto instrumento. Estas líneas de datos van desde el pin DIO7 al DIO0.
%
%\subsubsection{Líneas de control de transferencia de datos o \textit{handshaking}.}
%
%Obviamente, al ser un bus asíncrono, se necesitan ciertas líneas para coordinar la transferencia de datos  y asegurar que nada se emita si todos los receptores no están debidamente preparados para recibir o que la transmisión dure lo suficiente para que el dispositivo más lento reciba la información.
%
%En las líneas de handshaking se utiliza lógica negativa con cierto colector abierto, lo que aporta una serie de ventajas como la reducción del margen de ruido o la realización de operaciones AND entre líneas mediante el WIRED-OR.
%
%Éstas líneas de control son:
%
%\begin{itemize}
%
%\item \textbf{DAV:} (\textit{Data Valid}). Controlada por el emisor, indica que en el bus hay un dato correcto y estable que puede ser aceptado sin errores.
%
%\item \textbf{NRFD:} (\textit{Not Ready for Data}). Los receptores controlan esta línea de indicándole al emisor que están o no preparados para la recepción de datos por el bus.
%
%\item \textbf{NDAC:} (\textit{Not Data Accepted}). \textit{Flag} o bandera para que los receptores indiquen si han aceptado o no el dato enviado por el bus.
%
%\end{itemize}
%
%Mediante el cronograma de señales mostrado en la siguiente Figura \ref{fig:senal_cronogramasig}, podemos entender rápidamente el funcionamiento de estas líneas.
%
%\smallskip
%\begin{figure}[H]%here
%\noindent \begin{centering}
%\includegraphics[scale=0.5]{capitulo7/busGPIB/ch_gpib2}
%\par\end{centering}
%\smallskip
%\caption{\label{fig:senal_cronogramasig} Cronograma de señales de interfaz para handshaking \cite{WIKIPED}.}
%\end{figure} 
%
%
%\subsubsection{Líneas de control general del bus.}
%
%Las siguientes líneas son las encargadas de gestionar el flujo de información a través del bus GPIB:
%
%
%\begin{itemize}
%
%\item \textbf{IFC:} (\textit{Interface Clear}). Inicializa el bus a un estado conocido.
%
%\item \textbf{ATN:} (\textit{Attention}). Según el nivel lógico de esta línea los valores que circulen por el bus de datos serán considerados mensajes de interfaz o bien mensajes de datos ASCII correspondientes a comandos del instrumento.
%
%\item \textbf{REN:} (\textit{Remote Enable}). El controlador informa de que dispositivos deben situarse en estado remoto y atender al bus.
%
%\item \textbf{EOI:} (\textit{End Or Identify}). Esta línea tiene una doble función. Por una parte cuando un dispositivo actúa como transmisor de hacia el bus y quiere finalizar la transmisión, y por otro lado también se usa en los pollings paralelos para identificación.
%
%\item \textbf{SRQ:} (\textit{Request of Service}). Esta será la línea que utiliza un determinado instrumento para notificar al controlador del bus que necesita de sus servicios.
%
%\end{itemize}
%
%
%En cuanto un determinado instrumento emite un \textit{Service Request}, el controlador debe realizar un \textit{polling} serie o paralelo para detectar el instrumento demandando servicio y poder enlazarlo.
%
%
%Los mensaje que se transmiten a lo largo del bus pueden ser de dos tipos, atendiendo a su finalidad:
%
%\begin{itemize}
%
%\item \textbf{Mensajes de interfaz:} son los mensajes dirigidos a la interfaz del GPIB y que no interactúan con el instrumento propiamente dicho, sino solo con la interfaz de comunicación. Estos mensajes serán los que mantengan el protocolo en el estado adecuado que permita enviar y/o recibir mensajes de dispositivo en cada momento.
%
%\item \textbf{Mensajes de dispositivo:} son los mensajes que no interaccionan con la interfaz lógica del bus, por tanto no afectan al estado actual en el que se encuentre el bus, sino que actúan sobre el propio instrumento al que van dirigidos, configurando cualquier funcionalidad que ofrezca dicho dispositivo. De esta manera, si nos encontramos como \textit{listeners} (atentos a la información que se envía por el bus), nos encontramos en un estado lógico de la interfaz; si además recibimos un mensaje para efectuar una medida d corriente a través de uno de los terminales del dispositivo, el estado \textit{listener} no se habrá modificado, sin darse cambio alguno en ese sentido.
%
%\end{itemize}
%
%A continuación, en la Figura \ref{fig:tot_mens_interfaz}, se muestra el conjunto total de mensajes de interfaz, también llamados remotos. Estos mensajes se enviarán codificados en las líneas del bus. Por ejemplo, si se quisiese indicar al instrumento cuya dirección es la 10, que actúe como \textit{listener}, haría que activar la línea ATN (indicadora de transmisión de mensajes de interfaz) y enviar por las lineas DIO7-DIO0 el símbolo * codificado en ASCII, que corresponde a la instrucción MLA10 (\textit{MLA - My Listener Address}).
%
%\smallskip
%\begin{figure}[H]%here
%\noindent \begin{centering}
%\includegraphics[scale=0.6]{capitulo7/busGPIB/mensajes_interfaz_gpib}
%\par\end{centering}
%\smallskip
%\caption{\label{fig:tot_mens_interfaz} Totalidad de mensajes de interfaz GPIB \cite{WIKIPED}.}
%\end{figure} 
%
%De todos modos, no es necesario que utilicemos directamente estos comandos desde el punto de vista del programador, ya que dentro de Matlab y como se comentara en las siguientes secciones, todos estos comandos de interfaz se ejecutarán de forma automática dentro del driver del fabricante Agilent que utilizamos, por tanto no deben preocuparnos excesivamente.
%
%
%\subsubsection{Protocolo de direccionamiento.}
%
%Conviene destacar el protocolo de direccionamiento, el cual utilizan dispositivos \textit{controllers}, \textit{talkers} y \textit{listeners} para referenciarse entre ellos.
%
%Antes de que tenga lugar cualquier transferencia de datos en el bus, es necesario que algún dispositivo sea direccionado como \textit{talker} y algún otro como \textit{listener}. El controlador es el encargado de nombrar a quién habla y quién escucha, disponiendo un mensaje de interfaz con las direcciones específicas, tal y como los mostrados anteriormente.
%
%El formato del byte de mensaje de direccionamiento es el indicado en la Figura \ref{fig:direc_mens_interfaz}.
%
%\smallskip
%\begin{figure}[H]%here
%\noindent \begin{centering}
%\includegraphics[scale=0.95]{capitulo7/busGPIB/direcciones_gpib_formato}
%\par\end{centering}
%\smallskip
%\caption{\label{fig:direc_mens_interfaz} Formato del mensaje de direccionamiento GPIB \cite{WIKIPED}.}
%\end{figure} 
%
%El contenido de los bits de la posición 0-4 se utiliza para indicar la dirección del dispositivo. Los bits 5 y 6 informan de si es una dirección \textit{listener} o \textit{talker} respectivamente.
%
%Además de estos comandos existen también el comando \textit{Untalk} y \textit{Unlisten}, empleados para desdireccionar a los posibles \textit{listeners} y al posible \textit{talker} de un momento determinado.
%
%
%\subsubsection{Protocolo de \textit{Polling}.}
%
%Como se ha visto anteriormente, en el bus GPIB existe una línea de control, la SRQ, que activada informa al \textit{controller} sobr la necesidad de atención por parte de uno de los instrumentos conectados al bus. Para descubrir quién ha activado esta línea, existen dos métodos: el \textit{serial-polling} y el \textit{parallel-polling}.
%
%Al generarse un \textit{serial poll}, el \textit{controller} envía un mensaje de comando SPE (\textit{Serial Poll Enable}) a cada dispositivo de forma serializada y especificando su dirección. De esta forma cuando dicho dispositivo se direcciona como \textit{talker}, éste contesta al \textit{controller} indicando si ha sido él o no el que ha realizado un SRQ. Al recibir el \textit{controller} este byte, el controller devuelve a este mismo instrumento un SPD (\textit{Serial Poll Disable}) con el que el instrumento regresa a su estado normal de \textit{talker} o \textit{listener}. En caso de ser un \textit{parallel poll}, el envío del SPE se realiza en paralelo, pudiendo el \textit{controller} identificar al instrumento demandando servicio de forma más rápida y eficiente. 
 %
%\newpage
%\subsection{Conexión de interfaces IEEE 488.}\label{gpib_scheme}
%
%En la presente sección, se indicará al lector los pasos de instalación y estructura de conexionado que se ha empleado para habilitar un control centralizado de los 4 dispositivos dotados con este tipo de interfaz en un único ordenador personal con una \textit{GPIB 82350A PCI Interface Card} de Agilent Technologies (Figura \ref{fig:tarj_gpib}) basada en IEEE 488.
%
%\smallskip
%\begin{figure}[H]%here
%\noindent \begin{centering}
%\includegraphics[scale=0.65]{capitulo7/busGPIB/agilent82350A}
%\par\end{centering}
%\smallskip
%\caption{\label{fig:tarj_gpib} GPIB 82350A PCI Interface Card de Agilent \cite{AGILGPIB}.}
%\end{figure}
%
%Lo primero que debemos de realizar antes de montar la interfaz PCI o conectar algún dispositivo, es instalar la suite de librerías Agilent para el control de los puertos de entrada/salida del ordenador personal. Concretamente la suite se llama \textit{IO Agilent Libraries Suite v16.3}. Estas librerías habilitan la comunicación con instrumentos para una gran variedad de entornos de desarrollo compatibles con puertos GPIB, USB, LAN, RS-232, PXI, AXIe y VXI de diferentes fabricantes. Una vez instalado el software en el PC (Figura \ref{fig:iosuite_pic1}), podemos realizar el montaje de la \textit{PCI interface card} y ejecutar el software para activar la interfaz y comprobar la funcionalidad de la misma.
%
%\smallskip
%\begin{figure}[H]%here
%\noindent \begin{centering}
%\includegraphics[scale=0.8]{capitulo7/busGPIB/capture_iosuite0}
%\par\end{centering}
%\smallskip
%\caption{\label{fig:iosuite_pic1} Icono IO Suite 16.3 en el taskbar de Windows.}
%\end{figure}
%
%La conexión de la tarjeta GPIB PCIIA al PC se realiza mediante un zócalo de 80 conexiones presente en la gran mayoría de ordenadores. El bus de la tarjeta funciona a 16 bits, si bien, puede conectarse con un equipo cuyo bus sea de 32 bits si se dispone de un equipo en tal disposición, mediante el controlador gpib-32.dll incluido en las mencionadas librerías.
%
%Una vez ejecutado el software de librerías I/O, nos aparecerá una pantalla en la que se detallan los diferentes puertos detectados en el PC, el GPIB entre ellos (Figura \ref{fig:gpib_suiteinterface}(a)). Si seleccionamos la interfaz GPIB y hacemos click en el botón \textit{Properties} accedemos al panel de configuración de la interfaz GPIB 82350A, donde podemos modificar su dirección GPIB, los identificadores VISA y SICL, además de la unidad lógica (permite diferenciar entre varias tarjetas de interfaz GPIB PCI instaladas en el mismo equipo) (Figura \ref{fig:gpib_suiteinterface}(b)). En nuestro caso hemos dejado los valores por defecto, que asignan a esta interfaz la dirección 21 y la unidad lógica 8.
%
%\begin{figure}[H]%here
%\noindent \begin{centering}
%\subfloat[]{\includegraphics[scale=0.35]{capitulo7/busGPIB/capture_suite1}}
%\hspace{0.1cm}
%\subfloat[]{\includegraphics[scale=0.5]{capitulo7/busGPIB/prop}}
%\vspace{0.5cm}
%\smallskip
%\caption{\label{fig:gpib_suiteinterface} Interfaz GPIB detectada y configurada.}
%\par\end{centering}
%\end{figure}
%
%El siguiente paso es determinar una dirección GPIB para cada instrumento que vamos a conectar por medio de este bus. Generalmente, cada instrumento posee un conmutador DIP de 7 interruptores en el panel trasero, por medio del cual se fija la dirección con un binario equivalente a un decimal en el rango 0-30, ya que el máximo número de dispositivos conectados a la misma tarjeta es 31. Cada dirección tiene una dirección primaria y otra secundaria, aunque en este caso solo emplearemos la dirección primaria. A continuación incluimos las direcciones de los dispositivos utilizados en este proyecto (Tabla \ref{tab:devices_addr}).
%
%\begin{table}[H]
%\begin{center}
%\bigskip
%\includegraphics[scale=0.9]{capitulo7/busGPIB/gpibdirec}
%\smallskip
%\caption{Direcciones GPIB de los dispositivos utilizados.}
%\label{tab:devices_addr}
%\end{center}
%\end{table} 
%
%Para la conexión de los dispositivos se ha utilizado cables 10833A y 10833D de Agilent, de 1 metro y 0.5 metros respectivamente, acoplados según la forma indicada en la Figura \ref{fig:conex_pic}, que es la recomendada por el fabricante para la conexión de más de un instrumento.
%
%\smallskip
%\begin{figure}[H]%here
%\noindent \begin{centering}
%\includegraphics[scale=1]{capitulo7/busGPIB/conex4_gpib}
%\par\end{centering}
%\smallskip
%\caption{\label{fig:conex_pic} Esquema del conexionado de dispositivos mediante cables GPIB de 24-pines.}
%\end{figure}
%
%Agilent recomienda determinadas restricciones en cuanto a la longitud del cable y número de elementos conectados para alcanzar determinadas tasas de datos en la comunicación por el bus. Para minimizar el riesgo de sobrecarga, no se deben acoplar más de tres tomas de 24-pines una encima de otra. Si queremos alcanzar tasas de transferencia superiores a 500 Kbytes/sec la longitud total de cable utilizado debe ser menor que 1 metro por el número de instrumentos conectados juntos. 
%
%Una vez conectados los dispositivos, al encenderlo y pulsar el botón \textit{Refresh} del software, deberían aparecernos detallados bajo la interfaz a la que han sido conectados, en nuestro caso a la GPIB1, como muestra la Figura \ref{fig:gpib_suitedevices}. Podemos ver las propiedades de cada dispositivo en la parte derecha de la ventana. Notar que algunos dispositivos no son reconocidos mediante su nombre comercial, sino que algunos son nombrados con la primera cadena de caracteres que devuelven al mandarles un comando de petición de identificación (IDN?), y este en algunos casos es el nombre o una fecha de fabricación, etc, por lo que es conveniente determinar cuál es cada uno por medio de su dirección GPIB.
%
%\smallskip
%\begin{figure}[H]%here
%\noindent \begin{centering}
%\includegraphics[scale=0.35]{capitulo7/busGPIB/capture_suite2}
%\par\end{centering}
%\smallskip
%\caption{\label{fig:gpib_suitedevices} Dispositivos GPIB reconocidos.}
%\end{figure}
%
%Además de la tarjeta GPIB 82350A PCI de Agilent, disponemos de un conector GPIB/USB modelo 82357A (Figura \ref{fig:gpib_usbdevice}) de Agilent que permite la comunicación de hasta 14 dispositivos interconectados mediante conexiones GPIB de 24-pines a un PC por medio de un puerto USB. 
%
%\smallskip
%\begin{figure}[H]%here
%\noindent \begin{centering}
%\includegraphics[scale=0.65]{capitulo7/busGPIB/usb_gpibpic}
%\par\end{centering}
%\smallskip
%\caption{\label{fig:gpib_usbdevice} Conector GPIB/USB 82357A de Agilent \cite{AGILGPIB}.}
%\end{figure}
%
%Para tal utilización, la conexión debe realizarse siguiente el esquema propuesto en la Figura \ref{fig:usbgpib_esquema}. Notar también que para esta interfaz, la dirección GPIB de la interfaz es también 21, pero la unidad lógica es 7, para diferenciarse de la otra en caso de que se usen simultáneamente.
%
%\smallskip
%\begin{figure}[H]%here
%\noindent \begin{centering}
%\includegraphics[scale=1.2]{capitulo7/busGPIB/conexion_usbgpib}
%\par\end{centering}
%\smallskip
%\caption{\label{fig:usbgpib_esquema} Esquema de conexión GPIB/USB con el modelo 82357A \cite{AGILGPIB}.}
%\end{figure}
%
%
%\section{Matlab Instrument Control Toolbox.}
%
%Como el software será definitivamente implementado en Matlab, se ha hecho uso de la funciones ofrecidas por la herramienta \textit{Instrument Control Toolbox} para conectar con los equipos que vamos a utilizar de forma remota a través de nuestra estación de trabajo.
%
%\smallskip
%\begin{figure}[H]%here
%\noindent \begin{centering}
%\includegraphics[scale=0.3]{capitulo7/busGPIB/logo_mictg}
%\par\end{centering}
%\smallskip
%\caption{\label{fig:dibujo_mict} Instrument Control Toolbox \cite{ICTGPIB}.}
%\end{figure}
%
%Instrument Control Toolbox nos permite por tanto controlar y comunicarnos con dispositivos externos a través de los protocolos de comunicación GPIB y VXI directamente desde Matlab. En nuestro caso, hemos utilizado este toolbox para poder llevar a cabo el control de instrumentos a través de la programación directa del bus GPIB, Figura \ref{fig:tbmh_esquema}.
 %
%\smallskip
%\begin{figure}[H]%here
%\noindent \begin{centering}
%\includegraphics[scale=1.2]{capitulo7/busGPIB/matlab_control_toolbox}
%\par\end{centering}
%\smallskip
%\caption{\label{fig:tbmh_esquema} Esquema de control de instrumentos vía GPIB con Matlab \cite{ICTGPIB}.}
%\end{figure}
%
%\newpage
%
%Para realizar una comunicación con un instrumento podemos utilizar diferentes buses de comunicación siempre y cuando el instrumento disponga de ellos. Para acceder al bus desde un PC, es necesaria una tarjeta controlador del bus, ya sea GPIB, VXI o cualquiera de ellos. Para acceder a cada una de estas tarjetas podemos utilizar las funciones propias del bus, como por ejemplo los comandos SCPI para el bus GPIB. Pero la utilización de éstas, fuerza que las aplicaciones que desarrollamos para un determinado instrumento sirven únicamente para ese bus y ese instrumento. Por ejemplo, si utilizamos comandos SCPI para controlar un multímetro vía bus GPIB, no podremos utilizar dicho programa para controlar el mismo instrumento utilizando un bus diferente. Para solucionar este y otros problemas semejantes, en 1993 National Instruments junto con GenRad, Racal Instruments, Tektronix y Wavetek formaron un consorcio llamado VXI plug and play Systems Aliance. Uno de los estándares más desarrollados por este grupo fue VISA (Virtual Instrument Software Architecture), que es un conjunto de funciones de alto nivel que se encarga de hacer transparente los recursos software que estemos utilizando.
%
%\subsection{Funciones GPIB.}
%
%La herramienta Matlab Instrument Control Toolbox ofrece una amplia variedad de funciones para llevar a cabo la comunicación con el instrumento por medio de diferentes tipos de interfaces. En nuestro caso nos centraremos en aquellas destinadas a la interfaz GPIB y a la creación de objetos de este tipo.
%
%Entre todas ellas destacamos las siguientes:
%
%\begin{itemize}
%
%\item \textbf{\textit{gpib}:} permite crear objetos de tipo gpib indicándole el fabricante del instrumento, el número de tarjeta GPIB y la dirección del instrumento en cuestión.
%
%\item \textbf{\textit{fopen}:} una vez se ha creado el objeto de tipo gpib, se crea la conexión entre el objeto de la interfaz GPIB y el instrumento.
%
%\item \textbf{\textit{fclose}:} cierra la conexión entre el objeto de la interfaz GPIB creado y el instrumento en cuestión.
%
%\item \textbf{\textit{fprintf}:} escribe cadenas de texto ASCII en el instrumento.
%
%\item \textbf{\textit{fscanf}:} permite escanear el buffer del instrumento, obteniendo datos en formato texto ASCII.
%
%\item \textbf{\textit{clrdevice}:} borra el buffer del instrumento.
%
%\item \textbf{\textit{disp}:} muestra un resumen detallado del objeto creado.
%
%\end{itemize}
%
%
%\subsubsection{SCPI.}
%
%Una gran ventaja a la hora de mandar los comandos de instrucción a cada dispositivo enlazado vía GPIB por medio de las funciones Matlab Instrument Control Toolbox, es que la totalidad de los instrumentos utiliza cadenas de caracteres en formato ASCII para definir los códigos de instrucciones.
%
%La norma SCPI (Standard Commands for Programmable Instruments) aparece en 1991 para conseguir una estandarización de los comandos de control y el formato de los datos de los instrumentos. El objetivo es que, independientemente del fabricante, equipos que tienen la misma funcionalidad respondan de igual forma a un conjunto estándar de comandos. La norma SCPI es el escalón más alto dentro de la jerarquía normativa para el control de sistemas de instrumentación. Tal como se puede ver en la figura I.14, la norma SCPI se asienta sobre la IEEE-488.2 y esta, a su vez, en la IEEE-488.1, Figura \ref{fig:ieee488strc_esquema}.
%
%
%\smallskip
%\begin{figure}[H]%here
%\noindent \begin{centering}
%\includegraphics[scale=0.9]{capitulo7/busGPIB/estructura_ieee488}
%\par\end{centering}
%\smallskip
%\caption{\label{fig:ieee488strc_esquema} Esquema de control de instrumentos vía GPIB con Matlab.}
%\end{figure}
%
%A pesar de esta jerarquía los comandos y la estructura de datos basados en la norma SCPI pueden usarse, y se usan, en sistemas de instrumentación que no estén basados en IEEE-488, por ejemplo en sistemas basados en VXI, RS-232 o LAN.
%
%La norma SCPI reduce los costes de desarrollo y mantenimiento de programas de control de sistemas de instrumentación para pruebas automáticas. Esto se consigue ya que:
%
%\begin{itemize}
%
%\item Facilita el aprendizaje y uso de los comandos y los datos.
%
%\item Facilita el desarrollo y mantenimiento de los programas.
%
%\item Posibilita la sustitución de equipos con los mínimos cambios de software.
%
%\end{itemize}
%
 %
%Los manuales de programación de las instrucciones SCPI, para controlar las funcionalidades de los cuatro instrumentos a través del bus GPIB, se encuentran referenciados en el apéndice.
%
%\begin{itemize}
%
%\item HP 4145B $[$\hyperlink{codesHP4145B.1}{ref}$]$.
%
%\item HP 3478A $[$\hyperlink{codesHP3478A.1}{ref}$]$.
%
%\item KEPCO BOP 50-8 $[$\hyperlink{codesKTH220.1}{ref}$]$.
%
%\item Keithley 220 $[$\hyperlink{codesKEPCO508.1}{ref}$]$.
%
%\end{itemize}
%
%
%\newpage
%
%\section{Puerto serie RS-232.}
%
%Se trata de un antiguo estándar de comunicación serie que se diseñó para la comunicación entre un equipo terminal de datos (DTE), como un computador y un equipo de comunicaciones de datos (DCE), como un módem. 
%
%\smallskip
%\begin{figure}[H]%here
%\noindent \begin{centering}
%\includegraphics[scale=0.55]{capitulo7/busRS232/rs_232_logo}
%\par\end{centering}
%\smallskip
%\caption{\label{fig:rs232bus_logo} Estandar RS-232 \cite{WIKIPED}.}
%\end{figure}
%
%Es un enlace de tipo full dúplex y punto a punto. La distancia máxima que abarca sin ningún circuito de ampliación es de aproximadamente 15 m (en la práctica puede llegar a funcionar hasta con distancias de 100 m) y su velocidad típica ronda los 19000 baudios aunque puede ser superior. Este tipo de enlace dispone de unas líneas de datos y unas líneas de control sobre las que se implementa el protocolo de comunicación (ciertas señales presentes en el conector que permiten el bloqueo de la transmisión o la recepción). 
%
%La norma básica se ocupa de las especificaciones físicas del conector, los niveles de tensión de las señales y las señales de protocolo. Se suele utilizar un conector de 25 pines. Cuando no son necesarias todas las señales se puede adoptar un conector de 9 pines. Los niveles lógicos 0 y 1 se representan por los niveles físicos de tensión que muestra la Figura \ref{fig:rs232bus_crono}.
%
%\smallskip
%\begin{figure}[H]%here
%\noindent \begin{centering}
%\includegraphics[scale=0.9]{capitulo7/busRS232/rs232_cronogram}
%\par\end{centering}
%\smallskip
%\caption{\label{fig:rs232bus_crono} Cronograma de señales durante comunicación RS-232 \cite{WIKIPED}.}
%\end{figure}
%
%Cualquiera que sea el tipo de transmisión es necesario que el receptor se sincronice para saber en todo momento donde comienza la transmisión de un bit, un carácter o un bloque. Cada carácter va precedido de un bit de inicio (bit de Start) y finaliza con 1 ó 2 bits de parada (bits de Stop), que garantizan la sincronización del receptor y permiten el reconocimiento del comienzo y el final del carácter.
%
%Para el caso del gaussímetro GM08 de Hirst Magnetic, utilizamos una conexión vía este tipo de puerto para comunicarnos. Las especificaciones de configuración del puerto están recogidas en el manual de configuración del instrumento \cite{GM08FIELD}.
%
%\subsection{Protocolo de comunicación con Gaussímetro GM08 vía RS-232.}
%
%En esta sección se pretende dar una idea sobre el protocolo de comunicación empleado con el gaussímetro para configurarlo y obtener valores de campo magnético medidos por su sonda axial o transversal. 
%
%Lo primero que hicimos fue conectarlo al PC, siguiendo las indicaciones del fabricante (Figura \ref{fig:datos_para_conex_gm}).
%
%\begin{figure}[H]%here
%\noindent \begin{centering}
%\subfloat[]{\includegraphics[scale=0.8]{capitulo7/busRS232/pines_frabrica_gmo8}}
%\hspace{0.1cm}
%\subfloat[]{\includegraphics[scale=0.8]{capitulo7/busRS232/espec_frabrica_gmo8}}
%\vspace{0.5cm}
%\smallskip
%\caption{\label{fig:datos_para_conex_gm} a) Líneas del bus. b) Especificaciones del puerto.}
%\par\end{centering}
%\end{figure}
%
%Una vez conectado, utilizamos un rastreador de puerto serie para tratar de entender la comunicación controlando el gaussímetro manualmente, en primera instancia. El rastreador de puerto puede verse configurado y listo para iniciar una conexión en la Figura \ref{fig:rastreator_dpuerto}.
%
%\smallskip
%\begin{figure}[H]%here
%\noindent \begin{centering}
%\includegraphics[scale=0.4]{capitulo7/busRS232/rastreador_puerto}
%\par\end{centering}
%\smallskip
%\caption{\label{fig:rastreator_dpuerto} Rastreador de puerto serie RS-232.}
%\end{figure}
%
%A partir de un estudio analítico de la transferencia de datos por medio del puerto serie y ayudándonos de la librería GM0.dll escrita en C, pudimos realizar un proceso de ingeniería inversa para extraer el protocolo empleado por el dispositivo y el host durante la comunicación. Una breve introducción sobre dicho mecanismo de intercambio de datos se muestra en las siguientes líneas.
%
%El gaussímetro tiene dos modos de funcionamiento bien diferenciados:
%
%\begin{itemize}
%
%\item \textbf{Data Mode:} donde, si la sonda del gaussímetro está conectada, este realiza medidas del campo magnético constantemente, y las manda por el puerto serie, en nuestro caso hasta el PC. Este es en el modo en el que debemos dejar funcionando al instrumento, al terminar de configurar alguna de sus opciones. 
%
%\item \textbf{Command Mode:} donde el gaussímetro deja de mandar medidas y espera a que el host le mande un comando para configurar alguna función de medida. Una vez se le manda dicho comando, es necesario devolverlo al \textit{Data Mode}.
%
%\end{itemize}
%
%\smallskip
%\begin{figure}[H]%here
%\noindent \begin{centering}
%\includegraphics[scale=0.9]{capitulo7/busRS232/diagrama_sec_gmo8}
%\par\end{centering}
%\smallskip
%\caption{\label{fig:diagram_mensajes_gausim} Diagrama de flujo de mensajes PC-GM08.}
%\end{figure}
%
%El dispositivo GM08 posee un byte de estado o \textit{Status Byte} que suele utilizarse para describir en qué situación se encuentra el dispositivo o para detectar errores en la comunicación. El gaussímetro siempre contesta con su \textit{Status Byte} después de recibir cualquier dato proveniente del host, de tal manera que no se le puede enviar ningún comando sin haber recibido previamente su \textit{Status Byte}.
%
%En la Figura \ref{fig:diagram_mensajes_gausim}, hemos incluimos un diagrama de flujo de mensajes entre el PC y el gaussímetro donde se ejemplifican las transiciones entre ambos modos de funcionamiento.
%
%A través de la librería GM0.dll extrajimos la totalidad de comandos admitidos por el gaussímetro, a fin de elaborar nuestra propia librería empleando funciones propias de matlab para el control de comunicación sobre este puerto.
%
%Para finalizar este tema, hacemos un breve nombramiento del conjunto de funciones más representativas utilizadas en matlab para configurar la librería de control de comunicación del GM08:
%
%\begin{itemize}
%
%\item \textbf{\textit{serial}:} permite crear objetos de tipo \textit{serial port object} indicándole las propiedades como los baudios de conexión, bits de datos, bits de paridad, etc.
%
%\item \textbf{\textit{fopen}:} una vez se ha creado el objeto de tipo \textit{serial port object}, se crea la conexión entre el objeto de la interfaz RS-232 y el instrumento.
%
%\item \textbf{\textit{fclose}:} cierra la conexión entre el objeto de la interfaz RS-232 creado y el instrumento en cuestión.
%
%\item \textbf{\textit{fwrite}:} escribe datos de tipo binario en el instrumento.
%
%\item \textbf{\textit{fread}:} permite leer datos binarios del buffer del instrumento, pudiéndose especificar la cantidad de valores a leer.
%
%\end{itemize}
%

%\newpage
%\cleardoublepage