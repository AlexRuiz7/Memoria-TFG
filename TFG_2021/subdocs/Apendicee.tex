%Begin ----  Para que funcione bien el TOC en PDF
\cleardoublepage
\phantomsection
\addcontentsline{toc}{chapter}{Apéndices}
%END  ---- Para que funcione bien el TOC en PDF

\chapter*{Apéndices}
%\newpage
%\thispagestyle{empty}
%\chapter[Orden INT/316/2011 sobre seguridad privada]{Orden INT/316/2011, de 1 de febrero, sobre funcionamiento de los sistemas de alarma en el ámbito de la seguridad privada.}


%\label{cap:reglamento_seguridad}

%\section[Planos de zonas no seguras]{Planos de las posibles zonas accesibles por un intruso sin que se avise a la policía}
\section{Códigos SCPI de instrucciones para los equipos controlados por bus GPIB}
\label{cap:codigos_scpi_equipos_bus_gpib}

A continuación incluimos los principales códigos SCPI de instrucciones en texto ASCII para cada uno de los cuatro instrumentos con los que nos hemos comunicado remotamente vía bus GPIB:

\begin{itemize}

\item HP 4145B Parameter Analyzer (\hyperlink{codesHP4145B.1}{ref}).
\item HP 3478A Digital Multimeter (\hyperlink{codesHP3478A.1}{ref}).
\item Keithley 220 Current Source (\hyperlink{codesKTH220.1}{ref}).
\item Kepco bop 50-8 I/V Source (\hyperlink{codesKEPCO508.1}{ref}).

\end{itemize}

\includepdf[pages=-,link=true,linkname=codesHP4145B,addtotoc={1,section,1,SCPI Codes HP4145B,letter}]{apendices/codigos_scpi/codigos_HP4145_ok}

\includepdf[pages=-,link=true,linkname=codesHP3478A,addtotoc={1,section,1,SCPI Codes HP3478A,letter}]{apendices/codigos_scpi/codigos_3478A_ok}

\includepdf[pages=-,link=true,linkname=codesKTH220,addtotoc={1,section,1,SCPI Codes KEITH220,letter}]{apendices/codigos_scpi/codigos_220_ok}

\includepdf[pages=-,link=true,linkname=codesKEPCO508,addtotoc={1,section,1,SCPI Codes KEPCO508,letter}]{apendices/codigos_scpi/codigos_Kepco_ok}


%-------------------------------------------------------------------------------------%
%\section[Planos de zonas no seguras]{Planos de las posibles zonas accesibles por un intruso sin que se avise a la policía}
%\label{cap:planos_zonas_no_seguras}
%\subsection{Sin seguridad perimetral}
%\label{cap:planos_sin_seg_perimetral}
%
%A continuación se presenta una serie de planos en los que se ven reflejadas las zonas por las que podría transitar un intruso que accediera a la vivienda por todas las puertas y ventanas posibles...
%\newline
%Se han sombreado con colores diferentes las zonas no seguras accediendo por distintos lugares de la casa, tal y como se indica en la figura \ref{fig:diagrama_colores}.
%
%\begin{figure}[H]%here
%\noindent \begin{centering}
%\includegraphics[scale=0.8]{apendices/diagramas/Diagrama_colores.pdf}
%\par\end{centering}
%\caption{\label{fig:diagrama_colores}Zonas no seguras según la vía de acceso.}
%\end{figure}

%\subsubsection{Zonas no seguras: primer caso}
%
%\includepdf[pages=-]{apendices/planos/sin_seguridad_perimetral/Recorrido1/1.pdf}


%-------------------------------------------------------------------------------------%





%\includepdf[pages=-,link=true,linkname=RunSheet,addtotoc={1,section,1,Listado Equipos Laboratorio,letter}]{documentos/tablas/listado_equipos_laboratorio.pdf}
%\addcontentsline{toc}{section}{Listado Equipos Laboratorio}
%
%\clearpage{\pagestyle{plain}\cleardoublepage}%

%\input{subdocs/Anexo_ESPICE.tex}

%\fancyhead{}
%\fancyhead[RO]{\small{\nouppercase{Run Sheet}}}
%\fancyhead[LE]{\small{\nouppercase{Run Sheet}}}



%\clearpage{\pagestyle{plain}\cleardoublepage}%



%Glosario
%%%%%%%%%%\clearpage{\pagestyle{fancy}\cleardoublepage}%

%Begin ----  Para que funcione bien el TOC en PDF
%%%%%%%%%%\cleardoublepage
%%%%%%%%%%\phantomsection \label{Glosar}
%END  ---- Para que funcione bien el TOC en PDF

%%%%%%%%%%\printglossary[type=main,title=Glosario,toctitle=Glosario]


%Listado de Símbolos
%%%%%%%%%%\clearpage{\pagestyle{fancy}\cleardoublepage}%

%Begin ----  Para que funcione bien el TOC en PDF
%%%%%%%%%%\cleardoublepage
%%%%%%%%%%\phantomsection \label{ListadoSimbolos}
%END  ---- Para que funcione bien el TOC en PDF

%%%%%%%%%%\printglossary[style=long3col,type=symbols,title=Listado de Símbolos,toctitle=Listado de Símbolos]
%\addcontentsline{toc}{chapter}{Símbolos}

%%Índice Alfabético de apariciones
%\clearpage{\pagestyle{fancy}\cleardoublepage}
%
%%Begin ----  Para que funcione bien el TOC en PDF
%\cleardoublepage
%\phantomsection \label{IndiceApariciones}
%%END  ---- Para que funcione bien el TOC en PDF
%
%\printindex

%Listado de Acrónimos
%%%%%%%%%%\clearpage{\pagestyle{fancy}\cleardoublepage}%

%Begin ----  Para que funcione bien el TOC en PDF
%%%%%%%%%%\cleardoublepage
%%%%%%%%%%\phantomsection \label{ListadoAcronimos}
%END  ---- Para que funcione bien el TOC en PDF

%%%%%%%%%%\printglossary[style=long3col,type=\acronymtype,title=Acrónimos,toctitle=Acrónimos]
%\addcontentsline{toc}{chapter}{Símbolos}
