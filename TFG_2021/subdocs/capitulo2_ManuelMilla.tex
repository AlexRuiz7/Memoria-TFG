\chapter{Equipment, software and planning.}
\label{cap:equipment_software_and_planning}

In this chapter I pretend to explain in a general view the equipment used in this project. A noise measure is not too much different to any other measure, but, it has some particularities that we have to take in account. We will sketch the software used, that will be programmed in MATLAB and finally, the chapter will ends with a Gantt's diagram showing the planning.

\section{Equipment}

Any measure require an specific equipment. Noise is a very tricky signal, because most of the time seems to be placed out of view, noise signal has to be insulated from any other source of noise, amplified and processed in either frequency or time domain. Figure \ref{fig:general_measuring} shows a sketch of the process described.


Figure \ref{fig:general_measuring} shows an arbitrary noise measurement and its steps.

\smallskip	
\begin{figure}[H]%here
\noindent \begin{centering}
\includegraphics[scale=1.2]{capitulo2/general_measuring}
\par\end{centering}
\smallskip
\caption{\label{fig:general_measuring} Measure flow.}
\end{figure}

In the previous figure, we have sketched the equipment available to our measurements. As showed:

\begin{itemize}
	\item DUT: Device Under Test
	\item SR560: Stanford low-noise pre-amplifier.
	\item HP35670A: Dynamic Signal Analyzer.
	\item PC: A MATLAB software installed is required.
\end{itemize}

\subsection{Device Under Test}
\subsection{SR560: Stanford Low-Noise Pre-Amplifier}
\subsection{HP35670A: Dynamic Signal Analyzer}


\section{Software}

Some of this measurements are hard to make without an automatized system. In this case, is possible to control HP35670A remotely through a GIPB bus. It will help us to configure the equipment setup quickly and with all parameters included.

\begin{figure}[H]%here
\noindent \begin{centering}
\subfloat[]{\includegraphics[scale=0.6]{capitulo2/logotipo_gpib}}
\hspace{0.5cm}
\subfloat[]{\includegraphics[scale=0.35]{capitulo7/busRS232/rs_232_logo}}
\vspace{0.1cm}
\bigskip
\caption{\label{fig:gpib_and_rs232_cap2} Comunicación vía GPIB y RS-232 \cite{WIKIPED}.}
\par\end{centering}
\end{figure}


The whole program \textit{software} will be programmed in MATLAB\textsuperscript{\textregistered},  R2011a version and Windows 7 y XP (x32).

The software shall do:

\begin{itemize}
	\item Configure the setup of a measurement:
	\begin{itemize}
		\item Resistor.	
		\item Photo diode.
		\item Save results.
		\item Load data into Origin.
	\end{itemize}
 
\smallskip	
\begin{figure}[H]%here
\noindent \begin{centering}
\includegraphics[scale=0.65]{capitulo2/requerimientos_1}
\par\end{centering}
\smallskip
\caption{\label{fig:esquema_sistem_caract} Resumen gráfico de la utilización del software.}
\end{figure}

\section{Planning}

In this project, we will deal with:

\begin{itemize}
	\item Resistors.
	\item Photo diodes.
\end{itemize} 

Measure a resistor noise can be show as a training to measure photo diodes noise, it will help us to understand the problems we face when measuring noise. The noise we expect to observe is flicker and thermal noise in resistors and these two plus shot noise in photo diodes.

To protect the measurements from power line signals, a shielding is required. In this case we will use two of them as shown in figure \ref{fig:shielding}.

\begin{figure}[H]%here
\noindent \begin{centering}
\subfloat[Pista de corriente.]{\includegraphics[scale=0.15]{capitulo5/current_strap_and_sv}}
\hspace{0.5cm}
\subfloat[Bobinas de Helmholtz.]{\includegraphics[scale=0.3]{capitulo6/bobinas/helmholtz}}
\vspace{0.1cm}
\bigskip
\caption{\label{fig:shielding} Shielding used.}
\par\end{centering}
\end{figure} 

In L4 lab, we have the following electronic equipment. A list is showed below.

\begin{itemize}
  \item [-] SR560 Stanford Low-Noise Pre-Amplifier. 
	\item [-] Dynamic Signal Analizer HP35670A.
\end{itemize}

Just with this equipment we would be able to measure resistor and photo diode's noise.

A continuación expondremos las fases constitutivas del transcurso del proyecto, Figura \ref{fig:fases_proy_imagen}, describiendo su contenido de manera generalizada.

Una vez expuestas las fases, se considerarán fechas iniciales y finales para cada tarea, a fin de contemplar que alcance temporal que implica el desarrollo del proyecto al completo. Las fechas marcarán el transcurso del trabajo.

\smallskip	
\begin{figure}[H]%here
\noindent \begin{centering}
\includegraphics[scale=0.8]{capitulo2/requerimientos_3}
\par\end{centering}
\smallskip
\caption{\label{fig:fases_proy_imagen} Principales fases de un proyecto \cite{WIKIPED}.}
\end{figure}

\newpage
Las diferentes fases son:

\begin{enumerate}
 \item[-] \textbf{FASE I- Especificaciones del sistema:} realizaremos un estudio detallado de los requerimientos del sistema de caracterización. Para ello será necesario obtener documentación sobre el efecto magnetorresistivo y su estado del arte actual.

 \item[-] \textbf{FASE II- Análisis y Diseño:} en esta fase se agrupan todos los requerimientos y especificaciones de la fase anterior y se trata de concebir la composición del sistema. Repasamos el efecto magnetorresistivo en detalle, así como su estado del arte. Consideraremos todos aquellos elementos \textit{hardware}, \textit{software} y estructurales necesarios. Analizaremos las funcionalidades de cada equipo electrónico y/o mecánico utilizado, determinando su utilidad dentro del sistema. Diseñaremos y simularemos, mediante un \textit{software} específico, el comportamiento de las bobinas de Helmholtz y las pistas de corriente antes de fabricarlas. También analizaremos el protocolo de comunicación GPIB y RS-232.

 \item[-] \textbf{FASE III- Desarrollo e Implementación:} se fabrica todo el componente hardware detallado en la fase I y se realiza el conexionado de la instrumentación involucrada. Una vez analizados los diferentes equipos, comenzamos a programar sus librerías de comunicación, ya sean GPIB o RS-232. Tras realizar las primeras pruebas de comunicación y validar su correcto funcionamiento, implementaremos el \textit{software} definitivo en MATLAB\textsuperscript{\textregistered}. Desarrollaremos las interfaces gráficas de usuario, los algoritmos de medida y procesado de datos, e integraremos las librerías de comunicación de cada instrumento.

 \item[-] \textbf{FASE IV- Testeo y Validación:} una vez se ha concluido el proceso de fabricación e implementación, solo queda validar el funcionamiento global del conjunto, resolviendo defectos y anomalías \textit{hardware} y también depurando fallos \textit{software} y aportando algunas mejoras puntuales. Tras esto, la obtención de datos y resultados.
\end{enumerate}


\smallskip	
\begin{figure}[H]%here
\noindent \begin{centering}
\includegraphics[scale=0.5]{capitulo2/requerimientos_2}
\par\end{centering}
\smallskip
\caption{\label{fig:fases_proy_step} Analogía de fases.}
\end{figure}

Además de la analogía de fases expuesta, Figura \ref{fig:fases_proy_step}, se incluye un diagrama de Gantt, que muestra el tiempo de dedicación previsto para las diferentes tareas enmarcadas dentro de cada fase, ahora sí, desglosadas de forma más concreta.

Dicho diagrama de Gantt se ha realizado a través de una herramienta \textit{software} gratuita llamada Open Workbench. 

\clearpage{\cleardoublepage}

%\clearpage{\pagestyle{empty}\cleardoublepage}

%\includepdf[pages={34-40},nup=2x2,frame,landscape,scale=0.8,%
%pagecommand={}]{texbook.pdf}

%\newpage
\includepdf[pages=-,link=true,landscape,linkname=DiagramaDeGantt,addtotoc={1,section,1,Diagrama de Gantt,letter}]{capitulo2/Diagrama_Gantt_PFC}

\clearpage{\pagestyle{empty}\cleardoublepage}

