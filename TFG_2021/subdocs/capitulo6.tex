\chapter{Conclusions and Future Lines}\label{chap:chapter6}

In this document, we have presented the development of the Star Tracker used in the GranaSAT experiment. This subsystem, due to its complexity, is the most demanding of the three \gls{ADS} implemented. Firstly, we had to deal with a complete new field as it is the aerospace one. For a Telecommunication Engineering student as the author of this document is it was arduous to get used to all the attitude concepts, the star trackers, the star pattern recognition algorithms and its implementation to make then efficient and reliable.  

Not only the academic problems affected the development of this work, in our case, the budget restriction was also a main bone of contention. It affected all the stages of the project, i.e., from the material acquisition to the travel expenses to Holland and Kiruna. In spite of that, the whole team is proud of the work carried out. We have proven that the greatest hits can be achieved with effort rather than a huge budget. 

In this entire year working in an aerospace project, we have learnt a little of about this industry works, its standards and how important a good documentation is, as well as we have learnt how to work in teams. Students usually tend to think that they can work in groups, however, we all realised that it was an illusion. Only after getting involved in such a demanding project as the REXUS/BEXUS, one can say that the ability to work in groups has been learnt.

The Star Tracker developed is yet far from being implemented, as it is still a student desgin nano-satellite. However, we think we have lain the groundwork for this purpose. The \gls{LIS} algorithm implemented has proven reliable, it returned a 97.80\% of accuracy in identifying stellar fields. The non-correct ones are within the expected, as the reason that makes the Star Tracker \gls{LIS} routine fail cannot be controlled by us, i.e, gondola motion, aurora borealis and other objects as planets.

A large future work could be done with this work as a basis. From the author's point of view, it would be interesting to implement an algorithm to discard objects like planets and other spacecrafts if they are bright enough to appear within the three most brilliant objects in the image. We have seen our algorithm working well in absence of these objects. It also can discard them in the previous stages of the algorithm, however, a future \gls{LIS} algorithm should implement a procedure to discard them as stars in the previous stage of the algorithm, when the first triplet of stars is found.

The implementation of the tracking algorithm is the next step for the true implementation of this star tracker. In this document we have focused our effort on the programming and testing of the \gls{LIS} routine as well as one of the attitude determination method used, the SVD. The tracking routine will complete this work, unfortunately, its implementation was impossible due to scheduling constraints.

We would like to conclude this discussion by remarking the support of the \gls{ESA}, \gls{SNSB} and \gls{DLR} organizations for supporting and sponsoring the REXUS/BEXUS programme. This experience has been the most rewarding one for 10 students of the \gls{UGR}, in which is included the author of this document. Thank you all for sharing your knowledge, experience and invaluable time. 




























