\chapter{System Design} \label{chap:chapter4}

In this Chapter of the project will be specified the details of the design of every subsystem. In the design of each part, we should apply the requirements given in the Chapter \ref{chap:chapter2}, in the Section \ref{sec:requirements}.
In order to have the best solution for every subsystem, in chapter \ref{chap:chapter3} we have done an Analysis of other projects with different ideas. 

The Block diagram shown in the Chapter \ref{chap:chapter2}, in the Figure \ref{fig:diagram}, shows every subsystem that the \gls{GranaSAT} Testbed has. Now, we will see the design of the whole system, and the specifications of each part. 

\section{Electronic Design} \label{sec:electronicdesign}

The \acrshort{PCB} of the Testbed will have the objective of connect the sensors, actuators, LCD to show how is working, the microprocessor that controls everything, and the power supply part. That \acrshort{PCB} was designed in Altium Designer 14, a \acrshort{PCB} design tool which include the Schematics design, \acrshort{PCB} design and 3D model of the \acrshort{PCB}. Last option is very useful for the study of the dimension of that \acrshort{PCB}, because it can be used to prove if the design fits with the 3D model of the mechanical parts (see SECTIONOOONNON).

\subsection{Electronic Schematics and components descriptions} \label{ssec:SCHdesign}

In the following page is shown the schematic of the \acrshort{PCB}, with every subsystem included on it.

\includepdf[pages=-,link=true,landscape,linkname=gant]{capitulo4/TotalSCH}

Then, we will see every part of the schematic designed in more detail. 

\subsubsection{Microprocessor} \label{sssec:micro}
The microprocessor is from Atmel Corporation, model ATMEGA328P-AU. That microprocessor is the same as Arduino UNO cite{arduinoUNOweb}, but in the Testbed board, it will be in \acrshort{SMD} mode to save space and weight for the \acrshort{PCB}. The pinout of this microprocessor is shown in Figure \ref{fig:micropinout}.


\begin{figure}[H]
	\centering
		\includegraphics[scale=0.4]{capitulo4/micropinout.jpg}
	\caption{Pinout ATMEGA328P-AU}
	\label{fig:micropinout}
\end{figure}

The microprocessor will be the master for the communication with the sensors in I2C protocol, and the wiring of these devices is represented on Figure \ref{fig:i2cdiagram}.

The schematic designed for the control of the microprocessor is shown in Figure \ref{fig:microsch}.

  
\begin{figure}[H]
	\centering
		\includegraphics[scale=0.5]{capitulo4/microsch.jpg}
	\caption{Schematic for ATMEGA328P-AU}
	\label{fig:microsch}
\end{figure}

The first symbol that we can see in the left of the picture above, is the ICSP port. That port is used for the communication between a programmer and the microprocessor, with the objective of program it. The programmer used, as described in the Appendix \ref{cap:uploading}, is the USBtinyISP. 

The microprocessor is programmed for use it as an Arduino UNO, so the external clock installed is of 16MHz. That clock is made with a crystal oscillator and their capacitors connected to the pins ROSC1 and TOSC2 of the ATMEGA328P-AU.

Moreover, to reset the testbed, there is a button (shown in figure \ref{fig:touch}) with a pull-up resistor of $10k\Omega$. 
\begin{figure}[H]
	\centering
		\includegraphics[scale=0.4]{capitulo4/touch.jpg}
	\caption{Reset button}
	\label{fig:touch}
\end{figure}

Finnaly, the last part of the schematic of the microprocessor is the \acrshort{SMD} LEDs notificators:
\begin{itemize}
\item LED Blink (yellow color):  One of the most common LEDs in an Arduino Board is the LED Blink, a LED connected to the pin number 13 of the Arduino. It is very useful for the tests of the microprocessor, to show how it is working in each moment when a problem appears. 
\item LED ON (green color):  This LED is used to know if the ATMEGA328P-AU is turned ON. 
\end{itemize}
The resistors value used for these LEDs was calculated measuring the voltage and current consumption of each one:

\begin{table}[H]
\centering
\begin{tabular}{|l|l|l|}
\hline
						& \textbf{Green LED} & \textbf{Yellow LED} \\ \hline
Operating Voltage (V) & 2.1							& 2.1     \\ \hline
Datasheet Current (mA) & 20							& 20     \\ \hline
Measured Current (mA) & 17.4							& 16.5    \\ \hline
Theoretic Value Resistor ($\Omega$) & 166.67						& 166.67    \\ \hline
Normalized Value Resistor ($\Omega$) & 200						& 200   \\ \hline
\end{tabular}
\caption{Calculations of the resistor for the LEDs}\label{tab:LEDsvalue}
\end{table}

The Equation \ref{eq:LED} was used for these calculations.

\begin{equation}
R =  \frac{V_{dd}-V_{LED}}{I_{LED}}
\label{eq:LED}
\end{equation}

Where $ V_{dd}$ is the voltage supply for the LED and the resistor (5V), $ V_{LED}$ is the voltage that the LED supports for a correct operation (2.1V), and $I_{LED}$ is the current consumption that the LED have had in the measurements.

The device that controls the Blink LED (LMV358IDGKR) is an operational amplifier used as a comparator to turn ON/OFF the LED. \cite{LMV358IDGKR}.

\subsubsection{Display LCD} \label{sssec:LCDdesign}
The \acrshort{LCD} will be very useful in the prototype version of the Testbed. It will be used for seeing the data from the sensors and the \acrshort{GS} in real time. 
Moreover, the \acrshort{LCD} will be used to facilitate the visualization of the data without the use of the \acrshort{GS}, testing the Testbed without the wireless connection.  

The Figure \ref{fig:LCDsch} shows the schematic symbol for the \acrshort{LCD}, which consisted of 4 pins. That LCD has a I2C converter incorporated on it, because of the insufficient number of pins that the microprocessor has and its easy functionality (see Figure \ref{fig:lcdi2c}).

  
\begin{figure}[H]
	\centering
		\includegraphics[scale=0.4]{capitulo4/lcdsch.jpg}
	\caption{LCD Schematic}
	\label{fig:LCDsch}
\end{figure}

  
\begin{figure}[H]
	\centering
		\includegraphics[scale=0.4]{capitulo4/lcdi2c.JPG}
	\caption{LCD to I2C converter}
	\label{fig:lcdi2c}
\end{figure}

The pinout of these device is the following:
\begin{itemize}
\item VCC: 5V voltage supply.
\item GND: Pin connected to the ground of the circuit. 
\item SDA: I2C wire dedicated to the data signal. Connected to the PIN AD4 of the microprocessor.
\item SCL: I2C wire dedicated to the clock signal. Connected to the PIN AD5 of the microprocessor.
\end{itemize}

The LCD to I2C converter has a potentiometer to regulate the contrast of the LCD, and a jumper to choose if the back-light is turned ON or turned OFF. The schematic of this converter is shown in the Figure \ref{fig:schematicconverter}. A photo of the converter and the LCD is shown in the Figure \ref{fig:LCDphoto}.

The LCD choose for the Testbed has a HD4480 controller, whose datasheet is found on the reference \cite{HD4480}.


  
\begin{figure}[H]
	\centering
		\includegraphics[scale=0.4]{capitulo4/schematicconverter.jpg}
	\caption{LCD to I2C converter schematic \cite{LCDschematic}}
	\label{fig:schematicconverter}
\end{figure}

  
\begin{figure}[H]
	\centering
		\includegraphics[scale=0.6]{capitulo4/LCDfoto.jpg}
	\caption{Photo of the I2C converter and the LCD \cite{LCDphotoref}}
	\label{fig:LCDphoto}
\end{figure}

\subsubsection{Magnetometer and Accelerometer Sensor} \label{sssec:MagAc}

For the correcting use of the \acrshort{ACDS}, it is necessary some sensors.  The magnetorquers work with the Earth's Magnetic Field, so it should be measured by a sensor, in that case, a magnetometer. The magnetometer sensor chosen is the LSM303DLHC model (Figure \ref{fig:LSM303}), of STMicroelectronics. The datasheet of this sensor is found on the reference \cite{lsm303}. 
  
\begin{figure}[H]
	\centering
		\includegraphics[scale=0.4]{capitulo4/lsm303.jpg}
	\caption{LSM303DLHC chip \cite{lsm303}}
	\label{fig:LSM303}
\end{figure}
In order to have an easier soldering process, a test board (GY-511) for this sensor has been used. (see figure \ref{fig:GY511} and \ref{fig:GY511_2} for the schematic of this test board).
That sensor has a 3D digital linear acceleration sensor and a 3D digital magnetic sensor. The following table shows the main characteristics.
  
\begin{figure}[H]
	\centering
		\includegraphics[scale=0.27]{capitulo4/GY511.jpg}
	\caption{Photo of the the test board for LSM303DLHC}
	\label{fig:GY511}
\end{figure}

  
\begin{figure}[H]
	\centering
		\includegraphics[scale=0.5]{capitulo4/GY511_2.jpg}
	\caption{Schematic of the test board for LSM303DLHC \cite{LSM303board}}
	\label{fig:GY511_2}
\end{figure}

\begin{table}[H]
\centering
\begin{tabular}{|l|l|}
\hline
Number of channels & 3 x accel + 3 x magnet \\ \hline
Magnetic field scale        (gauss)     & ±1.3/±1.9/±2.5/±4.0/±4.7/±5.6/±8.1     \\ \hline
Linear acceleration scale     (G)         &  ±2/±4/±8/±16         \\ \hline
Data output (bits)          & 16                                               \\ \hline
Interface        & I2C                          \\ \hline
Analog supply voltage for the sensor (V) & 2.16-3.6                                      \\ \hline
Analog supply voltage for the board (V)     & 5         \\ \hline
Temperature range (°C)  & -40 to +85                                          \\ \hline
\end{tabular}
\caption{LSM303DLHC Specifications \cite{lsm303} }\label{tab:LSM303tab}
\end{table}

The schematic symbol that this sensor has is shown in figure \ref{fig:LSM303SCH}. It has 8 pins through hole, to place the test board shown in figure \ref{fig:GY511}. 

  
\begin{figure}[H]
	\centering
		\includegraphics[scale=0.4]{capitulo4/LSM303SCH.jpg}
	\caption{Schematic symbol for Testbed Board for LSM303DLHC}
	\label{fig:LSM303SCH}
\end{figure}

\subsubsection{Gyroscope and Accelerometer Sensor} \label{sssec:GyrAc}

In order to have the gyroscope sensor on the Testbed \acrshort{PCB}, we have include the MPU6050 sensor, which includes an gyroscope, an accelerometer and it is prepared to measure the temperature too. The MPU-6050 devices mixes a 3-axis gyroscope (3D) and a 3-axis accelerometer on the same chip. In addition, it includes Digital Motion Processor™ (DMP™), which processes complex 6-axis MotionFusion algorithms. \cite{MPUinfo}

As the LSM303DLHC sensor (see previous section \ref{sssec:MagAc}), we have used a test board for this sensor, GY-521 board in that case (see Figure \ref{fig:MPU6050photo}). It allow us a better soldering process and an easier prototype version to prove the different subsystems. The schematic of this board is shown in Figure \ref{fig:MPUsch}, where is represented the circuit necessary for the correct operation oh that chip, the MPU6050 sensor. 


\begin{figure}[H]
	\centering
		\includegraphics[scale=0.2]{capitulo4/GY-521.jpg}
	\caption{Photo of the test board for MPU6050}
	\label{fig:MPU6050photo}
\end{figure}

  
\begin{figure}[H]
	\centering
		\includegraphics[scale=0.4]{capitulo4/MPU6050SCH.jpg}
	\caption{Schematic of the test board for MPU6050 }
	\label{fig:MPUsch}
\end{figure}


And finally, the same as LSM303DLHC, it is placed on the Testbed \acrshort{PCB} an 8 pins schematic symbol (see figure \ref{fig:miMPU6050SCH}), for the placement of the test board shown in the Figure \ref{fig:MPU6050photo}.

  
\begin{figure}[H]
	\centering
		\includegraphics[scale=0.4]{capitulo4/miMPU6050SCH.jpg}
	\caption{Schematic of the test board for MPU6050 \cite{boardMPU}}
	\label{fig:miMPU6050SCH}
\end{figure}

The main characteristics of this sensor are the following:
\begin{table}[H]
\centering
\begin{tabular}{|l|l|}
\hline
Number of channels & 3 x accel + 3 x gyro \\ \hline
Gyroscope scale        (º/sec)     & ±250 ±500 ±1000 ±2000      \\ \hline
Linear acceleration scale     (G)         &  ±2/±4/±8/±16         \\ \hline
Data output (bits)          & 16                                               \\ \hline
Interface        & I2C                          \\ \hline
Analog supply voltage for the sensor (V) & 2.375–3.46                                      \\ \hline
Analog supply voltage for the board (V)     & 5         \\ \hline
Logic supply voltage (V)     & 1.71 to VDD (5)         \\ \hline
Temperature range (°C)  & -40 to +85                                          \\ \hline
\end{tabular}
\caption{MPU6050 Specifications \cite{MPU6050} }\label{tab:MPU6050tab}
\end{table}


For the communication with the microprocessor, I2C protocol is used, connected in parallel with the LSM303DLHC and the LCD described before. In the Figure \ref{fig:i2cdiagram} is shown how this connection is done and the directions that each device has. 

  
\begin{figure}[H]
	\centering
		\includegraphics[scale=0.35]{capitulo4/i2cdiagram.jpg}
	\caption{I2C diagram of the system}
	\label{fig:i2cdiagram}
\end{figure}

\subsubsection{Wireless Module} \label{sssec:xbee}

As we said in Section \ref{ssec:HWreq}, one of the most relevant requirements for the Testbed is the wireless communication. For that issue, the \acrshort{PCB} includes a specific device from Digi: Xbee Pro series 1 (see Figure \ref{fig:xbee1}. That device was chosen for the communication to the Testbed by Carlos Valenzuela, in his Final Project Degree about the Testbed communication. \cite{carlos}
  
\begin{figure}[H]
	\centering
		\includegraphics[scale=0.35]{capitulo4/xbee1.jpg}
	\caption{Module Xbee Pro Series 1}
	\label{fig:xbee1}
\end{figure}

From the datasheet of the device we can acquire the dimensions of the module Xbee PRO (in our case, the Series 1):

\begin{figure}[H]
	\centering
		\includegraphics[scale=1]{capitulo4/xbeedimensions.jpg}
	\caption{Dimensions of the Module Xbee Pro Series 1}
	\label{fig:xbeedimensions}
\end{figure}

Inside the family Xbee, we have selected the PRO Series 1, because of its good implementation to the Point-to-Point Link. The PRO selection is due to the improvements that it has in comparison with the Standard Series in respect of its range, sensibility and transmitted power. Moreover, the Series 2 of this family, has been designed for the communication in more complex networks, so for our link it will be inefficient. \cite{carlos}

The main features for this module are:

\begin{table}[H]
\centering
\begin{tabular}{|l|l|}
\hline
Dimensions (mm) & 324.38 x 32.94 \\ \hline
Operating Frequency        (GHz)     & 2.4     \\ \hline
RF Data Rate     (kbps)         &  250         \\ \hline
Indoor Range (m)          & Up to 90     \\ \hline
Antenna types        &  Chip, Wire, Whip, UFL, RPSMA                         \\ \hline
Supply Voltage (V) & 2.8-3.4                                      \\ \hline
Serial Interface     &  3.3V CMOS UART        \\ \hline
Logic supply voltage (V)     & 1.71 to VDD (5)         \\ \hline
Temperature range (°C)  & -40 to +85                                          \\ \hline
Transmit Current (mA)  & 150                                          \\ \hline
Receive Current (mA) & 55                                        \\ \hline
\end{tabular}
\caption{Xbee PRO Series S1  Specifications \cite{xbee} }\label{tab:XBEEtab}
\end{table}

 
\begin{figure}[H]
	\centering
		\includegraphics[scale=0.6]{capitulo4/xbeepinout.jpg}
	\caption{Pinout of the Xbee Module \cite{xbee}}
	\label{fig:xbeetable}
\end{figure}

In Figure \ref{fig:xbeetable} is shown how the pinout of this module is distributed. But, for our case, the only connections requires are: \cite{xbee}
\begin{itemize}
\item VCC: connected to 3.3V.
\item GND: connected to the circuit ground.
\item DOUT: connected to TXD of the microprocessor. There is a buffer in this line (SN74LVC1G125DCK) to adjust the levels between the Arduino and the Xbee Module \cite{carlos} \cite{74LVC}. 
\item DIN: connected to RXD of the microprocessor.
\end{itemize}

So, that connections were done in the Schematic of the project (see figure \ref{fig:xbeemio}). 

\begin{figure}[H]
	\centering
		\includegraphics[scale=0.7]{capitulo4/xbeemio.jpg}
	\caption{Schematic of the Xbee Module}
	\label{fig:xbeemio}
\end{figure}


On the other hand, if you want to upload the firmware of the module, the required connections would be: VCC, GND, DOUT, DIN, RTS \cite{xbee}, but we have an USB adapter to do this task, so we can save these connections.

Finally, in order to have more information about the status of the communication, some LEDs are placed. The resistors value, in this case for a VCC of 3.3V, were calculated like in the subsection \ref{sssec:micro}.

\begin{table}[H]
\centering
\begin{tabular}{|l|l|l|l}
\hline
						& \textbf{Green LED (TX and RX)} & \textbf{Yellow LED (RSSI)} & \textbf{Red LED (XBEE)} \\ \hline
Operating Voltage (V) & 2.1							& 2.1    & 2.1   \\ \hline
Datasheet Current (mA) & 20							& 20   & 20    \\ \hline
Measured Current (mA) & 17.4							& 16.5   & 21.43  \\ \hline
Theoretic Value Resistor ($\Omega$) & 68.966						& 72.727  & 55.996   \\ \hline
Normalized Value Resistor ($\Omega$) & 68						& 75 & 56   \\ \hline
\end{tabular}
\caption{Calculations of the resistors for the Xbee LEDs}\label{tab:XbeeLEDsvalue}
\end{table}
 
Again, the Equation \ref{eq:LED} was used for these calculations, but changing the $ V_{dd}$ voltage (now is 3.3V).

For more information about the Xbee module and the connection between the \acrshort{GS} and the Testbed platform, see the Master Thesis about this issue. \cite{carlos}

\subsubsection{Actuators Drivers} \label{sssec:drivers}

To control the actuators of \gls{GranaSAT} Testbed, there are placed on the \acrshort{PCB} two drivers, in this case the model L298P of STMicroelectronics, an \acrshort{SMD} dual full-bridge driver, in PowerSO20 package version. That device will accept standard TTL logic levels and drive inductive loads (in my case, DC Motors).\cite{L298P}

Each motor has 2 input signals to control the direction of the current, and an enable pin to enable or disable the motor independently of the input signals. Moreover, it has a current sense resistor to read the current consumption of each motor.

The pinout of this device is shown in Figure \ref{fig:pinoutdriver} and the function for each pin in the Figure \ref{fig:pinfunction}.

\begin{figure}[H]
	\centering
		\includegraphics[scale=0.25]{capitulo4/pinoutdriver.jpg}
	\caption{Pinout of the L298P \cite{L298P}}
	\label{fig:pinoutdriver}
\end{figure}

\begin{figure}[H]
	\centering
		\includegraphics[scale=0.5]{capitulo4/pinfunctions.jpg}
	\caption{Pin functions of the L298P \cite{L298P}}
	\label{fig:pinfunction}
\end{figure}

The schematic related to the drivers is shown in Figure \ref{fig:driversch}.

\begin{figure}[H]
	\centering
		\includegraphics[scale=0.7]{capitulo4/driversch.jpg}
	\caption{L298P Schematic \cite{L298P}}
	\label{fig:driversch}
\end{figure}

In order to protect the H-bridge of each driver, some protection diodes (Flyback Diodes) are placed in the outputs of the drivers. The reason of the protection is to create a low-impedance path for the coils (DC Motors) to discharge themselves through them. If they are not placed on the output, the discharge of the coils (stored magnetic energy), when the supply is stopped at each cycle, will generate arbitrarily a high reverse voltage across the \acrshort{MOSFET}s of the drivers, which can damage them. That diodes are the model S1M standard \acrshort{SMD}, from Fairchild \cite{S1M}.

That rectifier diodes have the following main features shown in Table \ref{tab:S1Mtab}:

\begin{table}[H]
\centering
\begin{tabular}{|l|l|}
\hline
Maximum Repetitive Reverse Voltage (V) & 1000 \\ \hline
Average Rectified Forward Current at $T_A = 100°C$       (A)     & 1     \\ \hline
Storage Temperature Range     (kbps)         &  -55 to +150         \\ \hline
Operating Junction Temperature (m)          & -55 to +150     \\ \hline
Forward Voltage at 1A (V)      &  1.1                         \\ \hline
\end{tabular}
\caption{Rectifier Diode S1M Specifications \cite{S1M} }\label{tab:S1Mtab}
\end{table}

The package that the S1M diode has is SMA/DO-214AC, see Figure \ref{fig:S1Mphoto}.

\begin{figure}[H]
	\centering
		\includegraphics[scale=0.3]{capitulo4/S1Mphoto.jpg}
	\caption{S1M package\cite{L298P}}
	\label{fig:S1Mphoto}
\end{figure}

Finally, to select the actuators (reaction wheels or magnetorquers), there are three switches between the output lines of the driver and the actuators. The switches are of the type DPDT (double pole double throw) and the model SS22SDP2 of NKK switches, and whose diagram, dimensions and a photo  can be seen in figure \ref{fig:switch}.


\begin{figure}[H]
\centering
\subfloat[Switch dimensions]{\label{fig:switchdimension} \includegraphics[width=70mm]{capitulo4/switchdimension.jpg}}
\subfloat[Switch diagram]{\label{fig:switchsch} \includegraphics[width=70mm]{capitulo4/switchsch.jpg}} //
\subfloat[Switch photo]{\label{fig:switchphoto} \includegraphics[width=60mm]{capitulo4/switch.jpg}} 
\caption{Switch NKK SS22SDP2} \label{fig:switch}
\end{figure}

\subsubsection{Power system} \label{sssec:powersystem}

The last section of the Schematic design for the Testbed \acrshort{PCB} of \gls{GranaSAT} project, is the power system. Schematics are shown in Figure \ref{fig:powersch}.
\begin{landscape}
\begin{figure}[H]
	\centering
		\includegraphics[scale=0.7]{capitulo4/powersch.jpg}
	\caption{Schematic for the power system of the Testbed}
	\label{fig:powersch}
\end{figure}
\end{landscape}

\paragraph{Battery} \label{battery}
The first part that is represented in the schematic is the battery. That battery (figure \ref{fig:battery}) is the same as used for the Cubesat \gls{GranaSAT}-I \cite{battery}, whose model is MGL2803 from Enix Energies. Some of the main characteristics that the battery has are shown in table \ref{tab:battery}. The battery is connected to the \acrshort{PCB} by a \acrshort{PCB} terminal connector placed on the bottom layer (see section \ref{ssec:PCBdesign}.

\begin{figure}[H]
	\centering
		\includegraphics[scale=0.7]{capitulo4/battery.jpg}
	\caption{ MP176065 battery}
	\label{fig:battery}
\end{figure}

\begin{table}[H]
\centering
\begin{tabular}{|l|l|}
\hline
Nominal voltage (V) & 3.75 \\ \hline
Nominal capacity 20°C (Ah)     & 6.8     \\ \hline
Technology            & Rechargeable Lithium Ion Battery Pack       \\ \hline
Discharge Cut off V Nominal (V)          & 2.8    \\ \hline
Maximum recommended discharge current (A)      &  5                       \\ \hline
Charge Voltage Nominal  (V)          & 4.2     \\ \hline
Charge Termination Current  (mA)          & 150    \\ \hline
Discharge temperature range (ºC)          & -10 to +60     \\ \hline
\end{tabular}
\caption{ENIX MGL2803 Battery Specifications \cite{S1M} }\label{tab:battery}
\end{table}

The battery charge will be described in the Chapter \ref{chap:chapter5}.

The following item in the circuit described before (figure \ref{fig:powersch}), is a jumper, which will turn on or off the whole system.

\paragraph{Voltage detectors} \label{detectors}

We have now 2 identical devices, MAX8211. That electronic device is a voltage monitor with programmable voltage detection. The application of the first one is detect the voltage that the battery has in each cycle, and notify to the microprocessor (see section \ref{sssec:micro}) when the voltage level was reach. For the second one, the objective is directly turn-off the whole system. 

The pinout of the device (in SO-8 package) is shown in the figure \ref{fig:MAX8211}.

\begin{figure}[H]
	\centering
		\includegraphics[scale=0.3]{capitulo4/MAX8211.jpg}
	\caption{ MAX8211 Voltage Detector}
	\label{fig:MAX8211}
\end{figure}

The schematic of the figure \ref{fig:powersch} shows three resistors for this device. The value of these resistors determine the value for the application of MAX8211. The calculations for the resistors for the first MAX8211 are represented in equations \ref{eq:MAX8211_1}, \ref{eq:MAX8211_2} and \ref{eq:MAX8211_3}
 First, it is necessary fix the value of the R20 resistor (see equation \ref{eq:MAX8211_1}).
\begin{equation}
R20=100k\Omega
\label{eq:MAX8211_1}
\end{equation}

\begin{equation}
R19= R20\times \frac{V_U - V_{TH}}{V_{TH}}=100k\Omega \times \frac{V_U - 1.15V}{1.15V}
\label{eq:MAX8211_2}
\end{equation}

\begin{equation}
R18= R19\times \frac{V_L - V_{TH}}{V_U - V_{L}}=R19 \times \frac{V_L - 1.15}{V_U - V_{L}}
\label{eq:MAX8211_3}
\end{equation}

Fixing the $V_U$=3.2V and $V_L$=2.4V values the resistors can be calculated. The values have to be changed to the closest normalised value. Where $V_U$ is the value that represents the start point once the battery have been discharged (when it is charging mode), and $V_L$ 
is the value from which the system turns off when it is in discharging mode.

The second MAX8211 is calculated in the same way, but with a more limiting values for the threshold voltages.


\paragraph{DC-DC converters} \label{DCconverters}

Finally, the power system has the voltage converters, in order to have 5V and 3.3V output with the 3.75V nominal input voltage from the battery.
 There are two different converters, one for each desired output. 
The first one is the MAX1771, from MAXIM. That step-up switching controller provides 90\% of efficiency over a 30mA to 2A load \cite{MAX1771}. It will be used the adjustable application for this device, in order to have the 5V in the output. 

The package mounted on the \acrshort{PCB} is SO-8, see figure \ref{fig:MAX1771}. In this figure it is shown the pinout for the MAX1771 too and in the figure \ref{fig:MAX1771pinfunction} the function of each pin.

\begin{figure}[H]
	\centering
		\includegraphics[scale=0.35]{capitulo4/MAX1771.jpg}
	\caption{ MAX1771 Step-up Converter}
	\label{fig:MAX1771}
\end{figure}


\begin{figure}[H]
	\centering
		\includegraphics[scale=0.6]{capitulo4/MAX1771pinfunction.jpg}
	\caption{ MAX1771 Pins function}
	\label{fig:MAX1771pinfunction}
\end{figure}

It is not necessary calculate the resistors values because there is an application circuit to supply 5V in the output with the adjustable way (Figure 5 of the Datasheet \cite{MAX1771}).

The second one is the LTC3440, from Linear technology. It is a Buck-Boost DC/DC Converter, which has the feature of having the fixed frequency operation with battery voltages above, below or equal to the output voltage, with a 96\% of efficiency. \cite{LTC}

The package chosen is the MSOP (10 leads), which has the pinout represented in the figure \ref{fig:LTC3440}.

\begin{figure}[H]
	\centering
		\includegraphics[scale=0.6]{capitulo4/LTC3440.jpg}
	\caption{ LTC3440 Pinout}
	\label{fig:LTC3440}
\end{figure}


\subsection{\acrshort{PCB} Design} \label{ssec:PCBdesign}
In this section, it will be presented the design process for the \acrshort{PCB} design of the circuit that has been explained in the section \ref{ssec:SCHdesign}. For the \acrshort{PCB} design, I have used Altium Designer, the same as the Schematic Design. That tool allows connect the schematic and \acrshort{PCB}, linking every component that you place on it. That design has some relevant steps which have to be followed in the right order for a proper \acrshort{PCB} design:

\begin{itemize}
\item \textbf{\acrshort{PCB} production Technology}. The first step that you have to take into account when build a \acrshort{PCB} is the production technology that you have available for the project. In my case, a prototyping \acrshort{PCB} machine was available thanks to the \acrshort{ECTD} (LPFK ProtoMat S62 \cite{LPFK}) for the \acrshort{PCB} production. So, the solution that we have available is a copper plate with the tracks, holes and vias made with drilling tools.
One of the most important features with this technology is the easy \acrshort{PCB} production in both sides, top and bottom layer and the drilling facilities that it has to make the holes and vias. The figure \ref{fig:LPFK} present the LPFK ProtoMat S62 that we have in the laboratory of the Faculty of Science of the University of Granada.

\begin{figure}[H]
	\centering
		\includegraphics[scale=0.6]{capitulo4/LPFK.jpg}
	\caption{LPFK ProtoMat S62}
	\label{fig:LPFK}
\end{figure}

\item \textbf{Mounting technology}. Once we know what type of technology is available for the \acrshort{PCB} production, we have to decide what mounting technology is the best solution for our design. The first issue that we have to consider is the minimum weight for the \acrshort{PCB}, because is one of the requirements (see section \ref{sec:requirements}), so the \acrshort{SMD} technology is the best solution. There are some devices which are not available in \acrshort{SMD}, so the \acrshort{PCB} will have \acrshort{SMD} and \acrshort{THT} technology.

\item \textbf{Package types}. Depending on the device chosen, different packages have been selected, trying to choose the smallest one because of the weight constraint. For the passive devices (resistors, capacitors, etc), in the majority of the cases, the size chosen was 0805 and 0603, but there are some 1206 and 0402, subject to the power consumption of each resistor and the maximum voltage for the capacitors.

\item \textbf{\acrshort{PCB} Library}. As soon as we have selected the components and their package it is time to make the PCB library. The library will include the footprints of every component placed on the \acrshort{PCB} and it will be created with the packages found in other libraries or made by the designer (measuring the device with a caliber or taking the measurements from the datasheet information).


\item \textbf{Transfering the components from the \acrshort{SCH} to the \acrshort{PCB}}. When the \acrshort{PCB} library was finished, the schematic should have every component linked to its footprint. The next step is transfer that information to a PCB, in order to start with the placing and the routing of the footprints. That step will create the netlists for the PCB, which will indicate the connections between the components in order to facilitate the next step.

\item \textbf{Placing and routing the \acrshort{PCB} components}. First, a good placing will facilitate the routing of the components(trace the tracks that connect the components). 

\item \textbf{Verify the \acrshort{PCB} design rules and connections}. Altium Designer has a really useful tool: \textit{Design Rule Check} (see figure \ref{fig:altium}). That tool have the objective of test if the rules for the design are being fulfilled. That rules will be explained in the section \ref{sssec:rulesaltium}.

\begin{figure}[H]
	\centering
		\includegraphics[scale=0.6]{capitulo4/altium.jpg}
	\caption{Design Rule Check -  Altium Designer}
	\label{fig:altium}
\end{figure}

\end{itemize}


\subsubsection{Design rules \acrshort{PCB}} \label{sssec:rulesaltium}

The ProtoMat S62 (PCB prototype machine) has some mechanical requirements that we have to take into account. The affected \acrshort{PCB} characteristics for that requirements are:

\begin{itemize}

\item \textbf{Track}. In every moment, the best option was the maximunm width possible for the tracks, in order to reduce the resistance of the track and allow a better operation for the \acrshort{PCB}, overall, for the power tracks. The minimum width for the tracks has been 15 mils, because of the operation mode that has the plotter, in order to have a minimum final width of the 10 mils, so the production phase reduces 5 mils approximately from the design. 
Moreover, it should be avoid the 90º angles, and the shortest tracks possible.

\item \textbf{Clearance}. The clearance is the minimum spacing between the tracks on a \acrshort{PCB}. The clearance should be as minimum 10 mils, in order to avoid the short-circuit between the tracks, because of the production errors that it can exist. Anyway, once the \acrshort{PCB} was ended, with a cutter tool you can remark the tracks in order to avoid the shavings generated by the production phase.

\item \textbf{Hole size}. The constraint of the hole size is imposed by the available drilling tools that we have in the laboratory. The minimum size fixed was 0.6 mm, and the available tools were: 0.6, 0.7, 0.8, 0.9, 1, 1.1, 1.3 y 1.5 mm. The biggest holes were made with a milling cutter drilling the perimeter of the hole.


\item \textbf{Annular ring}. The annular ring is the copper portion between the hole and the perimeter of the pad. That rule is the same for the vias and \acrshort{THT} pads. The minimum diameter of the annular ring is 80 mils, in order to have a good support for the soldering of this lead or via.

\item \textbf{Text layer}. The production type that we have available has not the silkscreen process. With the intection to give more information of the \acrshort{PCB}, a text tool (in both sides) has been created, on the ground plane, to avoid the short-circuits and the errors in the tracks. 

\item \textbf{Board Outline}. One of the most important things of the PCB is its dimension and shape. Because of the hemisphere chosen (see section \ref{sec:mechanicaldesign}), the PCB had to have circular shape, and the dimension was designed taking into account that the battery had to fit between the PCB and the lower part of the hemisphere.

\end{itemize}

\subsubsection{2D view of the \acrshort{PCB} Design} \label{sssec:2Ddesign}
The dimensions of the PCB were predefined once we had the semisphere od the Testbed. The board had to be 

In the following pages will be located the \acrshort{PCB} Design Schematics, where is described the tracks, vias, holes, footprints, and the position of each component. In this project, there are four \acrshort{PCB}s of the following three types:

\begin{itemize}
\item Testbed \acrshort{PCB}: Is the main \acrshort{PCB} of the project, where the most components described in the subsection \ref{ssec:SCHdesign} are placed. That PCB has component in top and bottom layer, so it will be included the PDF of both sides.
\item Motor \acrshort{PCB}: That \acrshort{PCB} will be used for support the motors in vertical position (for axes X and Y). There will be two units for that type of \acrshort{PCB}, one for each motor, and the third motor will be placed on the center of the Testbed \acrshort{PCB}, to control the Z axis (see section \ref{sec:coordinate}. It will be sold to the Testbed \acrshort{PCB} to ensure the correct support and connect it with a pad located on that hole. The motors placed on that \acrshort{PCB} have been explaines on section \ref{sec:actuators}. 
\item Magnetorquers \acrshort{PCB}: The function of this \acrshort{PCB} is only support the magnetorquers mounted in a Cubesat structure 3D printed.

\end{itemize}


\includepdf[fitpaper,pages=-,link=true,landscape,linkname=gant]{capitulo4/PCB1}
\includepdf[pages=-,link=true,landscape,linkname=gant]{capitulo4/PCB2}
\includepdf[pages=-,link=true,landscape,linkname=gant]{capitulo4/PCB3}
\includepdf[pages=-,link=true,landscape,linkname=gant]{capitulo4/PCB4}


\subsubsection{3D view of the \acrshort{PCB} Design} \label{sssec:3Ddesign}

In this section, the \acrshort{PCB}s shown in the section \ref{sssec:2Ddesign} will be represented in 3D view. That models will be used as reference in order to have an idea of the real size and aspect of the final \acrshort{PCB} implementation. 

That task has been carried out linking each footprint to a 3D model. The 3D models have been downloaded from \url{http://www.3dcontentcentral.es/} but some of then have been designed for the author of this project in Solidworks. 

%% PRINTING DOCUMENT
 
The figure show the 3D model of this project, one from each side of the \acrshort{PCB}.


\begin{figure}[H]
	\centering
		\includegraphics[scale=0.6]{capitulo4/PCB3D1.jpg}
	\caption{3D View of the top side}
	\label{fig:PCB3D1}
\end{figure}

\begin{figure}[H]
	\centering
		\includegraphics[scale=0.6]{capitulo4/PCB3D2.jpg}
	\caption{3D View of the bottom side}
	\label{fig:PCB3D2}
\end{figure}

\begin{figure}[H]
	\centering
		\includegraphics[scale=0.6]{capitulo4/PCB3D3.jpg}
	\caption{3D View of the right side}
	\label{fig:PCB3D3}
\end{figure}

\begin{figure}[H]
	\centering
		\includegraphics[scale=0.6]{capitulo4/PCB3D4.jpg}
	\caption{3D View of the left side}
	\label{fig:PCB3D4}
\end{figure}

This \url{https://www.youtube.com/watch?v=aF2nNklm4gU} have a video of the 3D view of that PCB.

%% PDF DOCUMENT
%\newcommand{\includemovie}[3]{%
%\includemedia[width=#1,height=#2,activate=pagevisible,deactivate=pageclose,addresource=#3,flashvars={src=#3 % same path as in addresource!
%&autoPlay=true % default: false; if =true, automatically starts playback after activation (see option ‘activation)’
%&loop=true % if loop=true, media is played in a loop
%&controlBarAutoHideTimeout=0 %  time span before auto-hide
%}]{}{capitulo4/PCB_Project.swf}%

%En el cuerpo
%\includemedia[<options>]{<poster text>}{
% <main Flash (SWF) file or URL | 3D (PRC, U3D) file>}
%  options: opciones
%   flashvars: especificaciones del entorno empleado
%  poster text: texto en lugar del video
%  main Flash (SWF) file or URL: nombre del entorno del video
%   (StrobeMediaPlayback.swf, VPlayer.swf, APlayer.swf)
%  Ejemplo insertando video mp4
%\includemedia[
  %activate=pageopen,
  %width=200pt,height=170pt,
  %addresource=example.mp4,
  %flashvars={%
%src=example
%&scaleMode=stretch}
%]{}{PCB_Project.swf}

\section{Mechanical Design} \label{sec:mechanicaldesign}

The first step in the mechanical design was acquire the hemisphere where the electronics were located, because that dimension was the decisive for the air-bearing, in order to perfectly fit with it, and for the PCB design (see section \ref{sssec:rulesaltium}).

In order to have a better design and improve the quality of the 3D model, a good option is to make the 3D model in Solidworks of the bought hemisphere, and adapt the air-bearing to it.

\subsection{Hemisphere} \label{ssec:hemisphere}

The hemisphere chosen have the real use for decoration, but it has the necessary characteristics for the correct operation in our project. 

The model is a fillable hollow acrylic ball with the hang tabs (see figure \ref{fig:hemisphere}), from Complex Plastics. It is a sphere divided in two hemispheres, using only one for the Testbed.

The hemisphere has the dimensions shown in the schematic of the mechanical design of this part in the following page.


\begin{figure}[H]
	\centering
		\includegraphics[scale=0.6]{capitulo4/balloriginal.jpg}
	\caption{Clear hemisphere}
	\label{fig:hemisphere}
\end{figure}


\includepdf[pages=-,link=true,landscape,linkname=gant]{capitulo4/hemisphere}

\subsection{Air-bearing} \label{ssec:airbearing}

The air-bearing is designed with the intention to print it in a 3D printer (see Appendix \ref{cap:printprocess}, where the printing process is described). 

The model that we have studied and adapted to our project is the FloatSat design, analyzed in chapter \ref{chap:chapter2}, sectiOOOOOOOOON, using the reverse engineering process.

Once the design was though, the next step is transfer that design to SolidWorks, in order to create a 3D model to print it in the 3D printers that we have in the \acrshort{ECTD} laboratories. The following pages show the schematics of the air-bearing design, with the most relevant dimensions and points of view.

\includepdf[pages=-,link=true,landscape,linkname=air1]{capitulo4/Airbearing1.pdf}
\includepdf[pages=-,link=true,landscape]{capitulo4/Airbearing2.pdf}
\includepdf[pages=-,link=true,landscape,linkname=air3]{capitulo4/Airbearing3.pdf}
\includepdf[pages=-,link=true,landscape,linkname=air3]{capitulo4/Airbearing4.pdf}

%PDF VERSION
The interactive 3D model, to have a better point of view of the design.
%\includemovie[
	%poster,
	%toolbar, %same as `controls'
	%label=airbearing.u3d,
	%text=(airbearing.u3d),
	%3Daac=60.000000, 3Droll=0.000000, 3Dc2c=-10.000000 -44.639999 -10.000000, 3Droo=46.826591, 3Dcoo=-10.000000 10.000000 10.000000, 3Dlights=CAD,
%]{15cm}{15cm}{capitulo4/airbearing.u3d}


For the design of the air-bearing, the most relevant limitation that exists is the building process, because of the technology that we are using, 3D printers. The printer works with \acrshort{FFF} depositing a filament of a certain material (\acrshort{ABS} in our case) on above the same material, in order to create a joint by heat and/or adhesion \cite{FFF}. Because of this way to build the models (from bottom to top of the model), there are some restrictions in the fabrication to control that the next layer is always on top of each other to form the object.

The triangles that shows the second schematic (section A of the air-bearing) and in figure \ref{fig:triangles}, are the support to avoid that mechanical problems. The calculation of the dimension of each triangle is made with some practical tests measuring how long can extrude the printer without support, depending on the extruder speed and the temperature for the \acrshort{ABS}. 


\begin{figure}[H]
	\centering
		\includegraphics[scale=0.6]{capitulo4/triangles.pdf}
	\caption{Air-bearing section to show the support triangles}
	\label{fig:triangles}
\end{figure}

\subsection{Reaction Wheels} \label{ssec:reactionwheels}

When the reaction wheels are designed for a Cubesat, there are two main limitations, volume and mass. In this case, we are going to use some limitations too, in order to have the most similar possible design for the future cubesat GranaSAT, but using the choice of the material instead of the mass, because of the limitations in cost and availability of the material.

The material chosen is the bronze, because it has a density of $8890 Kg/m^3$, one of the highest densities that we could acquire (in sectiOOOON there is a comparation of the different materials).

The second limitation is the volume, the dimension of the wheel. That limitation was fixed for two variables, the availability of the different sizes that we could buy and the dimension of the cubesat that we will build in a future(see section \ref{sec:Cubesat}), because a 1U-Cubesat has $10\times10\times10 cm^3$ and weights about 1Kg.

In order to name each dimension of the reaction wheel, the following page shows the parameters that are used for the design. 

\includepdf[pages=-,link=true,landscape,linkname=wheel]{capitulo4/wheelparameters.pdf}


The dimensions chosen are the shown in table \ref{tab:dimensionwheel}.

\begin{table}[H]
\centering
\begin{tabular}{|l|l|l|}
\hline\hline
\textbf{Parameter}		 & \textbf{Designed model}	& \textbf{Measured built model}\\ \hline
Rring (mm) 		& 18 & 17.95 \\ \hline
Rdisk (mm)    & 12.2  & 12.2      \\ \hline
Hring (mm)     & 10 & 10        \\ \hline
Hdisk (mm)     & 2 & 2.03    \\ \hline
Rhole (mm)     & 0.7 & 1.15    \\\hline\hline
\end{tabular}
\caption{Reaction wheels dimensions}\label{tab:dimensionwheel}
\end{table}
The measured dimension were made with a caliber with accuracy of 0.01mm, because the designed model had some differences with the build model (made by the \acrshort{CIC} of the \acrshort{UGR}, especially in the Rhole parameter, so in the implementation process was required a special glue to fix the reaction wheel to the motor axis. The assembly of the wheel with the motor is shown in the figure \ref{fig:wheelassembly}.
\begin{figure}[H]
	\centering
		\includegraphics[scale=0.6]{capitulo4/wheelassembly2.pdf}
	\caption{Assembly wheel-motor}
	\label{fig:wheelassembly}
\end{figure}
For the calculation of the wheel mass, the equations \ref{eq:calculemass} are used:

\begin{equation}
m_{total} =  m_{disk}+ m_{ring}- m_{hole}= \underbrace{\rho\pi R^2_{disk} H_{disk}}_{m_{disk}} + \underbrace{\rho\pi (R^2_{ring}-R^2_{disk}) H_{ring}}_{m_{ring}} - \underbrace{\rho\pi R^2_{hole} H_{disk}}_{m_{hole}} 
\label{eq:calculemass}
\end{equation}

Where $\rho$ is the density of the material chosen, in this case bronze with a density of $8890 Kg/m^3$ approximately, $m_{total}$ is the total mass of the reaction wheel, divided in $m_{disk}$(the mass of the disk), $m_{ring}$ the ring mass and finally the subtraction of the hole mass, $m_{hole}$. The equation \ref{eq:calculemass} can be solved now with the dimensions measured and compared with the mass measured with a balance graduated in grams. The table \ref{tab:masswheel} shows the comparative of this measurements.  

\begin{table}[H]
\centering
\begin{tabular}{|l|l|l|l|l|}
\hline\hline
\textbf{Parameter}		 & $m_{disk}$ (g) & $m_{ring}$ (g) & $m_{hole}$ (g) \textbf{$m_{total}$ (g)}	\\ \hline
Design calculation mass	& 8.31 & 48.92 & 0.16 & 57.06 \\ \hline
Build calculation mass   & 8.43 & 48.41 & 0.44 & 56.40        \\ \hline
Build measured mass    & - & - & - & 57       \\ \hline\hline
\end{tabular}
\caption{Reaction wheels mass}\label{tab:masswheel}
\end{table}

The difference between the measured mass and the calculation mass can be caused by the approximation done in the density, because the bronze is a copper alloy, and the value of its density may differ. Moreover, the balance only has grams accuracy, so maybe the value is 
rounded.

As the table \ref{tab:masswheel} prove, the majority of the mass in the reaction wheel is in the ring of the wheel. The reason of that design is that, with this characteristics, we will have a better inertia momentum with the minimum weight possible.

With the mass calculation we could acquire the inertial momentum that each designed wheel generates, see equations \ref{eq:Idisk}, \ref{eq:Iring} and \ref{eq:inertial}. That equations will be calculated with the mass of the calculation build model.

\begin{equation}
I_{disk} =  \frac{(m_{disk}-m_{hole})R^2_{disk}}{2}= 0.628\times 10^{-6} Kg m^2
\label{eq:Idisk}
\end{equation}

\begin{equation}
I_{ring} =  \frac{m_{ring}(R^2_{disk}+R^2_{ring})}{2}= 11.403\times 10^{-6} Kg m^2
\label{eq:Iring}
\end{equation}

\begin{equation}
I_{total} = I_{disk}+I_{ring}= 12.031\times 10^{-6} Kg m^2
\label{eq:inertial}
\end{equation}

As we said before, the most of the inertial momentum is located in the wheel ring.

In the next chapter, when the characterization of the motors was made, it will be calculated the maximum angular momentum with that values of inertial momentum.


\subsection{Assembly of the whole parts} \label{ssec:airbearing}

In this section, will be represented the whole 3D model design of the project, placing every part in its place.

The figure \ref{fig:totalmechanic} shows how the system is assembled, with every 3D model mounted in its correct position. 

\begin{figure}[H]
	\centering
		\includegraphics[scale=0.6]{capitulo4/totalmechanic.pdf}
	\caption{Testbed assembly}
	\label{fig:totalmechanic}
\end{figure}

%%printing version
There is a assembly video of the GranaSAT Testbed in this \url{https://www.youtube.com/watch?v=mojIS9s4XxM}.

%pdf version
%\includemedia[
  %activate=onclick,
  %width=500pt,height=490pt,
  %addresource=capitulo4/videoassembly.mp4,
  %flashvars={source=capitulo4/videoassembly.mp4}
%]{}{VPlayer9.swf}

For the assembly of Testbed \acrshort{PCB} and Magnetorquers \acrshort{PCB} four M4 screws have been used, as the figure \ref{fig:totalmechanic} shows.

\section{Software Design} \label{sec:softwaredesign}

In this section will be studied the necessary software design for the system. That design has two parts: the onboard software, and the \acrshort{GS} software. Both systems have to be synchronized in order to work properly, so the first issue that we have to explain is the communication protocol used.

\subsection{Communication protocol} \label{ssec:protocol}
The communication protocol was designed for Carlos Valenzuela Morales \cite{carlos}, another GranaSAT team member, anyway here is a summary of the main characteristics.

The first thing that we have to take into account is the quantity of data that we want to send to the \acrshort{GS}. That protocol is not only  prepared for the Testbed, but also for the Cubesat systems too, so the rest of the GranaSAT team have included their data in this transmission. Each parameter has the data type of the Arduino code more efficient, using the minimum bytes for each value. The table \ref{tab:bits} shows what type of data can be used in that protocol, but for the GranaSAT Testbed we will only use float and int type.


\begin{table}[H]
\centering
\begin{tabular}{|l|l|l|}
\hline\hline
\textbf{Data type (Arduino code)}		 & \textbf{Description}	& \textbf{Range}\\ \hline
byte		& unsigned, 8 bits & [0, 255] \\ \hline
char   & signed, 8 bits  & [-128, 127]     \\ \hline
int    & signed, 16 bits & [-32767, 32767]       \\ \hline
unsigned int     & unsigned, 16 bits & [0, 65535]    \\ \hline
unsigned long    & unsigned, 32 bits & [0, 4294967295 ]  \\ \hline
unsigned int (12)    & unsigned, 12 bits & [0, 4096]    \\ \hline
float    & signed, 32 bits & [-3.4028235E38 , 3.4028235E38]    \\ \hline
\hline
\end{tabular}
\caption{Data type used \cite{carlos}}\label{tab:bits}
\end{table}

In order to receive the data correctly, the transmission of the parameters should have an order in both systems, and it is used two delimiters for the sent frame. Moreover, it will have a CRC to detect the errors in each frame. Knowing the start and the end of the frame, will facilitate the read of the data and the placing of the values in the dashboard of the \acrshort{GS}.

The error control is made with the protocol 804.15.4 of the wireless communication, Xbee module. The errors correction is made in MAC level, with the ACK confirmations from the receptor \cite{ieeeprotocol}.When the maximum number of times to retransmit the frame without receive the ACK from the receptor occurs, that frame will be incorrigible. The CRC will act in that cases.

The figure \ref{fig:flowdiagram} shows the flow diagram designed for that protocol.

%\includepdf[pages=-,link=true,linkname=flow]{capitulo4/FLOW.pdf} 
\begin{figure}[H]
	\centering
		\includegraphics[scale=0.55]{capitulo4/FLOW.pdf}
	\caption{Flow diagram}
	\label{fig:flowdiagram}
\end{figure}

 Now, each phase of this communication protocol is described in the following enumeration:

\begin{itemize}
\item \textbf{TURN ON}. Each system has to be turned on to start the protocol. The \acrshort{GS} will be a \acrshort{PC}
 connected to a Xbee Module, and the Testbed will be connected to the battery to start the process flow.
\item \textbf{START CONNECTION}. Both Xbee modules (\acrshort{GS} and Testbed) are connected to each other. That function is in the \acrshort{GS} side of the flow diagram but is common to both systems at the same time.
\item \textbf{START TELEMETRY}. In the dashboard of the \acrshort{GS} the reception is started clicking in a button. Before of that phase, the Testbed is sending data but the receptor is not working yet. When the telemetry is started, the \acrshort{GS} wait until that the DELIMITER START is received.
\item \textbf{START TELECONTROL}. Exist a button to turn on the telecontrol. When the telecontrol is activated, a message from the \acrshort{GS} to the Testbed is sent to inform of that action, and the delimiter is sent to start the telecontrol data transmission.
\item \textbf{ACTUATORS CONTROL}. When the telecontrol commands are received in the Testbed, the microprocessor prepares and sends the parameters necessary to control the actuators. Now, the testbed start to send the next frame of telemetry.
\item \textbf{STOP TELEMETRY/TELECONTROL}. The transmission of the data is stopped in this phase, clicking on the button STOP in the dashboard of the \acrshort{GS}.
\end{itemize}


\subsection{On-board software} \label{ssec:onboard} 
On-board Software is controlled by the microprocessor in the Testbed \acrshort{PCB} and it has the following functions:
\begin{itemize}
\item Read the data sensors.
\item Display the values chosen in the \acrshort{LCD}.
\item Control the actuators.
\item Transmit the data to the \acrshort{GS} with wireless communication.
\end{itemize}

In order to have a more clear description of the software design, the state diagram of the on-board software is shown in the figure \ref{fig:onboardflow}.
\begin{figure}[H]
	\centering
		\includegraphics[scale=0.7]{capitulo4/onboardflow.pdf}
	\caption{On-board State diagram}
	\label{fig:onboardflow}
\end{figure}

In that state diagram we have five different states, which are going to be described now:
\begin{itemize}
\item \textbf{START}. In that state, the system is waiting to be turned on. It will only be occupied the first one time.
\item \textbf{TELEMETRY}. When the system is in that state, it will be measuring the data from the sensors until the reading process is ended and the system will have to decide if the next state is TELECONTROL or ESTIMATION, with a telecommand sent from the \acrshort{GS} (see \ref{fig:flowdiagram}). When telemetry state ends, the system send the data measured to the \acrshort{GS} through the wireless link created with the Xbee modules.
\item \textbf{TELECONTROL}. If the Telecontrol command from the \acrshort{GS} is ON, the TELECONTROL state will be the next in the state diagram, where the system will receive the data from the \acrshort{PC}. The system will stay here until the telecontrol data is completely received.
\item \textbf{ESTIMATION}. The estimation state will calculate the error of the system, using the \acrshort{PID} controller. If the last state is TELECONTROL, it will have the telecontrol commands to calculate it, however, if the system comes from the TELEMETRY state, the determination of the error will be calculated with the default values.
\item \textbf{ACTUATOR}. Finally, when the error is calculated, that state is the responsible for the control of the actuators. It will send the necessary commands and the system will come back to the TELEMETRY state, making a loop.
\end{itemize}



\subsection{\acrshort{GS} software} \label{ssec:ground} 
In the \acrshort{GS} of our project we will have a graphic interface, a dashboard, showing the telemetry values received from the Testbed and sending the telecontrol commands there. That graphic interface will be developed with Matlab, where the values will be shown in different tabs where the data will be organized and the configuration for the radio link can be changed. That dashboard was firstly designed by Carlos Valenzuela \cite{carlos}, but now the design of another new tab was necessary for the telecontrol of the Testbed, and the modifications of the values of the sensors chosen. 


TIENE QUE SER ACABADOOOOOOOO


