\section{Optimizing resistor divider for LDR-based sun sensor output}

	\begin{figure}[H] 
				\centering
				\includegraphics[width=60mm]{figurastfm/Chapter3/Imagenes/VoltageDividerLDR.png}
				\caption{Voltage divider for LDR-based sun sensor output (provisional)}      		
				\label{fig:dividerldr}
  		\end{figure}


The sun sensor will be implemented as an LDR connected to and ADC through a voltage divider. That voltage divider should maximize excursion output, in order to get the most resolution possible from the 12-bit ADC. In order to achieve that, resistor divider must be adequately optimized. In the diagram above, LDR is denoted as $R^*$ and R is another resistance completing the voltage divider. Depending on the light falling upon the LDR, $R^*$ varies, so does the voltage $U$. That voltage is directly connected to the ADC so it is converted into a discrete value. With this circuit, $U$ will increase when light level is high (facing sun) whereas will decrease when light reduces (eclipse).

Ideally, $U$ voltage should go through the range $[0 V,U_0]$ to maximize distinguishable values. Considering that the only degree of freedom is given by $R$ value, it must be the point of the optimization. On the other hand, extreme resistance values of the LDR must be known so the whole resistance of the voltage divider can be calculated. Therefore, for a certain resistance $R^* \in [R^*_{min}, R^*_{max}]$:

\begin{align}
U(R)=U_0 \frac{R}{R^* + R}
\end{align}

Since the aim is maximizing the voltage margin for $U$, it will be denoted as $\Delta U(R)$, which is the electric potential differential across $R^*$:

% <![CDATA[
\begin{align*}
\Delta U(R) &= U_{max}(R) - U_{min}(R) \\
&= U_0 \Bigg[\frac{R}{R^*_{min} + R} - \frac{R}{R^*_{max} + R} \Bigg] \\
&= U_0 \Bigg[\frac{R(R^*_{max} - R^*_{min})}{(R^*_{min} + R)(R^*_{max} + R)} \Bigg] \\
\end{align*} %]]>


In order to maximize the function $\Delta U(R)$, the derivative $\frac{\partial}{\partial R} \Delta U $ is calculated:

% <![CDATA[
\begin{align*}
\centering
\frac{\partial}{\partial R} \Delta U &= U_0\Bigg[\frac{(R^*_{min} - R^*_{max})\big[R^2-R^*_{min}R^*_{max}\big]}{(R^*_{min} + R)^2(R^*_{max} + R)^2}\Bigg]
\end{align*} %]]>

Then, it is solved $\frac{\partial}{\partial R} \Delta U = 0$ to find extreme point or extrema:

% <![CDATA[
\begin{align*}
\centering
U_0\Bigg[\frac{(R^*_{min} - R^*_{max})\big[R^2-R^*_{min}R^*_{max}\big]}{(R^*_{min} + R)^2(R^*_{max} + R)^2}\Bigg] =  0 \rightarrow 
(R^*_{min} - R^*_{max})\big[R^2-R^*_{min}R^*_{max}\big] =  0
\end{align*} %]]>

Assuming $R^*_{max}>R^*_{min}$, solving the last equation yields to:

% <![CDATA[	
\begin{align*}
R &= \sqrt{R^*_{min}R^*_{max}} \\
\end{align*} %]]>


Therefore, the optimum resistance in order to get high dynamic range in output voltage is given by the square root of the product of the LDR extreme values. The following figure shows the behaviour of the function $\Delta U(R)$ depending on the extreme resistance values of the photoresistor. Blue series corresponds to the currently available resistor, with an optimum resistance $ R = 1 k \Omega $


	
  	\begin{figure}[H]
			\centering
			\includegraphics[scale=0.75]{FigurasTFM/Chapter3/PDF/comparative.pdf}
			\caption{$\Delta{U}$ ratio for different LDR extreme values}      		
		\end{figure}